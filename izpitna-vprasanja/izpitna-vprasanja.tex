\documentclass{article}

\usepackage[T1]{fontenc} \usepackage[utf8]{inputenc} \usepackage[slovene]{babel}
\usepackage[margin=1cm]{geometry} \usepackage{times}
\usepackage{amssymb}

\parindent=0pt

\setlength{\fboxsep}{2ex}

\newcommand{\vprasanje}[2] {\framebox[\textwidth][l]{\parbox[t][3.5cm][c]{0.8\textwidth}{\medskip \LARGE \textbf{#1:} \raggedright #2
\medskip}}\par\medskip}

\pagestyle{empty}

\begin{document}
\vprasanje{Osnovno o preslikavah}{domena, kodomena, prirejanje, funkcijski predpis, vezane
spremenljivke, identiteta, kompozitum, funkcijske relacije, eksponentna množica}

\vprasanje{Osnovne konstrukcije množic}{končne množice, produkt, vsota, eksponent,
``aritmetična'' pravila za množice}

\vprasanje{Izomorfizmi}{definicija, inverz, ``aritmetična pravila'' za množice, povezava z
bijekcijami, izomorfnost potenčne množice in množice karakterističnih preslikav}

\vprasanje{Izjavni račun in Boolova algebra}{osnovni vezniki, pravila Boolove algebre,
tavtologija, zakon o izključeni tretji možnosti}

\vprasanje{Kvantifikatorja in logika prvega reda}{Univerzalni in eksistenčni
kvantifikator, kvantifikacija po prazni množici in po enojcu, vezane spremenljivke,
kvantifikatorji in negacija, kvantifikatorji in ostali vezniki}

\vprasanje{Dokazovanje izjav}{pravila dokazovanja, dokaz s protislovjem, zakon o zamenjavi
enakih elementov, zakon o zamenjavih ekvivalentnih izjav}

\vprasanje{Defincije in enolični opis}{definicija kot okrajšava, kvantifikator ``obstaja
natanko en'', zapis $\iota x . \phi(x)$}

\vprasanje{Podmnožice, karakteristične funkcije in potenčna množica}{Pojem podmnožice,
razmerje med podmnožicami in karakerističnimi preslikavami, potenčna množica, Cantorjev
izrek}

\vprasanje{Množice in razredi}{Russellov paradoks, pojem množice in razreda, pravi razred,
primeri pravih razredov}

\vprasanje{Družine množic}{Pojem družine, kartezični produkt družine, vsota družine, unija
in presek družine, pravila za računanje z njimi, aksom izbire, $\times$ kot $\Sigma$,
$\to$ kot $\Pi$}

\vprasanje{Slike in praslike}{definicija slike in praslike, slika in praslika kompozituma,
slika in praslika unije in preseka, pravila, ki veljajo za slike in praslike}

\vprasanje{Osnovne lastnosti preslikav}{injektivne, surjektivne, monomorfizmi,
epimorfizmi, bijekcije, izomorfizmi, zveza med njimi}

\vprasanje{Retrakcije in prerezi}{definicija, lastnosti retrakcije, lastnosti prereza, ali
ima vsaka surjektivna preslikava prerez in obratno}

\vprasanje{Aksiom izbire}{kaj pravi aksiom, osnovne posledice, Zornova lema, princip dobre
urejenosti, druge uporabe}

\vprasanje{Osnovne lastnosti relacij in operacije na njih}{relfeksivne, tranzitivne,
simetrične, sovisne, itd., osnovne operacije, kompozitum relacij, ovojnice relacij}

\vprasanje{Preslikave kot funkcijske relacije}{definicija, zveza med preslikavami in
funkcijskimi relacijami.}

\vprasanje{Ekvivalenčne relacije in kvocientne množice}{ekvivalenčna relacija,
ekvivalenčni razred, kvocientna množica, definicija preslikave na kvocientni domeni}

\vprasanje{Osnovno o strukturah urejenosti}{šibka ureditev, delna ureditev, linearna
  ureditev, zgornja in spodnja meja, supremum, maksimum, minimalni in maksimalni element,
  prvi in zadnji element, monotona preslikava}

\vprasanje{Indukcija in dobre ureditve}{Indukcija na $\mathbb{N}$, stroga ureditev, zveza
  z delno ureditvijo, progresivna množica, dobro osnovane ureditve, dobro urejena
  ureditev, primeri, karakterizacija dobre ureditve}

\vprasanje{Moč množic}{Končna množica, moč končne množice, primerjava med močmi množic z
  injektivimi in s surjektivnimi preslikavami, neskončna množica, kadrinalna števila,
  Cantorjev izrek, Cantor-Schröder-Bernsteinov izrek}

\vprasanje{Števne in neštevne množice}{Končne množice, števne množice, neštevne množice,
  $\aleph_0$, primeri števnih in neštevnih množic}

\vprasanje{Kumulativna hierarhija}{kodiranje matematičnih objektov z množicam, kumulativna
  hierarhija, števila kot množice, ordinalna števila.}

\vprasanje{Aksiomi teorije množic}{Zermelo-Fraenkelovi aksiomi teorije množic.}

\end{document}