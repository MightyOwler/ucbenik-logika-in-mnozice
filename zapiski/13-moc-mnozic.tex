\chapter{Aksiom izbire}

\section{Odvisna izbira}

V dokazu o karakterizaciji dobro osnovanih relacij smo uporabili

**Aksiom odvisne izbire:**
Naj bo `A` neprazna množica in `R ⊆ A × A` celovita relacija, t.j.,

    ∀ x ∈ A . ∃ y ∈ A . x R y.

Tedaj obstaja zaporedje `a : N → A`, da za vse `n ∈ N` velja `a(n) R a(n+1)`.

\section{Aksiom izbire}

Aksiom odvisne izbire sledi iz aksioma izbire (tega ne bomo dokazali):

**Aksiom izbire (AC):** Vsaka družina nepraznih množic ima funkcijo izbire.

Povedano z drugimi besedami:

* formulacija AC: družina nepraznih množic ima funkcijo izbire
* vsaka surjekcija ima prerez ⇔ AC
* propaganda: [The Banach–Tarski Paradox](https://www.youtube.com/watch?v=s86-Z-CbaHA)


\chapter{Moč množic}

\section{Končne množice}

**Definicija:** *Standardna* končna množica z `n` elementi je

    [n] = {k ∈ N | k < n}

Torej:

    [0] = {}
    [1] = {0}
    [2] = {0, 1}
    [3] = {0, 1, 2}

**Definicija:** Množica je **končna**, če je izomorfna kaki standardni končni množici.

Velja naslednje (ne bomo dokazali): če je `A ≅ [m]` in `A ≅ [n]`, je `m = n`. Torej za končno
množico `A` obstaja natanko en `n ∈ N`, da velja `A ≅ [n]`. Temu n pravimo **moč** množice `A`,
saj nam pove, koliko elementov ima `A`. Moč množice `A` označimo z `|A|`.

Zakoni za moč množic:

    |[n]| = n

    |A × B| = |A| × |B|

    |A + B| = |A| + |B|

    |B^A| = |B|^|A|

Pravilo vključitve/izključitve:

    |A ∪ B| = |A| + |B| - |A ∩ B|

    |A ∪ B ∪ C| = |A| + |B| + |C| - |A ∩ B| - |B ∩ C| - |C ∩ A| + |A ∩ B ∩ C|

In podobno za unijo štirih ali več množic.


\section{Neskončne množice in njihova moč}

**Definicija:** Množica je **neskončna**, če ni končna.

**Izrek:** Množica `A` je neskončna natanko tedaj, ko obstaja injektivna preslikava `N → A`.

Dokaz:

(⇒) Denimo da `A` ni končna. Injektivno preslikavo `e : N → A` definiramo s
pomočjo akisoma odvisne izbire. Ker `A` ni izomorfna `[0]`, ni prazna, torej
obstaja `e(0) ∈ A`. Denimo, da smo že definirali `e` kot injektivno preslikavo
`[n]` → A. Tedaj jo lahko razširimo na injektivno preslikavo `e : [n+1] → A`
takole: ker `e` ni surjektivna (če bi bila, bi veljalo `A ≅ [n]`), obstaja `x ∈
A`, ki ni v sliki `e`. Torej *izberemo* `e(n) ∈ A`, ki ni v sliki. Tako dobimo
`e : N → A`, ki je injektivna.

(⇐) Denimo, da obstaja injektivna preslikava `e : N → A`. Če bi veljalo `A ≅
[n]`, bi imeli izomorfizem `f : A → [n]`. Tedaj bi bil `f ∘ e : N → [n]`
injektivna preslikava, ta pa ne obstaja (dokaz opustimo). □

\subsection{Moč množic}

Tudi neskončnim množicam želimo prirediti *moč*. Potrebujem taka "števila", da
lahko vsaki množici `A` priredimo "število" `|A|`, ki pove, koliko elementov
ima. Za končne množice so to kar naravna števila. Za splošne množice so to
**kardinalna števila**. Zaenkrat še ne bomo povedali natančno, kaj kardinalna
števila so. Lahko pa jih primerjamo med sabo, ne da bi zares vedeli, kaj so!

**Definicija:** Naj bosta `A` in `B` poljubni množici. Pravimo:

1. `A` ima enako moč kot `B`, pišemo `|A| = |B|`, ko obstaja bijektivna preslikava `A → B`.
2. `A` ima moč manjšo ali enako `B`, pišemo `|A| ≤ |B|`, ko obstaja injektivna preslikava `A → B`.
3. `A` ima moč manjšo kot `B`, pišemo `|A| < |B|`, če velja `|A| ≤ |B|` in `|A| ≠ |B|`.

**Izrek:** `|A| ≤ |B|` natanko tedaj, ko je `A = ∅` ali obstaja surjekcija `B → A`.

*Dokaz.*

(⇒) Denimo, da je `f : A → B` injektivna in `A ≠ ∅`. Torej obstaja neki `x₀ ∈ A`.
Tedaj definiramo surjektivno preslikavo `g : B → A` takole:

    g(y) = x  ⇔  f(x) = y ali x = x₀.

(⇐) Denimo, da je `A` prazna ali obstaja surjekcija `f : B → A`. Če je `A`
prazna, je edina preslikava `∅ → B` injektivna. Če je `f : B → A` surjektivna,
ima prerez, ki je injektivna preslikava. □

\subsection{Cantorjev izrek}

**Izrek (Cantor):** `|A| < |P(A)|`.

*Dokaz:*

Najprej dokažimo `|A| ≤ |P(A)|`. Iščemo injektivno preslikava `f : A → P(A)`.
Vzemimo `f(x) = {x}`. Zlahka preverimo, da je `f` res injektivna.

Sedaj dokazujemo, da ne obstaja bijekcija `A → P(A)`. Dokazali bomo, da ne obstaja
surjekcija `A → P(A)`, kar zadostuje. Denimo, da je `g : A → P(A)` poljbuna preslikava.
Trdimo, da `g` ni surjekcija. Res, podmnožica

    S = {x ∈ A | x ∉ g(x) }

ni v sliki `g`. Če bi bila, bi za neki `y ∈ A` veljalo `g(y) = S`, a to bi
vodilo v protislovje:

1. velja `y ∉ S`: če `y ∈ S` potem `y ∉ g(y) = S` po definiciji `S`.
2. velja `¬ (y ∉ S)`: če `y ∉ S` potem `y ∉ g(y) = S`a. □


\subsection{Števne in neštevne množice}

Moč množice `N` označimo z `ℵ₀`. (Zaenkrat še vedno ne vemo, kaj točno so
kardinalna števila, a privzemimo, da imamo kardinalno število `ℵ₀`, ki ustreza
moči množice `N`.)

**Definicija:** Množica `A` je *števna*, če velja velja `|A| ≤ ℵ₀`.

**Definicija:** Množica `A` je *neštevna*, če ni števna.

**Izrek:** Za vsako množico `A` so ekvivalentne naslednje izjave:

1. `A` je števna
2. obstaja injektivna preslikava `A → N`
3. `A` je prazna ali obstaja surjektivna preslikava `N → A`
4. obstaja surjektivna preslikava `N → 1 + A`
5. `A` je končna ali izmoforna `N`

Dokaz.

(1 ⇒ 2) če je `A` števna, velja `|A| ≤ ℵ₀ = |N`, torej obstaja injektivna `A →
N` po definiciji relacije `≤`.

(2 ⇒ 3) To sledi neposredno iz zgornjega izreka

(3 ⇒ 4) Denimo, da je `A` prazna ali obstaja surjektivna preslikava `N → A`:

1. Če je `A = ∅`, potem seveda obstaja surjektivna preslikava `N → 1 + A`, in sicer
`n ↦ in₁()`.

2. Če obstaja surjektivna preslikava `f : N → A`, potem lahko definiramo surjektivno
preslikavo `g : N → 1 + A` s predpisom

        g(0) = in₁()
        g(n) = in₂(f(n-1)) za n > 1

(4 ⇒ 5) Denimo, da obstaja surjektivna preslikava `r : N → 1 + A`. Dokazali
bomo, da je `A` izomorfna `N`, če ni končna. Predpostavimo torej, da `A` ni
končna. Preslikva `r` ima prerez `s : 1 + A → N`, ki je seveda injektivna
preslikava. Preslikva `s ∘ in₂ : A → N` je torej kompozitum injektivnih
preslikav, zato je injektivna. Ker `A` ni končna, obstaja tudi injektivna
preslikava `N → A`. Po izreku Cantor-Schröder-Bernstein je torej `A` izomorfna
`N`.

(5 ⇒ 1) Če je `A` končna, je števna, ker očitno velja `A = |[n]| ≤ |N| = ℵ₀`. Če
je `A` izomorfna `N`, potem seveda velja `|A| = |N| ≤ |N| = ℵ₀`. □


**Izrek:** `N × N ≅ N`.

Pravimo, da je družina `A : I → Set` **števna**, če je števna njena indeksna
množica `I`.

**Izrek:** Unija števne družine števnih množic je števna.

*Dokaz.*

Najprej obravnavajmo unijo družine `A : N → Set`, kjer je `A_n` števna za vse `n ∈ N`.
Tu uporabimo aksiom izbire, da dokažemo števnost unije. Za vsak `n ∈ N` obstaja
surjektivna preslikava `N → A_n + 1`. Po aksiomu izbire obstaja preslikava

    e : ∏_{n ∈ N} { f : N → A_n + 1 | f surjekcija }.

Definiramo `e' : N × N → 1 + ⋃_n A_n`:

    e'(n, k) = e(n)(k).

Trdimo, da je `e'` surjekcija iz `N × N` na `1 + ⋃_n A_n`.

Nato obravnavamo še unijo družine `A : I → Set`, kjer je `I` števna in `A_i`
števna za vsak `i ∈ I`. □

\subsection{Cantor-Schröder-Bernsteinov izrek in zakon trihotomije}

**Izrek** (Cantor-Schröder-Bernstein): Če obstajata injektivni preslikava `A → B`
in `B → A`, potem obstaja bijektivna preslikava `A → B`.

*Dokaz.* Dokaz je v priloženi datoteki [csb.pdf](./csb.pdf)


**Posledica:** Če `|A| ≤ |B|` in `|B| ≤ |A|`, potem `|A| = |B|`.

*Dokaz.* To sledi neposredno iz izreka CSB in definicije `≤`. □

Brez dokaza omenimo še, da velja **zakon trihotomije**: za vsaki množici `A` in `B`
velja `|A| < |B|` ali `|A| = |B|` ali `|B| < |A|`, oziroma ekvivalentno

    |A| ≤ |B| ∨ |B| ≤ |A|.

Relacija `≤` potemtakem uredi moči množic linearno.

TODO: podaj referenco na dokaz (verjetno Prijateljeve Strukture 1).

\subsection{Moč kontinuuma in Cantorjeva hipoteza}

Vemo, da ima množica realnih števil `R` enako moč kot `P(N)`, potenčna množica
naravnih števil (to boste naredili na vajah). Tej moči pravimo **moč
kontinuuma** (ker je "kontinuum" tudi staro ime za `R`).

Že Goerg Cantor, utemelitelj teorije množic, še je vprašal naslednje vprašanje:

**Cantorjeva hipoteza:** Vsaka neštevna podmnožica realnih števil izomorfna `R`.

Povedano, z drugimi besedami, po moči ni nobene množice strogo med `N` in `R`.
Cantorjeva hipoteza je ostala odprta dlje časa. Dokončno je Cohen pred dobrega
pol stoletja dokazal naslednje:

**Izrek (Cohen):** Iz Zermelo-Fraenkelovih aksiomov teorije množic Cantorjeve
hipoteze ne moremo niti dokazati niti ovreči.

Pravimo, da je Cantorjeva hipoteza **neodvisna** od aksiomov teorije množic.

Poznamo še posplošeno Cantorjevo hipotezo, ki se glasi:

**Posplošena Cantorjeva hipoteza:** Če je `|A| ≤ |B| ≤ |P(A)|`, potem je `|B| =
|A|` ali `|B| = |P(A)|`.

Tudi posplošena Cantorjeva hipoteza je nedovisna od aksiomov teorije množic.
Danes vemo zelo veliko o tej hipotezi in poznamo, še mnoge druge izjave o
množicah, ki so neodvisne od Zermelo-Fraenkelovih aksiomov teorije množic. Ti
veljajo za nekakšno uradno različičo teorije množic in jih bomo obravnavali na
naslednjih predavanjih.

