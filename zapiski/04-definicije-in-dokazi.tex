\chapter{Enolični obstoj}

\section{Kvantifikator za enolični obstoj `∃!`}

S kvantifikatorjema `∀` in `∃` lahko izrazimo tudi druge kvantifikatorje.
Na primer, "Obstajata vsaj dva elementa `x` in `y` iz `A`, da velja `ϕ` lahko zapišemo

    ∃ x ∈ A . ∃ y ∈ A . x ≠ y ∧ ϕ

**Naloga:** S pomočjo `∀` in `∃` zapišite izjavo »Obstajata natanko dva elementa `x` in `y` iz `A`, da velja `ϕ`«.

Kako pa izrazimo »obstaja natanko en `x` iz `A`, da velja `ϕ(x)`«? Takole:

    (∃ x ∈ A . φ(x)) ∧ ∀ x y ∈ A . φ(x) ∧ φ(y) ⇒ x = y

ali ekvivalentno

    ∃ x ∈ A . φ(x) ∧ ∀ x ∈ A . φ(y) ⇒ x = y

To okrajšamo `∃! x ∈ A . φ(x)` in beremo »Obstaja natanko en `x` iz `A`, da velja `ϕ(x)`«.

Uporablja se tudi zapis `∃¹ x ∈ A . φ(x)`.

\section{Operator enoličnega opisa}

Če dokažemo, da obstaja natanko en `x ∈ A`, ki zadošča pogoju `ϕ(x)`, potem se lahko nanj smiselno sklicujemo z »tisti
`x` iz `A`, ki zadošča `ϕ(x)`«.

Primeri:

* »tisto realno število `x`, za katero je `x³ = 2`« (kubični koren 2)
* »tista množica `S`, ki nima nobenega elementa« (prazna množica)

Proti-primeri:

* »tisto racionalno število `x`, za katero je `x² = 2` (takega števila ni)
* »tisto realno število `x`, za katero je `x² = 2` (dve taki števili sta)
* »tista množica `S`, ki ima natanko en element` (takih množic je zelo veliko)

To je lahko zelo koristen način za opredelitev matematičnih objektov, zato uvedemo zanj simbolni zapis.
Če dokažemo

    ∃! x ∈ A . φ(x)

potem lahko pišemo

    ι x ∈ A . φ(x)

kar preberemo: "tisti `x ∈ A`, za katerega velja `φ(x)`". Torej velja:

    φ(ι x ∈ A . φ(x))

Spremenljivka `x` je *vezana* v `ι x ∈ A . φ(x)`.


**Primer:** Denimo, da še ne bi poznali simbola `√` za kvadatne korene. Tedaj bi
lahko kvadratni koren iz `2` zapisali kot

    ι x ∈ R . (x > 0 ∧ x^2 = 2)

Še več, preslikavo `√ : [0,∞) → [0,∞)` lahko definiramo s pomočjo zapisa `ι`:

    √ : x ↦ (ι y ∈ ℝ . (y ≥ 0 ∧ y² = x))


**Naloga:** Zapiši »limita zaporedja `a : ℕ → ℝ`« z operatorjem `ι` (pod predpostavko, da je `a` konvergentno
zaporedje). Najprej povej z besedami »limita zaporedja `a` je tisti `x ∈ ℝ`, ki ...«, nato pa zapiši še v obliki `ι x ∈ ℝ . ⋯`.

*Pomni:* zapis `ι x ∈ A . ϕ(x)` je dopusten samo v primeru, da velja `∃! x ∈ A . φ(x)`.


\chapter{Spremenljivke in definicije}


Preden v matematičnem ebsedilu uporabimo simbol ali spremenljivko, ga moramo *vpeljati*. To pomeni, da moramo pojasniti, kakšen je pomen simbola. Poznamo dva osnovna načina za vpeljavo novih simbolov:

* Nov simbol `s` lahko **definiramo** kot okrajšavo za neki drugi izraz (ki je lahko tudi logična formula).
* Nov simbol `s` je (prosta) **spremenljivka**, ki predstavlja neki (neznano, poljubno, nedoločeno) element dane množice `A`.

V obeh primerih dodamo simbol `s` v kontekst, se pravi v spisek znanih simbolov. Če smo simbol uvedli le začasno (na
primer v enem poglavju, ali v delu dokaza), ga iz konteksta odstranimo, ko ni več veljaven.

Matematiki zapisujejo definicije in vpeljujejo spremenljivke na razne načine.

\section{Vpeljava spremenljivke}

Če želimo vpeljavi spremljivko `x`, ki predstavlja neki poljuben ali neznani element množice `A`, zapišemo

    Naj bo x ∈ A.

S tem postane `x` veljavna spremenljivka, ki jo lahko uporabljamo. O njen vemo le to, da je element množice `A` – pravimo, da je `x` *prosta* spremenljivka. V matematičnih besedilih boste zasledili tudi naslednje fraze:

* »Naj bo `x ∈ A` poljuben.«
* »Obravnavajmo poljuben `x ∈ A`.«
* »Izberimo poljuben `x ∈ A`.«
* »Denimo, da imamo poljuben `x ∈ A`.«

Pozor, beseda "izberimo" bi komu dala misliti, da si lahko izbere neki konkretni `x`, a to preprečuje beseda "poljuben", ki jo matematik uporabi, kadar želi povedati, da je `x` neznana (poljubna) vrednost.

**Naloga:** Denimo, da učitelj reče »Naj bo `n` (poljubno) naravno število«, nato pa vas vpraša »Ali je `n` sodo število?« Kako boste odgovorili?


\section{Definicija simbola}

Definicija je v prvi vrsti **okrajšava** za neki izraz. Z njo uvedemo nov simbol `s` in mu pripišemo neko vrednost.
Simbol `s` je enak vrednosti, ki smo mu jo pripisali. Simbolni zapis za definicijo je

    s := ⋯

Na primer, v besedilu bi lahko napisali "Naj bo `s := √(log₂ 7 + π/6)`." S tem smo v kontekst dodali simbol `s` in
predpostavko, da je `s` enak `√(log₂ 7 + π/6)`. V matematičnih besedili boste zasledili tudi naslednje načine za definicijo:

* `s = √(log₂ 7 + π/6)` (namesto `:=` uporabimo `=`)
* `s ≡ √(log₂ 7 + π/6)` (namesto `:=` uporabimo `≡`)
* `s =̂ √(log₂ 7 + π/6)` (namesto `:=` uporabimo `=̂`)

Kadar definiramo simbol tako, da mu priredimo funkcijski predpis, recimo

    f := (x ↦ x² + 7)

to raje zapišemo kot

    f(x) := x² + 7

Kadar definiramo simbol s pomočjo enoličnega obstoja, recimo

    r := ι x ∈ ℝ . x³ = 2

to raje zapišemo z besedami:

    Naj bo r tisto realno število, ki zadošča r³ = 2.

Poglejmo še, kako definiramo okrajšave za logične formule. Defnimo, da želimo s `ϕ(x)` označiti izjavo `∃ y ∈ ℝ . y² =
x + 1`. Glede na zgornji dogovor, zapišemo

    ϕ := (x ↦ (∃ y ∈ ℝ . y² = x + 1))

ali

    ϕ(x) := (∃ y ∈ ℝ . y² = x + 1)

Vendar takega zapisa v praksi ne boste videli. Dosti bolj pogost je zapis

    ϕ(x) :⇔ ∃ y ∈ ℝ . y² = x + 1

ali pa kar `ϕ(x) ⇔ ∃ y ∈ ℝ . y² = x + 1`.


\subsection{Definicije novih matematičnih pojmov}

Kaj pa definicije novih pojmov, ki jih srečujete pri predavanjih, denimo pri analizi? Na primer:

**Definicija:** Zaporedje števil `a : ℕ → ℝ` je *neomejeno*, če za vsak `x ∈ ℝ` obstaja `i ∈ ℕ`, da je `aᵢ > x`.

S stališča simbolnega zapisa, je to le uveba novega simbola `neomejeno`:

    neomejeno(a) := (∀ x ∈ ℝ . ∃ i ∈ ℕ . aᵢ > x)

Seveda bistvo take definicije ni le krajši zapis izjave `(∀ x ∈ ℝ . ∃ i ∈ ℕ . aᵢ > x)`, ampak uporabna vrednost pojma
"neomejeno zaporedje".


\chapter{Konstrukcije in dokazi}

Matematiki v sklopu svojih aktivnosti **konstruiramo** matematične objekte:

* v geometriji so znane konstrukcije z ravnilom in šestilom
* računanje števk števila `π` je konstrukcija približka
* reševanje enačbe, je konstrukcija števila z želeno lastnostjo
* konstruiramo lahko elemente množice, pogosto kar tako, da jih zapišemo, na primer `(2, \inl(3)) ∈ ℕ × (ℤ + ℤ)`

Poleg tega **dokazujemo** matematične izjave. Na dokaz lahko gledamo kot na konstrukcijo, saj je to le še ena zvrst
matematičnega objekta. Ker pa so dokazi skoraj vedno zapisani v naravnem jeziku, jih matematiki pogosto dojemajo ločeno
od ostalih matematičnih objekotv (števila, preslikave, množice, ploskve, ...).

Kaj pravzaprav je dokaz? V prvi vrsti je dokaz utemeljitev matematične izjave. Zgrajen je po točno določenih *pravilih
sklepanja*, ki jih lahko podamo formalno in jih tudi implementiramo na računalniku, s čimer se pri tem predmetu ne bomo
ubadali, a kogar to zanima, si lahko ogleda »[The dawn of formalized mathematics](https://youtu.be/Z500sma3h90)«
([prosojnice](https://www.icloud.com/keynote/0Gkr1yM7XY-31aQleWf-fiW7A#The_Dawn_of_Formalized_Mathematics)) in se nauči
uporabljati kak [dokazovalni pomočnik](https://ncatlab.org/nlab/show/proof+assistant) (v zadnjem času hitro napreduje [Lean](https://leanprover.github.io)).

V praksi ljudje ne pišejo vseh podrobnosti v dokazu, ker bi bil tak dokaz nečitljiv in nerazumljiv. Pogosto podajo samo
glavno idejo, iz katere lahko izkušeni matematik sam rekonstruira dokaz. Iz dobro napisanega dokaza se lahko naučimo
marsikaj novega, poleg goleda dejstva, da dokaza izjava velja.

Mi bomo vadili podrobno pisanje dokazov. Pri ostalih predmetih boste videli "žive dokaze", ki imajo manj podrobnosti in so zapisani manj formalno. A vsi pravilni matematični dokazi se dajo zapisati na način, kot ga bomo predstavili mi (in celo zapisati povsem formalon z dokazovalnim pomočnikom).

\section{Kako pišemo dokaze}

Pravila sklepanja so kot pravila igre. Ne povedo, kako dobro igrati, samo kaj je dovoljeno. Seveda bomo hkrati s pravili
sklepanja povedali nekaj namigov in nasvetov, kako dokaz poiščemo. A kot pri vsaki igri velja, da vaja dela mojstra.

Dokaz ima vgnezdeno strukturo: sestoji iz delov in pod-dokazov, ki sestojijo iz nadaljnih pod-dokazov itn., ki se
zaključijo z osnovnimi dejstvi. Vsi ti kosi so s pomočjo pravil sklepanja zloženi v dokazno "drevo".

Ko pišemo dokaz, moramo v vsakem trenutku poznati

* **cilj**: kaj trenutno dokazujemo in
* **kontekst**: katere spremenljivke in predpostavke imamo trenutno na voljo.

Ko napravimo korak v dokazu, mora biti utemeljen z enim od pravil sklepanja. Dokaz je
popoln, ko smo utemeljili vse pod-dokaze, ki ga sestavljajo.

\subsubsection{Primer}

**Izjava:** `(p ∨ q) ∧ r ⇒ (p ∧ r) ∨ (q ∧ r).`

Dokaz:

```
Dokažimo (p ∨ q) ∧ r ⇒ (p ∧ r) ∨ (q ∧ r)
  Predpostavimo (p ∨ q) ∧ r.                         (1)
  Zaradi (1) velja p ∨ q.                            (2)
  Zaradi (1) velja r.                                (3)
  Dokažimo (p ∧ r) ∨ (p ∧ r).
     Zaradi (2) lahko obravnavamo dva primera:
         1. Če velja p:                              (4)
             Dokažimo (p ∧ r) ∨ (p ∧ r)
                Dokažimo p ∧ r:
                     1.1. p velja zaradi (4)
                     1.2. r velja zaradi (3)
         2. Če velja q:                              (5)
             Dokažimo (p ∧ r) ∨ (q ∧ r)
                Dokažimo q ∧ r:
                     1.1. q velja q zaradi (5)
                     1.2. r velja r zaradi (3)
Konec dokaza.
```

Dokaz bi bolj po človeško napisali takole:

> *Dokaz.* Predpostavimo `p ∨ q` in `r`. Če velja `p`, potem sledi `p ∧ r` ter od tod `(p ∧ r) ∨ (p ∧ r)`. Če pa velja `q`, sledi `q ∧ r` ter spet `(p ∧ r) ∨ (p ∧ r)`. □

Ali pa kar takole:

> *Dokaz:* Očitno.

\section{Pravila sklepanja}

Pravila sklepanja delimo na dve vrsti:

* **pravila vpeljave** povedo, kako dokažemo izjavo
* **pravila uporabe** povedo, kako lahko že znano izjavo uporabimo v dokazu neke druge izjave

Poleg tega poznamo še pravila o zamenjavi:

* **zamenjava enakih izrazov**: izraz lahko vedno zamenjamo z njim enakim
* **zamenjava ekvivalentnih izjav**: izjavo vedno lahko zamenjamo z njej ekvivalentno

Pravila sklepanja so zapisana v priloženi datoteki.

