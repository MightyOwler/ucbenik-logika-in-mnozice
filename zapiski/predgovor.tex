\chapter{Predgovor}
\label{chap:predgovor}

Glavni namen predmet Logika in množice v prvem letniku študija matematike je študente
naučiti osnov matematičnega izražanja: kako beremo in pišemo matematično besedilo, kako
uporabljamo simbolni zapis, kako zapišemo in preberemo dokaz itd. Drugi poglavitni namen
predmeta je spoznavanje osnov matematične logike in teorije množic.

Za semesterski predmet z dvema urama predavanj in dvema urama vaj ima predmet zelo
ambiciozen program. Najučinkovitejši recept za uspeh je tisti, ki ga študenti ne marajo:
učite se sproti, sprašujte predavatelja in asistente, trkajte na vrata njihovih pisarn
tudi takrat, ko nimajo govorilnih ur.

Ti zapiski s predavanj nastajajo sproti. Prvotno sem jih zapisoval v formatu Markdown, a napočil je čas, da jih prenesem v {\LaTeX} in nato izboljšujem. Opozarjam vas, da zapiski vsebujejo napake, ker so le grob zapis vsebine predavanj. Odkrivanje napak je sestavni del učnega procesa, čeprav si ne želim, da bi bi bilo napak toliko, da bi motile učenje. Zelo vam bom hvaležen, če mi boste odkrite napake sporočili, da jih popravim. Asistentom pri predmetu se zahvaljujem za skrbno odpravljanje napak. Vse ki so ostale, so moja last.

\bigskip

\begin{flushright}
Andrej Bauer \qquad\hbox{}
\end{flushright}

\bigskip

\paragraph{Zahvala.}
%
Pri urejanju zapiskov pomagali:
%
Matej Marinko,
Lev Rus,
Jakob Schrader,
Matija Sirk in
Marjetka Zupan.
%
Vsem se najlepše zahvaljujem.


%%% Local Variables: 
%%% mode: latex
%%% TeX-master: "lmn"
%%% End: 
