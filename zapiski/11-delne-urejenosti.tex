\chapter{Relacije urejenosti}

\section{Relacije urejenosti}
\begin{definicija}
  Relacija $R \subseteq A \times A$ je:
  %
  \begin{enumerate}
  \item \textbf{šibka urejenost}, ko je refleksivna in tranzitivna,
  \item \textbf{delna urejenost}, ko je refleksiva, tranzitivna in antisimetrična,
  \item \textbf{linearna urejenost}, ko je delna urejenost in je sovisna ($\all{x, y \in A} x \rel{R} y \lor y \rel{R} x$).
  \end{enumerate}
\end{definicija}

Za relacije urejenosti ponavadi uporabljamo simbole, ki spominjajo na znak $\leq$, kot so $\preceq$, $\subseteq$, $\sqsubseteq$ ipd.

\begin{primer}
  Primeri urejenosti:
  \begin{enumerate}
    \item Relacija deljivosti na naravnih številih je delna urejenost.
    \item Relacija deljivosti na celih številih je šibka urejenost, ni pa delna urejenost.
    \item Relacija $\leq$ na realnih številih je linearna urejenost.
    \item Relacija $\subseteq$ na $\pow{A}$ je delna urejenost. Za katere množice $A$ je linearna?
    \item Relacija $=$ je delna urejenost. Imenuje se tudi \textbf{diskretna urejenost}.
  \end{enumerate}
\end{primer}


\begin{definicija}
  V delni ureditvi $(P, {\leq})$ je \textbf{veriga} taka podmnožica $V \subseteq P$, ki je linearno urejena z relacijo~$\leq$, se pravi $\all{x, y \in V} x \leq y \lor y \leq x$. \textbf{Antiveriga} je taka podmnožica $A \subseteq P$, ki je diskretno urejena z relacijo~$\leq$, se pravi $\all{x, y \in A} x \leq y \lthen x = y$.
\end{definicija}

\begin{primer}
  Potence števila $2$ tvorijo verigo v $\NN$ glede na relacijo deljivosti.
  Praštevila tvorijo antiverigo.
\end{primer}


\subsection{Hassejev diagram}

Končno delno ureditev $(A, \leq)$ lahko predstavimo s \textbf{Hassejevim diagramom}: elemente
množice $A$ narišemo tako, da je $x$ pod $y$, kadar velja $x \leq y$. Nato povežemo vozlišči $x$ in $y$, če je $y$ neposredni naslednik $x$, se pravi, da velja $x \neq y$, $x \leq y$ in iz $x \leq z \leq y$ sledi $x = z \lor z = y$.

\begin{naloga}
  Narišite Hassejev diagram relacije deljivosti na množici $\set{0, 1, \dots, 10}$ ter
  Hassejev diagram relacije $\subseteq$ na množici $\pow(\{a,b,c\})$.
\end{naloga}

\begin{naloga}
  Kako v Hassejevem diagramu prepoznamo verigo? In kako prepoznamo antiverigo?
\end{naloga}


\subsection{Operacije na urejenostih}

\subsubsection{Obratna urejenost}

Če je $\leq$ delna urejenost na $P$ potem je tudi transponirana relacija $\geq$, definirana z
%
\begin{equation*}
    x \geq y \liff x \leq y,
\end{equation*}
%
delna urejenost na $P$. Če je $\leq$ linearna, je $\geq$ linearna.

\subsubsection{Produktna in leksikografska urejenost}

Naj bosta $(P, {\leq_P})$ in $(Q, {\leq_Q})$ delni urejenosti. Na kartezičnem produktu $P \times Q$ lahko definiramo dve urejenosti.

Prva je \textbf{produktna} urejenost
%
\begin{equation*}
  (x_1,y_1) \leq_{\times} (x_2,y_2) \defiff x_1 \leq_P x_2 \land y_1 \leq_Q y_2
\end{equation*}
%
in druga \textbf{leksikografska} urejenost
%
\begin{equation*}
  (x_1,y_1) \preceq_\mathrm{lex} (x_2,y_2)
  \defiff (x_1 \neq x_2 \land x_1 \leq_P x_2) \lor (x_1 = x_2 \land y_1 \leq_Q y_2).
\end{equation*}


\begin{naloga}
  Kako si predstavljamo produktno in leksikografsko ureditev na $[0,1] \times [0,1]$, če $[0,1]$ uredimo z običajno relacijo $\leq$? Na sliki označite območji
  %
  \begin{equation*}
    \set{(x,y) \in [0,1] \times [0, 1] \such (1/2,1/3) \leq_\times (x,y)}
  \end{equation*}
  %
  in
  %
  \begin{equation*}
    \set{(x,y) \in [0,1] \times [0, 1] \such (1/2,1/3) \leq_\mathrm{lex} (x,y)}.
  \end{equation*}
\end{naloga}

\begin{izjava}
  Produktna in leksikografska urejenosti sta delni urejenosti. Leksikografska urejenost linearnih urejenosti je linearna.
\end{izjava}

\begin{dokaz}
  Dejstvo, da je produktna urejenost refleksivna, tranzitivna in antisimetrična, pustimo za vajo. Preverimo, da je leksikografska urejenost $\leq_\mathrm{lex}$ delna urejenost.

  Dokaz, da je $\leq_\mathrm{lex}$ je refleksivna: za vsak $(x, y) \in P \times Q$ velja $x = x \land y \sqsubseteq y$, torej velja $(x, y) \sqsubseteq (x, y)$.

  Dokaz, da je $\leq_\mathrm{lex}$ je antisimetrična: naj bosta $(x_1,y_1), (x_2,y_2) \in P \times Q$ in denimo, da velja
  %
  \begin{equation*}
    (x_1, y_1) \leq_\mathrm{lex} (x_2, y_2) \land (x_2, y_2) \leq_\mathrm{lex} (x_1, y_1)
  \end{equation*}
  %
  To je ekvivalentno
  %
  \begin{align*}
  & (x_1 \neq x_2 \land x_1 \leq_P x_2 \land x_2 \neq x_1 \land x_2 \leq_P x_1) \lor {}\\
  & (x_1 \neq x_2 \land x_1 \leq_P x_2 \land x_2 = x_1 \land y_2 \leq_Q y_1) \lor {}\\
  & (x_1 = x_2 \land y_1 \leq_Q y_2 \land x_2 \neq x_1 \land x_2 \leq_P x_1) \lor {}\\
  & (x_1 = x_2 \land y_1 \leq_Q y_2 \land x_2 = x_1 \land y_2 \leq_Q y_1).
  \end{align*}
  %
  Če v zgornji formuli upoštevamo, da je $x_1 \neq x_2 \land x_1 = x_2$, vidimo, da sta drugi in tretji disjunkt ekvivalentna $\bot$, zato
  je izjava ekvivalentna:
  \begin{align*}
  &(x_1 \neq x_2 \land x_1 \leq_P x_2 \land x_2 \neq x_1 \land x_2 \leq_P x_1) \lor {}\\
  &(x_1 = x_2 \land y_1 \leq_Q y_2 \land x_2 = x_1 \land y_2 \leq_Q y_1).
  \end{align*}
  %
  A tudi prvi disjunkt je ekvivalenten $\bot$, ker iz $x_1 \leq_P x_2 \land x_2 \leq_P x_1$ sledi $x_1 = x_2$, saj je $\leq_P$ po predpostavki antisimetrična. Torej ostane samo zadnji disjunkt, ki je ekvivalenten
  \begin{equation*}
    x_1 = x_2 \land y_1 \leq_Q y_2 \land y_2 \leq_Q y_1.
  \end{equation*}
  %
  Ker je $\leq_Q$ antisimetrična, sledi $x_1 = x_2$ in $y_1 = y_2$, kar smo želeli dokazati.

  Dokaz, da je $\leq_\mathrm{lex}$ tranzitivna: naj bodo $(x_1,y_1), (x_2,y_2), (x_3, y_3) \in P \times Q$ in denimo, da velja
  %
  \begin{equation*}
    (x_1, y_1) \leq_\mathrm{lex} (x_2, y_2) \land (x_2, y_2) \leq_\mathrm{lex} (x_3, y_3).
  \end{equation*}
  %
  To je ekvivalentno
  %
  \begin{align*}
  & (x_1 \neq x_2 \land x_1 \leq_P x_2 \land x_2 \neq x_3 \land x_2 \leq_P x_3) \lor  {} \\
  & (x_1 \neq x_2 \land x_1 \leq_P x_2 \land x_2 = x_3 \land y_2 \leq_Q y_3) \lor {} \\
  & (x_1 = x_2 \land y_1 \leq_Q y_2 \land x_2 \neq x_3 \land x_2 \leq_P x_3) \lor {} \\
  & (x_1 = x_2 \land y_1 \leq_Q y_2 \land x_2 = x_3 \land y_2 \leq_Q y_3)
  \end{align*}
  %
  Obravnavajmo štiri primere in v vsakem od njih dokažimo $(x_1, y_1) \leq_\mathrm{lex} (x_3, y_3)$, se pravi
  $(x_1 \neq x_3 \land x_1 \leq_P x_3) \lor (x_1 = x_3 \land y_1 \leq_Q y_3)$:
  %
  \begin{enumerate}
  \item Če velja $x_1 \neq x_2 \land x_1 \leq_P x_2 \land x_2 \neq x_3 \land x_2 \leq_P x_3$: ker je $\leq$ tranzitivna sledi $x_1 \leq_P x_3$, poleg tega pv velja $x_1 \neq
    x_3$: če bi veljalo $x_1 = x_3$, bi iz predpostavk dobili $x_3 \leq_P x_2 \land x_2 \leq_P x_3$, od koder bi sledilo $x_2 = x_3$, kar je v
    protislovju s predpostavko $x_2 \neq x_3$.

  \item Če velja $x_1 \neq x_2 \land x_1 \leq_P x_2 \land x_2 = x_3 \land y_2 \leq_Q y_3$: ker je $x_2 = x_3$ iz prvih dveh predpostavk sledi $x_1 \neq x_3 \land x_1 \leq_P x_3$.

  \item Če velja $x_1 = x_2 \land y_1 \leq_Q y_2 \land x_2 \neq x_3 \land x_2 \leq_P x_3$: ker je $x_1 = x_2$ iz zadnjih dveh predpostavk sledi $x_1 \neq x_3 \land x_1 \leq_P x_3$.

  \item Če velja $x_1 = x_2 \land y_1 \leq_Q y_2 \land x_2 = x_3 \land y_2 \leq_Q y_3$: torej je $x_1 = x_3$ ker je $=$ tranzitivna in $y_1 \leq_Q y_3$ ker je $\leq_Q$ tranzitivna.
  \end{enumerate}
  %
  Nazadnje preverimo še, da je $\leq_\mathrm{lex}$ linearna, če sta $\leq$ in $\leq_Q$ linearni. Naj bosta $(x_1,y_1), (x_2,y_2) \in P \times Q$. Dokazati želimo
  %
  \begin{equation*}
    (x_1, y_1) \preceq (x_2, y_2) \lor (x_2, y_2) \preceq (x_1, y_1).
  \end{equation*}
  %
  To je ekvivalentno disjunkciji
  % 
  \begin{align*}
    & (x_1 \neq x_2 \land x_1 \leq_P x_2) \lor {} \\
    & (x_1 = x_2 \land y_1 \leq_Q y_2) \lor {} \\
    & (x_2 \neq x_1 \land x_2 \leq_P x_1) \lor {} \\
    & (x_2 = x_1 \land y_2 \leq_Q y_1),
  \end{align*}
  %
  kar je ekvivalentno
  %
  \begin{align*}
    & (x_1 \neq x_2 \land (x_1 \leq_P x_2 \lor x_2 \leq_P x_1)) \lor {} \\
    &(x_1 = x_2 \land (y_1 \leq_Q y_2 \lor y_2 \leq_Q y_1)).
  \end{align*}
  %
  Ker sta $\leq_P$ in $\leq_Q$ linearni, je to ekvivalentno
  %
  \begin{equation*}
    (x_1 \neq x_2 \land \top) \lor (x_1 = x_2 \land \top),
  \end{equation*}
  %
  kar je ekvivalentno
  \begin{equation*}
    (x_1 \neq x_2) \lor (x_1 = x_2).
  \end{equation*}
  %
  To pa drži po zakonu o izključeni tretji možnosti. S tem je linearnost $\leq_\mathrm{lex}$, dokazana.
\end{dokaz}

\subsubsection{Vsota urejenosti}

Naj bosta $(P, \leq_P)$ in $(Q, \leq_Q)$ delni urejenosti. Na vsoti $P + Q$ lahko
definiramo urejenost $\leq_{+}$ s predpisom:
%
\begin{equation*}
  u \leq_{+} v \defiff
  \begin{aligned}[t]
    & (\some{x, y \in P} u = \inl(x) \land v = \inl(y) \land x \leq_P y) \lor {} \\
    & (\some{s, t \in Q} u = \inr(s) \land v = \inr(t) \land s \leq_Q t).
  \end{aligned}
\end{equation*}

\subsubsection{Zaporedna vsota urejenosti}

Naj bosta $(P, \leq_P)$ in $(Q, \leq_Q)$ delni urejenosti. Na vsoti $P + Q$ lahko definiramo urejenost $\leq_{\to}$ s predpisom:
%
\begin{equation*}
  u \leq_{\to} v \defiff
  \begin{aligned}[t]
    &(\some{x, y \in P} u = \inl(x) \land v = \inl(y) \land x \leq_P y) \lor {} \\
    &(\some{x \in P} \some{s \in Q} u = \inl(x) \land v = \inr(s)) \lor {} \\
    &(\some{s, t \in Q} u = \inr(s) \land v = \inr(t) \land s \leq_Q t).
  \end{aligned}
\end{equation*}
%
Torej so vsi elementi $P$ pred vsemi elementi $Q$. Zaporedna vsota linearnih urejenosti je linearna.


\subsubsection{Potenca urejenosti}

Naj bo $(P, \leq)$ delna urejenost in $A$ množica. Na eksponentni množici $P^A$ lahko definiramo urejenost $\preceq$ s predpisom:
%
\begin{equation*}
  f \preceq g \defiff \all{x \in A} f(x) \leq g(x).
\end{equation*}

\begin{naloga}
  Ali je $\preceq$ linearna, kadar je $\leq$ linearna?
\end{naloga}


\subsubsection{Delna urejenost, inducirana s šibko ureditvijo}

Naj bo $(P, \leq)$ šibka ureditev. Relacija $\sim$ na $P$, definirana s predpisom
%
\begin{equation*}
  x \sim y \defiff x \leq y \land y \leq x,
\end{equation*}
%
je ekvivalenčna relacija. Na kvocientu $P/{\sim}$ lahko definiramo relacijo $\preceq$ s
predpisom
%
\begin{equation*}
  [x] \preceq [y] \defiff x \leq y.
\end{equation*}
%
Treba je preveriti, da je relacija dobro definirana, saj smo uporabili predstavnike ekvivalenčnih razredov. Se pravi, ali velja
\begin{equation*}
  x \sim x' \land y \sim y' \lthen (x \leq y \liff x' \leq y') ?
\end{equation*}
%
Pa preverimo. Denimo, da velja $x, y, x', y' \in P$ in $x \sim x'$ in $y \sim y'$.
Torej velja
\begin{equation*}
  x \leq x' \land x' \leq x \land y \leq y' \land y' \land x.
\end{equation*}
%
Sedaj dokažimo $x \leq y \liff x' \leq y'$:
%
\begin{enumerate}
\item Če velja $x \leq y$ potem $x' \leq x \leq y \leq y'$.
\item Če velja $x' \leq y'$, potem $x \leq x' \leq y' \leq y$.
\end{enumerate}
%
Torej je $\preceq$ dobro definirana.

\begin{izjava}
  Relacija, ki je inducirana s šibko ureditvijo, je delna ureditev.
\end{izjava}

\begin{dokaz}
  Refleksivnost in tranzitivnost $\preceq$ sledita iz refleksivnosti in tranzitivnosti~$\leq$. Preverimo antisimetričnost: denimo, da velja $[x] \leq [y]$ in $[y] \leq [x]$. Tedaj velja $x \leq y$ in $y \leq x$, torej velja $x \sim y$ in $[x] = [y]$.
\end{dokaz}

\begin{primer}
  Obravnavajmo cela števila $\ZZ$ in deljivost $\mid$, ki je šibka
  ureditev. Za vse $k, m \in \ZZ$ velja
  \begin{equation*}
    k \sim m \liff k \mid m \land m \mid k \liff |k| = |m|.
  \end{equation*}
  %
  Torej je $\ZZ/{\sim} \cong \NN$, kjer izomorfizem preslika $[k] \mapsto |k|$. Delna ureditev na $\ZZ/{\sim}$ inducirana z deljivostjo je spet deljivost (ko jo prenesemo iz $\ZZ/{\sim}$ na $\NN$ s pomočjo izomorfizma).
\end{primer}


\subsection{Monotone preslikave}

\begin{definicija}
  Preslikava $f : P \to Q$ med delnima urejenostma $(P, {\leq_P})$ in $(Q, {\leq_Q})$ je
  \textbf{monotona} (ali \textbf{naraščajoča}), ko velja $\all{x, y \in P} x \leq_P y \lthen f(x) \leq_Q f(y)$.
\end{definicija}

\begin{definicija}
  Preslikava $f : P \to Q$ med delnima urejenostma $(P, \leq_P)$ in $(Q, \leq_Q)$ je
  \textbf{antitona} (ali \textbf{padajoča}), ko velja $\all{x, y \in P} x \leq_P y \lthen f(y) \leq_Q f(x)$.
\end{definicija}

\begin{opomba}
  V analizi ">monotona"< pomeni ">monotona ali antitona". To ni nič
  čudnega, ker ">dan"< tudi pomeni ">dan in noč">.
\end{opomba}

\begin{izrek}
  Kompozicija monotonih preslikav je monotona. Identita je monotona.
\end{izrek}

\begin{dokaz}
  Naj bosta $f : P \to Q$ in $g : Q \to R$ monotoni preslikavi med delnimi
  urejenostmi $(P, {\leq_P})$, $(Q, {\leq_Q})$ in $(R, {\leq_R})$. Če je $x \leq_P y$, potem je zaradi monotonisti $f$ tudi $f(x) \leq_Q f(y)$, nato pa je zaradi monotonisti $g$ spet $g(f(x)) \leq_R g(f(y))$. Identiteta je očitno monotona.
\end{dokaz}

\begin{primer}
  Primeri monotonih preslikav:
  \begin{enumerate}
    \item Konstantna preslikava je monotona.
    \item Seštvanje ${+} : \RR \times \RR \to \RR$ je monotona operacija glede na produktno ureditev na $\RR \times \RR$.
    \item Množenje ${\times} : \RR \times \RR \to \RR$ ni monotona operacija.
  \end{enumerate}
\end{primer}


\subsection{Meje}

\begin{definicija}
  Naj bo $(P, {\leq})$ delna urejenost, $S \subseteq P$ in $x \in P$:
  \begin{itemize}

  \item $x$ je \textbf{spodnja meja} podmnožice $S$, ko velja $\all{y \in S} x \leq y$,

  \item $x$ je \textbf{zgornja meja} podmnožice $S$, ko velja $\all{y \in S} y \leq x$,

  \item $x$ je \textbf{infimum} ali \textbf{največja spodnja meja} ali \textbf{natančna spodnja meja} podmnožice $S$, ko je spodnja meja $S$ in velja: za vse $y \in P$, če je $y$ spodnja meja
    $S$, potem je $y \leq x$,

  \item $x$ je \textbf{supremum} ali \textbf{najmanjša zgornja meja} ali \textbf{natančna zgornja meja} podmnožice $S$, ko je zgornja meja $S$ in velja: za vse $y \in P$, če je $y$ zgornja meja $S$, potem je $x \leq y$,

  \item $x$ je \textbf{minimalni element} podmnožice $S$, ko velja $x \in S$ in $\all{y \in S} y \leq x \lthen x = y$,

  \item $x$ je \textbf{maksimalni element} podmnožice $S$, ko velja $x \in s$ in
      $\all{x \in S} x \leq y \lthen x = y$,

  \item $x$ je \textbf{najmnanjši} ali \textbf{prvi} element ali \textbf{minimum} podmnožice $S$, ko velja $x \in S$ in $\all{y \in S} x \leq y$,

  \item $x$ je \textbf{največji} ali \textbf{zadnji} element ali \textbf{maksimum} podmnožice $S$, ko velja $x \in S$ in $\all{y \in S} y \leq x$.
\end{itemize}
\end{definicija}

\begin{opomba}
  Minimalni element ni isto kot minimum (in maksimalni element ni isto kot maksimum).
\end{opomba}

Kadar govorimo o ">prvem elementu"< ali ">maksimalnem elementu"< in ne povemo, na
katero podmnožico se nanaša element, imamo običajno v mislih kar celotno delno
ureditev.

\begin{izrek}
  Naj bo $(P, {\leq})$ delna urejenost in $S \subseteq P$. Tedaj ima $S$ največ en
  infimum in največ en supremum, ki ju zapišemo $\inf S$ ter $\sup S$, kadar obstajata.
\end{izrek}

\begin{dokaz}
  Denimo, da sta $x$ in $y$ oba infimum $S$. Ker je $y$ spodnja meja za
  $S$ in $x$ njen infimum, velja $y \leq x$. Podobno velja $x \leq y$, torej $x = y$. Za
  supremum je dokaz podoben.
\end{dokaz}

\begin{primer}
  Supremum končne neprazne množice $S \subseteq \NN$ za relacijo deljivosti $\mid$
  je namanjši skupni večkratnik elementov iz $S$. Infimum je navečji skupni
  delitelj. Kaj pa, če je $S$ prazna ali neskončna?
\end{primer}

\subsection{Mreže}

\begin{definicija}
  Naj bo $(P, {\leq})$ delna urejenost:
  %
  \begin{enumerate}
  \item $(P, \leq)$ je \textbf{mreža}, ko imata vsaka dva elementa $x, y \in P$ infimum in supremum.

  \item $(P, \leq)$ je \textbf{omejena mreža}, ko ima vsaka končna podmnožica $P$ infimum in supremum.

  \item $(P, \leq)$ je \textbf{polna mreža}, ko ima vsaka podmnožica $P$ infimum in supremum.
  \end{enumerate}
  %
  Infimum in supremum elementov $x$ in $y$ pišemo $x \land y$ in $x \lor y$.
\end{definicija}

\begin{izrek}
  Delna urejenost $(P, {\leq})$ je omejena mreža natanko tedaj, ko ima
  najmanši element in največji element, ter imata vsaka sva elementa infimum in supremum.
\end{izrek}

\begin{dokaz}
  Denimo, da je $(P, \leq)$ omejana mreža. Tedaj $P$ ima najmanši element, namreč
  $\sup \emptyset$, in največji element, namreč $\inf \emptyset$. Infimum in supremum $x$ in $y$ sta seveda $\inf \set{x, y}$ in $\sup \set{x, y}$.

  Denimo, da ima $P$ najmanši element $\bot_P$ in največji element $\top_P$, vsaka dva
  elementa pa imata infimum in supremum. Naj bo $S \subseteq P$ končna množica:
  %
  \begin{enumerate}
  \item če je $S = \emptyset$, potem je $\inf S = \top_P$ in $\sup S = \bot_P$,
  \item če je $S = \set{x_1, \ldots, x_n}$ za $n > 0$, potem je $\inf S = \inf \set{x_1, \ldots, x_{n-1}} \lor x_n$ in $\sup S = \sup \set{x_1, \ldots, x_{n-1}} \lor x_n$.
  \end{enumerate}
\end{dokaz}

\begin{primer}
  Primeri mrež:
  %
  \begin{enumerate}
  \item Množica $\two = \set{\bot, \top}$ je omejena mreža za relacijo $\lthen$.
  \item Relacija deljivosti na množici pozitivnih naravnih števil je omejena mreža.
  \item Potenčna množica $\pow{A}$, urejena z $\subseteq$, je polna mreža.
  \item Zaprti interval $[a,b]$, urejen z $\leq$, je polna mreža.
  \item Relna števila $R$, urejena z $\leq$, so mreža.
  \end{enumerate}
\end{primer}
