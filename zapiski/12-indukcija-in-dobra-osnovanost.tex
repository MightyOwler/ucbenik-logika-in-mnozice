\chapter{Indukcija in dobra osnovanost}

\section{Dobra osnovanost}

\subsection{Indukcija na naravnih številih}

Poznamo že indukcijo na naravnih številih. Zapišemo jo lahko na dva načina,
kjer naslednika števila $n$ označimo $\suc{n}$:
%
\begin{enumerate}
\item Kot aksiom o predikatih na naravnih številih:
  %
  \begin{equation*}
  \phi(0) \land (\all{n \in \NN} \phi(n) \lthen \phi(\suc{n})) \lthen \all{m \in \NN} \phi(m)
  \end{equation*}

\item Kot lastnost podmnožic naravnih števil:
  %
  \begin{equation*}
    \all{S \in \pow{\NN}} 0 \in S \land (\all{k \in \NN} k \in S \lthen \suc{k} \in S) \lthen S = \NN
  \end{equation*}
\end{enumerate}
%
Uporabljali bomo verzijo s podmnožicami. Najprej jo predelajmo v ekvivalentno obliko:
%
\begin{align}
  &\all{S \in \pow{\NN}} 0 \in S \land (\all{k \in \NN} k \in S \lthen \suc{k} \in S) \lthen S = \NN \tag{$\liff$} \\
  &\all{S \in \pow{\NN}} 0 \in S \land (\all{m \in \NN} (\all{k \in \NN} \suc{k} = m \lthen k \in S) \lthen m \in S) \lthen S = \NN \tag{$\liff$} \\
  &\all{S \in \pow{\NN}} (\all{m \in \NN} (\all{k \in \NN} \suc{k} = m \lthen k \in S) \lthen m \in S) \lthen S = \NN. \notag
\end{align}
%
Kaj smo dosegli? Bazo indukcije in indukcijski korak smo združili v eno samo predpostavko
%
\begin{equation}
  \label{eq:ind-N}
  \all{m \in \NN} (\all{k \in \NN} \suc{k} = m \lthen k \in S) \lthen m \in S
\end{equation}
%
Če vstavimo $m \defeq 0$, dobimo:
%
\begin{align}
  &(\all{k \in \NN} \suc{k} = 0 \lthen k \in S) \lthen 0 \in S \tag{$\liff$} \\
  &(\all{k \in \NN} \bot \lthen k \in S) \lthen 0 \in S \tag{$\liff$} \\
  &(\all{k \in \NN} \top) \lthen 0 \in S \tag{$\liff$} \\
  &\top \lthen 0 \in S \tag{$\liff$} \\
  &0 \in S \notag
\end{align}
%
Če vstavimo $m \defeq \suc{n}$ dobimo:
%
\begin{align}
  &(\all{k \in \NN} \suc{k} = \suc{n} \lthen k \in S) \lthen \suc{n} \in S \tag{$\liff$} \\
  &(\all{k \in \NN} k = n \lthen k \in S) \lthen \suc{n} \in S \tag{$\liff$} \\
  &n \in S \lthen \suc{n} \in S \notag
\end{align}
%
To pa sta ravno običajna pogoja za indukcijo.

Ali lahko izrazimo indukcijo na naravnih številih tudi brez operacije naslednik?
Da, s pomočjo relacije $<$:
%
\begin{equation*}
    \all{S \in \pow{\NN}} (\all{m \in \NN} (\all{k \in \NN} k < m \lthen k \in S) \lthen m \in S) \lthen S = \NN
\end{equation*}
%
Temu principu pravimo tudi \textbf{krepka indukcija}, z besedami jo povemo takole: za podmnožico $S \subseteq \NN$ velja
$S = \NN$, če za vse $m \in \NN$ velja ">če so vsa števila manjša od $m$ v $S$, potem je tudi $m$ v $S$"<.

Denimo, da $S$ res ima dano lastnost. Ali je $0 \in S$? Da, ker za vse predhodnike $0$ velja, da
so $S$ (saj jih ni). Ali je $1 \in S$? Da, saj za vse predhodnike $1$ velja, da so v $S$. Ali je $2 \in
S$? Da, saj za vse predhodnike $2$ velja, da so v $S$. In tako naprej.


\subsection{Dobra osnovanost}

Princip indukcije na naravnih številih posplošimo, pri čemer izhajamo iz principa indukcije, izraženega s pomočjo lastnosti~\eqref{eq:ind-N}, v kateri relacijo ">neposredni predhodnik"< nadomestimo s splošno relacijo.

\begin{definicija}
  Relacija $R \subseteq A \times A$ je \textbf{dobro osnovana}, kadar velja
  %
  \begin{equation}
    \label{eq:ind-wf}%
    \all{S \in P(A)} (\all{y \in A} (\all{x \in A} x \rel{R} y \lthen x \in S) \lthen y \in S) \lthen S = A.
  \end{equation}
  %
  Množici $S \subseteq A$, ki zadošča pogoju
  %
  \begin{equation*}
  \all{y \in A} (\all{x \in A} x \rel{R} y \lthen x \in S) \lthen y \in S
  \end{equation*}
  %
  pravimo \textbf{$R$-progresivna} množica ali, da je $S$ \textbf{progresivna za $R$}.
\end{definicija}

Pogoj \eqref{eq:ind-wf} je \emph{indukcijski predpis} za dobro osnovano relacijo~$R$. Nekatere relacije temu predpisu zadoščajo in druge ne. Na primer, relacija ">neposredni predhodnik"< na $\NN$ mu zadošča, saj v tem primeru dobimo običajno indukcijo na~$\NN$.

\begin{primer}
  Preverimo, da je relacija ">neposredni predhodnik"< $P$ na množici $A = \set{0, 1, \ldots, 42}$ dobro osnovana.
  Natančneje, govorimo o relaciji
  %
  \begin{equation*}
    m \rel{P} n \defiff m + 1 = n.
  \end{equation*}
  %
  Naj bo $S \subseteq A$ progresivna množica, torej zadošča
  %
  \begin{equation*}
    \all{y \in A} (\all{x \in A} x + 1 = y \lthen x \in S) \lthen y \in S.
  \end{equation*}
  %
  Če vstavimo $y = 0$, dobimo
  %
  \begin{equation*}
    (\all{x \in A} x + 1 = 0 \lthen x \in S) \lthen 0 \in S,
  \end{equation*}
  %
  kar je ekvivalentno $0 \in S$. Torej je $0 \in S$. Nato vstavimo $y = 1$ in dobimo
  %
  \begin{equation*}
    (\all{x \in A} x + 1 = 1 \lthen x \in S) \lthen 1 \in S,
  \end{equation*}
  %
  kar se poenostavi v $0 \in S \lthen 1 \in S$. Ker smo že dokazali $0 \in S$, sledi tudi $1 \in S$. V naslednjem koraku vstavimo $y = 2$, poenostavimo in dobimo $1 \in S \lthen 2 \in S$, torej $2 \in S$. Tako nadaljujemo do $y = 42$ in ugotovimo, da res velja $S = A$. S tem smo pokazali, da je $P$ dobro osnovana. Seveda ni bistveno, da smo uporabili $42$.
\end{primer}

\subsection{Dvojiška drevesa}

Naravna števila $\NN$ so \textbf{induktivno definirana množica}. To pomeni, da elemente $\NN$
opredelimo s pravili, ki povedo, kako se gradi naravna števila:
%
\begin{itemize}
\item $0 \in \NN$,
\item če je $n \in \NN$, potem je $\suc{n} \in \NN$.
\end{itemize}
%
Množica $\NN$ vsebuje natanko tiste elemente, ki jih lahko zgradimo s pomočjo teh pravil:
%
\begin{equation*}
    0, 0^{+}, 0^{++}, 0^{+++}, 0^{++++}, \ldots
\end{equation*}
%
Tu sta $0$ in $\suc{{}}$ mišljena kot simbolni oznaki, podobno kot $\inl$ in $\inr$ v definiciji vsote množic. Dejstvo,
da $\NN$ vsebuje natanko tiste elemente, ki jih lahko zgradimo s pomočjo $0$ in $\suc{{}}$ ni nič drugega kot indukcija
na~$\NN$.

Podobno lahko definiramo tudi druge induktivne množice, ki tudi zadoščajo principu indukcije.
%
Na primer, \textbf{dvojiška drevesa} so induktivno definirana množica $\Tree$ s predpisoma:
%
\begin{itemize}
\item $\emptyTree \in \Tree$,
\item če je $t_1 \in \Tree$ in $t_2 \in \Tree$, potem je $\tree{t_1, t_2} \in \Tree$
\end{itemize}
%
Z besedami: drevo je bodisi prazno, bodisi je sestavljeno iz dveh \textbf{poddreves}. Ali znamo
našteti vsa drevesa, ali še bolje, jih narisati?
%
\begin{align*}
    & \emptyTree, \\
    & \tree{\emptyTree, \emptyTree} \\
    & \tree{\emptyTree, \tree{\emptyTree, \emptyTree}}, \\
    & \tree{\tree{\emptyTree, \emptyTree}, \emptyTree}, \\
    & \tree{\tree{\emptyTree, \emptyTree}, \tree{\emptyTree, \emptyTree}}, \\
    & \vdots
\end{align*}
%
Definirajmo relacijo $R \subseteq \Tree \times \Tree$ s predpisom:
%
\begin{equation*}
  t \rel{R} s \defiff \some{u \in \Tree} s = \tree{t, u} \lor s = \tree{u, t}.
\end{equation*}
%
To je relacija ">neposredno poddrevo"<. Je dobro osnovana, česar ne bomo dokazali, porodi pa naslednji princip indukcije za dvojiška drevesa.

\begin{izjava}[Indukcija za dvojiška drevesa]
  Naj bo $S \subseteq \Tree$ podmnožica dreves, za katero velja:
  %
  \begin{itemize}
  \item prazno drevo je v $S$,
  \item za vsa drevesa $t_1$ in $t_2$ velja: če je $t_1 \in S$ in $t_2 \in S$, potem je $\tree{t_1, t_2} \in S$.
  \end{itemize}
  %
  Tedaj je $S = \Tree$.
\end{izjava}

Princip povejmo še s pomočjo predikatov.

\begin{izjava}[Indukcija za dvojiška drevesa]
  Naj bo $\phi$ predikat na dvojiških drevesih, za katerega velja:
  %
  \begin{itemize}
  \item baza indukcije: $\phi(\emptyTree)$
  \item indukcijski korak: za vsa drevesa $t_1$ in $t_2$, če velja $\phi(t_1)$ in $\phi(t_2)$, potem
    $\phi(\tree{t_1, t_2})$.
  \end{itemize}
  %
  Tedaj $\all{t \in \Tree} \phi(t)$.
\end{izjava}

Kot vidimo, imamo v indukcijskem koraku \emph{dve} indukcijski predpostavki, ker ima vsako
sestavljeno drevo dve poddrevesi.


\subsubsection{Dobra osnovanost in padajoče verige}

Kako pa bi dobili kak proti-primer, se pravi, relacijo, ki ni dobra osnovanost? Poiskati
moramo kako lastnost, ki jo imajo vse dobre osnovanosti, nato pa relacijo, ki te lastnosti nima.

\begin{definicija}
  Naj bo $R \subseteq A \times A$ relacija na $A$. \textbf{Padajoča veriga} za relacijo $R$
  je zaporedje $a : \NN \to A$, za katerega velja $\all{i \in \NN} a(i+1) \rel{R} a(i)$.
\end{definicija}

Se pravi, da je padajoča veriga zaporedje, za katerega velja
%
\begin{equation*}
  \cdots a_4 \rel{R} a_3 \rel{R} a_2 \rel{R} a_1 \rel{R} a_0
\end{equation*}
%
\textbf{Cikel} za relacijo~$R$ je končna podmnožica $\set{a_0, \ldots, a_n} \subseteq A$ da velja
%
\begin{equation*}
  a_0 \rel{R} a_1 \rel{R} \cdots \rel{R} a_n \rel{R} a_0.
\end{equation*}
%
Iz cikla dobimo padajočo verigo, tako da cikel ponavljamo v nedogled:
%
\begin{equation*}
  \cdots \rel{R} a_0 \rel{R} \cdots \rel{R} a_n
         \rel{R} a_0 \rel{R} \cdots \rel{R} a_n \rel{R} a_0.
\end{equation*}

\begin{lema}
  V dobri osnovanosti ni ciklov in ni padajočih verig.
\end{lema}

\begin{dokaz}
  Dovolj je pokazati, da ni padajočih verig, saj iz cikla dobimo padajočo verigo.
  Denimo, da je $a : \NN \to A$ padajoča veriga za $R \subseteq A \times A$. Dokazali bomo, da $R$ ni dobro
  osnovana. Se pravi, da moramo poiskati $R$-progresivno podmnožico $S \subseteq A$, za katero velja
  $S \neq A$. Vzemimo $S \defeq A \setminus \set{ a(i) \mid i \in \NN}$. Očitno velja $S \neq A$, saj 
  $a(0) \not\in S$. Preverimo, da je $S$ progresivna, se pravi, da je
  %
  \begin{equation*}
    \all{y \in A} (\all{x \in A} x \rel{R} y \lthen x \in S) \lthen y \in S.
  \end{equation*}
  %
  Naj bo $y \in A$ in denimo, da velja
  \begin{equation}
    \label{eq:verige}
    \all{x \in A} x \rel{R} y \lthen x \in S
  \end{equation}
  %
  Dokazati moramo $y \in S$. Obravnavamo dve možnosti:
  %
  \begin{itemize}
  \item če $y \in S$, potem seveda sledi $y \in S$.
  \item če $y \not\in S$, potem obstaja $i \in \NN$, da je $y = a(i)$. Ker je $a(i+1) \rel{R} a(i)$, iz
    predpostavke~\eqref{eq:verige} sledi $y = a(i) \in S$.
  \end{itemize}
  Torej v vsakem primeru velja $y \in S$.
\end{dokaz}

\begin{primer}
  Sedaj lahko zlahka priskrbimo kak proti-primer. Na primer, cela števila $\ZZ$ z relacijo $R \subseteq \ZZ \times \ZZ$
  %
  \begin{equation*}
    a \rel{R} b \defiff a + 1 = b
  \end{equation*}
  %
  niso dobro osnovana, ker imajo padajočo verigo
  %
  \begin{equation*}
    \cdots \rel{R} (-3) \rel{R} (-2) \rel{R} (-1) \rel{R} 0
  \end{equation*}
  %
  Prav tako ni dobro osnovana relacija $<$ na intervalu $[0,1]$, ker imamo padajočo verigo
  $n \mapsto 2^{-n}$.
\end{primer}

\section{Dobra urejenost}

Posplošimo sedaj še krepko indukcijo na naravnih številih. Tokrat bomo najprej posplošili
strogo urejenost $<$.

\subsection{Stroge urejenosti}

\begin{definicija}
  Relacija $R \subseteq A \times A$ je \textbf{stroga urejenost}, če je
  %
  \begin{itemize}
  \item irefleksivna: $\all{x \in A} \lnot (x \rel{R} x)$ in
  \item tranzitivna: $\all{x, y, z \in A} x \rel{R} y \land y \rel{R} z \lthen x \rel{R} z$.
  \end{itemize}
  %
  Stroga urejenost je \textbf{linearna}, če je še
  %
  \begin{itemize}
  \item sovisna: $\all{x, y \in A} x \rel{R} y \lor x = y \lor y \rel{R} x$.
  \end{itemize}
  %
  Za stroge urejenosti uporabljamo simbole $<$, $\subset$, $\prec$, $\sqsubset$ ipd.
\end{definicija}

Relaciji $<$ in $\leq$ na številih sta med seboj povezani, saj denimo za realna števila velja
%
\begin{equation*}
  x < y \iff x \leq y \land x \neq y
\end{equation*}
%
in
%
\begin{equation}
  \label{eq:leq-iff-lteq}
  %
  x \leq y \iff x < y \lor x = y
\end{equation}
%
To velja v splošnem. Stroga urejenost $<$ na množici $A$ porodi delno urejenost $\leq$ na $A$,
definirano s predpisom:
%
\begin{equation*}
    x \leq y  \defiff x = y \lor x \leq y.
\end{equation*}
%
V obratno smer, delna urejenost $\sqsubseteq$ določa strogo urejenost $\sqsubset$, definirano s predpisom
%
\begin{equation}
  \label{eq:leq-iff-neqlt}
  a \sqsubset b  \defiff  a \neq b \land a \sqsubseteq b.
\end{equation}
%
Seveda je treba preveriti naslednja dejstva, ki jih postimo za vajo:
%
\begin{itemize}
\item če je $<$ stroga urejenost, potem je $\leq$ definirana s \eqref{eq:leq-iff-lteq} delna urejenost
\item če je $\sqsubseteq$ delna urejenost, potem je $\sqsubset$ definirana s \eqref{eq:leq-iff-neqlt} stroga urejenost.
\end{itemize}
%
Tako lahko prehajamo med delno in strogo urejenostjo.

\subsection{Dobra ureditev}

\begin{definicija}
  Relacija je \textbf{dobra ureditev}, če je dobro osnovana in stroga linearna ureditev.
\end{definicija}

\begin{izrek}
  Relacija je dobra ureditev natanko tedaj, ko je dobro osnovana in sovisna.
\end{izrek}

\begin{dokaz}
  V eno smer je ekvivalenca očitna, zato dokažimo samo obratno smer. Denimo, da je
  $R \subseteq A \times A$ dobro osnovana in sovisna relacija. Dokazujemo, da je dobra ureditev, se pravi,
  da potrebujemo še irefleksivnost in tranzitivnost $R$.

  Relacija $R$ je irefleksivna: če bi veljalo $x \rel{R} x$ za $x \in A$, potem $R$ ne bi bila dobro
  osnovana, ker bi vsebovala padajočo verigo $\cdots x \rel{R} x \rel{R} x$.

  Relacija $R$ je tranzitivna: denimo, da velja $x \rel{R} y$ in $y \rel{R} z$. Dokazujemo $x \rel{R} z$. Ker je $R$
  sovisna, velja $x \rel{R} z$ ali $x = z$ ali $z \rel{R} x$. Pokažimo, da $x = z$ in $z \rel{R} x$ nista
  možna:
  %
  \begin{itemize}
  \item Če je $x = z$, potem velja $x \rel{R} y$ in $y \rel{R} x$, torej $x$ in $y$ tvorita cikel, a
    $R$ je dobro osnovana, zato to ni možno.
  \item Če velja $z \rel{R} x$, potem dobimo cikel $x \rel{R} y \rel{R} z \rel{R} x$, kar spet ni možno.
  \end{itemize}
\end{dokaz}

\begin{lema}
  \label{lem:nepr-min-veriga}%
  Denimo, da je $<$ stroga urejenost na neprazni množici~$B$. Če $B$ nima
  $\leq$-minimalnega elementa, potem ima padajočo verigo.
\end{lema}

\begin{dokaz}
  Denimo, da $B$ nima minimalnega elementa, torej
  %a
  \begin{equation*}
    \lnot \some{x \in B} \all{y \in B} y \leq x \lthen y = x.
  \end{equation*}
  %
  To je ekvivalentno
  %
  \begin{equation*}
    \all{x \in B} \some{y \in B} y \leq x \land y \neq x
  \end{equation*}
  %
  kar je ekvivalentno
  \begin{equation}
    \label{eq:lema-min-elem}
    \all{x \in B} \some{y \in B} y < x.
  \end{equation}
  %
  Padajočo verigo $b : \NN \to B$ definiramo z zaporedjem izbir: ker je $B$ neprazna, lahko izberemo
  neki element $b(0) \in B$. Denimo, da smo za neki $i \in \NN$ že izbrali elemente $b(0), \ldots, b(i)$
  tako, da velja
  %
  \begin{equation*}
    b(i) < b(i-1) < \ldots < b(1) < b(0).
  \end{equation*}
  %
  Ker $B$ nima minimalnega elementa, $b(i)$ ni minimalni, torej po \eqref{eq:lema-min-elem} obstaja tak $y \in B$, da je $y < b(i)$. Torej lahko izberemo $b(i+1) \in B$, da velja $b(i+1) < b(i)$.
\end{dokaz}

\begin{opomba}
  V zgornjem dokazu smo uporabili \emph{aksiom odvisne izbire}, ki je poseben primer
  aksioma izbire in o katerem bomo še govorili.
\end{opomba}

\begin{izrek}
  \label{izr:dobr-osn-iff}
  Naj bo $\sqsubset$ relacija na $A$. Tedaj so ekvivalentne naslednje izjave:
  %
  \begin{enumerate}
  \item \label{it:dobr-osn-1}%
    $\sqsubset$ je dobro osnovana,
  \item \label{it:dobr-osn-2}%
    vsaka \emph{neprazna} $S \subseteq A$ ima $\sqsubseteq$-minimalni element,
  \item \label{it:dobr-osn-3}%
    $\sqsubset$ nima padajoče verige.
  \end{enumerate}
\end{izrek}

\begin{dokaz}
  ($1 \lthen 2$)
  %
  Denimo, da je $S \subseteq A$ neprazna. Če uporabimo \eqref{it:dobr-osn-1} na $A \setminus S$ dobimo
  %
  \begin{equation*}
    (\all{y \in A} (\all{x \in A} x \sqsubset y \lthen x \in A \setminus S) \lthen y \in A \setminus S) \lthen A \setminus S = A.
  \end{equation*}
  %
  Ker je $S$ neprazna, dobimo zaporedje ekvivalentnih izjav:
  \begin{align*}
    &(\all{y \in A} (\all{x \in A} x \sqsubset y \lthen x \in A \setminus S) \lthen y \in A \setminus S) \lthen \bot
    \tag{$\liff$} \\
    &\lnot (\all{y \in A} (\all{x \in A} x \sqsubset y \lthen x \in A \setminus S) \lthen y \in A \setminus S)
    \tag{$\liff$} \\
    &\some{y \in A} (\all{x \in A} x \sqsubset y \lthen x \in A \setminus S) \land y \not\in A \setminus S
    \tag{$\liff$} \\
    &\some{y \in A} (\all{x \in A} x \sqsubset y \lthen x \not\in S) \land y \in S
    \tag{$\liff$} \\
    &\some{y \in S} \all{x \in A} x \sqsubset y \lthen x \not\in S
    \tag{$\liff$} \\
    &\some{y \in S} (\all{x \in A} x \sqsubset y \lthen x \not\in S) \notag
  \end{align*}
  %
  Torej obstaja element $y \in S$ z lastnostjo, da pod njim ni nobenega elementa iz
  $S$, kar pa pomeni, da je $y$ iskani minimalni element.

  ($2 \lthen 3$) Denimo, da je $a : \NN \to A$ padajoča veriga. Tedaj slika $\set{ a(n) \mid n \in \NN }$ ne bi imela
  minimalnega elementa, v nasprotju z \eqref{it:dobr-osn-2}.

  ($3 \lthen 1$) Denimo, da je $S \subseteq A$ progresivna. Trdimo, da množica $C \defeq A \setminus S$ nima
  minimalnega elementa. Če bi bil $c \in C$ minimalni v $C$, bi to pomenilo
  %
  \begin{equation*}
    \all{x \in A} x \sqsubset c \lthen x \not\in C,
  \end{equation*}
  %
  kar je ekvivalentno
  %
  \begin{equation*}
    \all{x \in A} x \sqsubset c \lthen x \in S.
  \end{equation*}
  %
  Ker je $S$ progresivna, od tod sledi $c \in S$, kar ni mogoče.
  %
  Dokazati moramo, da je $C$ prazna. Če ne bi bila, bi lahko uporabili lemo~\ref{lem:nepr-min-veriga} in dobili padajočo verigo v $A$, kar je v nasprotju s \eqref{it:dobr-osn-3}.
\end{dokaz}

\begin{izrek}
  \label{izr:dobra-urejenost-karakterizacija}
  Naj bo $\sqsubset$ stroga urejenost na $A$. Tedaj so ekvivalentne naslednje izjave:
  %
  \begin{enumerate}
  \item[(1)] $\sqsubset$ je dobro urejena,
  \item[(2)] vsaka \emph{neprazna} množica $S \subseteq A$ ima $\sqsubset$-prvi element: to je tak $x \in S$, da velja
    $\all{y \in S} x \neq y \lthen x \sqsubset y$.
  \item[(3)] $A$ nima $\sqsubset$-padajoče verige in $\sqsubset$ je sovisna.
  \end{enumerate}
\end{izrek}

\begin{dokaz}
  Za nalogo predelajte dokaz prejšnjega izreka v dokaz tega izreka.
\end{dokaz}

\begin{primer}
  Primeri dobro urejenih množic:
  %
  \begin{enumerate}
  \item Končna množica $\set{0, \ldots, n}$ urejena z relacijo $<$.

  \item Naravna števila $\NN$ urejena z relacijo $<$.

  \item Če sta $(P, \leq_P)$ in $(Q, \leq_Q)$ dobri urejenosti, potem je dobro urejena tudi $P + Q$ z relacijo
    $\sqsubseteq$, ki~$P$ postavi pred~$Q$:
    %
    \begin{equation*}
      u \sqsubseteq v \defiff
      \begin{aligned}[t]
        &(\some{x \in P} \some{y \in Q} u = \inl(x) \land v = \inr(y)) \lor {}\\
        &(\some{x \in P} \some{y \in P} u = \inl(x) \land v = \inl(y) \lor x \leq_P y) \lor {}\\
        &(\some{x \in Q} \some{y \in Q} u = \inr(x) \land v = \inr(y) \lor x \leq_Q y).
      \end{aligned}
    \end{equation*}

  \item
    S prejšnjim primerom lahko seštevamo dobre urejenosti, na primer $\NN + \set{0, 1, 2}$ je dobra
    urejenost
    %
    \begin{equation*}
      \inl{0} < \inl{1} < \inl{2} < \cdots < \inr{0} < \inr{1} < \inr{2}.
    \end{equation*}
  \end{enumerate}
\end{primer}

\section{Ordinalna števila}
\label{sec:ordinalna-tevila}

Dobra urejenost na množici~$A$ postavi njene elemente v vrsto (strogo linearno urejenost), ki nima padajočih verig.
Končno množico lahko dobro uredimo na več načinov, na primer elemente $\set{0, 1, 2, \ldots, n-1}$ lahko postavimo v vrsto na $n!$ načinov. Množico vseh naravnih števil lahko postavimo v vrsto brez padajočih verig vsaj na tri načine,
%
\begin{equation*}
  0, 1, 2, 3, 4, 5, \ldots, n, n + 1, \ldots
\end{equation*}
%
in
%
\begin{equation*}
  1, 0, 3, 2, 5, 4, \ldots, 2 n + 1, 2 n, \ldots
\end{equation*}
%
in
%
\begin{equation*}
  0, 2, 4, 6, 8, \ldots, 1, 3, 4, 5, \ldots
\end{equation*}
%
Zdi se, da sta prvi in drugi način ">isti tip"< urejenosti in se razlikujeta od tretjega. Res, v tretji vrsti ima $1$ neskončno predhodnikov, v prvi in drugi pa takega elementa ni. Govorimo o naslednjem pojmu.

\begin{definicija}
  Dobri ureditvi $(P, {\leq_P})$ in $(Q, {\leq_Q})$ \textbf{izomorfni}, če obstajata monotoni preslikavi $f : P \to Q$ in $Q \to P$, da velja $f \circ g = \id[Q]$ in $g \circ f = \id[P]$.
\end{definicija}

Seveda je izomorfnost ekvivalenčna relacija, ki je definirana na pravem razredu vseh dobrih urejenosti. 
Koristno bi bilo imeti kak izbor predstavnikov zanjo, saj bi lahko z njimi merili ">dolžino"< dobre urejenosti. Takim predstavnikom pravimo \textbf{ordinalna števila}. A kako bi jih dobili? Pri 19.~letih je \href{https://en.wikipedia.org/wiki/John_von_Neumann}{John von Neumann} predlagal:
%
\begin{quote}
  \emph{">Ordinalno število je množica svojih predhodnikov, urejeno z relacijo $\in$."<}
\end{quote}
%
Poglejmo, kako deluje njegova ideja:
%
\begin{itemize}

\item Končna ordinalna števila sovpadajo z naravnimi števili:
  %
  \begin{align*}
    0 &\defeq \emptyset \\
    1 &\defeq \set{0} = \set{\emptyset} \\
    2 &\defeq \set{0, 1} = \set{\emptyset, \set{\emptyset}} \\
    3 &\defeq \set{0, 1, 2} = \set{\emptyset, \set{\emptyset}, \set{\emptyset, \set{\emptyset}}} \\
      &\vdots
  \end{align*}

\item Množica vseh končnih ordinalnih števil je prvo neskončno ordinalno število
  %
  \begin{equation*}
    \omega = \set{0, 1, 2, 3, \ldots}.
  \end{equation*}

\item Številu $\omega$ sledijo
  %
  \begin{align*}
    \omega + 1 &\defeq \set{0, 1, 2, \ldots, \omega} \\
    \omega + 2 &\defeq \set{0, 1, 2, \ldots, \omega, \omega + 1} \\
    \omega + 3 &\defeq \set{0, 1, 2, \ldots, \omega, \omega + 1, \omega + 2} \\
               &\vdots \\
    \omega + \omega &\defeq \set{0, 1, 2, \ldots, \omega, \omega + 1, \omega + 2, \ldots} \\
    \omega + \omega + 1 &\defeq \set{0, 1, 2, \ldots, \omega, \omega + 1, \omega + 2, \ldots, \omega + \omega} \\
               &\vdots
  \end{align*}
\end{itemize}

\begin{naloga}
  Kako bi si predstavljali naslednje ordinale: $\omega + \omega + \omega$, $\omega \cdot \omega$, $\omega^3$, $\omega^\omega$?
\end{naloga}

Von Neumann je imel pravo idejo, a pušča kanček dvoma, ker je definicija ordinalnega števila \emph{rekurzivna} (se nanaša sama nase). Če se malce potrudimo, da lahko von Neumannove ordinale opredelimo neposredno.

\begin{definicija}
  Množica $z$ je \textbf{tranzitivna}, če iz $x \in y$ in $y \in z$ sledi $x \in z$.
\end{definicija}

\noindent
%
Poimenovanje je smiselno, saj je pogoj v definiciji ravno tranzitivnost relacije~$\in$.
Ekvivalentno lahko pogoj izrazimo takole: množica $z$ je tranzitivna, če iz $y \in z$ sledi $y \subseteq z$.

\begin{primer}
  Množica $\set{\emptyset, \set{\emptyset}, \set{\set{\emptyset}}}$ je tranzitivna, niso pa vsi njeni elementi tranzitivne množice, saj $\set{\set{\emptyset}}$ ni tranzitivna, ker $\emptyset \in \set{\emptyset} \in \set{\set{\emptyset}}$ vendar $\emptyset \not\in \set{\set{\emptyset}}$.
\end{primer}

\begin{naloga}
  Dokažite, da so ekvivalentni pogoji:
  %
  \begin{enumerate}
  \item $A$ je tranzitivna množica,
  \item $\bigcup A \subseteq A$,
  \item $A \subseteq \pow{A}$.
  \end{enumerate}
\end{naloga}

Sedaj lahko zapišemo definicijo von Neumannovih ordinalov, ki ni rekurzivna.

\begin{definicija}
  \label{def:von-neuman-ordinal}
  \textbf{(Von Neumannov) ordinal} je tranzitivna množica, ki je z relacijo $\in$ dobro urejena.
\end{definicija}

Razred vseh von Neumannovih ordinalov označimo z $\On$ (v angleščini ">ordinal number"<). To je pravi razred, česar ne bomo dokazali. Kogar zanima dokaz, naj poišče ">Burali-Fortijev paradoks"<, ki je celo starejši od Russellovega paradoksa.

\begin{naloga}
  Poiščite množico, ki \emph{ni} tranzitivna in je dobro urejena z relacijo $\in$.
\end{naloga}

Ali definicija~\ref{def:von-neuman-ordinal} res sovpada z idejo, da je ordinal množica svojih prednikov? To potrjuje naslednja izjava.

\begin{izjava}
  Če je $\alpha$ ordinal in $\beta \in \alpha$, potem je $\beta$ ordinal.
\end{izjava}

\begin{dokaz}
  Ker je $\alpha$ tranzitivna množica, je $\beta \subseteq \alpha$, zato je $\beta$ z relacijo~$\in$ dobro urejen. Dokazati moramo še, da je $\beta$ tranzitivna množica. Denimo, da je $\gamma \in \beta$.
  Tedaj je $\gamma \in \alpha$ in ker je $\alpha$ z $\in$ linearno urejen, velja bodisi $\gamma \in \beta$ bodisi $\gamma = \beta$ bodisi $\beta \in \gamma$. A ker druga in tretja možnost ne prideta v poštev, saj bi dobili cikel $\gamma \in \beta \in \gamma$, velja prva, kar smo želeli dokazati.
\end{dokaz}

\begin{naloga}
  V zgornjem dokazu smo uporabili naslednje dejstvo: če je $(P, {<})$ dobra ureditev in $Q \subseteq P$, tedaj je $Q$ z relacijo $<$ zoženo na~$Q$ tudi dobra ureditev. Zapišite dokaz.
\end{naloga}

Brez dokaza navedimo, da so von Neumannovi ordinali izbor predstavnikov za dobre urejenosti.

\begin{izrek}
  Vsaka dobra ureditev je izomorfna natanko enemu von Neumannovemu ordinalu.
\end{izrek}

