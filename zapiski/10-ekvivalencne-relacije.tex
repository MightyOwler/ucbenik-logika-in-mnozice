\chapter{Ekvivalenčne relacije}

\section{Ekvivalenčne relacije}

\begin{definicija}
  Relacija $R \subseteq A \times A$ je \textbf{ekvivalenčna relacija}, če je refleksivna, tranzitivna in simetrična. Kadar velja $x \rel{R} y$, pravimo, da sta $x$ in $y$ \textbf{ekvivalentna} glede na~$R$.
\end{definicija}

\begin{opomba}
  Kdor reče ">ekvivalentna relacija"<, je noob. Kdor reče, da sta ">$x$ in $y$
  ekvivalenčna"<, je rookie.
\end{opomba}

Ekvivalenčne relacije se običajno označuje s simboli, ki so podobni znaku za enakost:
$\equiv$, $\sim$, $\simeq$, $\cong$.

\begin{primer}
  Primeri ekvivalenčnih relacij:
  \begin{enumerate}
    \item Relacija ">vzporednost"< med premicami v ravnini.
    \item Relacija ">skladnost"< med trikotniki v ravnini.
    \item Relacija ">podobnost"< med trikotniki v ravnini.
    \item Relacija ">isti ostanek pri deljenju s 7"< na množici $\NN$.
    \item Prazna relacija $\emptyset \subseteq A \times A$ je ekvivalenčna le v primeru, da je $A = \emptyset$.
    \item Polna relacija $A \times A$ je ekvivalenčna.
    \item Diagonala (enakost) je ekvivalenčna relacija.
  \end{enumerate}
\end{primer}

\subsection{Ekvivalenčna relacija porojena s preslikavo}

Posebej pomemben je primer ekvivalenčne relacije \textbf{porojene (ali inducirane) s preslikavo}:
naj bo $f : A \to B$ preslikava in definirajmo relacijo $\sim_f$ na $A$ s predpisom
%
\begin{equation*}
  x \sim_f y \liff f(x) = f(y)
\end{equation*}
%
Tedaj je $\sim_f$ ekvivalenčna relacija:
%
\begin{itemize}
\item refleksivnost: $x \sim_f x$ velja, ker velja $f(x) = f(x)$,
\item tranzitivnost: če je $x \sim_f y$ in $y \sim_f z$, potem je $f(x) = f(y)$ in $f(y) = f(z)$, torej $f(x) = f(z)$ in $x \sim_f z$,
\item simetričnost: če je $x \sim_f y$, potem je $f(x) = f(y)$, torej $f(y) = f(x)$ in $y \sim_f x$.
\end{itemize}
%
Ali je vsaka ekvivalenčna relacija porojena z neko preslikavo?

\begin{primer}
  Premici sta vzporedni natanko tedaj, ko imata enaka smerna vektorja. Če je
  torej $P$ množica vseh premic, $\RR^2$ množica vektorjev v ravnini, in $s : P \to \RR^2$
  preslikava, ki premici $P$ priredi njen enotski smerni vektor, ki leži v zgornji polravnini ali
  na pozitivnem delu osi $x$, tedaj velja
  \begin{equation*}
    p \parallel q \liff s(p) = s(q).
  \end{equation*}
  %
  Torej je vzporednost porojena s preslikavo $s$.
\end{primer}

\section{Ekvivalenčni razredi in kvocientne množice}

\begin{definicija}
  Naj bo $E \subseteq A \times A$ ekvivalenčna relacija. \textbf{Ekvivalenčni razred} elementa $x \in A$ je množica
  $[x]_E \defeq \set{ y \in A \such x \rel{E} y }$. Z besedami: ekvivalenčni razred~$x$ je množica vseh elementov, ki so mu
  ekvivalentni.
\end{definicija}

\begin{opomba}
  Kdor reče ">ekvivalentni razred"<, je newbie.
  Če pustimo šalo ob strani: ekvivalenčni razredi se tako imenujejo zaradi zgodovinskih razlogov. Beseda ">razred"< nakazuje dejstvo, da so imajo elementi ekvivalenčnega razredi vsi nekaj skupnega (">delavski razred"<, ">Tina Maze je razred zase"<) in ne, da niso množice (saj očitno so).
\end{opomba}

\begin{definicija}
  Naj bo $E \subseteq A \times A$ ekvivalenčna relacija. \textbf{Kvocientna ali faktorska množica} ali \textbf{kvocient} $A/E$ je množica vseh ekvivalenčnih razredov:
  %
  \begin{equation*}
    A/E \defeq \set{ \xi \in \pow{A} \such \some{x \in A} \xi = [x]_E }.
  \end{equation*}
  %
  Z izpeljanimi množicami lahko to zapišemo bolj razumljivo
  % 
  \begin{equation*}
    A/E = \set{ [x]_E \such x \in A }.
  \end{equation*}
  %
  \textbf{Kanonična kvocientna preslikava} $q_E : A \to A/E$ je preslikava, ki vsakemu elementu
  priredi njegov ekvivalenčni razred: $q_E(x) \defeq [x]_E$.
\end{definicija}

\begin{izrek}
  Vsaka ekvivalenčna relacija je porojena z neko preslikavo.
\end{izrek}

\begin{dokaz}
  Dokažimo, da je ekvivalenčna relacija porojena s svojo kvocientno preslikavo.

  Naj bo $E$ ekvivalenčna relacija na $A$. Najprej ugotovimo naslednje: za vse $x,
  y \in A$ velja
  %
  \begin{equation*}
    x \rel{E} y \liff [x]_E = [y]_E.
  \end{equation*}

  ($\lthen$) Če je $x \rel{E} y$ potem je $[x]_E \subseteq [y]_E$, ker iz $z \rel{E} x$ in $x \rel{E} y$ sledi $z \rel{E} y$.
  Podobno dokažemo $[y]_E \subseteq [x]_E$.

  ($\Leftarrow$) Če je $[x]_E = [y]_E$ potem je $y \in [y]_E = [x]_E$, torej po definiciji $[x]_E$
  dobimo $x \rel{E} y$.

  Sedaj izrek sledi zlahka: $q_E(x) = q_E(y) \liff [x]_E = [y]_E \liff x \rel{E} y$.
\end{dokaz}


\subsection{Razdelitev množice}

\begin{definicija}
  \textbf{Razdelitev} ali \textbf{particija} množice $A$ je množica nepraznih, paroma
  disjunktnih množic, ki tvorijo pokritje $A$ (kar pomeni, da je $A$ enaka njihovi uniji). Se
  pravi, to je množica $S \subseteq \pow{A}$, za katero velja:
  %
  \begin{enumerate}
  \item Elementi razdelitve so neprazni: $\all{B \in S} B \neq \emptyset$.
  \item Vsaka dva elementa razdelitve sta bodisi enaka bodisi disjunktna:
    %
    \begin{equation*}
      \all{B, C \in S} B = C \lor B \cap C = \emptyset.
    \end{equation*}
  \item Elementi razdelitve tvorijo pokritje $A$, se pravi $A = \bigcup S$.
  \end{enumerate}
\end{definicija}

\begin{primer}
  Primeri razdelitev:
  %
  \begin{enumerate}
  \item Navpične premice tvorijo razdelitev ravnine.
  \item Množici sodih in lihih števil tvorita razdelitev naravnih števil.
  \item Množica $\set{\set{1,2}, \set{3,5}, \set{4,6,7}}$ tvori razdelitev $\set{1,2,3,4,5,6,7}$.
  \item Množica $\set{\set{1,2,3,4,5,6,7}}$ tvori razdelitev $\set{1,2,3,4,5,6,7}$.
  \end{enumerate}
\end{primer}

\begin{izrek}
  Naj bo $E \subseteq A \times A$ ekvivalenčna relacija. Njeni ekvivalenčni razredi tvorijo
  razdelitev množice $A$.
\end{izrek}

\begin{dokaz}
  Dokažimo, da so ekvivalenčni razredi neprazni, paroma disjunktni in da tvorijo pokritje.

  Naj bo $\xi \in \pow{A}$ ekvivalenčni razred za $E$. Tedaj obstaja $x \in A$, da je $\xi = [x]_E$,
  torej je $x \in \xi$ in zato $\xi \neq \emptyset$.

  Naj bosta $\zeta, \xi \in \pow{A}$. Dokazali bomo $\zeta \cap \xi \neq \emptyset \lthen \zeta = \xi$. Če je $x \in \zeta \cap \xi$, potem velja $\zeta \subseteq \xi$ ker: naj bo $y \in \zeta$, tedaj je $y \rel{E} x$ in ker je $x \in \xi$ velja $y \in \xi$. Simetrično dokažemo $\xi \subseteq \zeta$.

  Očitno je unija vseh ekvivalenčnih razredov podmnožica $A$, saj je vsak ekvivalenčni razred podmnožica $A$. Zagotovo
  pa je vsak $x \in A$ v kakem ekvivalenčnem razredu, namreč $x \in [x]_E$.
\end{dokaz}

Torej vsaka ekvivalenčna relacija na $A$ določa razdelitev množice $A$, namreč na
ekvivalenčne razrede. Velja pa tudi obrat: vsaka razdelitev $S \subseteq \pow{A}$ določa ekvivalenčno
relacijo na $A$, namreč $\simeq_S$ definiran s predpisom
\begin{equation*}
    x \simeq_S y \defiff \some{B \in S} x \in B \land y \in B.
\end{equation*}
%
Z besedami: $x$ in $y$ sta ekvivalentna, kadar sta v istem elementu razdelitve. Pravzaprav
smo ugotovili, da imamo izomorfizem množic
%
\begin{equation*}
  \set{ E \subseteq A \times A \such \text{$E$ je ekvivalenčna relacija na $A$} } \iso
  \set{ S \subseteq \pow{A} \such \text{$S$ je razdelitev $A$} }.
\end{equation*}
%
V eno smer izomorfizem ekvivalenčni relaciji $E$ priredi njeno razdelitev, v drugo pa razdelitvi priredimo ekvivalenčno
relacijo, kakor smo to opisali zgoraj. (Premislite, da sta ti preslikavi inverza.)


\subsection{Prerezi kvocientne preslikave in aksiom izbire}

Ekvivalenčni razred je natanko določen že z enim od svojih elementov, zato pogosto želimo
namesto ekvivalenčnih razredov navesti le njihove predstavnike.

\begin{definicija}
  Naj bo $E$ ekvivalenčna relacija na $A$. Množico $C \subseteq A$, ki vsak
  ekvivalenčni razred relacije $E$ seka natanko enkrat, imenujemo \textbf{izbor predstavnikov}
  (ekvivalenčnih razredov) za relacijo $E$.
\end{definicija}

Izbor predstavnikov $C \subseteq A$ za $E$ določa preslikavo $c : A/E \to A$, ki priredi
ekvivalenčnemu razredu $\xi$ tisti $x \in \xi$, ki je element $C$:
%
\begin{align*}
  c &: A/E \to A \\
  c &: \xi \mapsto \descr{x \in \xi} x \in C
\end{align*}
%
Preslikava $c : A/E \to A$ je \emph{prerez} kvocientne preslikave $q_E : A \to A/E$.

\begin{izjava}
  Če je $s : A/E \to A$ prerez kvocientne preslikave $q_E : A \to A/E$, potem je
  njegova slika $\img{s}(A/E) = \set{ c(\xi) \such \xi \in A/E }$ izbor predstavnikov za $E$.
\end{izjava}

\begin{dokaz}
  Vaja.
\end{dokaz}

Ker izbor predstavnikov in prerez kvocientne preslikave določata drug drugega, včasih tudi
prerez imenujemo ">izbor predstavnikov"<.

\begin{primer}
  Definirajmo $\sim$ na množici celih števil $Z$ s predpisom
  %
  \begin{equation*}
    a \sim b \defiff 7 \mathrel{|} a - b.
  \end{equation*}
  %
  Torej sta števili $a$ in $b$ ekvivalentni, če dasta enak ostanek pri deljenju s~$7$,
  na primer $13 \sim 20$ in $\lnot (13 \sim 15)$.
  %
  Ekvivalenčni razred števila $a$ dobimo tako, da $a$ prištejemo vse večkratnike števila $7$:
  %
  \begin{equation*}
    [a]_{\sim} = \set{ a + 7 \cdot k \such k \in \ZZ }.
  \end{equation*}
  %
  Na primer,
  \begin{equation*}
    [13]_\sim = \set{ 7 \cdot k + 13 \such k \in \ZZ }
           = \set{ \ldots, -22, -15, -8, -1, 6, 13, 20, 27, 34, 41, \ldots}.
  \end{equation*}
  %
  Koliko pa je ekvivalenčnih razredov? Toliko, kot je ostankov pri deljenju s~$7$, torej sedem. Množica
  $\set{0, 1, 2, 3, 4, 5, 6}$ je izbor predstavnikov za $\sim$, saj je vsako celo število ekvivalentno natanko enemu od
  teh števil po modulu $7$.
  %
  Ni pa to edini izbor! Tudi $\set{0, 1, 2, 3, 4, 5, 13}$ je izbor in prav tako $\set{-7, -6, -5, -4, -3, -2, -1}$.
\end{primer}

Ali ima vsaka ekvivalenčna relacija izbor predstavnikov? Da to vprašanje ni tako
enostavno, kot se zdi na prvi pogled, doma premislite o naslednji nalogi.

\begin{naloga}
  Na množici realnih števil $\RR$ definiramo relacijo $E$ s predpisom
  %
  \begin{equation*}
    x \rel{E} y  \defiff  x - y \in \QQ.
  \end{equation*}
  %
  Se pravi, da sta števili ekvivalentni, če je njuna razlika racionalno število. Podajte kak
  izbor predstavnikov za $E$.
\end{naloga}

\begin{izrek}
  Naslednje izjave so ekvivalentne:
  %
  \begin{enumerate}
  \item Vsaka surjektivna preslikava ima desni inverz (prerez).
  \item Vsaka ekvivalenčna relacija ima izbor predstavnikov.
  \item Vsaka družina nepraznih množic ima funkcijo izbire.
  \item Produkt družine nepraznih množic je neprazen.
  \end{enumerate}
\end{izrek}

\begin{dokaz}
  ($1 \lthen 2$):
  %
  Naj bo $E \subseteq A \times A$ ekvivalenčna relacija na $A$. Tedaj je $q_E : A \to A/E$
  surjektivna, zato ima po predpostavki (1) prerez, ki določa izbor predstavnikov.

  ($2 \lthen 3$):
  %
  Naj bo $A : I \to \Set$ družina nepraznih množic. Naj bo $\sim$ ekvivalenčna relacija
  na koproduktu $K \defeq \sum_{i \in I} A_i$, porojena s prvo projekcijo $\fst : S \to I$, t.j.,
  %
  \begin{equation*}
    \inj[i](x) \sim \inj[j](y) \liff i = j.
  \end{equation*}
  %
  Po predpostavki (2) obstaja izbor predstavnikov za $\sim$, se pravi taka množica $C \subseteq K$, da
  za vsak $u \in K$ obstaja natanko en $v \in C$, da je $\fst(u) = \fst(v)$. Definirajmo $f : I \to
  \bigcup A$ s predpisom
  %
  \begin{equation*}
    f(i) \defeq \descr{x \in A_i} \inj[i](x) \in C
  \end{equation*}
  %
  Očitno je $f$ funkcija izbire za družino $A$, če je izraz na desni veljaven:
  %
  \begin{itemize}
  \item Enoličnost: iz $\inj[i](x) \in C$ in $\inj[i](y) \in C$ sledi $\inj[i](x) = \inj[j](y)$.
  \item Celovitost: ker je $A_i$ neprazna, obstaja $z \in A_i$, torej obstaja $v \in C$, da je
    $i = \fst(\inj[i](z)) = \fst(v)$, in je potemtakem $\snd(v) \in A_i$ element, za katerega velja
    $\inj[i](\snd(v)) \in C$.
  \end{itemize}

  ($3 \lthen 4$):
  %
  Elementi produkta so funkcije izbire, zato je produkt res neprazen, če obstaja
  kaka funkcija izbire.

  ($4 \lthen 1$):
  %
  Naj bo $f : X \to Y$ surjektivna. Definirajmo družino $A : Y \to \Set$ s
  predpisom $A_y = \invimg{f}(\set{y})$. Ker je $f$ surjektivna, je $A$ družina nepraznih
  množic. Po predpostavki (4) je produkt te družine neprazen, torej vsebuje neko
  funkcijo izbire $c : Y \to \bigcup A$, se pravi, da je $f(c(y)) = y$ za vsak $y \in Y$.
  Opazimo še, da je $\bigcup A = Y$, torej je $c$ prerez $f$.
\end{dokaz}

Izbor predstavnikov je torej ekvivalenten še nekaterim drugim trditvam. Pa te veljajo? Za
to potrebujemo aksiom.

\begin{aksiom}[Aksiom izbire]
  Vsaka družina nepraznih množic ima funkcijo izbire.
\end{aksiom}

Se pravi, če je $A : I \to \Set$ taka družina množica, da za vsak $i \in I$ velja $A_i \neq \emptyset$,
tedaj obstaja $f : I \to \bigcup A$, za katerega je $f(i) \in A_i$ za vse $i \in I$.
%
O aksiomu izbire bomo še govorili.


\subsection{Univerzalna lastnost kvocientne množice}

Naj bo $E$ ekvivalenčna relacija na $A$ in $B$ množica. Pogosto želimo definirati
preslikavo
%
\begin{equation*}
    f : A/E \to B
\end{equation*}
%
s pomočjo preslikave $A \to B$. Kdaj lahko to naredimo?

\begin{izrek}
  Naj bo $E$ ekvivalenčna relacija na $A$ in $g : A \to B$ preslikava, ki je \emph{skladna} z $E$, kar pomeni da $g$
  slika ekvivalentne elemente v enake: $\all{x, y \in A} x \rel{E} y \lthen g(x) = g(y)$. Tedaj obstaja natanko ena
  preslikava $f : A/E \to B$, da je $f([x]_E) = g(x)$ za vse $x \in A$, ali drugače povedano, $f \circ q_E = g$.
\end{izrek}

\begin{dokaz}
  Dokažimo najprej, da imamo največ eno tako preslikavo. Denimo da za $f_1 : A/E \to B$ in
  $f_2 : A/E \to B$ velja $f_1 \circ q_E = f_2 \circ q_E$. Ker je $q_E$ surjektivna, je epi in jo smemo
  krajšati na desni, od koder res sledi $f_1 = f_2$.

  Sedaj dokažimo, da $f$ obstaja. V ta namen naj bo $\phi \subseteq A/E \times B$ relacija
  %
  \begin{equation*}
    \phi(\xi, y) \defiff \some{x \in A} x \in \xi \land g(x) = y.
  \end{equation*}
  %
  Trdimo, da je $\phi$ funkcijska relacija:
  %
  \begin{itemize}
  \item
    Enoličnost: če je $\phi(\xi, y_1)$ in $\phi(\xi, y_2)$, potem obstajata $x_1, x_2 \in \xi$, da je $g(x_1) = y_1$
    in $g(x_2) = y_2$. Ker pa velja $x_1 \rel{E} x_2$ in je $g$ skladna z $E$, sledi $y_1 = g(x1) = g(x_2) = y_2$.

  \item  Celovitost: naj bo $\xi \in A/E$. Tedaj obstaja $x \in \xi$. Očitno velja $g(\xi, g(x))$.
  \end{itemize}
  %
  Naj bo $f : A/E \to B$ preslikava, ki je določena s funkcijsko relacijo $\phi$. Za $x \in A$
  velja $\phi([x]_E, f([x]_E))$, od tod pa iz definicije $\phi$ sledi tudi $g(x) = f([x]_E)$.
\end{dokaz}

\begin{opomba}
  Profesorja prosite, da pojasni ali sem zapiše, zakaj se reče ">univerzalna lastnost"< kvocientne množice.
\end{opomba}


\section{Kanonična razčlenitev preslikave}

Naj bo $f : A \to B$ preslikava. Naj bo $\sim_f$ ekvivalenčna relacija na $A$, ki jo porodi
$f$, in $q_f : A \to A/E$ kanonična kvocientna preslikava (morali bi jo pisati $q_{\sim_f}$,
kar je nečitljivo). Naj bo $i : \img{f}(A) \to B$ kanonična inkluzija slike $f$ v kodomeno.
Preslikava $f : A \to \img{f}(A)$ je skladna s $\sim_f$, zato obstaja (natanko ena) preslikava
$b_f : A/f \to \img{f}(A)$, da velja $b_f([x]_\sim) = f(x)$. Trdimo:
%
\begin{enumerate}
\item $f = i_f \circ b_f \circ q_f$ in
\item $q_f$ je surjektivna, $b_f$ je bijektivna in $i_f$ je injektivna.
\end{enumerate}
%
Računajmo: $f(x) = b_f([x]_\sim) = i_f(b_f([x]_\sim)) = i_f(b_f(q_f(x)))$, za vse $x \in A$, od
koder sledi prva trditev.

Vemo že, da je kanonična kvocientna preslikava surjektivna in kanonična inkluzija
injektivna. Ostane nam še bijektivnost preslikave $b_f$:
%
\begin{itemize}
\item $b_f$ je injektivna: naj bosta $\xi, \zeta \in A/(\sim_f)$ in denimo, da velja $b_f(\xi) = b_f(\zeta)$.
  Obstajata $x, y \in A$, da je $\xi = [x]_\sim$ in $\zeta = [y]_\sim$. Velja
  %
  \begin{equation*}
    f(x) = i_f(b_f(q_f(x))) = i_f(b_f(\xi)) = i_f(b_f(\zeta)) = i_f(b_f(q_f(y))) = f(y),
  \end{equation*}
  %
  torej je $x \sim_f y$ in zato $\xi = [x]_\sim = [y]_\sim = \zeta$.

  \item $b_f$ je surjektivna: naj bo $u \in \img{f}(A)$. Tedaj obstaja $x \in A$, da je $u = f(x)$.
  Vzemimo $\xi = [x]_E$ in preverimo: $b_f(\xi) = b_f([x]_\sim) =f(x) = u$.
\end{itemize}