\documentclass[11pt,a4paper]{book}

\usepackage[T1]{fontenc}
\usepackage[utf8]{inputenc}
\usepackage[slovene]{babel}
\usepackage[colorlinks]{hyperref}
\usepackage{lmodern}
\usepackage{xcolor}
\usepackage{amsmath}
\usepackage{amsfonts}
\usepackage{bbold}
\usepackage{fancyhdr}
\usepackage{theorem}
\usepackage{mathpartir}
\usepackage{proof}
\usepackage{xypic}

%--------------------------------------------------------------------
%-- Barve hiper povezav

\hypersetup{
    colorlinks,
    linkcolor={red!50!black},
    citecolor={blue!50!black},
    urlcolor={blue!80!black}
}

%--------------------------------------------------------------------
%-- Okolja

{
  \theorembodyfont{\itshape}

  \newtheorem{izrek}{Izrek}[chapter]
  \newtheorem{lema}[izrek]{Lema}
  \newtheorem{izjava}[izrek]{Izjava}
  \newtheorem{posledica}[izrek]{Posledica}
  \newtheorem{hipoteza}[izrek]{Hipoteza}
  \newtheorem{aksiom}[izrek]{Aksiom}
}

{
  \theorembodyfont{\rmfamily}
  \newtheorem{definicija}[izrek]{Definicija}
  \newtheorem{primer}[izrek]{Primer}
  \newtheorem{opomba}[izrek]{Opomba}
  \newtheorem{naloga}[izrek]{Naloga}
}

\newcommand{\qedsign}{{\vrule width 1ex height 1ex depth 0ex}}
\newcommand{\qed}{\hfill\qedsign}

\newenvironment{dokaz}{
  \goodbreak\par
  \textit{Dokaz.}%
}{%
  \nopagebreak
  \qed
  \medskip
  \goodbreak
}

%--------------------------------------------------------------------
%%%%%%%%%%%%%%%%%%%%%%%%%%%%%%%%%%%%%%%%%%%%%%%%%%%%%%%%%%%%%%%%%%%%%%%%%%%%%%%%%%%%%%%%%%%%%%%%%%%%%%%%%%%%%%%%%%%%%%
%%%  Commands
%%%%%%%%%%%%%%%%%%%%%%%%%%%%%%%%%%%%%%%%%%%%%%%%%%%%%%%%%%%%%%%%%%%%%%%%%%%%%%%%%%%%%%%%%%%%%%%%%%%%%%%%%%%%%%%%%%%%%%


%%%%%%%%%%%%%%%%%%%%%%%%%%%%%%%%%%%%%%%%%%%%%%%%%%%%%%%%%%%%%
%%%  Theorems etc.
%%%%%%%%%%%%%%%%%%%%%%%%%%%%%%%%%%%%%%%%%%%%%%%%%%%%%%%%%%%%%
{
\theoremstyle{theorem}
\newtheorem{izrek}{Izrek}[chapter]
\newtheorem{lema}[izrek]{Lema}
\newtheorem{trditev}[izrek]{Trditev}
\newtheorem{posledica}[izrek]{Posledica}
\newtheorem{pravilo}[izrek]{Pravilo}
}

{
\theoremstyle{definition}
\newtheorem{definicija}[izrek]{Definicija}
\newtheorem{opomba}[izrek]{Opomba}
\newtheorem{primer}[izrek]{Primer}
\newtheorem{zgled}[izrek]{Zgled}
\newtheorem{naloga}[izrek]{Naloga}
}


%%%%%%  Proofs
%%%%%%%%%%%%%%%%%%%%%%%%%%%%%%%%%%%%%%%%%%%%%%%%%%%%%%%%%%%%%
% Za dokaze uporabimo amsmath proof, sicer ne deluje \qedhere.


%%%%%%  Auxiliary
%%%%%%%%%%%%%%%%%%%%%%%%%%%%%%%%%%%%%%%%%%%%%%%%%%%%%%%%%%%%%
\newcommand{\sizedescriptor}[2]
{
\ifthenelse{\equal{#1}{0}}{}{
\ifthenelse{\equal{#1}{1}}{\big}{
\ifthenelse{\equal{#1}{2}}{\Big}{
\ifthenelse{\equal{#1}{3}}{\bigg}{
\ifthenelse{\equal{#1}{4}}{\Bigg}{
#2}}}}}
}

\newcommand{\someref}{{\small\textcolor{blue}{[\textbf{ref.}]}}}
\newcommand{\intermission}{\bigskip\medskip}
\newcommand{\ltc}[1]{$\backslash$\texttt{#1}}  % LaTeX command
\newcommand{\nls}[1]{``\textit{#1}''}  % sentence in a natural language

%%%%%%  Logical Quantifiers, λ- and ι-Expressions
%%%%%%%%%%%%%%%%%%%%%%%%%%%%%%%%%%%%%%%%%%%%%%%%%%%%%%%%%%%%%

\newcommand{\all}[1]{\forall #1 .\,}
\newcommand{\some}[1]{\exists #1 .\,}
\newcommand{\exactlyone}[1]{\exists{!} #1 .\,}
\newcommand{\lam}[1]{\lambda #1 .\,}
\newcommand{\that}[1]{\iota #1 .\,}

%%%%%%  Logic
%%%%%%%%%%%%%%%%%%%%%%%%%%%%%%%%%%%%%%%%%%%%%%%%%%%%%%%%%%%%%
\newcommand{\tvs}{\Omega}  % set of truth values
\newcommand{\true}{\top}  % truth
\newcommand{\false}{\bot}  % falsehood
\newcommand{\etrue}{\boldsymbol{\top}}  % emphasized truth
\newcommand{\efalse}{\boldsymbol{\bot}}  % emphasized falsehood
\newcommand{\impl}{\Rightarrow}  % implication sign
\newcommand{\revimpl}{\Leftarrow}  % reverse implication sign
\newcommand{\lequ}{\Leftrightarrow}  % equivalence sign
\newcommand{\xor}{\mathbin{\veebar}}  % exclusive disjunction sign
\newcommand{\shf}{\mathbin{\uparrow}}  % Sheffer connective
\newcommand{\luk}{\mathbin{\downarrow}}  % Łukasiewicz connective


%%%%%%  Sets
%%%%%%%%%%%%%%%%%%%%%%%%%%%%%%%%%%%%%%%%%%%%%%%%%%%%%%%%%%%%%
%  \set{1, 2, 3}         ->  {1, 2, 3}
%  \set{a \in X}{a < 1}  ->  {a ∈ X | a < 1}
\NewDocumentCommand{\set}
{O{auto} m G{\empty}}
{\sizedescriptor{#1}{\left}\{ {#2} \ifthenelse{\equal{#3}{}}{}{ \; \sizedescriptor{#1}{\middle}| \; {#3}} \sizedescriptor{#1}{\right}\}}
%\newcommand{\vsubset}{\Mapstochar\cap}
%\newcommand{\finseq}[1]{{#1}^*}
\newcommand{\pst}{\mathcal{P}}
\renewcommand{\complement}[1]{{#1}^C}


%%%%%%  Number Sets, Intervals
%%%%%%%%%%%%%%%%%%%%%%%%%%%%%%%%%%%%%%%%%%%%%%%%%%%%%%%%%%%%%
\newcommand{\NN}{\mathbb{N}}
\newcommand{\ZZ}{\mathbb{Z}}
\newcommand{\QQ}{\mathbb{Q}}
\newcommand{\RR}{\mathbb{R}}
\newcommand{\CC}{\mathbb{C}}
\newcommand{\HH}{\mathbb{H}}
\newcommand{\OO}{\mathbb{O}}
\newcommand{\intoo}[3][\RR]{{#1}_{(#2, #3)}}
\newcommand{\intcc}[3][\RR]{{#1}_{[#2, #3]}}
\newcommand{\intoc}[3][\RR]{{#1}_{(#2, #3]}}
\newcommand{\intco}[3][\RR]{{#1}_{[#2, #3)}}


%%%%%%  Maps and Relations
%%%%%%%%%%%%%%%%%%%%%%%%%%%%%%%%%%%%%%%%%%%%%%%%%%%%%%%%%%%%%
\newcommand{\id}[1][]{\mathrm{id}_{#1}}  % identity map
\newcommand{\argbox}{{\;\!\fbox{\phantom{M}}\;\!}}  % box for a function argument
\newcommand{\konst}[1]{\mathrm{k}_{#1}} % constant map
\newcommand{\rstr}[1]{\left.{#1}\right|}  % map restriction
\newcommand{\im}{\mathrm{im}}  % map image
\newcommand{\parto}{\mathrel{\rightharpoonup}}  % partial mapping sign
\NewDocumentCommand{\rel}
{O{\empty} O{\empty}}
{\ifthenelse{\equal{#1}{}}{\mathscr{R}}{{#1} \mathrel{\mathscr{R}} {#2}}}  % a relation
\NewDocumentCommand{\srel}
{O{\empty} O{\empty}}
{\ifthenelse{\equal{#1}{}}{\mathscr{S}}{{#1} \mathrel{\mathscr{S}} {#2}}}  % a second relation
\newcommand{\dom}{\mathrm{dom}}  % domain
\newcommand{\cod}{\mathrm{cod}}  % codomain
\newcommand{\dd}[1]{D_{#1}}  % domain of definition
\newcommand{\rn}[1]{Z_{#1}}  % range
\newcommand{\graph}[1]{\Gamma_{#1}}  % graph of a (partial) function
\NewDocumentCommand{\img}  % image
{O{\empty} m G{\empty}}
{{#2}_*\ifthenelse{\equal{#3}{}}{}{\!\sizedescriptor{#1}{\left}( {#3} \sizedescriptor{#1}{\right})}}
\NewDocumentCommand{\pim}  % preimage
{O{\empty} m G{\empty}}
{{#2}^*\ifthenelse{\equal{#3}{}}{}{\!\sizedescriptor{#1}{\left}( {#3} \sizedescriptor{#1}{\right})}}
\newcommand{\ec}[2][]{[\:\!{#2}\:\!]_{#1}}  % equivalence class
\newcommand{\transposed}[1]{\widehat{#1}}


%%%%%%  Projections and Injections
%%%%%%%%%%%%%%%%%%%%%%%%%%%%%%%%%%%%%%%%%%%%%%%%%%%%%%%%%%%%%
\NewDocumentCommand{\fst}
{O{\empty} O{\empty}}
{\pi_1^{{#1}\ifthenelse{\equal{#2}{}}{}{,}{#2}}}
\NewDocumentCommand{\snd}
{O{\empty} O{\empty}}
{\pi_2^{{#1}\ifthenelse{\equal{#2}{}}{}{,}{#2}}}
\NewDocumentCommand{\inl}
{O{\empty} O{\empty}}
{\iota_1^{{#1}\ifthenelse{\equal{#2}{}}{}{,}{#2}}}
\NewDocumentCommand{\inr}
{O{\empty} O{\empty}}
{\iota_2^{{#1}\ifthenelse{\equal{#2}{}}{}{,}{#2}}}


%%%%%%  Categories
%%%%%%%%%%%%%%%%%%%%%%%%%%%%%%%%%%%%%%%%%%%%%%%%%%%%%%%%%%%%%
\newcommand{\ct}[1]{\mathbf{#1}}
\newcommand{\mnoz}{\ct{Mno\check{z}}}
\newcommand{\pkol}{\ct{PKol}}  % category of semirings
\newcommand{\upkol}{\pkol_1}  % category of unital semirings
\newcommand{\kol}{\ct{Kol}}  % category of rings
\newcommand{\ukol}{\kol_1}  % category of unital rings


%%%%%%  Exercises and Solutions
%%%%%%%%%%%%%%%%%%%%%%%%%%%%%%%%%%%%%%%%%%%%%%%%%%%%%%%%%%%%%
\Newassociation{resitev}{Resitev}{resitve}
\renewcommand{\Resitevlabel}[1]{\emph{Re\v{s}itev~#1}}
{
\theoremstyle{definition}
\newtheorem{vaja}{Vaja}[chapter]
}


%%%%%%  Misc.
%%%%%%%%%%%%%%%%%%%%%%%%%%%%%%%%%%%%%%%%%%%%%%%%%%%%%%%%%%%%%
\renewcommand{\divides}{\,|\,}
% Načeloma bi morala biti navpična črta v \divides obdana z \mathrel, ampak to vodi do prevelikih presledkov.
\newcommand{\df}[1]{\emph{\textbf{#1}}}  % defined notion
\newcommand{\oper}{\mathop{\circledast}\nolimits}  % symbol for a generic operation
\newcommand{\soper}{\mathop{\boxasterisk}\nolimits}  % symbol for a second generic operation
\newcommand{\qo}[1]{\;\!\widetilde{#1}\;\!}  % quotient operation
\newcommand{\tconc}{\mathop{\bullet}\nolimits}  % symbol for binary tree concatenation
\newcommand{\conc}{\mathop{::}\nolimits}  % symbol for string concatenation
\newcommand{\ism}{\cong}  % isomorphic
\newcommand{\inv}[1]{#1^{-1}} % inverz preslikave
\newcommand{\equ}{\sim}  % equivalent
\newcommand{\dfeq}{\mathrel{\mathop:}=}  % definitional equality
\newcommand{\revdfeq}{=\mathrel{\mathop:}}  % reverse definitional equality
\newcommand{\isdefined}[1]{{#1}\!\downarrow}  % given value is defined
\newcommand{\kleq}{\simeq}  % Kleene equality
\newcommand{\claim}[3]{{#1} \;\colon\; \frac{#2}{#3}}  % claim, divided on context, assumptions, conclusions
\newcommand{\one}{\mathtt{\mathbf{1}}}  % generic singleton
\newcommand{\unit}{\mathord{()}}  % element in a generic singleton
\newcommand{\nul}{\mathtt{N}}  % null map
\newcommand{\suc}{\mathtt{S}}  % successor
\newcommand{\prd}{\mathtt{P}}  % predecessor
\newcommand{\tprd}{\tilde{\prd}}  % predecessor as a total function
\newcommand{\monus}{\mathbin{\vphantom{+}\text{\mathsurround=0pt \ooalign{\noalign{\kern-.35ex}\hidewidth$\smash{\cdot}$\hidewidth\cr\noalign{\kern.35ex}$-$\cr}}}}
% Definicija za monus pobrana s TeX Stack Exchange
\newcommand{\wf}{\prec}  % well-founded order
\NewDocumentEnvironment{implproof}  % proof of an implication
{O{\empty} G{\empty} O{=>} G{\empty}}
{
\begin{description}
\item[\quad$\sizedescriptor{#1}{\left}({#2}
\ifthenelse{\equal{#3}{=>}}{\impl}{
\ifthenelse{\equal{#3}{<=}}{\revimpl}{
\ifthenelse{\equal{#3}{->}}{\rightarrow}{
\ifthenelse{\equal{#3}{<-}}{\leftarrow}{
#3
}}}} {#4}\sizedescriptor{#1}{\right})$]\ \vspace{0.3em}\\
}
{
\end{description}
}
\NewDocumentCommand{\pres}  % presentation of an algebraic structure with generators and relations
{O{auto} m G{\empty}}
{\sizedescriptor{#1}{\left}\langle {#2} \ifthenelse{\equal{#3}{}}{}{ \; \sizedescriptor{#1}{\middle}| \; {#3}} \sizedescriptor{#1}{\right}\rangle}


%%%%%%%%%%%%%%%%%%%%%%%%%%%%%%%%%%%%%%%%%%%%%%%%%%%%%%%%%%%%%%%%%%%%%%%%%%%%%%%%%%%%%%%%%%%%%%%%%%%%%%%%%%%%%%%%%%%%%%

%%% Local Variables:
%%% mode: latex
%%% TeX-master: "ucbenik-lmn"
%%% End:


%--------------------------------------------------------------------
%-- Velikost strani

%% A4 stran = 210mm x 297mm
%% sirino besedila nastavimo na 170mm, visino na 247mm

\setlength{\textwidth}{15cm}
\setlength{\textheight}{224mm}

\setlength{\topmargin}{0cm}
\setlength{\evensidemargin}{0cm}
\setlength{\oddsidemargin}{\paperwidth}
\addtolength{\oddsidemargin}{-\textwidth}
\addtolength{\oddsidemargin}{-2in}

%--------------------------------------------------------------------
%-- Glava in dno

\pagestyle{fancyplain}

%\setlength{\headrulewidth}{0.2pt}
%\addtolength{\headheight}{2pt}

\renewcommand{\chaptermark}[1]{\markboth{#1}{}}
\renewcommand{\sectionmark}[1]{\markright{\thesection\ #1}}

\lhead[\fancyplain{}{{\thepage}}]%
      {\fancyplain{}{{\rightmark}}}
\rhead[\fancyplain{}{{\leftmark}}]%
      {\fancyplain{}{\thepage}}
\cfoot{\footnotesize [verzija \today]}
\lfoot[]{}
\rfoot[]{}

%--------------------------------------------------------------------
% NASLOV

\author{Andrej Bauer}
\title{Logika in množice \\ ZAPISKI V NASTAJANJU}

\begin{document}

\maketitle

\cleardoublepage

%--------------------------------------------------------------------
% KAZALO
\pagestyle{fancyplain}

{
\renewcommand{\markboth}[2]{}
\tableofcontents
}

\cleardoublepage

%--------------------------------------------------------------------
% VSEBINA

\chapter{Predgovor}
\label{chap:predgovor}

Glavni namen predmet Logika in množice v prvem letniku študija matematike je študente
naučiti osnov matematičnega izražanja: kako beremo in pišemo matematično besedilo, kako
uporabljamo simbolni zapis, kako zapišemo in preberemo dokaz itd. Drugi poglavitni namen
predmeta je spoznavanje osnov matematične logike in teorije množic.

Za semesterski predmet z dvema urama predavanj in dvema urama vaj ima predmet zelo
ambiciozen program. Najučinkovitejši recept za uspeh je tisti, ki ga študenti ne marajo:
učite se sproti, sprašujte predavatelja in asistente, trkajte na vrata njihovih pisarn
tudi takrat, ko nimajo govorilnih ur.

Ti zapiski s predavanj nastajajo sproti. Prvotno sem jih zapisoval v formatu Markdown, a napočil je čas, da jih prenesem v {\LaTeX} in nato izboljšujem. Opozarjam vas, da zapiski vsebujejo napake, ker so le grob zapis vsebine predavanj. Odkrivanje napak je sestavni del učnega procesa, čeprav si ne želim, da bi bi bilo napak toliko, da bi motile učenje. Zelo vam bom hvaležen, če mi boste odkrite napake sporočili, da jih popravim. Asistentom pri predmetu se zahvaljujem za skrbno odpravljanje napak. Vse ki so ostale, so moja last.

\bigskip

\begin{flushright}
Andrej Bauer \qquad\hbox{}
\end{flushright}

\bigskip

\paragraph{Zahvala.}
%
Pri urejanju zapiskov pomagali:
%
Matej Marinko,
Lev Rus,
Jakob Schrader,
Matija Sirk in
Marjetka Zupan.
%
Vsem se najlepše zahvaljujem.


%%% Local Variables: 
%%% mode: latex
%%% TeX-master: "lmn"
%%% End: 

\chapter{Osnovni podatki o predmetu}

\paragraph{Gradivo:}
%
Osnovni podatki o predmetu in gradivo je na \href{https://ucilnica.fmf.uni-lj.si/}{spletni učilnici}, kjer najdete:
\begin{itemize}
\item povezavo do video posnetkov predavanj in zapiskov s table,
\item naloge z vaj, ki so objavljenje v naprej,
\item prejšnje kolokvije in izpite,
\item povezo na Discord server za predmet.
\end{itemize}

\paragraph{Izpitni režim.}
%
Predmet opravite z izpitom, ki ima dva dela:
%
\begin{enumerate}
\item \textbf{pisni izpit}
\item \textbf{ustni izpit}
\end{enumerate}
%
Namesto pisnega izpita lahko opravite dva kolokvija (s povprečno oceno obeh kolokvijev skupaj vsaj 50\%). Na ustni izpit pridete šele, ko ste opravili pisni izpit. Če ustnega izpita ne opravite, vam pisni izpit propade in ga
morate ponovno opravljati.

\chapter{Množice in preslikave}

Pri predmetu Logika in množice se bomo učili, kako matematiki komuniciramo in razmišljamo. Spoznali bomo osnove logike
in teorije množic, tako iz povsem praktičnega vidika kot tudi matematičnega. Pri tem predmetu cenimo ne le matematično
razmišljanje, ampak tudi razmišljanje o matematiki.

Za uvod povejmo nekaj osnovnega o množicah in spoznajmo nekatere osnovne konstrukcije.

\section{Osnovno o množicah}

\subsection{Množice kot skupki elementov, relacija $\in$}

Naivno bi rekli, da je množica kakršnakoli zbirka ali skupek matematičnih objektov. Le-ti so lahko števila, funkcije,
množice, množice števil ipd., skratka karkoli.
%
Najbolj preprosti primeri množic so končne množice, katerih elemente naštejemo. Zapišemo jih na primer takole:
%
\begin{gather*}
  \{1, 2, 3\}
  \{\sin, \cos, \tan\}
  \{\{1\}, \{2\}, \{3\}\}
\end{gather*}
%
Objektom, ki tvorijo množico, pravimo \textbf{elementi}. Na primer, elementi množice $\{1, \{4\}, 7/3\}$ so število $1$, množica $\{4\}$, in število $7/4$.

Kadar je $a$ element množice $M$, to zapišemo $a \in M$ in beremo ">$a$ je element $M$"<.

Ali sta množici $\{1, 4, 10\}$ in $\{4, 10, 1, 10\}$ enaki? Da, saj množice obravnavamo kot \emph{neurejene} skupke, v katerih ni pomembno, kolikokrat se pojavi kak element. Da vrstni red in število pojavitev nista pomembna, sledi iz \textbf{aksioma
ekstenzionalnosti}. Aksiom je matematična izjava, ki jo vzamemo za osnovno, se pravi, da je ne dokazujemo. Aksiomi opredeljujejo matematično teorijo, ki jo želimo študirati. Tako bom pri tem predmetu spoznali aksiome teorije množic, pri algebri aksiome za vektorski prostor in grupo itd.

\begin{aksiom}[Ekstenzionalnost množic]
  Množici sta enaki, če imata iste elemente.
\end{aksiom}

Povedano drugače: če je vsak element množice $A$ tudi element množice $B$ in je vsak element množice $B$ tudi element množice $A$, potem velja $A = B$.

Z uporabo ekstenzionalnosti, lahko \emph{dokažemo}, da sta $\{1, 4, 10\}$ in $\{4, 10, 1, 10\}$ enaki:
%
\begin{enumerate}
\item 
  Vsak element $\{1, 4, 10\}$ je tudi element $\{4, 10, 1, 10\}$:
  \begin{enumerate}
    \item velja $1 \in \{4, 10, 1, 10\}$
    \item velja $4 \in \{4, 10, 1, 10\}$
    \item velja $10 \in \{4, 10, 1, 10\}$
  \end{enumerate}
\item
Vsak element $\{4, 10, 1, 10\}$ je tudi element $\{1, 4, 10\}$:
  \begin{enumerate}
     \item velja $4 \in \{1, 4, 10\}$
     \item velja $10 \in \{1, 4, 10\}$
     \item velja $1 \in \{1, 4, 10\}$
     \item velja $10 \in \{1, 4, 10\}$
  \end{enumerate}
\end{enumerate}

Iz zgornjih dveh preverjanj z uporabo ekstenzionalnosti sledi, da $\{1, 4, 10\} = \{4, 10, 1, 10\}$.

\begin{naloga}
  Zapišite podroben dokaz, da sta množici $\{x, y\}$ in $\{y, x\}$ enaki.
\end{naloga}

\begin{opomba}
  Poznamo tudi skupke, pri katerih je pomembno, kolikokrat se pojavi vsak element. Imenujejo se \textbf{multimnožice}.
\end{opomba}

Opozorimo takoj, da v praksi pogosto uporabljamo zapise, ki niso povsem natančni. Takrat se zanašamo, da bodo ostali pravilni uganili, kaj imamo v mislih. Na primer, katere elemente vsebuje množica
%
\begin{equation*}
    \{1, 2, 3, ..., 2021\} \ ?
\end{equation*}
%
Verjetno bi vsi ">uganili"<, da so mišljena vsa naravna števila med $1$ in $2021$, ali ne? Zavedati se je treba, da zgornji zapis tega ne določa! Morda smo imeli v mislih vsa števila med $1$ in $2021$, ki pri deljenju s~$5$ ne dajo ostanka~$4$.

Pri tem predmetu bomo pogosto opozarjali na razne nejasnosti in nenatančne zapise, ki jih uporabljajo matematiki v praksi.
Ni mišljeno, da bi se pretvarjali, da je kaj narobe s ">človeško matematiko"<. Želimo se predvsem zavedati, kje se nejasnosti v praksi pojavljajo in kako bi jih lahko odpravili (tudi če jih v praksi dejansko ne odpravimo). Ko bo torej asistent pri analizi na tablo napisal
%
\begin{equation*}
    1, 2, 4, 8, \ldots
\end{equation*}
%a
imate tri možnosti:
%
\begin{enumerate}
  \item Ste zmedeni.
  \item Uganete, da ima v mislih potence števila 2.
  \item Vprašate, ali je $n$-ti člen število regij, na katerega lahko razdelimo prostor z $(n-1)$ ravninami?
\end{enumerate}
%
Sami se odločite, kakšen odnos želite vzpostaviti z asistentom.

\subsection{Prazna množica $\emptyset$}

Verjetno ni treba izgubljati besed o prazni množici. To je množica, ki nima nobenega elementa. Zapišemo jo $\emptyset$ ali $\{\}$.

\begin{naloga}
  Ali je kakšna razlika med $\{\}$ in $\{\emptyset\}$?
\end{naloga}


\subsection{Standardni enojec $\one$}

Množici, ki ima natanko en element, pravimo \textbf{enojec}.

Ali znamo pojasniti, kaj pomeni, da ima množica natanko en element, ne da bi pri tem omenili število $1$ ali katerokoli drugo število? Takole: množica $A$ ima natanko en element če velja:
%
\begin{enumerate}
\item obstaja $x \in A$ in
\item če je $x \in A$ in $y \in A$, potem $x = y$.
\end{enumerate}

\begin{naloga}
  Kako bi opredelili ">množica ima natanko dva elementa"< brez uporabe števil?
\end{naloga}

Pogosto bomo potrebovali kak enojec (že na naslednjih predavanjih). Seveda se ni težko domisliti enojca, na primer $\{42\}$ ali $\{\sin\}$. Da pa ne bomo vedno znova izgubljali časa z izbiro enojca, se dogovorimo da je \textbf{standardni enojec $\one$} množica $\{\unit\}$. To je zelo čudno, ker smo označili množico s številko\footnote{Ali ločite med ">števka"<, ">številka"< in ">število<"?} $1$ in ker je element standardnega enojca $\unit$, česar še nikoli nismo videli.

Glede oznake $\one$ povejmo, da imamo kot matematiki \emph{načelno svobodo} pri izbiri zapisa, a je smiselno in vljudno, da se ne zafrkavamo. Ali se torej predavatelj zafrkava, ko standardni enojec označi s številko $\one$? Ne, saj gresta ">ena"< in ">enojec"< lepo skupaj, poleg tega pa bomo na naslednjih predavanjih spoznali tudi matematične razloge za tak zapis.

Glede oznake $\unit$ se bo kmalu izkazalo, da je zapis smiseln, ker je $()$ pravzaprav ">urejena ničterica"<.

\subsection{Številske in ostale množice}

Seveda si bomo privoščili uporabo raznih množic, ki jih že poznate, kot so na primer številske množice $\NN$, $\ZZ$, $\QQ$, $\RR$ itd. Opozorimo pa na naslednji dilemo:
v osnovni in srednji šoli z $\NN$ označimo množico celih števil, ki so večja ali enaka $1$, vendar pa pogosto v matematiki, še posebej pa v logiki, tudi število $0$ obravnavamo kot naravno število. V takih primerih $\NN$ izenačuje množico celih števil, ki so večja ali enaka $0$.

Kaj je torej prav $\NN = \{0, 1, 2, \ldots\}$ ali $\NN = \{1, 2, 3, \ldots\}$? To je napačno vprašanje! Lahko vprašamo le ">kako se bomo dogovorili?"<. Pri tem predmetu se
dogovorimo, da je $0$ naravno število, ker vadimo ">matematično svobodo"<, imamo dobre matematične razloge, da $0$ uvrstimo med naravna števila, in ker je predavatelj tako zapovedal.

\begin{naloga}
  Zberite pogum in predavatelja vprašate, kakšni so ti dobri matematični razlogi, zaradi katerih je zapovedal, da je $0$ naravno število, bo sledila filozofska razprava, ki vam bo pokvarila odmor.
\end{naloga}

\section{Konstrukcije množic}

Ena od osnovnih matematičnih aktivnosti so \textbf{konstrukcije}. Poznamo na primer geometrijske konstrukcije z ravnilom in šestilom. Ko računamo rešite enačbe, bi lahko rekli, da konstruiramo število, ki zadošča enačbi. Ko pišemo dokaz, konstruiramo objekt, iz katerega je razvidna resničnost neke izjave. Tudi računalniški programi so le matematični konstrukti.

Spoznajmo nekatere osnovne konstrukcije množic, se pravi, načine, kako iz množic naredimo nove množice.

\subsection{Zmnožek ali kartezični produkt}

\textbf{Urejeni par} $\pair{x,y}$ je matematični objekt, ki da dobimo tako, da združimo dva matematična objekta~$x$ in~$y$. V srednji šoli ste večinoma pisali urejene pare števil (ki ste jih imenovali ">koordinate"<). V urejenem paru je vrstni red \emph{pomemben}: urejena para $(1, 3)$ in $(3, 1)$ \emph{nista} enaka. (Množici $\{1, 3\}$ in $\{3, 1\}$ sta enaki.)

Urejeni par $\pair{x, y}$ ima \textbf{prvo komponento~$x$} in \textbf{drugo komponento~$y$}. Če imamo neki urejeni par~$u$, njegovi komponenti pišemo tudi $\fst(u)$ in $\snd(u)$. Velja torej:
%
\begin{equation*}
    \fst(x,y) = x
    \iinn
    \snd(x,y) = y.  
\end{equation*}
%
Simboloma $\mathsf{pr}_1$ in $\mathsf{pr}_2$ pravimo \textbf{prva} in \textbf{druga projekcija}. Običajne oznake za projekciji so tudi $\pi_1$ in $\pi_2$, v
programiranju $\mathtt{fst}$ in $\mathtt{snd}$, lahko pa tudi $\pi_0$ in $\pi_1$.

Spoznajmo sedaj \textbf{zmnožek} ali \textbf{kartezični produkt} množic $A$ in $B$. Opis nove konstrukcijo množic mora navesti zapis za konstruirano množico, katere elemente ima, in kdaj sta elementa konstruirane množice enaka:
%
\begin{enumerate}
\item Zmnožek množic $A$ in $B$ zapišemo $A \times B$.
\item Elementi množice $A \times B$ so urejeni pari $\pair{x, y}$, pri čemer je $x \in A$ in $y \in B$.
\item Enakost elementov (princip ekstenzionalnosti za pare): $u \in A \times B$ in $v \in A \times B$ sta enaka, če velja $\fst(u) = \fst(v)$ in $\snd(u) = \snd(v)$.
\end{enumerate}

\begin{primer}
Zmnožek množic $\{1,2,3\}$ in $\{\Box, \diamond\}$ je
%
\begin{equation*}
    \{1, 2, 3\} \times {\Box, \diamond} =
    \{\pair{1, \Box},
      \pair{2, \Box},
      \pair{3, \Box},
      \pair{1, \diamond},
      \pair{2, \diamond},
      \pair{3, \diamond}
     \}.
\end{equation*}
%
Iz principa ekstenzionalnosti za pare sledi, da je vrstni red v urejenem paru pomemben, saj $\pair{1, 3} \neq \pair{3, 1}$, ker $\fst(1,3) = 1 \neq 3 = \fst(3,1)$.
\end{primer}


\subsubsection{Zmnožek več množic}

Tvorimo lahko tudi zmnožek več množic. Na primer, zmnožek množic $A$, $B$ in $C$ je množica $A \times B \times C$, katerih elementi so \textbf{urejene trojke} $\pair{x, y, z}$, kjer je $x \in A$, $y \in B$ in $z \in C$. V tem primeru imamo tri projekcije $\mathsf{pr}_1$, $\mathsf{pr}_2$ in $\mathsf{pr}_3$. Podobno lahko tvorimo zmnožek štirih, petih, šestih, \dots množic.

\begin{naloga}
  Ali lahko tvorimo zmnožek ene množice? Kaj pa zmnožek nič množic?
\end{naloga}


\subsection{Vsota ali koprodukt}

Naslednja osnovna konstrukcija je \textbf{vsota} ali \textbf{koprodukt} množic $A$ in $B$:
%
\begin{enumerate}
\item vsoto množic $A$ in $B$ označimo z $A + B$,
\item elementi množice $A + B$ so $\inl{x}$ za $x \in A$ in $\inr{y}$ za $y in B$,
\item elementa $u \in A + B$ in $v \in A + B$ sta enaka, kadar velja
  %
  \begin{enumerate}
    \item bodisi za neki $a \in A$ velja $u = \inl{a} = v$,
    \item bodisi za neki $b \in B$ velja $u = \inr{b} = v$.
  \end{enumerate}
\end{enumerate}

\begin{primer}
Primeri vsote množic:
%
\begin{enumerate}

\item $\{1, 2, 3\} + \{\square, \diamond\} = \{\inl{1}, \inl{2}, \inl{3}, \inr{\Box}, \inr{\diamond}\}$

\item $\{a, b\} + \{b, c\} = \{\inl{a}, \inl{b}, \inr{b}, \inr{c}\}$

\item Vsota \emph{ni} unija! Po eni strani je
      $\{3, 5\} \cup \{3, 5\} = \{3, 5\}$ in po drugi
      $\{3, 5\} + \{3, 5\} = \{\inl{3}, \inl{5}, \inr{3}, \inr{5}\}$.
\end{enumerate}
\end{primer}

Vsoti pravimo tudi ">disjunktna unija"<, a se bomo temu izrazu izogibali, ker obravnavamo vsoto kot osnovno operacijo in ne kot poseben primer unije.

Oznakama $\mathsf{in}_1$ in $\mathsf{in}_2$ pravimo \textbf{prva in druga injekcija}. Uporabljajo se tudi oznake $\iota_1$ in $\iota_2$, v funkcijskem
programiranju $\mathtt{inl}$ in $\mathtt{inr}$, pa tudi $\iota_0$ in $\iota_1$. Pravzaprav ni pomembno, kakšne oznake uporabimo, poskrbeti moralo
le, da sta to različna simbola, s katerima razločimo elemente prvega in drugega sumanda.

Tvorimo lahko vsoto več množic, na primer $A + B + C$. V tem primeru imamo tri injekcije $\mathsf{in}_1$, $\mathsf{in}_2$ in $\mathsf{in}_3$.

\section{Preslikave ali funkcije}

Poleg množic so preslikave še en osnovni matematični pojem, ki mu bomo posvetili veliko pozornosti. \textbf{Preslikava} ali \textbf{funkcija} sestoji iz treh sestavin:
%
\begin{itemize}
\item množice, ki ji pravimo \textbf{domena},
\item množice, ki ji pravimo \textbf{kodomena},
\item \textbf{prirejanja}, ki vsakemu elementu domene priredi natanko en element kodomene.
\end{itemize}
%
Če je $f$ funkcija z domeno $A$ in kodomeno $B$, to zapišemo
%
\begin{equation*}
  f : A \to B
\end{equation*}
%
ali
%
\begin{equation*}
  \xymatrix{
    {A} \ar[rr]^{f} & & {B}
  }
\end{equation*}
%
Rišemo lahko tudi diagrame, ki prikazujejo več funkcij hkrati, na primer
%
\begin{equation*}
  \xymatrix{
    {A} \ar[r]^{f} &
    {B} \ar[r]^{g} &
    {C} \ar[d]^{h} \\
    & & {D}
  }
\end{equation*}
%
Ta diagram prikazuje tri preslikave: $f : A \to B$, $g : B \to C$ in $h : C \to D$.

V srednji šoli ste spoznavali posamične zvrsti funkcij, na primer linearne funkcije, trigonometrične funkcije, eksponentno funkcijo itd. Le-te so običajno slikale števila v števila, bile so \emph{številske funkcije}. Mi se bomo ukvarjali s preslikavami na splošno, se pravi s poljubnimi preslikavami med poljubnimi množicami.

\subsubsection{Princip ekstenzionalnosti preslikav}

\textbf{Princip ekstenzionalnosti za preslikave}, pove, kdaj sta dve funkciji enaki, namreč takrat, ko prirejata enake vrednosti:  če za preslikavi $f : A \to B$ in $g : C \to D$ velja $A = C$, $B = D$ in $f(x) = g(x)$ za vse $x \in A$, potem velja $f = g$.

Kasneje bomo videli, da princip ekstenzionalnosti za preslikave sledi iz principa ekstenzionalnosti za množice.

\subsection{Prirejanje in funkcijski predpisi}

Dejstvo, da mora prirejanje vsakemu elementu domene prirediti ">natanko en"< element kodomene, lahko izrazimo tako, da se izognemo uporabi števila ">ena"< ali kateregakoli števila. Poglejmo kako.

Prirejanje z domeno $A$ in kodomeno $B$ mora biti:
%
\begin{enumerate}
\item \textbf{celovito:} vsakemu $x \in A$ je prirejen vsaj en $y \in B$ (priredimo vsaj en element),
\item \textbf{enolično:} če sta $x \in A$ prirejena $y \in B$ in $z \in B$, potem velja $y = z$ (priredimo največ en element).
\end{enumerate}
%
Res, celovitost zagotavlja, da vsakemu elementu domene priredimo \emph{vsaj en} element kodomene, enoličnost pa zagotavlja, da priredimo \emph{kvečjemu enega}.

\begin{opomba}
  Pozor, celovitost \emph{ni} surjektivnost in enoličnost \emph{ni} injektivnost!
\end{opomba}

Kako pravzaprav podamo prirejanje? Kaj to pravzaprav je? Čez kak mesec bomo znali odgovoriti na to vprašanje natančno, zaenkrat pa le povejmo, da je prirejanje kakršnakoli metoda, tabela, postopek, prikaz, ali konstrukcija, ki zagotavlja celovitost in enoličnost prirejanja elementov kodomene elementom domene.

Običajni način za podajanje prirejanja je \textbf{funkcijski predpis}, ki ga pišemo
%
\begin{equation*}
  x \mapsto \cdots
\end{equation*}
%
Pri čemer za $\cdots$ na desni postavimo neki smiseln izraz, ki določa enolično vrednost za vsak $x$ iz domene. Spremenljivki~$x$ na levi pravimo \textbf{parameter}, izrazu $\cdots$ na desni pa \textbf{prirejena vrednost}.

\begin{primer}
  Primeri prirejanj:
  %
  \begin{itemize}
  \item prirejanje ">prištej 7 in kvadriraj"< zapišemo s funkcijskim predpisom  $x \mapsto (x + 7)^2$,
  \item prirejanje ">kvadriraj in prištej 7"< zapišemo s funkcijskim predpisom  $x \mapsto x^2 + 7$,
  \item prirejanje ">prištej kvadrat 7"< zapišemo s funkcijskim predpisom  $x \mapsto x + 7^2$.
  \end{itemize}
\end{primer}

\begin{opomba}
  Pozor: če podamo \emph{samo} funkcijski predpis brez domene in kodomene, še nismo podali preslikave! Preslikava sestoji iz \emph{treh} delov: domena, kodomena in prirejanje.
  Torej zgornji trije primeri \emph{ne} podajajo preslikav, ker nismo podali domen in
  kodomen.
\end{opomba}

Domeno, kodomeno in funkcijski predpis lahko zapišemo na različne načine:
%
\begin{align*}
  f &: \ZZ \to \NN \\
  f &: x \mapsto x^2 + 7
\end{align*}
%
ali
%
\begin{align*}
    f &: \ZZ \to \NN \\
    f(x) &\defeq x^2 + 7
\end{align*}
%
ali
%
\begin{align*}
    f &: \ZZ \to \NN \\
    f &= (x \mapsto x^2 + 7)
\end{align*}
%
Simbol $x$ je \textbf{vezana spremenljivka}, če jo preimenujemo, se predpis ne spremeni. Naslednji funkcijski predpisi so \emph{enaki}:
%
\begin{align*}
  x &\mapsto x^2 + 7 \\
  y &\mapsto y^2 + 7 \\
  \textit{banana} &\mapsto \textit{banana}^2 + 7
\end{align*}
%
Funkcijski predpis $x \mapsto 5 + x \cdot x + 2$ pa \emph{ni enak} zgornjim trem, čeprav vrača enake vrednosti in torej določa \emph{enako} funkcijo.

\subsubsection{Aplikacija ali uporaba}

Preslikavo $f : A \to B$ \textbf{uporabimo} ali \textbf{apliciramo} na elementu $a \in A$, da dobimo \textbf{vrednost} $f(a) \in B$. V izrazu $f(a)$ se imenuje $a$ \textbf{argument}
%
Kadar je $f$ podana s predpisom, izračunamo vrednost $f(a)$ tako, da $a$ vstavimo v predpis (vezano spremenljivo zamenjamo z argumentom~$A$).
%
O zamenjavi vezane spremenljivke z argumentom bomo več povedali v razdelku~\ref{sec:substitucija} o substituciji.

\begin{primer}
  Če je $f : \NN \to \NN$ podana s predpisom $f = (x \mapsto x^3 + 4)$, tedaj je $f(5)$ enako $5^3 + 4$. Lahko bi celo pisali
  %
  \begin{equation*}
    (x \mapsto x^3 + 4)(5) = 5^3 + 4.
  \end{equation*}
\end{primer}


\subsection{Eksponentna množica}

Tretja konstrukcija množic, ki jo bomo spoznali v uvodnem poglavju, je \textbf{eksponent} ali \textbf{eksponentna množica}:
%
\begin{enumerate}
\item eksponent množic $A$ in $B$ označimo $B^A$, in preberemo ">$B$ na $A$"<,
\item elementi $B^A$ so preslikave z domeno $A$ in kodomeno $B$,
\item preslikavi $f : A \to B$ in $g : A \to B$ sta enaki, če imate enake vrednosti: za
  vse $x \in A$ velja $f(x) = g(x)$, potem je $f = g$.
\end{enumerate}
%
Eksponent $B^A$ pišemo tudi $A \to B$. To pomeni, da bi lahko namesto $f : A \to B$ pisali tudi $f \in B^A$ ali celo $f \in A \to B$, vendar je ta zadnji zapis neobičajen.

\begin{primer}
Eksponent $\{1, 2\}^{\{a, b\}}$ ima štiri elemente:
%
\begin{equation*}
  \{1, 2\}^{\{a, b\}} =
  \{
     (a \mapsto 1, b \mapsto 1),
     (a \mapsto 1, b \mapsto 2),
     (a \mapsto 2, b \mapsto 1),
     (a \mapsto 2, b \mapsto 2)
  \}.
\end{equation*}
\end{primer}



\chapter{Aritmetika množic}

Nadaljujmo s študijem splošnih preslikav.

\section{Preslikave in prazna množica}

Naj bo $A$ množica. Kaj vemo povedati o preslikavah $\emptyset \to A$?

Čez nekaj tednov bomo spoznali naslednji dejstvi, ki ju zaenkrat vzemimo v zakup:

\begin{itemize}
\item Vsaka izjava oblike ">za vsak element $\emptyset$ ..."< je resnična.
\item Vsaka izjava oblike ">obstaja element $\emptyset$ ..."< je neresnična.
\end{itemize}

Primeri resničnih izjav:
%
\begin{enumerate}
\item ">Vsak element prazne množice je sodo število"<
\item ">Vsak element prazne množice je liho število"<
\item ">Vsak element prazne množice je hkrati sodo in liho število"<
\item ">Vsak element prazne množice \dots"<
\end{enumerate}

Primeri neresničnih izjav:
%
\begin{enumerate}
\item ">Obstaja element prazne množice, ki je sodo število"<
\item ">Obstaja element prazne množice, ki je enak sam sebi"<
\item ">Obstaja element prazne množice, ki \dots"<
\end{enumerate}
%
Denimo, da imamo preslikave $f :\emptyset \to A$ in $g : \emptyset \to A$. Tedaj sta enaki, saj velja: ">za vsak element $x \in \emptyset$ velja $f(x) = g(x)$".
Torej imamo kvečjemu eno preslikavo $\emptyset \to A$. Ali pa imamo sploh kakšno? Da, pravimo ji \textbf{prazna preslikava}, ker je njeno prirejanje ">prazno"<, oziroma ga sploh ni treba podati (saj ni nobenega elementa domene $\emptyset$, ki bi mu morali prirediti kak element kodomene $A$).

Kaj pa preslikave $A \to \emptyset$?
%
Če je $A = \emptyset$, potem imamo natanko eno preslikavo $A \to \emptyset$, namreč prazno preslikavo, $\emptyset^A = \{ \textrm{prazna-preslikava} \}$.
%
Če $A$ vsebuje kak element, potem ni nobene preslikave $A \to \emptyset$, se pravi  $\emptyset^A = \emptyset$.

Zakaj ni preslikave $A \to \emptyset$, kadar $A$ vsebuje kak element? Denimo da je $x \in A$. Če bi bila kaka preslikava $f : A \to \emptyset$, bi
veljalo $f(x) \in \emptyset$, kar pa ni res. Torej take preslikave ni.

\begin{naloga}
  Koliko je preslikav $1 \to A$ in koliko je preslikav $A \to 1$?
  Ali je odgovor odvisen od~$A$?
\end{naloga}

\section{Identiteta in kompozicija}

Spoznajmo nekaj osnovnih preslikav in operacij na preslikavah.

\textbf{Identiteta} na $A$ je preslikava $id[A] : A \to A$, podana s predpisom $x \mapsto x$.

\textbf{Kompozitum} preslikav
%
\begin{equation*}
  \xymatrix{
    {A} \ar[r]^f & {B} \ar[r]^g & {C}
  }
\end{equation*}
%
je preslikava $g \circ f : A \to C$, podana s predpisom $x \mapsto g(f(x))$.

\textbf{Kompozitum je asociativen:} za preslikave
%
\begin{equation*}
  \xymatrix{
    {A} \ar[r]^f & {B} \ar[r]^g & {C} \ar[r]^h & {D}
  }
\end{equation*}
%
velja $(h \circ g) \circ f = h \circ (g \circ f)$. Res, za vsak $x \in A$ velja
%
\begin{align*}
  ((h \circ g) \circ f)(x)
  &= (h \circ g) (f x) \\
  &=  h (g (f (x)) \\
  &= h ((g \circ f)(x)) \\
  &= (h \circ (g \circ f))(x),
\end{align*}
%
torej želena\footnote{Piše se ">želen"< in ne ">željen"<, ker je ">želen"< deležnik na
  ">n"< glagola ">želeti"<. V slovenščini ni glagola ">željeti"<. Hitro boste spoznali, da na FMF profesorji za matematiko radi popravljajo slovnico.} enačba sledi iz principa ekstenzionalnosti za funkcije.

\textbf{Identiteta je nevtralni element za kompozitum:} za vsako preslikavo $f : A \to B$ velja
%
\begin{equation*}
  \id[B] \circ f = f
  \iinn
  f \circ id[A] = f.
\end{equation*}
%
To preverimo z uporabo ekstenzionalnosti za funkcije: za vsak $x \in A$ velja
%
\begin{equation*}
    (\id[B] \circ f)(x) = \id[B] (f(x)) = f(x)
\end{equation*}
%
in
\begin{equation*}
  (f \circ id[A])(x) = f (id[A](x)) = f(x).
\end{equation*}
%
Kompizicija $\circ$ in identiteta $id$ se torej obnašata podobno kot nekatere operacije v algebri, na primer $+$ in $0$ ter $×$ in $1$.

\begin{naloga}
  Seštevanje je komutativno, velja $a + b = b + a$. Ali je kompozicija preslikav tudi komutativna?
\end{naloga}

\section{Funkcijski predpisi na zmnožku in vsoti}

Pogosto želimo definirati preslikavo, katere kodomena je zmnožek množic, denimo $f : A \times B \to C$. V takem primeru lahko
podamo funkcijski predpis takole:
%
\begin{equation*}
  (x, y) \mapsto \cdots
\end{equation*}
%
pri čemer je $x \in A$ in $y \in B$. To je dovoljeno, ker je vsak element domene $A \times B$ urejeni par $(x, y)$ za natanko določena $x \in A$ in $y \in B$.

\begin{primer}

Preslikavo
%
\begin{align*}
  \RR \times \RR  &\to  \RR \\
  u &\mapsto  \fst{u}² + 3 \cdot \snd{u}
\end{align*}
%
lahko bolj čitljivo podamo s predpisom
%
\begin{align*}
  \RR \times \RR  &\to  \RR \\
  (x, y) &\mapsto  x^2 + 3 \cdot y
\end{align*}
\end{primer}

\begin{primer}
  Seveda lahko podobno podajamo tudi preslikave na zmnožkih večih preslikav, denimo
  %
  \begin{align*}
  A \times B \times C &\to A \times A \\
  (a, b, c) &\mapsto (a, a)
  \end{align*}
  in
  \begin{align*}
  X \times (Y \times Z) &\to (X \times Y) \times Z \\
  (x, (y, z)) &\mapsto ((x, y), z).
  \end{align*}
\end{primer}

Kako pa zapišemo funkcijski predpis funkcije z domeno $A + B$? V tem primeru je vsak element domene bodisi oblike
$\inl{x}$ za enolično določeni $x \in A$, bodisi oblike $\inr{y}$ za enolično določeni $y \in B$, zato funkcijski predpis podamo v dveh vrsticah:
%
\begin{align*}
    A + B &\to C \\
    \inl{x} &\mapsto \cdots \\
    \inr{y} &\mapsto \cdots
\end{align*}

\begin{primer}
  Primer take preslikave je
  %
  \begin{align*}
    \RR + \ZZ &\to \RR \\
    \inl{x} &\mapsto x \\
    \inr{y} &\mapsto y + 3
  \end{align*}
  %
  Seveda lahko podobno podajamo tudi preslikave na vsotah večih preslikav:
  %
  \begin{align*}
    A + B + C &\to \{u, v\} \\
    \inl{x} &\mapsto u \\
    \inr{y} &\mapsto u \\
    \mathsf{in}_3(z) &\mapsto v
  \end{align*}
  in
  \begin{align*}
    A + (B + C) &\to \{u, v\} \\
    \inl{x} &\mapsto u \\
    \inr{\inl{y}} &\mapsto u \\
    \inr{\inr{y}} &\mapsto v
  \end{align*}
\end{primer}

Zapisa za zmnožek in vsoto lahko tudi kombiniramo:
%
\begin{align*}
  (A \times B \times C) + (D \times E) &\to \{0, 1, 2\} \\
  \inl{(a, b, c)} &\mapsto 1 \\
  \inr{(d, e)} &\mapsto 2
\end{align*}
%
in
\begin{align*}
  (A + B) \times C &\to \{0, 1, 2\} \\
  (\inl{a}, c) &\mapsto 0 \\
  (\inr{b}, c) &\mapsto 1
\end{align*}
%
Izraz na levi strani $\mapsto$ sestoji iz vezanih spremelnjivk in operacij, s katerimi gradimo elemente množic (urejeni par, kanonična injekcija). Imenuje se tudi \textbf{vzorec}. Predpis je podan pravilno, če so vzorci napisani tako, da vsak element domene ustreza natanko enemu vzorcu.
%
S tem zagotovimo, da predpis obravnava vse možne primere (celovitost) in da ne obravnava nobenega primera večkrat (enoličnost).


\subsection{Funkcijski predpis, podan po kosih}

Omenimo še en pogost način podajanja funkcij, namreč s predpisom po kosih.

\begin{primer}
  Preslikava ">absolutno"< je definirana po kosih za negativna in nenegativna števila:
  %
  \begin{align*}
    \RR &\to \RR \\
    x &\mapsto
    \begin{cases}
      -x & \text{če $x < 0$,}\\
       x & \text{če $x \geq 0$.}
    \end{cases}
  \end{align*}
\end{primer}

\begin{primer}
  Preslikava ">predznak"< je definirana po kosih:
  %
  \begin{align*}
    \RR &\to \RR \\
    x &\mapsto
      \begin{cases}
        -1 & \text{če $x < 0$,}\\
        0 & \text{če $x = 0$,}\\
        1 & \text{če $x \geq 0$.}
      \end{cases}
  \end{align*}
\end{primer}


Pri takem zapisu moramo paziti, da kosi skupaj pokrivajo domeno (vsi elementi domene so obravnavani) in da se kosi ne prekrivajo (vsak element domene je obravnavan natanko enkrat). Pravzaprav se smejo kosi prekrivati, a moramo v tem primeru preveriti, da se na skupnih delih skladajo, se pravi, da vsi kosi podajajo enake vrednosti na preseku.

\begin{primer}
  Preslikavo ">absolutno"< bi lahko podali takole:
  %
  \begin{align*}
    \RR &\to \RR \\
    x &\mapsto
    \begin{cases}
      -x & \text{če $x < 0$,}\\
       x & \text{če $x \geq 0$.}
    \end{cases}
  \end{align*}
  %
  Kosa se prekrivata pri $x = 0$, vendar to ni težava, ker je $-0 = 0$.
\end{primer}


\section{Nekatere preslikave na eksponentnih množicah}

Poglejmo si nekaj preslikav, ki slikajo iz in v eksponente množice.

\textbf{Evalvacija} ali \textbf{aplikacija} ali \textbf{uporaba} je preslikava, ki sprejme preslikavo in argument, ter preslikavo uporabi na argumentu:
%
\begin{align*}
  \mathsf{ev} &: B^A \times A \to B \\
  \mathsf{ev} &: (f, x) \mapsto f(x)
\end{align*}
%
Pravimo, da je $\mathsf{ev}$ \textbf{preslikava višjega reda}, ker slika preslikave v vrednosti.

\begin{primer}
  Določeni integral $\int_0^1$ je funkcija višjega reda, ker
  sprejme funkcijo $[0,1] \to \RR$ in vrne realno število. Je torej preslikava
  $\RR^{[0,1]} \to \RR$, če se pretvarjamo, da lahko integriramo vsako funkcijo.
  Bolj pravilno bi bilo reči, da je $\int_0^1$ preslikava iz množice \emph{integrabilnih funkcij $[0,1] \to \RR$} v realna števila.
\end{primer}

Kompozitum preslikav je tudi preslikava višjega reda:
%
\begin{align*}
    {\circ} &: C^B \times B^A \to C^A \\
    {\circ} &: (g, f) \mapsto (x \mapsto g(f(x)))
\end{align*}
%
Tretja splošna preslikava višjega reda je ">currying"< (ali zna kdo to prevesi v slovenščino?):
%
\begin{align*}
  A^{(B \times C)} &\to (A^B)^C \\
  f &\mapsto (c \mapsto (b \mapsto f(b, c))).
\end{align*}
%
Pravzaprav je to izomorfizem, katerega inverz je ">uncurrying"<:
%
\begin{align*}
  (A^B)^C &\to A^{B \times C} \\
  g       &\mapsto ((b, c) \mapsto f(b)(c))
\end{align*}


\section{Izomorfizmi in artimetika množic}

\subsection{Inverz}

\begin{definicija}
  Preslikava $f : A \to B$ je \textbf{inverz} preslikave $g : B \to A$, če velja $f \circ g = \id[B]$ in $g \circ f = \id[A]$.
\end{definicija}

\begin{naloga}
  Utemelji: če je $f$ inverz $g$, potem je $g$ inverz $f$.
\end{naloga}

\begin{primer}
  Kub in kubični koren sta inverza
  %
  \begin{align*}
    \RR &\to \RR    &     \RR &\to \RR \\
    x &\mapsto x^3  &     y &\mapsto \sqrt[3]{y}
  \end{align*}
\end{primer}

\begin{naloga}
  Naj bo~$S$ množica nenegativnih realnih števil, se pravi, $\RR_{\geq 0} = \{x \in \RR \mid x \geq 0\}$. Ali sta kvadriranje in kvadratni koren inverza?
  %
  \begin{align*}
    \RR &\to \RR_{\geq 0}    &     \RR_{\geq 0} &\to \RR \\
    x &\mapsto x^2           &     y &\mapsto \sqrt[2]{y}
  \end{align*}
  %
\end{naloga}

\begin{izjava}
  Če sta $f : A \to B$ in $g : A \to B$ oba inverza preslikave $h : B \to A$, potem je $f = g$.
\end{izjava}

\begin{dokaz}
  Denimo, da sta $f : A \to B$ in $g : A \to B$ inverza preslikave $h : B \to A$. Tedaj velja
  %
  \begin{equation*}
    f =
    f \circ \id[A] =
    f \circ (h \circ g) =
    (f \circ h) \circ g =
    \id[B] \circ g =
    g.
  \end{equation*}
\end{dokaz}

Ali znate utemeljiti vsakega od zgornjih korakov?

\begin{definicija}
  Preslikava, ki ima inverz, se imenuje \textbf{izomorfizem}.
\end{definicija}

Če je $f : A \to B$ izomorfizem, potem ima natanko en inverz $B \to A$, ki ga označimo $\inv{f}$.

\begin{primer}
  Identiteta $id[A] : A \to A$ je izomorfizem, saj je sama sebi inverz.
  Torej $\inv{id[A]} = \id[A]$.
\end{primer}

\begin{primer}
  Eksponentna preslikava $\exp : \RR \to \RR_{> 0}$, $exp : x \mapsto e^x$ je
  izomorfizem, njen inverz je naravni logaritem $\ln : \RR_{> 0} \to \RR$, torej $\inv{\exp} = \ln$.
\end{primer}

\begin{primer}
  Eksponentna preslikava $\exp : \RR \to \RR$ \emph{ni} izomorfizem.
\end{primer}

\begin{izjava}
  Če sta $f : A \to B$ in $g : B \to C$ izomorfizma, potem je tudi $g \circ f : A \to C$ izomorfizem. Velja torej $\inv{(g \circ f)} = \inv{f} \circ \inv{g}$.
\end{izjava}

\begin{dokaz}
  Dokazati moramo, da ima $g \circ f$ inverz. Trdimo, da je $\inv{f} \circ \inv{g} : C \to A$ inverz preslikave $g \circ f$. Računajmo:
  %
  \begin{align*}
    (g \circ f) \circ (\inv{f} \circ \inv{g}) \\
     &= ((g \circ f) \circ \inv{f}) \circ \inv{g} \\
     &= (g \circ (f \circ \inv{f})) \circ \inv{g} \\
     &= (g \circ \id[B]) \circ \inv{g} \\
     &= g \circ \inv{g} \\
     &= \id[C].
  \end{align*}
  %
  Doma sami preverite, da velja tudi $(\inv{f} \circ \inv{g}) \circ (g \circ f) = \id[A]$.
\end{dokaz}

\subsection{Izomorfne množice}


\begin{definicija}
  Množici $A$ in $B$ sta \textbf{izomorfni}, če obstaja izomorfizem $f : A \to B$. Kadar sta $A$ in $B$ izomorfni, to zapišemo $A \iso B$.
\end{definicija}

\begin{izjava}
  \textbf{Izjava:} Za vse množice $A$, $B$ in $C$ velja:
  %
  \begin{enumerate}
    \item $A \iso A$,
    \item če $A \iso B$, potem $B \iso A$,
    \item če $A \iso B$ in $B \iso C$, potem $A \iso C$.
  \end{enumerate}
\end{izjava}

\begin{dokaz}
  %
  \begin{enumerate}
     \item $id[A]$ je izomorfizem $A \to A$,
     \item če je $f : A \to B$ izomorfizem, potem je tudi $\inv{f} : B \to A$ izomorfizem,
     \item če je $f : A \to B$ izomorfizem in $g : B \to C$ izomorfizem, potem je $g \circ f : A \to C$ izomorfizem.
  \end{enumerate}
\end{dokaz}

\begin{primer}
  $A \times B \iso B \times A$, ker imamo izomorfizem in njegov inverz
  %
  \begin{align*}
    A \times B  &\to  B \times A      &    B \times A  &\to  A \times B \\
    (x, y) &\mapsto  (y, x)           &    (b, a) &\mapsto  (a, b)
  \end{align*}
\end{primer}

\subsection{Aritmetika množic}

Veljajo naslednji izomorfizmi, ki nas seveda spomnijo na zakone aritmetike, ki
veljajo za števila. Ali gre tu za kako globjo povezavo?
%
\begin{enumerate}
\item Vsota in $\emptyset$:
  \begin{enumerate}
    \item $A + \emptyset \iso A$
    \item $A + B \iso B + A$
    \item $(A + B) + C \iso A + (B + C)$
  \end{enumerate}

\item Zmnožek in $\one$:
  \begin{enumerate}
    \item $A \times 1 \iso A$
    \item $A \times B \iso B \times A$
    \item $(A \times B) \times C \iso A \times (B \times C)$
  \end{enumerate}

\item Distributivnost:
  \begin{enumerate}
    \item $A \times (B + C) \iso (A \times B) + (A \times C)$
    \item $A \times \emptyset \iso \emptyset$
  \end{enumerate}

\item Eksponenti:
  \begin{enumerate}
    \item $A^1 \iso A$
    \item $1^A \iso 1$
    \item $A^{\emptyset }\iso 1$
    \item $\emptyset^A \iso \emptyset , če  A \neq \emptyset$
    \item $A^{(B \times C)} \iso (A^B)^C$
    \item $A^{(B + C)} \iso A^B \times A^C$
    \item $(A \times B)^C \iso A^C \times B^C$
  \end{enumerate}
\end{enumerate}

\begin{naloga}
  Zapišite vseh 15 izomorfizmov, ki potrjujejo pravilnost zgornjega seznama.
\end{naloga}


\chapter{Logika}\label{poglavje:logika}





\section{Logični simboli}\label{razdelek:logicni-simboli}

Preproste izjave, kot na primer \nls{$n$ je sodo število.}, že znamo zapisati s simboli: $2 \divides n$. Povečini pa delamo z bolj kompleksnimi, sestavljenimi izjavami. Tudi za te obstaja simbolni zapis; na primer, izjavo \nls{Če je $n$ sodo število, je tudi kvadrat števila $n$ sod.}, zapišemo kot $2 \divides n \implies 2 \divides n^2$. Seveda ta izjava velja za vsa naravna števila (znaš to dokazati?). To zapišemo takole: $\all{n \in \NN} (2 \divides n \implies 2 \divides n^2)$. V tem razdelku si bomo ogledali, kako povezati preprostejše izjave v bolj sestavljene in kako to v splošnem simbolno zapisati.

Kot smo navajeni iz naravnih jezikov, posamične stavke povežemo v sestavljeno poved z \emph{vezniki}. Najpogosteje uporabljeni matematični vezniki so v tabeli~\ref{tabela:standardni-izjavni-vezniki}.

\begin{table}[!ht]
\centering
\begin{tabular}{|ccc|}
\hline
\textbf{Izjavni veznik} & \textbf{Oznaka} & \textbf{Kako preberemo} \\
\hline
negacija & $\lnot{p}$ & ne $p$ \\
konjunkcija & $p \land q$ & $p$ in $q$ \\
disjunkcija & $p \lor q$ & $p$ ali $q$ \\
implikacija & $p \impl q$ & če $p$, potem $q$ \\
ekvivalenca & $p \lequ q$ & $p$ natanko tedaj, ko $q$ \\
\hline
\end{tabular}
\caption{Standardni izjavni vezniki}\label{tabela:standardni-izjavni-vezniki}
\end{table}

\begin{opomba}
V matematiki se za izjavne veznike običajno uporabljajo zgoraj navedene tujke, ampak vsaka od njih seveda ima svoj pomen. Dobesedni prevodi teh tujk so:
\begin{itemize}
\item
negacija $\to$ zanikanje,
\item
konjunkcija $\to$ vezava,
\item
disjunkcija $\to$ ločitev,
\item
implikacija $\to$ vpletenost,
\item
ekvivalenca $\to$ enakovrednost.
\end{itemize}
Za primerjavo: spomnite se vezalnega in ločnega priredja iz slovenščine!
\end{opomba}

\begin{zgled}
Naj $p$ označuje stavek \nls{Zunaj dežuje.} in $q$ stavek \nls{Vzamem dežnik.}. Tedaj $\lnot{p}$ pomeni \nls{Zunaj ne dežuje.} in $p \impl q$ pomeni \nls{Če zunaj dežuje, potem vzamem dežnik.}.
\end{zgled}

Kose sestavljene izjave lahko veže več kot en veznik. V tem primeru se (tako kot pri računanju s števili) dogovorimo o prednosti veznikov. Po dogovoru je vrstni red veznikov tak, kot v tabeli~\ref{tabela:standardni-izjavni-vezniki}, tj.~najmočneje veže negacija, nato konjunkcija, nato disjunkcija, nato implikacija, nato ekvivalenca. Kadar želimo, da se najprej izvede veznik z nižjo prednostjo, uporabimo oklepaje.

\begin{zgled}
Označimo sledeče stavke:
\begin{quote}
$p$ \ \ldots\ldots\ \nls{Imam čas.} \\
$q$ \ \ldots\ldots\ \nls{Ostanem doma.}
\end{quote}
Tedaj $\lnot{p} \land q$ pomeni isto kot $(\lnot{p}) \land q$, to je \nls{Nimam časa in ostanem doma.}, medtem ko $\lnot(p \land q)$ pomeni \nls{Ni res, da imam čas in ostanem doma.}.
\end{zgled}
\davorin{Če komu pade na pamet primer boljših stavkov, je zaželjeno, da popravi\ldots}

Poleg zgoraj navedenih izjavnih veznikov se včasih uporabljajo še sledeči (tabela~\ref{tabela:nadaljnji-izjavni-vezniki}).

\begin{table}[!ht]
\centering
\begin{tabular}{|ccc|}
\hline
\textbf{Izjavni veznik} & \textbf{Oznaka} & \textbf{Kako preberemo} \\
\hline
stroga disjunkcija & $p \xor q$ & bodisi $p$ bodisi $q$ \\
Shefferjev\tablefootnote{Henry Maurice Sheffer (1882 -- 1964) je bil ameriški logik.} veznik & $p \shf q$ & ne hkrati $p$ in $q$ \\
Łukasiewiczev\tablefootnote{Jan Łukasiewicz (beri: \hill{u}ukaśj\^{e}vič) (1878 -- 1956) je bil poljski logik in filozof.} veznik & $p \luk q$ & niti $p$ niti $q$ \\
\hline
\end{tabular}
\caption{Nekateri nadaljnji izjavni vezniki}\label{tabela:nadaljnji-izjavni-vezniki}
\end{table}

Za strogo disjunkcijo (tudi: ekskluzivna disjunkcija, izključitvena disjunkcija) se uporabljajo še druge oznake: $p \oplus q$, $p + q$. Razlika med navadno in strogo disjunkcijo je sledeča: $p \lor q$ pomeni, da \emph{vsaj eden} od $p$ in $q$ velja, medtem ko $p \xor q$ pomeni, da velja \emph{natanko eden}.

\begin{zgled}
Stavek \nls{Pisni del predmeta je potrebno opraviti s kolokviji ali pisnim izpitom.} je primer navadne disjunkcije (seveda se vam prizna pisni del predmeta tudi, če uspešno odpišete tako kolokvije kot pisni izpit), stavek \nls{Grem bodisi na morje bodisi v hribe.} pa je primer stroge disjunkcije (ne da se biti na dveh mestih hkrati).
\end{zgled}

Pogosto veznike iz tabele~\ref{tabela:nadaljnji-izjavni-vezniki} (in vse preostale, ki jih nismo navedli) kar izrazimo s standardnimi na sledeči način.
\begin{center}
\begin{tabular}{|ccc|}
\hline
\textbf{Izjavni veznik} & \multicolumn{2}{c|}{\textbf{Nekatere izražave s standardnimi vezniki}} \\
\hline
$p \xor q$ & $(p \lor q) \land \lnot(p \land q)$ & $(p \land \lnot{q}) \lor (\lnot{p} \land q)$ \\
$p \shf q$ & $\lnot(p \land q)$ & $\lnot{p} \lor \lnot{q}$ \\
$p \luk q$ & $\lnot(p \lor q)$ & $\lnot{p} \land \lnot{q}$ \\
\hline
\end{tabular}
\end{center}

Včasih pa vendarle raje delamo neposredno z dodatnimi vezniki. Služijo lahko kot koristna okrajšava, so pa še drugi razlogi. Na primer, stroga disjunkcija igra vlogo seštevanja v Boolovem kolobarju (glej~\note{razdelek o Boolovih kolobarjih}), Shefferjev in Łukasiewiczev veznik pa se uporabljata pri preklopnih vezjih, saj je z vsakim od njiju možno izraziti vse izjavne veznike (glej vajo~\ref{naloga:polni-nabori-z-enim-veznikom}). V računalništvu imajo ti trije vezniki standardne oznake XOR, NAND, NOR.

\davorin{Nekje tukaj povejmo, kakšno prednost damo tem trem veznikom v primerjavi s standardnimi. Kateremu dogovoru sledimo?}

Včasih so izjave odvisne od kakšnih parametrov. Na primer, naj $\phi(x)$ pomeni \nls{$x$ je zelen.}; tedaj $\phi(\text{trava})$ pomeni \nls{Trava je zelena.}. Simbol $\phi$ torej predstavlja lastnost določenih objektov. Takšne primere smo imeli že v razdelku~\ref{razdelek:mnozice}, kjer smo navedli oznako za podmnožico tistih elementov, ki zadoščajo dani lastnosti.

Lastnosti, odvisne od spremenljivk, lahko \emph{kvantificiramo} po njihovih spremenljivkah, tj.~povemo, ``kako pogosto'' velja lastnost. Tabela~\ref{tabela:kvantifikatorji} podaja najpogosteje uporabljane kvantifikatorje in njihove oznake.

\begin{table}[!ht]
\centering
\begin{tabular}{|ccc|}
\hline
\textbf{Kvantifikator} & \textbf{Oznaka} & \textbf{Kako preberemo} \\
\hline
univerzalni kvantifikator & $\all{x \in X} \phi(x)$ & za vsak $x$ iz $X$ velja lastnost $\phi$ \\
eksistenčni kvantifikator & $\some{x \in X} \phi(x)$ & obstaja $x$ iz $X$ z lastnostjo $\phi$ \\
\note{enolični eksistenčni kvantifikator?} & $\exactlyone{x \in X} \phi(x)$ & obstaja natanko en $x$ iz $X$ z lastnostjo $\phi$ \\
\hline
\end{tabular}
\caption{Kvantifikatorji}\label{tabela:kvantifikatorji}
\end{table}

Oznaki $\forall$ in $\exists$ sta narobe obrnjena A in E in izhajata iz nemščine (\textbf{a}ll, \textbf{e}xistiert).

Seveda je tudi kvantificirana spremenljivka vezana in jo lahko poljubno preimenujemo. Izjavi $\all{x \in X} \phi(x)$ in $\all{y \in X} \phi(y)$ povesta natanko isto: vsi elementi množice $X$ imajo lastnost $\phi$.

\begin{zgled}
Vemo, da za vsako nenegativno realno število obstaja enolično določen nenegativen kvadratni koren; to izjavo lahko zapišemo na sledeči način.
\[\all{a \in \RR_{\geq 0}} \exactlyone{b \in \RR_{\geq 0}} b^2 = a\]
Zaradi tega lahko definiramo kvadratni koren kot funkcijo $\sqrt{\phantom{I}}\colon \RR_{\geq 0} \to \RR_{\geq 0}$ \note{več o tem kasneje}.
\end{zgled}

Po dogovoru kvantifikatorji vežejo močneje kot izjavni vezniki. Izjavo, da je vsako celo število bodisi sodo bodisi liho, torej zapišemo takole.
\[\all{a \in \ZZ} (2 \divides a \xor 2 \divides (a-1))\]

\begin{zgled}
Za poljubno naravno število $n \in \NN$ naj $P(n)$ označuje izjavo, da je $n$ praštevilo. Torej, $P$ definiramo takole.
\[P(n) \dfeq \all{x \in \NN_{\geq 1}} (x \divides n \implies x = 1 \xor x = n)\]
(Premisli, kaj bi se zgodilo, če bi namesto stroge disjunkcije vzeli navadno. Bi še vedno dobili pravilni pojem praštevila?)

Naj $S(n)$ označuje, da je $n$ sestavljeno število.
\[S(n) \dfeq \some{x, y \in \intoo[\NN]{1}{n}} x \cdot y = n\]
(Kadar imamo več zaporednih kvantifikatorjev iste vrste, jih po dogovoru lahko strnemo kot zgoraj. Dana formula za $S(n)$ je krajši zapis za $\some{x \in \intoo[\NN]{1}{n}} \some{y \in \intoo[\NN]{1}{n}} x \cdot y = n$.)

Zdaj lahko na pregleden način zapišemo, da je vsako naravno število od $2$ naprej bodisi praštevilo bodisi sestavljeno.
\[\all{n \in \NN_{\geq 2}} (P(n) \xor S(n))\]
\end{zgled}

\section{Definicije}

\davorin{Predlagam, da v definicijah konsistentno uporabljamo `kadar' namesto `če' (``Funkcija je zvezna, kadar velja to in to.''). V definicijah gre za ekvivalenco, ne implikacijo.}

\davorin{Verjetno je smiselno v tem razdelku razložiti definicijsko enakost $\dfeq$ (oz.~$\revdfeq$). Če se tako odločimo, odstranimo zgornje uporabe teh simbolov.}


% TU SE KONČA PRESTAVLJENA ROBA


        \note{uvod}


        \section{Izjavni vezniki}

                V razdelku~\ref{razdelek:logicni-simboli} smo omenili nekaj izjavnih veznikov, podali oznake zanje in opisali njihov intuitivni pomen. Ampak če se hočemo zanašati na pravilnost naših sklepov, moramo tem oznakam dati \emph{formalni matematični pomen}.

                Če imamo neko izjavo, lahko določimo njeno resničnost, tj.~povemo, do kolikšne mere je resnična. Temu rečemo \df{resničnostna vrednost} izjave. Množico vseh možnih resničnostnih vrednosti označimo z $\tvs$. Seveda ni kaj dosti možnih resničnostnih vrednosti: to sta \df{resnica} (dogovorimo se, da bomo zanjo uporabljali oznako $\true$) in \df{neresnica} (oznaka $\false$). Se pravi, $\tvs = \set{\true, \false}$.

                \begin{opomba}
                        Logiki, kjer sta edini resničnostni vrednosti resnica in neresnica, rečemo \df{dvovrednostna} oziroma \df{klasična logika}. Obstajajo splošnejše vrste logike, kjer je $\set{\true, \false}$ prava podmnožica $\tvs$, ampak v tej knjigi se bomo omejili na klasično logiko, na katero ste navajeni in ki se uporablja v večjem delu matematike.
                \end{opomba}

                \davorin{Kako izrecno bomo ločevali med izjavami in njihovimi logičnimi vrednostmi?}

                Izjavne veznike lahko potem formalno podamo kot preslikave. Na primer, negacija je preslikava $\lnot\colon \tvs \to \tvs$ (vsaki resničnostni vrednosti pripišemo njeno nasprotno vrednost). Preslikavo, definirano na majhni končni množici, lahko preprosto podamo s tabelo vseh njenih vrednosti. V primeru izjavnih veznikov takim tabelam rečemo \df{resničnostne tabele}. Resničnostna tabela za negacijo je videti takole.
                \begin{center}
                        \begin{tabular}{c|c}
                                $p$ & $\lnot{p}$ \\
                                \hline
                                $\true$ & $\false$ \\
                                $\false$ & $\true$
                        \end{tabular}
                \end{center}
                Ta tabela povsem natančno definira negacijo kot preslikavo $\lnot\colon \tvs \to \tvs$. Seveda smo negacijo definirali tako, kot bi pričakovali: negacija resnice je neresnica, negacija neresnice je resnica.

                Podobno lahko naredimo z ostalimi izjavnimi vezniki, le da preostali vežejo dve izjavi. Se pravi, npr.~konjunkcija vzame dve resničnostni vrednosti in vrne resničnostno vrednost, ki pove, ali sta obe dani vrednosti resnični. Konjunkcijo lahko torej interpretiramo kot preslikavo $\land\colon \tvs \times \tvs \to \tvs$ (ali na kratko $\land\colon \tvs^2 \to \tvs$).

                V splošnem definiramo, da je \df{$n$-mestni izjavni veznik} preslikava oblike $\tvs^n \to \tvs$. Negacija je torej enomestni izjavni veznik, ostali vezniki, ki smo jih do zdaj omenili, pa so dvomestni.

                Definirajmo zdaj konjunkcijo natančno s pomočjo resničnostne tabele. Množica $\tvs \times \tvs$ ima štiri elemente --- vse možne pare, sestavljene iz $\true$ oz.~$\false$. Intuitivni pomen konjunkcije razumemo: konjunkcija dveh izjav je resnična natanko tedaj, ko sta obe izjavi resnični. To nas vodi do naslednje tabele.
                \begin{center}
                        \begin{tabular}{cc|c}
                                $p$ & $q$ & $p \land q$ \\
                                \hline
                                $\true$ & $\true$ & $\true$ \\
                                $\true$ & $\false$ & $\false$ \\
                                $\false$ & $\true$ & $\false$ \\
                                $\false$ & $\false$ & $\false$
                        \end{tabular}
                \end{center}

                Za disjunkcijo smo že rekli, da pride v dveh različicah: navadna pomeni, da vsaj ena od izjav velja, izključitvena pa pomeni, da velja natanko ena od izjav. Posledično je torej smiselno definirati funkciji $\lor, \xor\colon \tvs \times \tvs \to \tvs$ na sledeči način.
                \begin{center}
                        \begin{tabular}{cc|cc}
                                $p$ & $q$ & $p \lor q$ & $p \xor q$ \\
                                \hline
                                $\true$ & $\true$ & $\true$ & $\false$ \\
                                $\true$ & $\false$ & $\true$ & $\true$ \\
                                $\false$ & $\true$ & $\true$ & $\true$ \\
                                $\false$ & $\false$ & $\false$ & $\false$
                        \end{tabular}
                \end{center}
                Bodi pozoren na razliko med zadnjima dvema stolpcema!

                Obenem lahko še na hitro opravimo z veznikoma $\shf$ in $\luk$. Spomnimo se, da $p \shf q$ pomeni ``ne hkrati $p$ in $q$'', medtem ko $p \luk q$ pomeni ``niti $p$ niti $q$''.
                \begin{center}
                        \begin{tabular}{cc|cc}
                                $p$ & $q$ & $p \shf q$ & $p \luk q$ \\
                                \hline
                                $\true$ & $\true$ & $\false$ & $\false$ \\
                                $\true$ & $\false$ & $\true$ & $\false$ \\
                                $\false$ & $\true$ & $\true$ & $\false$ \\
                                $\false$ & $\false$ & $\true$ & $\true$
                        \end{tabular}
                \end{center}

                Implikacija je nekoliko bolj subtilna. Kaj točno trdimo z izjavo $p \impl q$, se pravi, kakor hitro velja $p$, mora veljati tudi $q$? No, če $p$ ne velja, potem sploh nismo postavili nobenega pogoja --- izjava je avtomatično izpolnjena. Če $p$ velja, pa zraven zahtevamo še $q$. Resničnostna tabela za implikacijo je potemtakem sledeča.
                \begin{center}
                        \begin{tabular}{cc|c}
                                $p$ & $q$ & $p \impl q$ \\
                                \hline
                                $\true$ & $\true$ & $\true$ \\
                                $\true$ & $\false$ & $\false$ \\
                                $\false$ & $\true$ & $\true$ \\
                                $\false$ & $\false$ & $\true$
                        \end{tabular}
                \end{center}

                Ekvivalenca je spet preprosta --- izjavi sta ekvivalentni, kadar imata isto resničnostno vrednost. Od tod dobimo sledečo resničnostno tabelo.
                \begin{center}
                        \begin{tabular}{cc|c}
                                $p$ & $q$ & $p \lequ q$ \\
                                \hline
                                $\true$ & $\true$ & $\true$ \\
                                $\true$ & $\false$ & $\false$ \\
                                $\false$ & $\true$ & $\false$ \\
                                $\false$ & $\false$ & $\true$
                        \end{tabular}
                \end{center}

                Za lažjo referenco zberimo resničnostne tabele vseh do zdaj omenjenih veznikov na eno mesto (tabela~\ref{tabela:resnicnostna-tabela-osnovnih-izjavnih-veznikov}).

                \begin{table}[!ht]
                        \centering
                        \begin{tabular}{c|c}
                                $p$ & $\lnot{p}$ \\
                                \hline
                                $\true$ & $\false$ \\
                                $\false$ & $\true$
                        \end{tabular}
                        \qquad\quad
                        \begin{tabular}{cc|ccccccc}
                                $p$ & $q$ & $p \land q$ & $p \lor q$ & $p \xor q$ & $p \shf q$ & $p \luk q$ & $p \impl q$ & $p \lequ q$ \\
                                \hline
                                $\true$ & $\true$ & $\true$ & $\true$ & $\false$ & $\false$ & $\false$ & $\true$ & $\true$ \\
                                $\true$ & $\false$ & $\false$ & $\true$ & $\true$ & $\true$ & $\false$ & $\false$ & $\false$ \\
                                $\false$ & $\true$ & $\false$ & $\true$ & $\true$ & $\true$ & $\false$ & $\true$ & $\false$ \\
                                $\false$ & $\false$ & $\false$ & $\false$ & $\false$ & $\true$ & $\true$ & $\true$ & $\true$
                        \end{tabular}
                        \caption{Resničnostna tabela osnovnih izjavnih veznikov}\label{tabela:resnicnostna-tabela-osnovnih-izjavnih-veznikov}
                \end{table}

                Zdaj ko imamo natančno definicijo izjavnih veznikov, lahko trditve v zvezi z njimi tudi formalno utemeljimo. Na primer, spomnimo se, da smo že malo po omembi veznikov $\xor$, $\shf$, $\luk$ podali njihovo izražavo z vezniki $\lnot$, $\land$, $\lor$. Če na glas preberemo vse izjave, nam je intuitivno jasno, katere se ujemajo in zakaj, ampak zdaj lahko dejansko preverimo, da te izražave veljajo.

                Na primer, kaj pomeni, da se $p \luk q$ lahko izrazi kot $\lnot(p \lor q)$? To pomeni, da sta funkciji $\tvs \times \tvs \to \tvs$, dani s predpisoma $(p, q) \mapsto p \luk q$ in $(p, q) \mapsto \lnot(p \lor q)$, enaki. (Slednja funkcija je sestavljena, tj.~sklop dveh funkcij. Lahko bi tudi zapisali, da velja $\luk = \lnot \circ \lor$.) Funkciji z isto domeno in kodomeno sta enaki, kadar pri vsakem argumentu vrneta isti vrednosti, kar v našem primeru pomeni, da imata enaka stolpca v resničnostni tabeli. Poračunajmo torej vse izraze v danih izražavah. Ko dobimo enake rezultate, bomo vedeli, da izražave dejansko veljajo.

                \begin{center}
                        \begin{tabular}{cc|cccccc}
                                $p$ & $q$ & $p \shf q$ & $p \land q$ & $\lnot(p \land q)$ & $\lnot{p}$ & $\lnot{q}$ & $\lnot{p} \lor \lnot{q}$ \\
                                \hline
                                $\true$ & $\true$ & $\efalse$ & $\true$ & $\efalse$ & $\false$ & $\false$ & $\efalse$ \\
                                $\true$ & $\false$ & $\etrue$ & $\false$ & $\etrue$ & $\false$ & $\true$ & $\etrue$ \\
                                $\false$ & $\true$ & $\etrue$ & $\false$ & $\etrue$ & $\true$ & $\false$ & $\etrue$ \\
                                $\false$ & $\false$ & $\etrue$ & $\false$ & $\etrue$ & $\true$ & $\true$ & $\etrue$
                        \end{tabular}
                \end{center}

                \begin{center}
                        \begin{tabular}{cc|cccccc}
                                $p$ & $q$ & $p \luk q$ & $p \lor q$ & $\lnot(p \lor q)$ & $\lnot{p}$ & $\lnot{q}$ & $\lnot{p} \land \lnot{q}$ \\
                                \hline
                                $\true$ & $\true$ & $\efalse$ & $\true$ & $\efalse$ & $\false$ & $\false$ & $\efalse$ \\
                                $\true$ & $\false$ & $\efalse$ & $\true$ & $\efalse$ & $\false$ & $\true$ & $\efalse$ \\
                                $\false$ & $\true$ & $\efalse$ & $\true$ & $\efalse$ & $\true$ & $\false$ & $\efalse$ \\
                                $\false$ & $\false$ & $\etrue$ & $\false$ & $\etrue$ & $\true$ & $\true$ & $\etrue$
                        \end{tabular}
                \end{center}

                \begin{center}
                        \begin{tabular}{cc|ccccc}
                                $p$ & $q$ & $p \xor q$ & $p \lor q$ & $p \land q$ & $\lnot(p \land q)$ & $(p \lor q) \land \lnot(p \land q)$  \\
                                \hline
                                $\true$ & $\true$ & $\efalse$ & $\true$ & $\true$ & $\false$ & $\efalse$ \\
                                $\true$ & $\false$ & $\etrue$ & $\true$ & $\false$ & $\true$ & $\etrue$ \\
                                $\false$ & $\true$ & $\etrue$ & $\true$ & $\false$ & $\true$ & $\etrue$ \\
                                $\false$ & $\false$ & $\efalse$ & $\false$ & $\false$ & $\true$ & $\efalse$
                        \end{tabular}
                \end{center}

                \begin{center}
                        \begin{tabular}{cc|cccccc}
                                $p$ & $q$ & $p \xor q$ & $\lnot{q}$ & $p \land \lnot{q}$ & $\lnot{p}$ & $\lnot{p} \land q$ & $(p \land \lnot{q}) \lor (\lnot{p} \land q)$  \\
                                \hline
                                $\true$ & $\true$ & $\efalse$ & $\false$ & $\false$ & $\false$ & $\false$ & $\efalse$ \\
                                $\true$ & $\false$ & $\etrue$ & $\true$ & $\true$ & $\false$ & $\false$ & $\etrue$ \\
                                $\false$ & $\true$ & $\etrue$ & $\false$ & $\false$ & $\true$ & $\true$ & $\etrue$ \\
                                $\false$ & $\false$ & $\efalse$ & $\true$ & $\false$ & $\true$ & $\false$ & $\efalse$
                        \end{tabular}
                \end{center}

                Kako simbolno zapisati, da sta dve izražavi enaki? Lahko bi pisali
                \[\big((p, q) \mapsto p \shf q\big) = \big((p, q) \mapsto \lnot(p \land q)\big),\]
                ampak to je nekoliko nerodno in nepregledno. Kasneje (v razdelku~\note{o anonimnih funkcijah}) se bomo naučili $\lambda$-notacijo, s katero dobimo
                \[(\lam{(p, q) \in \tvs^2} p \shf q) = (\lam{(p, q) \in \tvs^2} \lnot(p \land q)),\]
                ampak to je še vedno nepregledno. Uveljavil se je običaj, da se izraze, ki so enakovredni v smislu, da dajo isti rezultat pri vsaki izbiri argumentov, poveže s simbolom $\equiv$, torej zapišemo
                \[p \shf q \equiv \lnot(p \land q).\]
                Konkretno za izraze v logiki se uporablja tudi $\sim$, se pravi, zapišemo lahko tudi
                \[p \shf q \sim \lnot(p \land q).\]
                V tej knjigi se bomo držali uporabe simbola $\equiv$. \davorin{Recimo. Po mojem je to boljše, ker lahko $\equiv$ uporabljamo še za druge funkcije (npr.~$f(x) \equiv 0$ pomeni, da je $f$ konstantno enaka $0$, medtem ko $f(x) = 0$ predstavlja enačbo, s katero iščemo ničle funkcije) in ker bomo kasneje $\sim$ uporabljali za ekvivalenčne relacije.}

                Med drugim smo s temi tabelami izpeljali tako imenovana \df{de Morganova zakona} za izjavno logiko \davorin{Verjetno je smiselno specificirati ``za izjavno logiko''. Imeli bomo namreč še zakona za predikatno logiko (za $\forall$ in $\exists$) ter za množice (za preseke in unije).}, ki povesta, kako negacija vpliva na konjunkcijo in disjunkcijo:
                \[\lnot(p \land q) \equiv \lnot{p} \lor \lnot{q},\]
                \[\lnot(p \lor q) \equiv \lnot{p} \land \lnot{q}.\]
                To je smiselno: kadar ni res, da veljata oba $p$ in $q$, vsaj eden od njiju ne velja. Kadar ni res, da velja vsaj eden od njiju, nobeden od njiju ne velja.

                Z resničnostnimi tabelami lahko preverimo še mnoge druge formule. \df{Zakon dvojne negacije} pravi $\lnot\lnot{p} \equiv p$, tj.~če dvakrat zanikamo izjavo, dobimo izjavo, enakovredno začetni. Poračunajmo tabelo.

                \begin{center}
                        \begin{tabular}{c|ccc}
                                $p$ & $\lnot{p}$ & $\lnot\lnot{p}$ & $p$ \\
                                \hline
                                $\true$ & $\false$ & $\etrue$ & $\etrue$ \\
                                $\false$ & $\true$ & $\efalse$ & $\efalse$
                        \end{tabular}
                \end{center}

                Spomnimo se: za poljubno dvomestno operacijo $\oper$ na neki množici $X$ rečemo, da je
                \begin{itemize}
                        \item
                                \df{izmenljiva} ali \df{komutativna}, kadar velja $a \oper b = b \oper a$ za vse $a, b \in X$ (na kratko: $a \oper b \equiv b \oper a$),
                        \item
                                \df{družilna} ali \df{asociativna}, kadar velja $(a \oper b) \oper c = a \oper (b \oper c)$ za vse $a, b, c \in X$ (na kratko: $(a \oper b) \oper c \equiv a \oper (b \oper c)$),
                        \item
                                \df{idempotentna} \davorin{a imamo slovenski izraz za to?}, kadar velja $a \oper a = a$ za vse $a \in X$ (torej $a \oper a \equiv a$).
                \end{itemize}

                Preverimo z resničnostno tabelo, da je konjunkcija komutativna, torej $p \land q \equiv q \land p$.

                \begin{center}
                        \begin{tabular}{cc|ccccc}
                                $p$ & $q$ & $p \land q$ & $q \land p$ \\
                                \hline
                                $\true$ & $\true$ & $\etrue$ & $\etrue$ \\
                                $\true$ & $\false$ & $\efalse$ & $\efalse$ \\
                                $\false$ & $\true$ & $\efalse$ & $\efalse$ \\
                                $\false$ & $\false$ & $\efalse$ & $\efalse$
                        \end{tabular}
                \end{center}

                Še hitreje lahko preverimo, da je konjunkcija idempotentna.

                \begin{center}
                        \begin{tabular}{c|cc}
                                $p$ & $p \land p$ & $p$ \\
                                \hline
                                $\true$ & $\etrue$ & $\etrue$ \\
                                $\false$ & $\efalse$ & $\efalse$
                        \end{tabular}
                \end{center}

                Kako pa preveriti, da je konjunkcija asociativna, torej $(p \land q) \land r \equiv p \land (q \land r)$? Vidimo, da v teh izrazih nastopajo tri spremenljivke in torej potrebujemo resničnostno tabelo, kjer upoštevamo vseh osem možnosti za izbiro $p$, $q$, $r$.

                \begin{center}
                        \begin{tabular}{ccc|cccc}
                                $p$ & $q$ & $r$ & $p \land q$ & $(p \land q) \land r$ & $q \land r$ & $p \land (q \land r)$ \\
                                \hline
                                $\true$ & $\true$ & $\true$ & $\true$ & $\etrue$ & $\true$ & $\etrue$ \\
                                $\true$ & $\true$ & $\false$ & $\true$ & $\efalse$ & $\false$ & $\efalse$ \\
                                $\true$ & $\false$ & $\true$ & $\false$ & $\efalse$ & $\false$ & $\efalse$ \\
                                $\true$ & $\false$ & $\false$ & $\false$ & $\efalse$ & $\false$ & $\efalse$ \\
                                $\false$ & $\true$ & $\true$ & $\false$ & $\efalse$ & $\true$ & $\efalse$ \\
                                $\false$ & $\true$ & $\false$ & $\false$ & $\efalse$ & $\false$ & $\efalse$ \\
                                $\false$ & $\false$ & $\true$ & $\false$ & $\efalse$ & $\false$ & $\efalse$ \\
                                $\false$ & $\false$ & $\false$ & $\false$ & $\efalse$ & $\false$ & $\efalse$
                        \end{tabular}
                \end{center}

                To pomeni, da lahko v izrazih, kjer nastopa več zaporednih konjunkcij, spuščamo oklepaje: namesto $p \land (\lnot{q} \land r)$ pišemo kar $p \land \lnot{q} \land r$.

                Enako velja tudi za disjunkcijo.

                \begin{naloga}
                        Dokaži, da je disjunkcija komutativna, asociativna in idempotentna!
                \end{naloga}

                Preostali dvomestni vezniki, ki smo jih omenili, ne zadoščajo vsem trem lastnostim naenkrat.

                \begin{naloga}
                        Preveri, kateri znani dvomestni izjavni vezniki so komutativni, asociativni oziroma idempotentni!
                \end{naloga}

                Ko rešite zgornjo vajo, boste med drugim opazili: implikacija ni komutativna. To pomeni, da lahko definiramo nov izjavni veznik $\revimpl$ na naslednji način: $p \revimpl q \dfeq q \impl p$ za vse $p, q \in \tvs$. Z drugimi besedami, $\revimpl$ je dan s sledečo resničnostno tabelo.
                \begin{center}
                        \begin{tabular}{cc|c}
                                $p$ & $q$ & $p \revimpl q$ \\
                                \hline
                                $\true$ & $\true$ & $\true$ \\
                                $\true$ & $\false$ & $\true$ \\
                                $\false$ & $\true$ & $\false$ \\
                                $\false$ & $\false$ & $\true$
                        \end{tabular}
                \end{center}

                \note{dokazi s pomočjo resničnostnih tabel še vseh ostalih formul, ki jih hočemo imeti, med drugim distributivnosti}

                Do zdaj smo omenili zgolj nekaj posamičnih izjavnih veznikov. Koliko pa je vseh skupaj? Spomnimo se, da je $n$-mestni izjavni veznik definiran kot preslikava $\tvs^n \to \tvs$. Množica $\tvs^n$ vsebuje vse urejene $n$-terice elementov $\true$ in $\false$; teh je $2^n$ (za vsako od $n$ mest v $n$-terici imamo dve možnosti in vse te izbire so neodvisne med sabo). Za vsako od teh $2^n$ večteric imamo dve možnosti, kam jo preslikamo: v $\true$ ali v $\false$. Vseh možnosti --- torej vseh $n$-mestnih veznikov --- je potemtakem $2^{2^n}$. (Vseh izjavnih veznikov, ko dopuščamo vse možne $n$, je seveda neskončno.)

                Za boljšo predstavo si oglejmo vse $n$-mestne veznike za majhne $n \in \NN$. Prva možnost je $n = 0$. Formula nam pravi, da je število ničmestnih izjavnih veznikov enako $2^{2^0} = 2^1 = 2$. Kaj pomeni, da pri nič vhodnih podatkih vrnemo $\true$ ali $\false$? To pomeni, da preprosto izberemo resničnostno vrednost --- z drugimi besedami, ničmestni izjavni vezniki so isto kot resničnostne vrednosti.

                Koliko je vseh enomestnih izjavnih veznikov? Formula pravi $2^{2^1} = 2^2 = 4$. Zapišimo vse možnosti.

                \begin{center}
                        \begin{tabular}{c|cccc}
                                $p$ &&&& \\
                                \hline
                                $\true$ & $\true$ & $\false$ & $\true$ & $\false$ \\
                                $\false$ & $\true$ & $\false$ & $\false$ & $\true$
                        \end{tabular}
                \end{center}

                Vidimo: enomestni izjavni vezniki so obe konstantni funkciji na $\tvs$, identiteta na $\tvs$ in negacija.

                Kar se dvomestnih veznikov tiče, vidimo, da jih je $2^{2^2} = 2^4 = 16$.

                \begin{naloga}
                        Preveri, da so vsi dvomestni vezniki natanko: konstanta z vrednostjo $\top$, projekcija na prvo komponento (tj.~$(p, q) \mapsto p$), projekcija na drugo komponento (tj.~$(p, q) \mapsto q$), konjunkcija $\land$, disjunkcija $\lor$, implikacija $\impl$, povratna implikacija $\revimpl$, ekvivalenca $\lequ$ in negacije vseh teh.
                \end{naloga}

                Tromestnih veznikov je že $2^{2^3} = 2^8 = 256$ in ne bomo vseh naštevali. Kako pa bi kakega dobili? Preprost način je, da vzamemo tri spremenljivke in jih združimo z večimi znanimi vezniki, na primer $(p, q, r) \mapsto p \land \lnot{q} \impl r$.\footnote{Načeloma sploh ni nujno, da vse tri spremenljivke dejansko uporabimo. Na primer, $(p, q, r) \mapsto p \land q$ še vedno podaja tromestni veznik, saj gre za preslikavo $\tvs^3 \to \tvs$.}

                Seveda se pojavi vprašanje, ali obstajajo izjavni vezniki, ki jih ne bi mogli sestaviti iz osnovnih. Izkaže se, da je odgovor nikalen: \emph{vsak veznik (ne glede na mestnost) je možno izraziti z osnovnimi}; pravzaprav zadostujejo že $\lnot$, $\land$ in $\lor$.

                Ideja je sledeča. Katerikoli izjavni veznik je oblike $V\colon \tvs^n \to \tvs$ in v celoti podan z resničnostno tabelo. Vzemimo konkreten primer; naj bo $V$ tromestni veznik, podan z naslednjo tabelo.

                \begin{center}
                        \begin{tabular}{ccc|c}
                                $p$ & $q$ & $r$ & $V(p, q, r)$ \\
                                \hline
                                $\true$ & $\true$ & $\true$ & $\false$ \\
                                $\true$ & $\true$ & $\false$ & $\true$ \\
                                $\true$ & $\false$ & $\true$ & $\true$ \\
                                $\true$ & $\false$ & $\false$ & $\false$ \\
                                $\false$ & $\true$ & $\true$ & $\true$ \\
                                $\false$ & $\true$ & $\false$ & $\true$ \\
                                $\false$ & $\false$ & $\true$ & $\false$ \\
                                $\false$ & $\false$ & $\false$ & $\false$
                        \end{tabular}
                \end{center}

                Tedaj lahko rečemo: $V$ je resničen tedaj, ko smo v 2., 3., 5.~ali 6.~vrstici. Kdaj smo v drugi vrstici? Točno tedaj, ko $p$ in $q$ veljata, $r$ pa ne, se pravi, ko velja $p \land q \land \lnot{r}$. Podobno naredimo še za preostale vrstice: tretja je določena s $p \land \lnot{q} \land r$, peta z $\lnot{p} \land q \land r$ in šesta z $\lnot{p} \land q \land \lnot{r}$. Potemtakem lahko zapišemo:
                \[V(p, q, r) \equiv (p \land q \land \lnot{r}) \lor (p \land \lnot{q} \land r) \lor (\lnot{p} \land q \land r) \lor (\lnot{p} \land q \land \lnot{r}).\]
                Temu rečemo \df{disjunktivna normalna oblika} (s kratico DNO) veznika $V$.

                Obstaja še dualna oblika take izražave. Lahko si rečemo tudi, da je $V$ resničen, kadar nismo v 1., 4., 7.~oz.~8.~vrstici. Kdaj nismo v prvi vrstici? Kadar niso vsi $p$, $q$, $r$ resnični, torej ko je vsaj eden od njih neresničen --- s formulo $\lnot{p} \lor \lnot{q} \lor \lnot{r}$. Kdaj nismo v četrti vrstici? Ko ni res, da je $p$ resničen, $q$ in $r$ pa ne, torej ko prekršimo vsaj enega teh pogojev, kar nam da formulo $\lnot{p} \lor q \lor r$. Podobno sklepamo, da nismo v sedmi vrstici, kadar velja $p \lor q \lor \lnot{r}$, in da nismo v osmi vrstici, kadar velja $p \lor q \lor r$. To nam da sledečo izražavo za $V$:
                \[V(p, q, r) \equiv (\lnot{p} \lor \lnot{q} \lor \lnot{r}) \land (\lnot{p} \lor q \lor r) \land (p \lor q \lor \lnot{r}) \land (p \lor q \lor r).\]
                Temu rečemo \df{konjunktivna normalna oblika} (s kratico KNO) veznika $V$.

                Spremenljivkam in njihovim negacijam z eno besedo rečemo \df{literali}. Disjunktivna normalna oblika je torej disjunkcija konjunkcij literalov, konjunktivna normalna oblika pa konjunkcija disjunkcij literalov.

                Iz tega primera je jasno, kako postopamo za poljuben izjavni veznik in zanj zapišemo DNO ali KNO. Opazimo: dolžina posamičnega člena, ki ga omejujejo oklepaji, je vedno enaka (vsebuje toliko literalov, kolikor je mestnost veznika), število teh členov pa razberemo iz stolpca, ki podaja vrednosti veznika v resničnostni tabeli. V primeru DNO je to število enako številu resnic $\true$, v primeru KNO pa številu neresnic $\false$. V zgornjem primeru sta bili DNO in KNO enako dolgi, ker smo imeli štiri $\true$ in $\false$, v splošnem pa se nam morda bolj splača uporabiti eno obliko kot drugo. Na primer, DNO implikacije se glasi $p \impl q \equiv (p \land q) \lor (\lnot{p} \land q) \lor (\lnot{p} \land \lnot{q})$, KNO pa je precej krajša: $p \impl q \equiv \lnot{p} \lor q$.

                Vidimo pa, da tu naletimo na problem: kaj se zgodi, če se katera resničnostna vrednost v stolpcu veznika sploh ne pojavi --- z drugimi besedami, kaj če je funkcija, ki podaja veznik, konstantna? Najprej dajmo takim veznikom ime: izjavni veznik, ki je pri vseh argumentih resničen, se imenuje \df{istorečje} ali \df{tavtologija}, izjavni veznik, ki je vedno neresničen, pa se imenuje \df{protislovje} ali \df{kontradikcija}.

                Za istorečje lahko vedno (ne glede na mestnost) zapišemo DNO (ki je sicer najdaljša možna), medtem ko bi KNO načeloma bila konjunkcija nič členov. Je to smiselno? V bistvu ja: če zahtevamo, da hkrati velja nič pogojev, je naša zahteva vedno izpolnjena. V tem smislu je konjunkcija nič členov enaka $\true$.

                Poglejmo podobne primere iz računstva. Kaj je vsota nič členov? Odgovor je seveda $0$. To je enota za seštevanje, kar je smiselno: če nič členom prištejemo en člen, moramo imeti zgolj ta člen. Podobno sklepamo: zmnožek nič členov je enota za množenje $1$ --- če nič faktorjem dodamo še en faktor, imamo skupaj zgolj ta faktor. Spomni se tudi: $a^0 = 1$ in $0! = 1$. To, da je ničkratna uporabe neke operacije enaka enoti za to operacijo, se izide tudi za konjunkcijo: dejansko velja $p \land \true \equiv p \equiv \true \land p$ (preveri z resničnostno tabelo!).

                Enak razmislek velja za protislovje. Zanj lahko zapišemo KNO na običajen način, medtem ko bi DNO bila disjunkcija nič členov. Smiselno je, da je disjunkcija nič členov enaka $\false$, tako zaradi tega, ker je $\false$ enota za disjunkcijo (preveri!), kot zaradi čisto intuitivnega razmisleka: kdaj je vsaj en člen od nič členov resničen? Nikoli.

                Vseeno je nekoliko nerodno delati s konjunkcijo ali disjunkcijo nič členov --- kako točno bi to zapisali? Da velja $V(p_1, p_2, \ldots, p_n) \equiv $? Če nič ne zapišemo, kako sploh vemo, ali smo mislili na ničkratno konjunkcijo, disjunkcijo ali katerokoli drugo operacijo? Nekateri se zato preprosto dogovorijo, da ne dopuščajo ničkratnih operacij v DNO oz.~KNO in potem štejejo, da istorečja nimajo KNO, protislovja pa ne DNO.

                Tudi če ne dopuščamo ničkratnih operacij, pa še vedno velja: vsak izjavni veznik z mestnostjo vsaj $1$ ima vsaj eno od DNO oz.~KNO in ga torej lahko izrazimo samo z negacijo, konjunkcijo in disjunkcijo. Družini izjavnih veznikov, s katerimi lahko izrazimo vse veznike z mestnostjo vsaj $1$, rečemo \df{poln nabor}. Na kratko lahko torej rečemo, da je $\set{\lnot, \land, \lor}$ poln nabor.

                Jasno, če je neka množica veznikov poln nabor, je tudi vsaka njena nadmnožica poln nabor. Sledi, da je tudi na primer $\set{\lnot, \land, \lor, \impl}$ poln nabor.

                Spomnimo se zdaj de Morganovih zakonov in zakona o dvojni negaciji --- iz njih lahko izpeljemo $p \land q \equiv \lnot(\lnot{p} \lor \lnot{q})$ in $p \lor q \equiv \lnot(\lnot{p} \land \lnot{q})$. Se pravi, konjunkcijo lahko izrazimo z disjunkcijo in negacijo in prav tako lahko disjunkcijo izrazimo s konjunkcijo in negacijo. To pomeni, da sta že $\set{\lnot, \lor}$ in $\set{\lnot, \land}$ polna nabora! Se pravi, vse veznike s pozitivno mestnostjo je možno izraziti že samo z dvema.

                Je možno iti še dlje in najti en sam veznik, s katerim lahko izrazimo ostale? Odgovor je da: $\set{\shf}$ in $\set{\luk}$ sta polna nabora. (Izkaže se, da sta to edina taka veznika med dvomestnimi vezniki.)

                \begin{naloga}\label{naloga:polni-nabori-z-enim-veznikom}
                        \
                        \begin{enumerate}
                                \item
                                        Izrazi negacijo samo z veznikom $\shf$. Izrazi še konjunkcijo ali disjunkcijo samo z veznikom $\shf$. Sklepaj, da je $\set{\shf}$ poln nabor.
                                \item
                                        Izrazi negacijo samo z veznikom $\luk$. Izrazi še konjunkcijo ali disjunkcijo samo z veznikom $\luk$. Sklepaj, da je $\set{\luk}$ poln nabor.
                        \end{enumerate}
                \end{naloga}

                \davorin{Bi na tem mestu predebatirali preklopna vezja?}

                \davorin{Mogoče lahko zavoljo celovitosti podamo karakterizacijo polnih naborov kot izrek (in se za dokaz skličemo na literaturo). Nabor je poln, kadar za vsako sledečih lastnosti obstaja veznik v njem, ki jo prekrši: ohranjanje resnice, ohranjanje neresnice, monotonost, sebi-dualnost, afinost (kot polinom Žegalkina).}


        \section{Predikati in kvantifikatorji}

                \note{``Lastnostim'' elementov množic, ki smo jih prej uporabljali za podajanje podmnožic in pri kvantifikatorjih, zdaj ``uradno'' rečemo \df{predikati} in jih formalno definiramo: predikat na množici $X$ je preslikava $X \to \tvs$. Karakteristične preslikave podmnožic. Spomnimo se kvantifikatorjev in jih definiramo kot preslikave $\tvs^X \to \tvs$. Povemo, da lahko imajo predikati več spremenljivk in da lahko kvantificiramo po samo nekaterih (dobimo torej preslikave oblike $\tvs^{X \times Y} \to \tvs^Y$). Vezane, proste spremenljivke. Pravila, ki veljajo za kvantifikatorje (de Morgan itd.).}


\section{Vaje}

\begin{vaja}
  Preverite, da je $(p \Rightarrow q) \lor (q \Rightarrow p)$ tavtologija z resničnostno tabelo in s
  poenostavljanjem.
\end{vaja}

\anja{Ali želimo imeti toliko nalog iz polnih naborov? Jaz sem samo skopirala te naloge od prejšnjih vaj.}

\begin{vaja}
Pokaži, da so naslednji nabori izjavnih povezav polni.
\begin{enumerate}
 \item $\set{ \land, \xor, \true }$
 \item $\set{ \impl, \lnot }$
 \item $\set{ \impl, \false }$
 \item $\set{ \land, \false, \true, \Delta }$, kjer je $\Delta(p,q,r) \equiv p \xor q \xor r$.
\end{enumerate}
\end{vaja}

\begin{vaja}
 Naslednje izjave izrazi le z veznikoma $\lnot$ in $\impl$.
\begin{itemize}
  \item $p\land q$
  \item $(p\xor q)\lequ (p\lor q)$
  \item $p\luk q$
\end{itemize}
\end{vaja}

\begin{vaja}
Pokaži, da spodnja nabora izjavnih veznikov {\em nista} polna:
\begin{itemize}
 \item  $\set{\land, \lequ }$,
 \item  $\set{ \land, \xor }$.
\end{itemize}
\end{vaja}

%\begin{vaja}
%Trimestna izjavna povezava $\ifthen{p}{q}{r}$ je določena takole:
%\[
%\ifthen{p}{q}{r} = \left\{ \begin{array}{ll}
% q, & p = 1 \\
% r, & p = 0.
%\end{array} \right.
%\]
%\begin{itemize}
% \item Izrazi povezavo $\ifthen{p}{q}{r}$ čim krajše z običajnimi izjavnimi povezavami.
%  Pokaži, da velja enakovrednost $\ifthen{p}{q}{r} \sim (p \Rightarrow q) \land (\lnot p \Rightarrow r)$.
% \item Pokaži, da je nabor $\{ 0, 1, \ifthen{p}{q}{r} \}$ poln.
% \item Izrazi z naborom iz prejšnje točke izjavo $p \Leftrightarrow q$.
%\end{itemize}
%\end{vaja}

\begin{vaja}
Kateri izmed naslednjih izjavnih veznikov sestavlja poln nabor?
\begin{itemize}
 \item $\Lambda(p,q,r) \equiv p \impl (q \lor r)$
 \item $\Lambda(p,q,r) \equiv (p \shf q) \luk r$
 \item $\Lambda(p,q,r) \equiv (\lnot p \land \lnot r) \impl q$
 \item $\Lambda(p,q,r) \equiv p \impl (q \impl \lnot r)$
\end{itemize}
\end{vaja}


\begin{vaja}
Ali sestavljata izjavni povezavi $\set{ \impl, \not\impl }$, kjer je $p \not\impl q \equiv \lnot (p \impl q)$, poln nabor?
\end{vaja}

\begin{vaja}
Izjavna povezava $\square$ je določena z $p \square q \equiv p \land \lnot q$. Ugotovi, kateri nabori od spodnjih naborov izjavnih povezav so polni.
\begin{itemize}
 \item $\set{ \square }$
 \item $\set{ \square, \lnot }$
 \item $\set{ \square, \impl }$
\end{itemize}
\end{vaja}

\begin{vaja}
Preklopna vezja so sestavljena iz stikal in žic. Stikala so lahko vklopljena ali izklopljena, glede na njihovo stanje pa je odvisno, ali bo tok tekel po žici ali ne. Denimo, da imamo dve stikali $A$ in $B$. Če sta stikali vezani zaporedno, tj. 
\begin{center}
\begin{circuitikz} \draw
(0,0) to [switch, l^=$A$] (2,0) to[switch, l^=$B$] (4,0);
\end{circuitikz}
\end{center}
potem tok teče, kadar sta obe stikali vklopljeni. Če pa sta stikali vezani vzporedno, tj.
\begin{center}
\begin{circuitikz} \draw
(0,0) to [short, -*] (1,0) to [short] (1,1) to [switch, l^=$A$] (3,1) to[short] (3,0)
(1,0) to [short] (1,-1) to [switch, l^=$B$] (3,-1) to[short] (3,0) to[short, *-] (4,0);
\end{circuitikz}
\end{center}
potem tok teče, če je vklopljeno stikalo $A$ ali stikalo $B$. Tako lahko simuliramo logične veznike. Stikala so izjavne spremenljivke, takšni bloki pa predstavljajo vrata ``in'' ter ``ali''. Vrata za logične veznike predstavljamo z naslednjimi simboli:
\begin{center}
\begin{circuitikz} \draw
(0,0) node[or port] (myor1) {}
(0,2) node[and port] (myand1) {}
(6,0) node[not port](mynot1){}
(6,2) node[nor port](mynor){}
(2,0) node(o) {ali}
(2,2) node(a) {in}
(9,0) node(n) {ne}
(9,2) node(no) { niti};
%(myand1.out) -- (myxnor.in 1)
%(myand2.out) -- (myxnor.in 2);
\end{circuitikz}
\end{center}

Prvi dve vezji lahko torej z vrati zapišemo takole
\begin{center}
\begin{circuitikz} \draw
(0,4) to[switch, l^=$A$] (1,4)
(3,2) node[and port] (myand1) {}
(0,0) to[switch,  l^=$B$] (1,0)
(1,4) -- (myand1.in 1)
(1,0)-- (myand1.in 2)
(5,2) node(in){ter}
(6,4) to[switch, l^=$A$] (7,4)
(9,2) node[or port] (myand1) {}
(6,0) to[switch,  l^=$B$] (7,0)
(7,4) -- (myand1.in 1)
(7,0)-- (myand1.in 2);
\end{circuitikz}
\end{center}

Andrej prenavlja stanovanje in načrtuje električno napeljavo. Ker se mu pred spanjem ne ljubi vstajati, da bi ugasnil luč, si želi v spalnici imeti dve stikali, eno ob postelji in eno pri vhodu. Seveda pa morata obe stikali delovati, torej ko pritisnemo na katero koli stikalo, se mora luč prižgati ali pa ugasniti, če je že prižgana. Pri izdelavi električnega omrežja sme Andrej uporabiti le vrata ``in'', ``ali'' ter ``ne''. Ker pa so vrata draga, si želi uporabiti čim manj vrat. Pomagajte Andreju načrtovati vezje za njegovo spalnico. Kaj pa če lahko uporabi samo vrata ``in'' ter ``ali''? Ali lahko uporabi le vrata ``niti'' ($\downarrow$)?
\begin{resitev}
Imamo dve stikali, imenujmo ju $p$ in $q$. Opazujemo, kdaj luč sveti. Na začetku sta obe stikali ugasnjeni in luč ne sveti. Če prižgemo eno stikalo, mora luč zasvetiti. Če prižgemo nato še drugo stikalo mora luč ugasniti. Ugotovimo, da je luč prižgana, ko je prižgano natanko eno stikalo. To ponazorimo v naslednji tabeli:

\begin{center}
                        \begin{tabular}{cc|c}
                                $p$ & $q$ & \text{ luč sveti } \\
                                \hline
                                $\true$ & $\true$& $\false$ \\
                                $\true$ & $\false$  & $\true$ \\
                                $\false$ & $\true$ & $\true$ \\
                                $\false$ & $\false$  & $\false$
                        \end{tabular}
\end{center}
Opazimo, da ima to enako tabelo, kot izjava $p  \xor q$. Torej moramo to izjavo izraziti z izjavnimi vezniki $\land, \lor$ in $\neg$. En način, kako to naredimo je, da zapišemo $p \xor q \equiv (p \lor q) \land \neg (p \land q)$, in tako konstruiramo vezja z vrati ``in'', ``ali'' in negacijo takole:
\begin{center}
\begin{circuitikz} \draw
(2,0) node[anchor=north] (q) {}
(2,8) node[anchor=south] (p) {}
(4 ,2) node[or port] (myor1) {}
(4,6) node[and port] (myand1) {}
(6,6) node[not port](mynot1){}
(8,4) node[and port](myand2){}
(0,0) to[switch, l^=$q$, -*] (2,0)
(0,8) to[switch,  l^=$p$, -*] (2,8)
(p) -- (myor1.in 1)
(q) -- (myor1.in 2)
(p) -- (myand1.in 1)
(q) -- (myand1.in 2)
(myand1.out) -- (mynot1.in)
(mynot1.out) -- (myand2.in 1)
(myor1.out) -- (myand2.in 2)
(myand2.out) to [lamp] (10,4);
%(myand1.out) -- (myxnor.in 1)
%(myand2.out) -- (myxnor.in 2);
\end{circuitikz}
\end{center}
Le z veznikoma $\land$ in $\lor$ tega ne moremo storiti, saj veznika ne predstavljata polnega nabora. Z uporabo zgolj Łukasiewiczevega veznika pa je to mogoče, saj predstavlja poln nabor. 
\end{resitev}
\end{vaja}


%%% Local Variables:
%%% mode: latex
%%% TeX-master: "ucbenik-lmn"
%%% End:

% \chapter{Množice}
\label{chap:mnozice}

V drugem delu predmeta bomo spoznali osnove teorije množic. Najprej pa
se bomo posvetili še naravnim številom in Peanovim aksiomom.

%%%%%%%%%%%%%%%%%%%%%%%%%%%%%%%%%%%%%%%%%%%%%%%%%%%%%%%%%%%%%%%%%%%%%%
\section{Naravna števila}
\label{sec:naravna-stevila}

Naravna števila
%
\begin{equation*}
  0, 1, 2, 3, 4, 5, 6, 7, 8, 9, 10, 11, 12, \ldots
\end{equation*}
%
vsi že dobro poznamo iz osnovne šole.\footnote{V teh zapiskih in v
  logiki nasploh vzamemo za prvo naravno število $0$. V osnovni šoli
  in drugje pa ponavadi za prvo naravno število jemljemo~$1$.} V tem
razdelku pokažimo, kako uvedemo naravna števila kot formalno teorijo v
logiki. V splošnem \emph{formalna teorija} opisuje neko matematično
strukturo ali družino struktur in je podana z osnovnimi simboli
(konstantami in operacijami), aksiomi in pravili sklepanja.

Teorijo naravnih števil, ki jo imenujemo tudi \emph{Peanova
  aritmetika}, sestoji iz konstante $0$, enočlene operacije
naslednik~$\suc{n}$ ter dvočlenih operacij seštevanje~$m + n$ in
množenje~$m \cdot n$. Množenje ia prednost pred seštevanjem, se pravi,
da je $k + m \cdot n = k + (m \cdot n)$ in ne $(k + m) \cdot n$.
Aksiomi in pravila sklepanja se glasijo:
%
\begin{enumerate}
  \item Nič ni naslednik:
  %
  \begin{equation*}
    \inferrule{ }{\suc{n} \neq 0}    
  \end{equation*}
  %
  \item Če sta naslednika enaka, sta števili enaki:
  %
  \begin{equation*}
    \inferrule{\suc{m} = \suc{n}}{m = n}
  \end{equation*}
  %
  \item Pravili za seštevanje:
  %
  \begin{mathpar}
    \inferrule{ }{0 + n = n}
    \and
    \inferrule{ }{\suc{m} + n = (m + n\suc{)}}    
  \end{mathpar}
  %
  \item Pravili za množenje:
  %
  \begin{mathpar}
    \inferrule{ }{0 \cdot n = 0}
    \and
    \inferrule{ }{\suc{m} \cdot n = m \cdot n + n}
  \end{mathpar}
  %
  \item Princip indukcije:
  %
  \begin{equation*}        
    \inferrule{\phi(0) \\ \xall{m}{\NN}{\phi(m) \lthen \phi(\suc{m})}}{\phi(n)}
  \end{equation*}
\end{enumerate}
%
Pri običajnem računanju z naravnimi števili uporabljamo vse znanje, ki
smo ga pridobili v šoli. Ko pa obravnavamo naravna števila kot
formalno teorijo, smemo uporabljati \emph{samo} konstante in simbole,
ki jih vpeljemo v teoriji, in se sklicevati \emph{samo} na Peanove
aksiome. Denimo, ker teorija ne vpelje simbolov $1$ in $2$, ju ne
smemo uporabljati, razen če ju prej definiramo kot okrajšavi za
$\suc{0}$ in $\suc{(\suc{0})}$. Prav tako ne smemo omenjati odštevanja
števil, ker to ni ena od operacij $+$ in $\cdot$, ne smemo govoriti o
parnosti števil, ne da bi prej ta pojem definirali, itn. Tudi osnovne
lastnosti seštevanja in množenja, kot sta komutativnost in
asociativnost, ne smemo uporabiti, če ju prej ne dokažemo. Matematiki
so seveda preverili, da vse običajne lastnosti števil dejansko sledijo
iz Peanovih aksiomov.

Glavno orodje pri dokazovanju lastnosti naravnih števil je princip
indukcije. V besedilu ga uporabimo takole:
%
\begin{quote}
  \em
  %
  Dokazujemo $\phi(n)$ z indukcijo po~$n$:
  %
  \begin{enumerate}
    \item Baza indukcije: (Dokaz, da velja $\phi(0)$.)
    \item Indukcijski korak: denimo, da za naravno število $m$ velja
      $\phi(m)$. (Dokaz, da velja $\phi(\suc{m})$.)
  \end{enumerate}
\end{quote}
%
Za zgled dokažimo, da je seštevanje komutativno. To naredimo v nekaj
korakih.

\begin{izjava}
  \label{izjava:peano-n-plus-0}
  Za vsako naravno število $m$ velja $m + 0 = m$.
\end{izjava}

\begin{dokaz}
  Dokazujemo z indukcijo. Baza indukcije: $0 + 0 = 0$ po enem od
  Peanovih aksiomov.
  %
  Indukcijski korak: denimo, da za naravno število $k$ velja $k + 0 =
  k$. Tedaj je $\suc{k} + 0 = \suc{(k + 0)} = \suc{k}$, kjer smo v
  prvem koraku uporabili enega od Peanovih aksiomov in v drugem
  indukcijsko predpostavko.
\end{dokaz}


\begin{izjava}
  \label{izjava:peano-m-plus-suc-n}
  Za vsaki naravni števili $m$ in $n$ velja $m + \suc{n} = \suc{(m + n)}$.
\end{izjava}

\begin{dokaz}
  Izjavo dokažemo z indukcijo po $m$.
  Baza indukcije: $0 + \suc{n} = \suc{n} = \suc{(0 + n)}$.
  %
  Indkucijski korak: denimo, da za naravno število $k$ velja $k +
  \suc{n} = \suc{(k + n)}$. Tedaj je
  %
  \begin{equation*}
    \suc{k} + \suc{n} = 
    \suc{(k + \suc{n})} =
    \suc{{\suc{(k + n)}}} =
    \suc{(\suc{k} + n)}.
  \end{equation*}
  %
\end{dokaz}

\begin{izjava}
  Za vsaki naravni števili $m$ in $n$ velja $m + n = n + m$.
\end{izjava}

\begin{dokaz}
  Izjavo dokažemo z indukcijo po $m$.
  Baza indukcije: $0 + n = n = n + 0$, kjer smo v prvem koraku uporabili Peanov aksiom in v drugem Izjavo~\ref{izjava:peano-n-plus-0}.
  %
  Indukcijski korak: denimo, da za naravno število $k$ velja $k + n = n + k$. Tedaj je
  %
  \begin{equation*}
    \suc{k} + n =
    \suc{(k + n)} =
    \suc{(n + k)} =
    n + \suc{k}.
  \end{equation*}
  %
  V prvem koraku smo uporabili Peanov aksiom, v drugem indukcijsko predpostavko, v tretjem pa Izjavo~\ref{izjava:peano-m-plus-suc-n}.
\end{dokaz}

%%%%%%%%%%%%%%%%%%%%%%%%%%%%%%%%%%%%%%%%%%%%%%%%%%%%%%%%%%%%%%%%%%%%%%
\section{Množice}
\label{sec:naivne-mnozice}

Množice so osnovni gradniki matematičnih objektov in struktur. V tem razdelku obravnavamo množice \emph{naivno}, se pravi s pomočjo neformalnih razlag. Samo formalno teorijo množic in aksiome bomo obravnavali v razdelku~\ref{sec:zfc}.

Množico si predstavljamo kot skupek ali zbirko poljubnih objektov, jim pravimo \emph{elementi} množice. Dejstvo, da je $x$ element množice $A$ zapišemo $x \in A$. Če $x$ ni element $S$, pišemo $x \not\in S$ kot okrajšavo za $\lnot (x \in S)$. Množica ni odvisna od tega, kako jo opišemo ali skonstruiramo, ampak le od tega, kateri elementi so v njej. To dejstvo izraža \emph{aksiomom o ekstenzionalnosti}, ki pravi, da sta množici $A$ in $B$ enaki natanko tedaj, ko vsebujeta iste elemente, kar zapišemo s formulo kot
%
\begin{equation*}
  A = B \liff \uall{x}{x \in A \liff x \in B}.
\end{equation*}
%
Množice gradimo iz osnovnih množic s pomočjo operacij.

\subsection{Osnovne množice}
\label{sec:osnovne-mnozice}

Najpreprostejša osnovna množica je \emph{prazna množica}, ki jo označimo z $\emptyset$. Dejstvo, da prazna množica ne vsebuje nobenih elementov izrazimo z aksiomom o prazni množici,
%
\begin{equation*}
  \xuall{x}{x \not\in \emptyset}.
\end{equation*}
%
V zvezi s prazno množico omenimo, da za vsako izjavo $\phi$ velja
%
\begin{equation*}
  \xall{x}{\emptyset}{\phi},
\end{equation*}
%
kar dokažemo takole: naj bo $x \in \emptyset$ poljuben. Ker velja $x \not\in \emptyset$, je to protislovje, od koder smemo sklepati $\phi$. Podobno za vsako izjavo $\phi$ velja
%
\begin{equation*}
  \lnot\xsome{x}{\emptyset}{\phi}.
\end{equation*}

\begin{naloga}
  Za katere množice $S$ velja $(\xall{x}{S}{\phi(x)}) \lthen   \xsome{x}{S}{\phi(x)}$?
\end{naloga}

Za osnovno množico vzamemo tudi množico naravnih števil~$\NN$, ki smo jo že spoznali v razdelku~\ref{sec:naravna-stevila}.

\subsection{Konstrukcije množic}
\label{sec:konstrukcije-mnozic}

Iz osnovnih množic lahko konstruiramo nove s pomočjo naslednjih operacij.

\subsubsection{Končne množice}
\label{sec:koncne-mnozice}

Naj bodo $a_1, \ldots, a_n$ poljubni objekti. Tedaj lahko tvorimo množico
%
\begin{equation*}
  \set{a_1, a_2, \ldots, a_n}
\end{equation*}
%
ki sestoji iz naštetih elementov, to je
%
\begin{equation*}
  \uall{x}{x \in \set{a_1, \ldots, a_n} \liff
    x = a_1 \lor \cdots x = a_n}.
\end{equation*}
%
Poseben primer take množice je \emph{enojec} $\set{a}$, za katerega velja
%
\begin{equation*}
  \uall{x}{x \in \set{a} \liff x = a}.
\end{equation*}


\begin{naloga}
  Ali je $\set{a, b} = \set{b, a}$? Ali je $\set{a, a, b} = \set{a, b}$? Uporabi aksiom o ekstenzionalnosti.
\end{naloga}

\subsubsection{Unija in presek}
\label{sec:unija-presek}

Družina množic.

Unije, preseki.

\subsubsection{Podmnožica}
\label{sec:podmnozica}

Podmnožica (separacija).

\subsubsection{Potenčna množica}
\label{sec:potencna-mnozica}

Potenčna množica.

\subsubsection{Kartezični produkt}
\label{sec:kartezicni-produkt}

Kartezični produkt. Produkt s prazno.

\subsubsection{Eksponentna množica}
\label{sec:eksponentna-mnozica}

Eksponentna množica. Eksponent s prazno.

\subsubsection{Vsota}
\label{sec:vsota-mnozic}

Disjunktna unija.

\subsubsection{Razlika in komplement}
\label{sec:vsota-mnozic}


\section{Funkcije}
\label{sec:funkcije}


Funkcija, neformalna definicija.

Kompozitum, asociativnost kompozituma. Identiteta.

Inverz funkcije. Inverz je enoličen, če obstaja.

Slika in inverzna slika.

Kdaj obstaja inverz? Surjektivna, injektivna, bijektivna funckija.

Epi in mono.

Sekcija in retrakcija.

Sekcija je mono, retrakcija je epi.

Standardne bijekcije za vsoto, produkt in eksponent.

\section{Relacije}
\label{sec:relacije}

Definicija relacije.

Nasprotna relacija. Komplement, unija.

\subsection{Funkcijske relacije}
\label{sub:funkcijske_relacije}


\subsection{Ekvivalenčne relacije}
\label{sub:ekvivalencne_relacije}


%Definirali smo pojem ekvivalenčne relacije in kvocienta množice po
%ekvivalenčni relaciji. Pokazali smo razne primere. Dokazali smo, da
%smemo definirati $f : A/{\sim} \to B$ na kvocientu tako, da definiramo
%$g : A \to B$, ki je skladen s~$\sim$.




Defincije. Primeri.

Ekvivalenčna relacija, generirana z relacijo.

Faktorska množica. Kako definiramo preslikavo na faktorski množici.

Kanonični razcep funkcije.

\subsection{Delna ureditev}
\label{sub:delna_ureditev}

Definicija delne ureditve. Primeri.

Linearna ureditev. Stroga linearna ureditev. Veriga.

Zgornja meja, spodna meja, infimum, supremum, maksimum, minimum, minimalni element, maksimalni element.



%%% Local Variables: 
%%% mode: latex
%%% TeX-master: "lmn"
%%% End: 



%--------------------------------------------------------------------
% BIBLIOGRAFIJA

%\bibliographystyle{plain}
%\addcontentsline{toc}{chapter}{\numberline{}Literatura}
%\markboth{}{Literatura}
%
%{
%\raggedright
%\renewcommand{\markboth}[2]{}
%\bibliography{literatura}
%}


\end{document}

%%% Local Variables: 
%%% mode: latex
%%% TeX-master: t
%%% End: 
