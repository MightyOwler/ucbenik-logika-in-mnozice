\documentclass[11pt,a4paper]{book}

\usepackage[T1]{fontenc}
\usepackage[utf8]{inputenc}
\usepackage[slovene]{babel}
\usepackage[colorlinks]{hyperref}
\usepackage{lmodern}
\usepackage{xcolor}
\usepackage{amsfonts,amssymb,amsmath}
\usepackage{bbold}
\usepackage{fancyhdr}
\usepackage{theorem}
\usepackage{mathpartir}
\usepackage{proof}
\usepackage{xypic}
\usepackage{tikz}
\usepackage{booktabs}

%--------------------------------------------------------------------
%-- Barve hiper povezav

\hypersetup{
    colorlinks,
    linkcolor={red!50!black},
    citecolor={blue!50!black},
    urlcolor={blue!80!black}
}

%--------------------------------------------------------------------
%-- Okolja

{
  \theorembodyfont{\itshape}

  \newtheorem{izrek}{Izrek}[chapter]
  \newtheorem{lema}[izrek]{Lema}
  \newtheorem{izjava}[izrek]{Izjava}
  \newtheorem{posledica}[izrek]{Posledica}
  \newtheorem{hipoteza}[izrek]{Hipoteza}
  \newtheorem{aksiom}[izrek]{Aksiom}
}

{
  \theorembodyfont{\rmfamily}
  \newtheorem{definicija}[izrek]{Definicija}
  \newtheorem{primer}[izrek]{Primer}
  \newtheorem{opomba}[izrek]{Opomba}
  \newtheorem{naloga}[izrek]{Naloga}
}

\newcommand{\qedsign}{{\vrule width 1ex height 1ex depth 0ex}}
\newcommand{\qed}{\hfill\qedsign}

\newenvironment{dokaz}{
  \goodbreak\par
  \textit{Dokaz.}%
}{%
  \nopagebreak
  \qed
  \medskip
  \goodbreak
}

%--------------------------------------------------------------------
%%%%%%%%%%%%%%%%%%%%%%%%%%%%%%%%%%%%%%%%%%%%%%%%%%%%%%%%%%%%%%%%%%%%%%%%%%%%%%%%%%%%%%%%%%%%%%%%%%%%%%%%%%%%%%%%%%%%%%
%%%  Commands
%%%%%%%%%%%%%%%%%%%%%%%%%%%%%%%%%%%%%%%%%%%%%%%%%%%%%%%%%%%%%%%%%%%%%%%%%%%%%%%%%%%%%%%%%%%%%%%%%%%%%%%%%%%%%%%%%%%%%%


%%%%%%%%%%%%%%%%%%%%%%%%%%%%%%%%%%%%%%%%%%%%%%%%%%%%%%%%%%%%%
%%%  Theorems etc.
%%%%%%%%%%%%%%%%%%%%%%%%%%%%%%%%%%%%%%%%%%%%%%%%%%%%%%%%%%%%%
{
\theoremstyle{theorem}
\newtheorem{izrek}{Izrek}[chapter]
\newtheorem{lema}[izrek]{Lema}
\newtheorem{trditev}[izrek]{Trditev}
\newtheorem{posledica}[izrek]{Posledica}
\newtheorem{pravilo}[izrek]{Pravilo}
}

{
\theoremstyle{definition}
\newtheorem{definicija}[izrek]{Definicija}
\newtheorem{opomba}[izrek]{Opomba}
\newtheorem{primer}[izrek]{Primer}
\newtheorem{zgled}[izrek]{Zgled}
\newtheorem{naloga}[izrek]{Naloga}
}


%%%%%%  Proofs
%%%%%%%%%%%%%%%%%%%%%%%%%%%%%%%%%%%%%%%%%%%%%%%%%%%%%%%%%%%%%
% Za dokaze uporabimo amsmath proof, sicer ne deluje \qedhere.


%%%%%%  Auxiliary
%%%%%%%%%%%%%%%%%%%%%%%%%%%%%%%%%%%%%%%%%%%%%%%%%%%%%%%%%%%%%
\newcommand{\sizedescriptor}[2]
{
\ifthenelse{\equal{#1}{0}}{}{
\ifthenelse{\equal{#1}{1}}{\big}{
\ifthenelse{\equal{#1}{2}}{\Big}{
\ifthenelse{\equal{#1}{3}}{\bigg}{
\ifthenelse{\equal{#1}{4}}{\Bigg}{
#2}}}}}
}

\newcommand{\someref}{{\small\textcolor{blue}{[\textbf{ref.}]}}}
\newcommand{\intermission}{\bigskip\medskip}
\newcommand{\ltc}[1]{$\backslash$\texttt{#1}}  % LaTeX command
\newcommand{\nls}[1]{``\textit{#1}''}  % sentence in a natural language

%%%%%%  Logical Quantifiers, λ- and ι-Expressions
%%%%%%%%%%%%%%%%%%%%%%%%%%%%%%%%%%%%%%%%%%%%%%%%%%%%%%%%%%%%%

\newcommand{\all}[1]{\forall #1 .\,}
\newcommand{\some}[1]{\exists #1 .\,}
\newcommand{\exactlyone}[1]{\exists{!} #1 .\,}
\newcommand{\lam}[1]{\lambda #1 .\,}
\newcommand{\that}[1]{\iota #1 .\,}

%%%%%%  Logic
%%%%%%%%%%%%%%%%%%%%%%%%%%%%%%%%%%%%%%%%%%%%%%%%%%%%%%%%%%%%%
\newcommand{\tvs}{\Omega}  % set of truth values
\newcommand{\true}{\top}  % truth
\newcommand{\false}{\bot}  % falsehood
\newcommand{\etrue}{\boldsymbol{\top}}  % emphasized truth
\newcommand{\efalse}{\boldsymbol{\bot}}  % emphasized falsehood
\newcommand{\impl}{\Rightarrow}  % implication sign
\newcommand{\revimpl}{\Leftarrow}  % reverse implication sign
\newcommand{\lequ}{\Leftrightarrow}  % equivalence sign
\newcommand{\xor}{\mathbin{\veebar}}  % exclusive disjunction sign
\newcommand{\shf}{\mathbin{\uparrow}}  % Sheffer connective
\newcommand{\luk}{\mathbin{\downarrow}}  % Łukasiewicz connective


%%%%%%  Sets
%%%%%%%%%%%%%%%%%%%%%%%%%%%%%%%%%%%%%%%%%%%%%%%%%%%%%%%%%%%%%
%  \set{1, 2, 3}         ->  {1, 2, 3}
%  \set{a \in X}{a < 1}  ->  {a ∈ X | a < 1}
\NewDocumentCommand{\set}
{O{auto} m G{\empty}}
{\sizedescriptor{#1}{\left}\{ {#2} \ifthenelse{\equal{#3}{}}{}{ \; \sizedescriptor{#1}{\middle}| \; {#3}} \sizedescriptor{#1}{\right}\}}
%\newcommand{\vsubset}{\Mapstochar\cap}
%\newcommand{\finseq}[1]{{#1}^*}
\newcommand{\pst}{\mathcal{P}}
\renewcommand{\complement}[1]{{#1}^C}


%%%%%%  Number Sets, Intervals
%%%%%%%%%%%%%%%%%%%%%%%%%%%%%%%%%%%%%%%%%%%%%%%%%%%%%%%%%%%%%
\newcommand{\NN}{\mathbb{N}}
\newcommand{\ZZ}{\mathbb{Z}}
\newcommand{\QQ}{\mathbb{Q}}
\newcommand{\RR}{\mathbb{R}}
\newcommand{\CC}{\mathbb{C}}
\newcommand{\HH}{\mathbb{H}}
\newcommand{\OO}{\mathbb{O}}
\newcommand{\intoo}[3][\RR]{{#1}_{(#2, #3)}}
\newcommand{\intcc}[3][\RR]{{#1}_{[#2, #3]}}
\newcommand{\intoc}[3][\RR]{{#1}_{(#2, #3]}}
\newcommand{\intco}[3][\RR]{{#1}_{[#2, #3)}}


%%%%%%  Maps and Relations
%%%%%%%%%%%%%%%%%%%%%%%%%%%%%%%%%%%%%%%%%%%%%%%%%%%%%%%%%%%%%
\newcommand{\id}[1][]{\mathrm{id}_{#1}}  % identity map
\newcommand{\argbox}{{\;\!\fbox{\phantom{M}}\;\!}}  % box for a function argument
\newcommand{\konst}[1]{\mathrm{k}_{#1}} % constant map
\newcommand{\rstr}[1]{\left.{#1}\right|}  % map restriction
\newcommand{\im}{\mathrm{im}}  % map image
\newcommand{\parto}{\mathrel{\rightharpoonup}}  % partial mapping sign
\NewDocumentCommand{\rel}
{O{\empty} O{\empty}}
{\ifthenelse{\equal{#1}{}}{\mathscr{R}}{{#1} \mathrel{\mathscr{R}} {#2}}}  % a relation
\NewDocumentCommand{\srel}
{O{\empty} O{\empty}}
{\ifthenelse{\equal{#1}{}}{\mathscr{S}}{{#1} \mathrel{\mathscr{S}} {#2}}}  % a second relation
\newcommand{\dom}{\mathrm{dom}}  % domain
\newcommand{\cod}{\mathrm{cod}}  % codomain
\newcommand{\dd}[1]{D_{#1}}  % domain of definition
\newcommand{\rn}[1]{Z_{#1}}  % range
\newcommand{\graph}[1]{\Gamma_{#1}}  % graph of a (partial) function
\NewDocumentCommand{\img}  % image
{O{\empty} m G{\empty}}
{{#2}_*\ifthenelse{\equal{#3}{}}{}{\!\sizedescriptor{#1}{\left}( {#3} \sizedescriptor{#1}{\right})}}
\NewDocumentCommand{\pim}  % preimage
{O{\empty} m G{\empty}}
{{#2}^*\ifthenelse{\equal{#3}{}}{}{\!\sizedescriptor{#1}{\left}( {#3} \sizedescriptor{#1}{\right})}}
\newcommand{\ec}[2][]{[\:\!{#2}\:\!]_{#1}}  % equivalence class
\newcommand{\transposed}[1]{\widehat{#1}}


%%%%%%  Projections and Injections
%%%%%%%%%%%%%%%%%%%%%%%%%%%%%%%%%%%%%%%%%%%%%%%%%%%%%%%%%%%%%
\NewDocumentCommand{\fst}
{O{\empty} O{\empty}}
{\pi_1^{{#1}\ifthenelse{\equal{#2}{}}{}{,}{#2}}}
\NewDocumentCommand{\snd}
{O{\empty} O{\empty}}
{\pi_2^{{#1}\ifthenelse{\equal{#2}{}}{}{,}{#2}}}
\NewDocumentCommand{\inl}
{O{\empty} O{\empty}}
{\iota_1^{{#1}\ifthenelse{\equal{#2}{}}{}{,}{#2}}}
\NewDocumentCommand{\inr}
{O{\empty} O{\empty}}
{\iota_2^{{#1}\ifthenelse{\equal{#2}{}}{}{,}{#2}}}


%%%%%%  Categories
%%%%%%%%%%%%%%%%%%%%%%%%%%%%%%%%%%%%%%%%%%%%%%%%%%%%%%%%%%%%%
\newcommand{\ct}[1]{\mathbf{#1}}
\newcommand{\mnoz}{\ct{Mno\check{z}}}
\newcommand{\pkol}{\ct{PKol}}  % category of semirings
\newcommand{\upkol}{\pkol_1}  % category of unital semirings
\newcommand{\kol}{\ct{Kol}}  % category of rings
\newcommand{\ukol}{\kol_1}  % category of unital rings


%%%%%%  Exercises and Solutions
%%%%%%%%%%%%%%%%%%%%%%%%%%%%%%%%%%%%%%%%%%%%%%%%%%%%%%%%%%%%%
\Newassociation{resitev}{Resitev}{resitve}
\renewcommand{\Resitevlabel}[1]{\emph{Re\v{s}itev~#1}}
{
\theoremstyle{definition}
\newtheorem{vaja}{Vaja}[chapter]
}


%%%%%%  Misc.
%%%%%%%%%%%%%%%%%%%%%%%%%%%%%%%%%%%%%%%%%%%%%%%%%%%%%%%%%%%%%
\renewcommand{\divides}{\,|\,}
% Načeloma bi morala biti navpična črta v \divides obdana z \mathrel, ampak to vodi do prevelikih presledkov.
\newcommand{\df}[1]{\emph{\textbf{#1}}}  % defined notion
\newcommand{\oper}{\mathop{\circledast}\nolimits}  % symbol for a generic operation
\newcommand{\soper}{\mathop{\boxasterisk}\nolimits}  % symbol for a second generic operation
\newcommand{\qo}[1]{\;\!\widetilde{#1}\;\!}  % quotient operation
\newcommand{\tconc}{\mathop{\bullet}\nolimits}  % symbol for binary tree concatenation
\newcommand{\conc}{\mathop{::}\nolimits}  % symbol for string concatenation
\newcommand{\ism}{\cong}  % isomorphic
\newcommand{\inv}[1]{#1^{-1}} % inverz preslikave
\newcommand{\equ}{\sim}  % equivalent
\newcommand{\dfeq}{\mathrel{\mathop:}=}  % definitional equality
\newcommand{\revdfeq}{=\mathrel{\mathop:}}  % reverse definitional equality
\newcommand{\isdefined}[1]{{#1}\!\downarrow}  % given value is defined
\newcommand{\kleq}{\simeq}  % Kleene equality
\newcommand{\claim}[3]{{#1} \;\colon\; \frac{#2}{#3}}  % claim, divided on context, assumptions, conclusions
\newcommand{\one}{\mathtt{\mathbf{1}}}  % generic singleton
\newcommand{\unit}{\mathord{()}}  % element in a generic singleton
\newcommand{\nul}{\mathtt{N}}  % null map
\newcommand{\suc}{\mathtt{S}}  % successor
\newcommand{\prd}{\mathtt{P}}  % predecessor
\newcommand{\tprd}{\tilde{\prd}}  % predecessor as a total function
\newcommand{\monus}{\mathbin{\vphantom{+}\text{\mathsurround=0pt \ooalign{\noalign{\kern-.35ex}\hidewidth$\smash{\cdot}$\hidewidth\cr\noalign{\kern.35ex}$-$\cr}}}}
% Definicija za monus pobrana s TeX Stack Exchange
\newcommand{\wf}{\prec}  % well-founded order
\NewDocumentEnvironment{implproof}  % proof of an implication
{O{\empty} G{\empty} O{=>} G{\empty}}
{
\begin{description}
\item[\quad$\sizedescriptor{#1}{\left}({#2}
\ifthenelse{\equal{#3}{=>}}{\impl}{
\ifthenelse{\equal{#3}{<=}}{\revimpl}{
\ifthenelse{\equal{#3}{->}}{\rightarrow}{
\ifthenelse{\equal{#3}{<-}}{\leftarrow}{
#3
}}}} {#4}\sizedescriptor{#1}{\right})$]\ \vspace{0.3em}\\
}
{
\end{description}
}
\NewDocumentCommand{\pres}  % presentation of an algebraic structure with generators and relations
{O{auto} m G{\empty}}
{\sizedescriptor{#1}{\left}\langle {#2} \ifthenelse{\equal{#3}{}}{}{ \; \sizedescriptor{#1}{\middle}| \; {#3}} \sizedescriptor{#1}{\right}\rangle}


%%%%%%%%%%%%%%%%%%%%%%%%%%%%%%%%%%%%%%%%%%%%%%%%%%%%%%%%%%%%%%%%%%%%%%%%%%%%%%%%%%%%%%%%%%%%%%%%%%%%%%%%%%%%%%%%%%%%%%

%%% Local Variables:
%%% mode: latex
%%% TeX-master: "ucbenik-lmn"
%%% End:


%--------------------------------------------------------------------
%-- Velikost strani

%% A4 stran = 210mm x 297mm
%% sirino besedila nastavimo na 170mm, visino na 247mm

\setlength{\textwidth}{15cm}
\setlength{\textheight}{224mm}

\setlength{\topmargin}{0cm}
\setlength{\evensidemargin}{0cm}
\setlength{\oddsidemargin}{\paperwidth}
\addtolength{\oddsidemargin}{-\textwidth}
\addtolength{\oddsidemargin}{-2in}

%--------------------------------------------------------------------
%-- Glava in dno

\pagestyle{fancyplain}

%\setlength{\headrulewidth}{0.2pt}
%\addtolength{\headheight}{2pt}

\renewcommand{\chaptermark}[1]{\markboth{#1}{}}
\renewcommand{\sectionmark}[1]{\markright{\thesection\ #1}}

\lhead[\fancyplain{}{{\thepage}}]%
      {\fancyplain{}{{\rightmark}}}
\rhead[\fancyplain{}{{\leftmark}}]%
      {\fancyplain{}{\thepage}}
\cfoot{\footnotesize [verzija \today]}
\lfoot[]{}
\rfoot[]{}

%--------------------------------------------------------------------
% NASLOV

\author{Andrej Bauer}
\title{Logika in množice \\ ZAPISKI V NASTAJANJU}

\begin{document}

\maketitle

\cleardoublepage

%--------------------------------------------------------------------
% KAZALO
\pagestyle{fancyplain}

{
\renewcommand{\markboth}[2]{}
\tableofcontents
}

\cleardoublepage

%--------------------------------------------------------------------
% VSEBINA

\chapter{Predgovor}
\label{chap:predgovor}

Glavni namen predmet Logika in množice v prvem letniku študija matematike je študente
naučiti osnov matematičnega izražanja: kako beremo in pišemo matematično besedilo, kako
uporabljamo simbolni zapis, kako zapišemo in preberemo dokaz itd. Drugi poglavitni namen
predmeta je spoznavanje osnov matematične logike in teorije množic.

Za semesterski predmet z dvema urama predavanj in dvema urama vaj ima predmet zelo
ambiciozen program. Najučinkovitejši recept za uspeh je tisti, ki ga študenti ne marajo:
učite se sproti, sprašujte predavatelja in asistente, trkajte na vrata njihovih pisarn
tudi takrat, ko nimajo govorilnih ur.

Ti zapiski s predavanj nastajajo sproti. Prvotno sem jih zapisoval v formatu Markdown, a napočil je čas, da jih prenesem v {\LaTeX} in nato izboljšujem. Opozarjam vas, da zapiski vsebujejo napake, ker so le grob zapis vsebine predavanj. Odkrivanje napak je sestavni del učnega procesa, čeprav si ne želim, da bi bi bilo napak toliko, da bi motile učenje. Zelo vam bom hvaležen, če mi boste odkrite napake sporočili, da jih popravim. Asistentom pri predmetu se zahvaljujem za skrbno odpravljanje napak. Vse ki so ostale, so moja last.

\bigskip

\begin{flushright}
Andrej Bauer \qquad\hbox{}
\end{flushright}

\bigskip

\paragraph{Zahvala.}
%
Pri urejanju zapiskov pomagali:
%
Matej Marinko,
Lev Rus,
Jakob Schrader,
Matija Sirk in
Marjetka Zupan.
%
Vsem se najlepše zahvaljujem.


%%% Local Variables: 
%%% mode: latex
%%% TeX-master: "lmn"
%%% End: 

\chapter{Osnovni podatki o predmetu}

\paragraph{Gradivo:}
%
Osnovni podatki o predmetu in gradivo je na \href{https://ucilnica.fmf.uni-lj.si/}{spletni učilnici}, kjer najdete:
\begin{itemize}
\item povezavo do video posnetkov predavanj in zapiskov s table,
\item naloge z vaj, ki so objavljenje v naprej,
\item prejšnje kolokvije in izpite,
\item povezo na Discord server za predmet.
\end{itemize}

\paragraph{Izpitni režim.}
%
Predmet opravite z izpitom, ki ima dva dela:
%
\begin{enumerate}
\item \textbf{pisni izpit}
\item \textbf{ustni izpit}
\end{enumerate}
%
Namesto pisnega izpita lahko opravite dva kolokvija (s povprečno oceno obeh kolokvijev skupaj vsaj 50\%). Na ustni izpit pridete šele, ko ste opravili pisni izpit. Če ustnega izpita ne opravite, vam pisni izpit propade in ga
morate ponovno opravljati.

\chapter{Množice in preslikave}

Pri predmetu Logika in množice se bomo učili, kako matematiki komuniciramo in razmišljamo. Spoznali bomo osnove logike
in teorije množic, tako iz povsem praktičnega vidika kot tudi matematičnega. Pri tem predmetu cenimo ne le matematično
razmišljanje, ampak tudi razmišljanje o matematiki.

Za uvod povejmo nekaj osnovnega o množicah in spoznajmo nekatere osnovne konstrukcije.

\section{Osnovno o množicah}

\subsection{Množice kot skupki elementov, relacija $\in$}

Naivno bi rekli, da je množica kakršnakoli zbirka ali skupek matematičnih objektov. Le-ti so lahko števila, funkcije,
množice, množice števil ipd., skratka karkoli.
%
Najbolj preprosti primeri množic so končne množice, katerih elemente naštejemo. Zapišemo jih na primer takole:
%
\begin{gather*}
  \{1, 2, 3\}
  \{\sin, \cos, \tan\}
  \{\{1\}, \{2\}, \{3\}\}
\end{gather*}
%
Objektom, ki tvorijo množico, pravimo \textbf{elementi}. Na primer, elementi množice $\{1, \{4\}, 7/3\}$ so število $1$, množica $\{4\}$, in število $7/4$.

Kadar je $a$ element množice $M$, to zapišemo $a \in M$ in beremo ">$a$ je element $M$"<.

Ali sta množici $\{1, 4, 10\}$ in $\{4, 10, 1, 10\}$ enaki? Da, saj množice obravnavamo kot \emph{neurejene} skupke, v katerih ni pomembno, kolikokrat se pojavi kak element. Da vrstni red in število pojavitev nista pomembna, sledi iz \textbf{aksioma
ekstenzionalnosti}. Aksiom je matematična izjava, ki jo vzamemo za osnovno, se pravi, da je ne dokazujemo. Aksiomi opredeljujejo matematično teorijo, ki jo želimo študirati. Tako bom pri tem predmetu spoznali aksiome teorije množic, pri algebri aksiome za vektorski prostor in grupo itd.

\begin{aksiom}[Ekstenzionalnost množic]
  Množici sta enaki, če imata iste elemente.
\end{aksiom}

Povedano drugače: če je vsak element množice $A$ tudi element množice $B$ in je vsak element množice $B$ tudi element množice $A$, potem velja $A = B$.

Z uporabo ekstenzionalnosti, lahko \emph{dokažemo}, da sta $\{1, 4, 10\}$ in $\{4, 10, 1, 10\}$ enaki:
%
\begin{enumerate}
\item 
  Vsak element $\{1, 4, 10\}$ je tudi element $\{4, 10, 1, 10\}$:
  \begin{enumerate}
    \item velja $1 \in \{4, 10, 1, 10\}$
    \item velja $4 \in \{4, 10, 1, 10\}$
    \item velja $10 \in \{4, 10, 1, 10\}$
  \end{enumerate}
\item
Vsak element $\{4, 10, 1, 10\}$ je tudi element $\{1, 4, 10\}$:
  \begin{enumerate}
     \item velja $4 \in \{1, 4, 10\}$
     \item velja $10 \in \{1, 4, 10\}$
     \item velja $1 \in \{1, 4, 10\}$
     \item velja $10 \in \{1, 4, 10\}$
  \end{enumerate}
\end{enumerate}

Iz zgornjih dveh preverjanj z uporabo ekstenzionalnosti sledi, da $\{1, 4, 10\} = \{4, 10, 1, 10\}$.

\begin{naloga}
  Zapišite podroben dokaz, da sta množici $\{x, y\}$ in $\{y, x\}$ enaki.
\end{naloga}

\begin{opomba}
  Poznamo tudi skupke, pri katerih je pomembno, kolikokrat se pojavi vsak element. Imenujejo se \textbf{multimnožice}.
\end{opomba}

Opozorimo takoj, da v praksi pogosto uporabljamo zapise, ki niso povsem natančni. Takrat se zanašamo, da bodo ostali pravilni uganili, kaj imamo v mislih. Na primer, katere elemente vsebuje množica
%
\begin{equation*}
    \{1, 2, 3, ..., 2021\} \ ?
\end{equation*}
%
Verjetno bi vsi ">uganili"<, da so mišljena vsa naravna števila med $1$ in $2021$, ali ne? Zavedati se je treba, da zgornji zapis tega ne določa! Morda smo imeli v mislih vsa števila med $1$ in $2021$, ki pri deljenju s~$5$ ne dajo ostanka~$4$.

Pri tem predmetu bomo pogosto opozarjali na razne nejasnosti in nenatančne zapise, ki jih uporabljajo matematiki v praksi.
Ni mišljeno, da bi se pretvarjali, da je kaj narobe s ">človeško matematiko"<. Želimo se predvsem zavedati, kje se nejasnosti v praksi pojavljajo in kako bi jih lahko odpravili (tudi če jih v praksi dejansko ne odpravimo). Ko bo torej asistent pri analizi na tablo napisal
%
\begin{equation*}
    1, 2, 4, 8, \ldots
\end{equation*}
%a
imate tri možnosti:
%
\begin{enumerate}
  \item Ste zmedeni.
  \item Uganete, da ima v mislih potence števila 2.
  \item Vprašate, ali je $n$-ti člen število regij, na katerega lahko razdelimo prostor z $(n-1)$ ravninami?
\end{enumerate}
%
Sami se odločite, kakšen odnos želite vzpostaviti z asistentom.

\subsection{Prazna množica $\emptyset$}

Verjetno ni treba izgubljati besed o prazni množici. To je množica, ki nima nobenega elementa. Zapišemo jo $\emptyset$ ali $\{\}$.

\begin{naloga}
  Ali je kakšna razlika med $\{\}$ in $\{\emptyset\}$?
\end{naloga}


\subsection{Standardni enojec $\one$}

Množici, ki ima natanko en element, pravimo \textbf{enojec}.

Ali znamo pojasniti, kaj pomeni, da ima množica natanko en element, ne da bi pri tem omenili število $1$ ali katerokoli drugo število? Takole: množica $A$ ima natanko en element če velja:
%
\begin{enumerate}
\item obstaja $x \in A$ in
\item če je $x \in A$ in $y \in A$, potem $x = y$.
\end{enumerate}

\begin{naloga}
  Kako bi opredelili ">množica ima natanko dva elementa"< brez uporabe števil?
\end{naloga}

Pogosto bomo potrebovali kak enojec (že na naslednjih predavanjih). Seveda se ni težko domisliti enojca, na primer $\{42\}$ ali $\{\sin\}$. Da pa ne bomo vedno znova izgubljali časa z izbiro enojca, se dogovorimo da je \textbf{standardni enojec $\one$} množica $\{\unit\}$. To je zelo čudno, ker smo označili množico s številko\footnote{Ali ločite med ">števka"<, ">številka"< in ">število<"?} $1$ in ker je element standardnega enojca $\unit$, česar še nikoli nismo videli.

Glede oznake $\one$ povejmo, da imamo kot matematiki \emph{načelno svobodo} pri izbiri zapisa, a je smiselno in vljudno, da se ne zafrkavamo. Ali se torej predavatelj zafrkava, ko standardni enojec označi s številko $\one$? Ne, saj gresta ">ena"< in ">enojec"< lepo skupaj, poleg tega pa bomo na naslednjih predavanjih spoznali tudi matematične razloge za tak zapis.

Glede oznake $\unit$ se bo kmalu izkazalo, da je zapis smiseln, ker je $()$ pravzaprav ">urejena ničterica"<.

\subsection{Številske in ostale množice}

Seveda si bomo privoščili uporabo raznih množic, ki jih že poznate, kot so na primer številske množice $\NN$, $\ZZ$, $\QQ$, $\RR$ itd. Opozorimo pa na naslednji dilemo:
v osnovni in srednji šoli z $\NN$ označimo množico celih števil, ki so večja ali enaka $1$, vendar pa pogosto v matematiki, še posebej pa v logiki, tudi število $0$ obravnavamo kot naravno število. V takih primerih $\NN$ izenačuje množico celih števil, ki so večja ali enaka $0$.

Kaj je torej prav $\NN = \{0, 1, 2, \ldots\}$ ali $\NN = \{1, 2, 3, \ldots\}$? To je napačno vprašanje! Lahko vprašamo le ">kako se bomo dogovorili?"<. Pri tem predmetu se
dogovorimo, da je $0$ naravno število, ker vadimo ">matematično svobodo"<, imamo dobre matematične razloge, da $0$ uvrstimo med naravna števila, in ker je predavatelj tako zapovedal.

\begin{naloga}
  Zberite pogum in predavatelja vprašate, kakšni so ti dobri matematični razlogi, zaradi katerih je zapovedal, da je $0$ naravno število, bo sledila filozofska razprava, ki vam bo pokvarila odmor.
\end{naloga}

\section{Konstrukcije množic}

Ena od osnovnih matematičnih aktivnosti so \textbf{konstrukcije}. Poznamo na primer geometrijske konstrukcije z ravnilom in šestilom. Ko računamo rešite enačbe, bi lahko rekli, da konstruiramo število, ki zadošča enačbi. Ko pišemo dokaz, konstruiramo objekt, iz katerega je razvidna resničnost neke izjave. Tudi računalniški programi so le matematični konstrukti.

Spoznajmo nekatere osnovne konstrukcije množic, se pravi, načine, kako iz množic naredimo nove množice.

\subsection{Zmnožek ali kartezični produkt}

\textbf{Urejeni par} $\pair{x,y}$ je matematični objekt, ki da dobimo tako, da združimo dva matematična objekta~$x$ in~$y$. V srednji šoli ste večinoma pisali urejene pare števil (ki ste jih imenovali ">koordinate"<). V urejenem paru je vrstni red \emph{pomemben}: urejena para $(1, 3)$ in $(3, 1)$ \emph{nista} enaka. (Množici $\{1, 3\}$ in $\{3, 1\}$ sta enaki.)

Urejeni par $\pair{x, y}$ ima \textbf{prvo komponento~$x$} in \textbf{drugo komponento~$y$}. Če imamo neki urejeni par~$u$, njegovi komponenti pišemo tudi $\fst(u)$ in $\snd(u)$. Velja torej:
%
\begin{equation*}
    \fst(x,y) = x
    \iinn
    \snd(x,y) = y.  
\end{equation*}
%
Simboloma $\mathsf{pr}_1$ in $\mathsf{pr}_2$ pravimo \textbf{prva} in \textbf{druga projekcija}. Običajne oznake za projekciji so tudi $\pi_1$ in $\pi_2$, v
programiranju $\mathtt{fst}$ in $\mathtt{snd}$, lahko pa tudi $\pi_0$ in $\pi_1$.

Spoznajmo sedaj \textbf{zmnožek} ali \textbf{kartezični produkt} množic $A$ in $B$. Opis nove konstrukcijo množic mora navesti zapis za konstruirano množico, katere elemente ima, in kdaj sta elementa konstruirane množice enaka:
%
\begin{enumerate}
\item Zmnožek množic $A$ in $B$ zapišemo $A \times B$.
\item Elementi množice $A \times B$ so urejeni pari $\pair{x, y}$, pri čemer je $x \in A$ in $y \in B$.
\item Enakost elementov (princip ekstenzionalnosti za pare): $u \in A \times B$ in $v \in A \times B$ sta enaka, če velja $\fst(u) = \fst(v)$ in $\snd(u) = \snd(v)$.
\end{enumerate}

\begin{primer}
Zmnožek množic $\{1,2,3\}$ in $\{\Box, \diamond\}$ je
%
\begin{equation*}
    \{1, 2, 3\} \times {\Box, \diamond} =
    \{\pair{1, \Box},
      \pair{2, \Box},
      \pair{3, \Box},
      \pair{1, \diamond},
      \pair{2, \diamond},
      \pair{3, \diamond}
     \}.
\end{equation*}
%
Iz principa ekstenzionalnosti za pare sledi, da je vrstni red v urejenem paru pomemben, saj $\pair{1, 3} \neq \pair{3, 1}$, ker $\fst(1,3) = 1 \neq 3 = \fst(3,1)$.
\end{primer}


\subsubsection{Zmnožek več množic}

Tvorimo lahko tudi zmnožek več množic. Na primer, zmnožek množic $A$, $B$ in $C$ je množica $A \times B \times C$, katerih elementi so \textbf{urejene trojke} $\pair{x, y, z}$, kjer je $x \in A$, $y \in B$ in $z \in C$. V tem primeru imamo tri projekcije $\mathsf{pr}_1$, $\mathsf{pr}_2$ in $\mathsf{pr}_3$. Podobno lahko tvorimo zmnožek štirih, petih, šestih, \dots množic.

\begin{naloga}
  Ali lahko tvorimo zmnožek ene množice? Kaj pa zmnožek nič množic?
\end{naloga}


\subsection{Vsota ali koprodukt}

Naslednja osnovna konstrukcija je \textbf{vsota} ali \textbf{koprodukt} množic $A$ in $B$:
%
\begin{enumerate}
\item vsoto množic $A$ in $B$ označimo z $A + B$,
\item elementi množice $A + B$ so $\inl{x}$ za $x \in A$ in $\inr{y}$ za $y in B$,
\item elementa $u \in A + B$ in $v \in A + B$ sta enaka, kadar velja
  %
  \begin{enumerate}
    \item bodisi za neki $a \in A$ velja $u = \inl{a} = v$,
    \item bodisi za neki $b \in B$ velja $u = \inr{b} = v$.
  \end{enumerate}
\end{enumerate}

\begin{primer}
Primeri vsote množic:
%
\begin{enumerate}

\item $\{1, 2, 3\} + \{\square, \diamond\} = \{\inl{1}, \inl{2}, \inl{3}, \inr{\Box}, \inr{\diamond}\}$

\item $\{a, b\} + \{b, c\} = \{\inl{a}, \inl{b}, \inr{b}, \inr{c}\}$

\item Vsota \emph{ni} unija! Po eni strani je
      $\{3, 5\} \cup \{3, 5\} = \{3, 5\}$ in po drugi
      $\{3, 5\} + \{3, 5\} = \{\inl{3}, \inl{5}, \inr{3}, \inr{5}\}$.
\end{enumerate}
\end{primer}

Vsoti pravimo tudi ">disjunktna unija"<, a se bomo temu izrazu izogibali, ker obravnavamo vsoto kot osnovno operacijo in ne kot poseben primer unije.

Oznakama $\mathsf{in}_1$ in $\mathsf{in}_2$ pravimo \textbf{prva in druga injekcija}. Uporabljajo se tudi oznake $\iota_1$ in $\iota_2$, v funkcijskem
programiranju $\mathtt{inl}$ in $\mathtt{inr}$, pa tudi $\iota_0$ in $\iota_1$. Pravzaprav ni pomembno, kakšne oznake uporabimo, poskrbeti moralo
le, da sta to različna simbola, s katerima razločimo elemente prvega in drugega sumanda.

Tvorimo lahko vsoto več množic, na primer $A + B + C$. V tem primeru imamo tri injekcije $\mathsf{in}_1$, $\mathsf{in}_2$ in $\mathsf{in}_3$.

\section{Preslikave ali funkcije}

Poleg množic so preslikave še en osnovni matematični pojem, ki mu bomo posvetili veliko pozornosti. \textbf{Preslikava} ali \textbf{funkcija} sestoji iz treh sestavin:
%
\begin{itemize}
\item množice, ki ji pravimo \textbf{domena},
\item množice, ki ji pravimo \textbf{kodomena},
\item \textbf{prirejanja}, ki vsakemu elementu domene priredi natanko en element kodomene.
\end{itemize}
%
Če je $f$ funkcija z domeno $A$ in kodomeno $B$, to zapišemo
%
\begin{equation*}
  f : A \to B
\end{equation*}
%
ali
%
\begin{equation*}
  \xymatrix{
    {A} \ar[rr]^{f} & & {B}
  }
\end{equation*}
%
Rišemo lahko tudi diagrame, ki prikazujejo več funkcij hkrati, na primer
%
\begin{equation*}
  \xymatrix{
    {A} \ar[r]^{f} &
    {B} \ar[r]^{g} &
    {C} \ar[d]^{h} \\
    & & {D}
  }
\end{equation*}
%
Ta diagram prikazuje tri preslikave: $f : A \to B$, $g : B \to C$ in $h : C \to D$.

V srednji šoli ste spoznavali posamične zvrsti funkcij, na primer linearne funkcije, trigonometrične funkcije, eksponentno funkcijo itd. Le-te so običajno slikale števila v števila, bile so \emph{številske funkcije}. Mi se bomo ukvarjali s preslikavami na splošno, se pravi s poljubnimi preslikavami med poljubnimi množicami.

\subsubsection{Princip ekstenzionalnosti preslikav}

\textbf{Princip ekstenzionalnosti za preslikave}, pove, kdaj sta dve funkciji enaki, namreč takrat, ko prirejata enake vrednosti:  če za preslikavi $f : A \to B$ in $g : C \to D$ velja $A = C$, $B = D$ in $f(x) = g(x)$ za vse $x \in A$, potem velja $f = g$.

Kasneje bomo videli, da princip ekstenzionalnosti za preslikave sledi iz principa ekstenzionalnosti za množice.

\subsection{Prirejanje in funkcijski predpisi}

Dejstvo, da mora prirejanje vsakemu elementu domene prirediti ">natanko en"< element kodomene, lahko izrazimo tako, da se izognemo uporabi števila ">ena"< ali kateregakoli števila. Poglejmo kako.

Prirejanje z domeno $A$ in kodomeno $B$ mora biti:
%
\begin{enumerate}
\item \textbf{celovito:} vsakemu $x \in A$ je prirejen vsaj en $y \in B$ (priredimo vsaj en element),
\item \textbf{enolično:} če sta $x \in A$ prirejena $y \in B$ in $z \in B$, potem velja $y = z$ (priredimo največ en element).
\end{enumerate}
%
Res, celovitost zagotavlja, da vsakemu elementu domene priredimo \emph{vsaj en} element kodomene, enoličnost pa zagotavlja, da priredimo \emph{kvečjemu enega}.

\begin{opomba}
  Pozor, celovitost \emph{ni} surjektivnost in enoličnost \emph{ni} injektivnost!
\end{opomba}

Kako pravzaprav podamo prirejanje? Kaj to pravzaprav je? Čez kak mesec bomo znali odgovoriti na to vprašanje natančno, zaenkrat pa le povejmo, da je prirejanje kakršnakoli metoda, tabela, postopek, prikaz, ali konstrukcija, ki zagotavlja celovitost in enoličnost prirejanja elementov kodomene elementom domene.

Običajni način za podajanje prirejanja je \textbf{funkcijski predpis}, ki ga pišemo
%
\begin{equation*}
  x \mapsto \cdots
\end{equation*}
%
Pri čemer za $\cdots$ na desni postavimo neki smiseln izraz, ki določa enolično vrednost za vsak $x$ iz domene. Spremenljivki~$x$ na levi pravimo \textbf{parameter}, izrazu $\cdots$ na desni pa \textbf{prirejena vrednost}.

\begin{primer}
  Primeri prirejanj:
  %
  \begin{itemize}
  \item prirejanje ">prištej 7 in kvadriraj"< zapišemo s funkcijskim predpisom  $x \mapsto (x + 7)^2$,
  \item prirejanje ">kvadriraj in prištej 7"< zapišemo s funkcijskim predpisom  $x \mapsto x^2 + 7$,
  \item prirejanje ">prištej kvadrat 7"< zapišemo s funkcijskim predpisom  $x \mapsto x + 7^2$.
  \end{itemize}
\end{primer}

\begin{opomba}
  Pozor: če podamo \emph{samo} funkcijski predpis brez domene in kodomene, še nismo podali preslikave! Preslikava sestoji iz \emph{treh} delov: domena, kodomena in prirejanje.
  Torej zgornji trije primeri \emph{ne} podajajo preslikav, ker nismo podali domen in
  kodomen.
\end{opomba}

Domeno, kodomeno in funkcijski predpis lahko zapišemo na različne načine:
%
\begin{align*}
  f &: \ZZ \to \NN \\
  f &: x \mapsto x^2 + 7
\end{align*}
%
ali
%
\begin{align*}
    f &: \ZZ \to \NN \\
    f(x) &\defeq x^2 + 7
\end{align*}
%
ali
%
\begin{align*}
    f &: \ZZ \to \NN \\
    f &= (x \mapsto x^2 + 7)
\end{align*}
%
Simbol $x$ je \textbf{vezana spremenljivka}, če jo preimenujemo, se predpis ne spremeni. Naslednji funkcijski predpisi so \emph{enaki}:
%
\begin{align*}
  x &\mapsto x^2 + 7 \\
  y &\mapsto y^2 + 7 \\
  \textit{banana} &\mapsto \textit{banana}^2 + 7
\end{align*}
%
Funkcijski predpis $x \mapsto 5 + x \cdot x + 2$ pa \emph{ni enak} zgornjim trem, čeprav vrača enake vrednosti in torej določa \emph{enako} funkcijo.

\subsubsection{Aplikacija ali uporaba}

Preslikavo $f : A \to B$ \textbf{uporabimo} ali \textbf{apliciramo} na elementu $a \in A$, da dobimo \textbf{vrednost} $f(a) \in B$. V izrazu $f(a)$ se imenuje $a$ \textbf{argument}
%
Kadar je $f$ podana s predpisom, izračunamo vrednost $f(a)$ tako, da $a$ vstavimo v predpis (vezano spremenljivo zamenjamo z argumentom~$A$).
%
O zamenjavi vezane spremenljivke z argumentom bomo več povedali v razdelku~\ref{sec:substitucija} o substituciji.

\begin{primer}
  Če je $f : \NN \to \NN$ podana s predpisom $f = (x \mapsto x^3 + 4)$, tedaj je $f(5)$ enako $5^3 + 4$. Lahko bi celo pisali
  %
  \begin{equation*}
    (x \mapsto x^3 + 4)(5) = 5^3 + 4.
  \end{equation*}
\end{primer}


\subsection{Eksponentna množica}

Tretja konstrukcija množic, ki jo bomo spoznali v uvodnem poglavju, je \textbf{eksponent} ali \textbf{eksponentna množica}:
%
\begin{enumerate}
\item eksponent množic $A$ in $B$ označimo $B^A$, in preberemo ">$B$ na $A$"<,
\item elementi $B^A$ so preslikave z domeno $A$ in kodomeno $B$,
\item preslikavi $f : A \to B$ in $g : A \to B$ sta enaki, če imate enake vrednosti: za
  vse $x \in A$ velja $f(x) = g(x)$, potem je $f = g$.
\end{enumerate}
%
Eksponent $B^A$ pišemo tudi $A \to B$. To pomeni, da bi lahko namesto $f : A \to B$ pisali tudi $f \in B^A$ ali celo $f \in A \to B$, vendar je ta zadnji zapis neobičajen.

\begin{primer}
Eksponent $\{1, 2\}^{\{a, b\}}$ ima štiri elemente:
%
\begin{equation*}
  \{1, 2\}^{\{a, b\}} =
  \{
     (a \mapsto 1, b \mapsto 1),
     (a \mapsto 1, b \mapsto 2),
     (a \mapsto 2, b \mapsto 1),
     (a \mapsto 2, b \mapsto 2)
  \}.
\end{equation*}
\end{primer}



\chapter{Aritmetika množic}

Nadaljujmo s študijem splošnih preslikav.

\section{Preslikave in prazna množica}

Naj bo $A$ množica. Kaj vemo povedati o preslikavah $\emptyset \to A$?

Čez nekaj tednov bomo spoznali naslednji dejstvi, ki ju zaenkrat vzemimo v zakup:

\begin{itemize}
\item Vsaka izjava oblike ">za vsak element $\emptyset$ ..."< je resnična.
\item Vsaka izjava oblike ">obstaja element $\emptyset$ ..."< je neresnična.
\end{itemize}

Primeri resničnih izjav:
%
\begin{enumerate}
\item ">Vsak element prazne množice je sodo število"<
\item ">Vsak element prazne množice je liho število"<
\item ">Vsak element prazne množice je hkrati sodo in liho število"<
\item ">Vsak element prazne množice \dots"<
\end{enumerate}

Primeri neresničnih izjav:
%
\begin{enumerate}
\item ">Obstaja element prazne množice, ki je sodo število"<
\item ">Obstaja element prazne množice, ki je enak sam sebi"<
\item ">Obstaja element prazne množice, ki \dots"<
\end{enumerate}
%
Denimo, da imamo preslikave $f :\emptyset \to A$ in $g : \emptyset \to A$. Tedaj sta enaki, saj velja: ">za vsak element $x \in \emptyset$ velja $f(x) = g(x)$".
Torej imamo kvečjemu eno preslikavo $\emptyset \to A$. Ali pa imamo sploh kakšno? Da, pravimo ji \textbf{prazna preslikava}, ker je njeno prirejanje ">prazno"<, oziroma ga sploh ni treba podati (saj ni nobenega elementa domene $\emptyset$, ki bi mu morali prirediti kak element kodomene $A$).

Kaj pa preslikave $A \to \emptyset$?
%
Če je $A = \emptyset$, potem imamo natanko eno preslikavo $A \to \emptyset$, namreč prazno preslikavo, $\emptyset^A = \{ \textrm{prazna-preslikava} \}$.
%
Če $A$ vsebuje kak element, potem ni nobene preslikave $A \to \emptyset$, se pravi  $\emptyset^A = \emptyset$.

Zakaj ni preslikave $A \to \emptyset$, kadar $A$ vsebuje kak element? Denimo da je $x \in A$. Če bi bila kaka preslikava $f : A \to \emptyset$, bi
veljalo $f(x) \in \emptyset$, kar pa ni res. Torej take preslikave ni.

\begin{naloga}
  Koliko je preslikav $1 \to A$ in koliko je preslikav $A \to 1$?
  Ali je odgovor odvisen od~$A$?
\end{naloga}

\section{Identiteta in kompozicija}

Spoznajmo nekaj osnovnih preslikav in operacij na preslikavah.

\textbf{Identiteta} na $A$ je preslikava $id[A] : A \to A$, podana s predpisom $x \mapsto x$.

\textbf{Kompozitum} preslikav
%
\begin{equation*}
  \xymatrix{
    {A} \ar[r]^f & {B} \ar[r]^g & {C}
  }
\end{equation*}
%
je preslikava $g \circ f : A \to C$, podana s predpisom $x \mapsto g(f(x))$.

\textbf{Kompozitum je asociativen:} za preslikave
%
\begin{equation*}
  \xymatrix{
    {A} \ar[r]^f & {B} \ar[r]^g & {C} \ar[r]^h & {D}
  }
\end{equation*}
%
velja $(h \circ g) \circ f = h \circ (g \circ f)$. Res, za vsak $x \in A$ velja
%
\begin{align*}
  ((h \circ g) \circ f)(x)
  &= (h \circ g) (f x) \\
  &=  h (g (f (x)) \\
  &= h ((g \circ f)(x)) \\
  &= (h \circ (g \circ f))(x),
\end{align*}
%
torej želena\footnote{Piše se ">želen"< in ne ">željen"<, ker je ">želen"< deležnik na
  ">n"< glagola ">želeti"<. V slovenščini ni glagola ">željeti"<. Hitro boste spoznali, da na FMF profesorji za matematiko radi popravljajo slovnico.} enačba sledi iz principa ekstenzionalnosti za funkcije.

\textbf{Identiteta je nevtralni element za kompozitum:} za vsako preslikavo $f : A \to B$ velja
%
\begin{equation*}
  \id[B] \circ f = f
  \iinn
  f \circ id[A] = f.
\end{equation*}
%
To preverimo z uporabo ekstenzionalnosti za funkcije: za vsak $x \in A$ velja
%
\begin{equation*}
    (\id[B] \circ f)(x) = \id[B] (f(x)) = f(x)
\end{equation*}
%
in
\begin{equation*}
  (f \circ id[A])(x) = f (id[A](x)) = f(x).
\end{equation*}
%
Kompizicija $\circ$ in identiteta $id$ se torej obnašata podobno kot nekatere operacije v algebri, na primer $+$ in $0$ ter $×$ in $1$.

\begin{naloga}
  Seštevanje je komutativno, velja $a + b = b + a$. Ali je kompozicija preslikav tudi komutativna?
\end{naloga}

\section{Funkcijski predpisi na zmnožku in vsoti}

Pogosto želimo definirati preslikavo, katere kodomena je zmnožek množic, denimo $f : A \times B \to C$. V takem primeru lahko
podamo funkcijski predpis takole:
%
\begin{equation*}
  (x, y) \mapsto \cdots
\end{equation*}
%
pri čemer je $x \in A$ in $y \in B$. To je dovoljeno, ker je vsak element domene $A \times B$ urejeni par $(x, y)$ za natanko določena $x \in A$ in $y \in B$.

\begin{primer}

Preslikavo
%
\begin{align*}
  \RR \times \RR  &\to  \RR \\
  u &\mapsto  \fst{u}² + 3 \cdot \snd{u}
\end{align*}
%
lahko bolj čitljivo podamo s predpisom
%
\begin{align*}
  \RR \times \RR  &\to  \RR \\
  (x, y) &\mapsto  x^2 + 3 \cdot y
\end{align*}
\end{primer}

\begin{primer}
  Seveda lahko podobno podajamo tudi preslikave na zmnožkih večih preslikav, denimo
  %
  \begin{align*}
  A \times B \times C &\to A \times A \\
  (a, b, c) &\mapsto (a, a)
  \end{align*}
  in
  \begin{align*}
  X \times (Y \times Z) &\to (X \times Y) \times Z \\
  (x, (y, z)) &\mapsto ((x, y), z).
  \end{align*}
\end{primer}

Kako pa zapišemo funkcijski predpis funkcije z domeno $A + B$? V tem primeru je vsak element domene bodisi oblike
$\inl{x}$ za enolično določeni $x \in A$, bodisi oblike $\inr{y}$ za enolično določeni $y \in B$, zato funkcijski predpis podamo v dveh vrsticah:
%
\begin{align*}
    A + B &\to C \\
    \inl{x} &\mapsto \cdots \\
    \inr{y} &\mapsto \cdots
\end{align*}

\begin{primer}
  Primer take preslikave je
  %
  \begin{align*}
    \RR + \ZZ &\to \RR \\
    \inl{x} &\mapsto x \\
    \inr{y} &\mapsto y + 3
  \end{align*}
  %
  Seveda lahko podobno podajamo tudi preslikave na vsotah večih preslikav:
  %
  \begin{align*}
    A + B + C &\to \{u, v\} \\
    \inl{x} &\mapsto u \\
    \inr{y} &\mapsto u \\
    \mathsf{in}_3(z) &\mapsto v
  \end{align*}
  in
  \begin{align*}
    A + (B + C) &\to \{u, v\} \\
    \inl{x} &\mapsto u \\
    \inr{\inl{y}} &\mapsto u \\
    \inr{\inr{y}} &\mapsto v
  \end{align*}
\end{primer}

Zapisa za zmnožek in vsoto lahko tudi kombiniramo:
%
\begin{align*}
  (A \times B \times C) + (D \times E) &\to \{0, 1, 2\} \\
  \inl{(a, b, c)} &\mapsto 1 \\
  \inr{(d, e)} &\mapsto 2
\end{align*}
%
in
\begin{align*}
  (A + B) \times C &\to \{0, 1, 2\} \\
  (\inl{a}, c) &\mapsto 0 \\
  (\inr{b}, c) &\mapsto 1
\end{align*}
%
Izraz na levi strani $\mapsto$ sestoji iz vezanih spremelnjivk in operacij, s katerimi gradimo elemente množic (urejeni par, kanonična injekcija). Imenuje se tudi \textbf{vzorec}. Predpis je podan pravilno, če so vzorci napisani tako, da vsak element domene ustreza natanko enemu vzorcu.
%
S tem zagotovimo, da predpis obravnava vse možne primere (celovitost) in da ne obravnava nobenega primera večkrat (enoličnost).


\subsection{Funkcijski predpis, podan po kosih}

Omenimo še en pogost način podajanja funkcij, namreč s predpisom po kosih.

\begin{primer}
  Preslikava ">absolutno"< je definirana po kosih za negativna in nenegativna števila:
  %
  \begin{align*}
    \RR &\to \RR \\
    x &\mapsto
    \begin{cases}
      -x & \text{če $x < 0$,}\\
       x & \text{če $x \geq 0$.}
    \end{cases}
  \end{align*}
\end{primer}

\begin{primer}
  Preslikava ">predznak"< je definirana po kosih:
  %
  \begin{align*}
    \RR &\to \RR \\
    x &\mapsto
      \begin{cases}
        -1 & \text{če $x < 0$,}\\
        0 & \text{če $x = 0$,}\\
        1 & \text{če $x \geq 0$.}
      \end{cases}
  \end{align*}
\end{primer}


Pri takem zapisu moramo paziti, da kosi skupaj pokrivajo domeno (vsi elementi domene so obravnavani) in da se kosi ne prekrivajo (vsak element domene je obravnavan natanko enkrat). Pravzaprav se smejo kosi prekrivati, a moramo v tem primeru preveriti, da se na skupnih delih skladajo, se pravi, da vsi kosi podajajo enake vrednosti na preseku.

\begin{primer}
  Preslikavo ">absolutno"< bi lahko podali takole:
  %
  \begin{align*}
    \RR &\to \RR \\
    x &\mapsto
    \begin{cases}
      -x & \text{če $x < 0$,}\\
       x & \text{če $x \geq 0$.}
    \end{cases}
  \end{align*}
  %
  Kosa se prekrivata pri $x = 0$, vendar to ni težava, ker je $-0 = 0$.
\end{primer}


\section{Nekatere preslikave na eksponentnih množicah}

Poglejmo si nekaj preslikav, ki slikajo iz in v eksponente množice.

\textbf{Evalvacija} ali \textbf{aplikacija} ali \textbf{uporaba} je preslikava, ki sprejme preslikavo in argument, ter preslikavo uporabi na argumentu:
%
\begin{align*}
  \mathsf{ev} &: B^A \times A \to B \\
  \mathsf{ev} &: (f, x) \mapsto f(x)
\end{align*}
%
Pravimo, da je $\mathsf{ev}$ \textbf{preslikava višjega reda}, ker slika preslikave v vrednosti.

\begin{primer}
  Določeni integral $\int_0^1$ je funkcija višjega reda, ker
  sprejme funkcijo $[0,1] \to \RR$ in vrne realno število. Je torej preslikava
  $\RR^{[0,1]} \to \RR$, če se pretvarjamo, da lahko integriramo vsako funkcijo.
  Bolj pravilno bi bilo reči, da je $\int_0^1$ preslikava iz množice \emph{integrabilnih funkcij $[0,1] \to \RR$} v realna števila.
\end{primer}

Kompozitum preslikav je tudi preslikava višjega reda:
%
\begin{align*}
    {\circ} &: C^B \times B^A \to C^A \\
    {\circ} &: (g, f) \mapsto (x \mapsto g(f(x)))
\end{align*}
%
Tretja splošna preslikava višjega reda je ">currying"< (ali zna kdo to prevesi v slovenščino?):
%
\begin{align*}
  A^{(B \times C)} &\to (A^B)^C \\
  f &\mapsto (c \mapsto (b \mapsto f(b, c))).
\end{align*}
%
Pravzaprav je to izomorfizem, katerega inverz je ">uncurrying"<:
%
\begin{align*}
  (A^B)^C &\to A^{B \times C} \\
  g       &\mapsto ((b, c) \mapsto f(b)(c))
\end{align*}


\section{Izomorfizmi in artimetika množic}

\subsection{Inverz}

\begin{definicija}
  Preslikava $f : A \to B$ je \textbf{inverz} preslikave $g : B \to A$, če velja $f \circ g = \id[B]$ in $g \circ f = \id[A]$.
\end{definicija}

\begin{naloga}
  Utemelji: če je $f$ inverz $g$, potem je $g$ inverz $f$.
\end{naloga}

\begin{primer}
  Kub in kubični koren sta inverza
  %
  \begin{align*}
    \RR &\to \RR    &     \RR &\to \RR \\
    x &\mapsto x^3  &     y &\mapsto \sqrt[3]{y}
  \end{align*}
\end{primer}

\begin{naloga}
  Naj bo~$S$ množica nenegativnih realnih števil, se pravi, $\RR_{\geq 0} = \{x \in \RR \mid x \geq 0\}$. Ali sta kvadriranje in kvadratni koren inverza?
  %
  \begin{align*}
    \RR &\to \RR_{\geq 0}    &     \RR_{\geq 0} &\to \RR \\
    x &\mapsto x^2           &     y &\mapsto \sqrt[2]{y}
  \end{align*}
  %
\end{naloga}

\begin{izjava}
  Če sta $f : A \to B$ in $g : A \to B$ oba inverza preslikave $h : B \to A$, potem je $f = g$.
\end{izjava}

\begin{dokaz}
  Denimo, da sta $f : A \to B$ in $g : A \to B$ inverza preslikave $h : B \to A$. Tedaj velja
  %
  \begin{equation*}
    f =
    f \circ \id[A] =
    f \circ (h \circ g) =
    (f \circ h) \circ g =
    \id[B] \circ g =
    g.
  \end{equation*}
\end{dokaz}

Ali znate utemeljiti vsakega od zgornjih korakov?

\begin{definicija}
  Preslikava, ki ima inverz, se imenuje \textbf{izomorfizem}.
\end{definicija}

Če je $f : A \to B$ izomorfizem, potem ima natanko en inverz $B \to A$, ki ga označimo $\inv{f}$.

\begin{primer}
  Identiteta $id[A] : A \to A$ je izomorfizem, saj je sama sebi inverz.
  Torej $\inv{id[A]} = \id[A]$.
\end{primer}

\begin{primer}
  Eksponentna preslikava $\exp : \RR \to \RR_{> 0}$, $exp : x \mapsto e^x$ je
  izomorfizem, njen inverz je naravni logaritem $\ln : \RR_{> 0} \to \RR$, torej $\inv{\exp} = \ln$.
\end{primer}

\begin{primer}
  Eksponentna preslikava $\exp : \RR \to \RR$ \emph{ni} izomorfizem.
\end{primer}

\begin{izjava}
  Če sta $f : A \to B$ in $g : B \to C$ izomorfizma, potem je tudi $g \circ f : A \to C$ izomorfizem. Velja torej $\inv{(g \circ f)} = \inv{f} \circ \inv{g}$.
\end{izjava}

\begin{dokaz}
  Dokazati moramo, da ima $g \circ f$ inverz. Trdimo, da je $\inv{f} \circ \inv{g} : C \to A$ inverz preslikave $g \circ f$. Računajmo:
  %
  \begin{align*}
    (g \circ f) \circ (\inv{f} \circ \inv{g}) \\
     &= ((g \circ f) \circ \inv{f}) \circ \inv{g} \\
     &= (g \circ (f \circ \inv{f})) \circ \inv{g} \\
     &= (g \circ \id[B]) \circ \inv{g} \\
     &= g \circ \inv{g} \\
     &= \id[C].
  \end{align*}
  %
  Doma sami preverite, da velja tudi $(\inv{f} \circ \inv{g}) \circ (g \circ f) = \id[A]$.
\end{dokaz}

\subsection{Izomorfne množice}


\begin{definicija}
  Množici $A$ in $B$ sta \textbf{izomorfni}, če obstaja izomorfizem $f : A \to B$. Kadar sta $A$ in $B$ izomorfni, to zapišemo $A \iso B$.
\end{definicija}

\begin{izjava}
  \textbf{Izjava:} Za vse množice $A$, $B$ in $C$ velja:
  %
  \begin{enumerate}
    \item $A \iso A$,
    \item če $A \iso B$, potem $B \iso A$,
    \item če $A \iso B$ in $B \iso C$, potem $A \iso C$.
  \end{enumerate}
\end{izjava}

\begin{dokaz}
  %
  \begin{enumerate}
     \item $id[A]$ je izomorfizem $A \to A$,
     \item če je $f : A \to B$ izomorfizem, potem je tudi $\inv{f} : B \to A$ izomorfizem,
     \item če je $f : A \to B$ izomorfizem in $g : B \to C$ izomorfizem, potem je $g \circ f : A \to C$ izomorfizem.
  \end{enumerate}
\end{dokaz}

\begin{primer}
  $A \times B \iso B \times A$, ker imamo izomorfizem in njegov inverz
  %
  \begin{align*}
    A \times B  &\to  B \times A      &    B \times A  &\to  A \times B \\
    (x, y) &\mapsto  (y, x)           &    (b, a) &\mapsto  (a, b)
  \end{align*}
\end{primer}

\subsection{Aritmetika množic}

Veljajo naslednji izomorfizmi, ki nas seveda spomnijo na zakone aritmetike, ki
veljajo za števila. Ali gre tu za kako globjo povezavo?
%
\begin{enumerate}
\item Vsota in $\emptyset$:
  \begin{enumerate}
    \item $A + \emptyset \iso A$
    \item $A + B \iso B + A$
    \item $(A + B) + C \iso A + (B + C)$
  \end{enumerate}

\item Zmnožek in $\one$:
  \begin{enumerate}
    \item $A \times 1 \iso A$
    \item $A \times B \iso B \times A$
    \item $(A \times B) \times C \iso A \times (B \times C)$
  \end{enumerate}

\item Distributivnost:
  \begin{enumerate}
    \item $A \times (B + C) \iso (A \times B) + (A \times C)$
    \item $A \times \emptyset \iso \emptyset$
  \end{enumerate}

\item Eksponenti:
  \begin{enumerate}
    \item $A^1 \iso A$
    \item $1^A \iso 1$
    \item $A^{\emptyset }\iso 1$
    \item $\emptyset^A \iso \emptyset , če  A \neq \emptyset$
    \item $A^{(B \times C)} \iso (A^B)^C$
    \item $A^{(B + C)} \iso A^B \times A^C$
    \item $(A \times B)^C \iso A^C \times B^C$
  \end{enumerate}
\end{enumerate}

\begin{naloga}
  Zapišite vseh 15 izomorfizmov, ki potrjujejo pravilnost zgornjega seznama.
\end{naloga}

\chapter{Simbolni zapis}

V matematiki uporabljamo **simbolni zapis** – matematične objekte, konstrukcije in dokaze opišemo s pomočjo izrazov kot
so `3 + 4`, `x ↦ x² + 3`, `∀ x ∈ ℝ . x² + x + 1 ≥ 1/4` itd. Matematično besedilo je mešanica naravnega jezika
(slovenščine) in simbolnega zapisa. Načeloma bi lahko pisali matematiko *samo* s simbolnim zapisom (kar dejansko
počnemo, kadar matematiko *formaliziramo* z računalnikom, a o tem kdaj drugič), a bi bilo to ljudem preveč nerazumljivo.

Danes bomo spoznali pravila simbolnega zapisa in se učili razumeti, brati in pisati logične formule (matematične izjave,
izražene s simbolnim zapisom).

\section{Izrazi}

**(Simbolni) izraz** je zaporedje znakov, ki predstavlja neki matematični pojem, na primer

    3 + 5
    S ∩ (T ∪ V)
    2 x y ≤ x² + y²

Izraz je *pravilno formiran* ali *sintaktično pravilen*, če ustreza pravilom, ki določajo kako postavljamo oklepaje,
vejice, pike, kako uporabljamo razne posebne simbole (`+`, `∨`, `∫`) itd. Na primer, izraz `3 + ) x · 4` ni sintaktično
pravilen, ker ima narobe postavljen zaklepaj.

Natančna sintaktična pravila za pisanje matematičnih izrazov so precej zapletena, ker je matematični zapis raznovrsten
in se je razvijal skozi zgodovino. Na srečo skoraj vsa pravila že poznate ("vsak oklepaj mora imeti ustrezni zaklepaj",
"piše se `a + b` in ne `a b +`, ...). Tu se ne bomo ukvarjali s podajanjem vseh pravil – to je delo za računalničarje,
ki želijo taka pravila naprogramirati. Kljub temu pa velja omeniti nekatere pojme.


\subsection{Prefiksne, postfiksne in infiksne operacije}

V simbolnem zapisu uporabljamo *operacije*, ki jih pišemo pred, za ali med argumente:

* **prefiksne operacije** so take, ki jih pišemo *pred* argument:
    * `-x` za nasprotno vrednost `x`
    * `¬P` za negacijo izjave `P`
* **infiksne operacije** so take, ki jih pišemo *med* argumenta:
    * aritmetične operacije `x + y`, `x - y`, `x · y` itn.
    * logični vezniki `P ∧ Q`, `P ∨ Q`, `P ⇒ Q` itn.
* **postfiksne operacije** so take, ki jih pišemo *za* argument:
    * `n!` za faktorielo števila `n`

Včasih uporabljamo tudi druge oblike zapisa:

* *potenciranje* `Aᴮ`
* *ulomki* `a/b` (mišljen je ulomek, kjer so `a`, črta in `b` zapisani navpično)
* integrali in vsote
* zapis podmnožice `{ x ∈ ℝ | x² + x > 2}`

Operacija je lahko celo "nevidna", oziroma jo pišemo kot presledek med argumentoma:

* `x y` kot zmnožek `x` in `y`
* `sin x` kot uporabe funkcije `sin` na argumentu `x`


\subsection{Oklepaji, prioriteta in asociiranost}

Z oklepaji ponazorimo, katera operacija ima prednost. Na primer, če ne bi imeli dogovora, da ima množenje prednost pred
seštevanjem, potem bi lahko izraz

    3 + 4 × 5

razumeli kot `3 + (4 × 5)` ali kot `(3 + 4) × 5`. Oklepajev ne smemo opustiti, kadar bi lahko prišlo do take zmede.
Nikoli pa ne škodi, če zapišemo kak oklepaj več, kot je to potrebno (v mejah normale).

Da se izognemo pisanju oklepajev, se dogovorimo, da imajo nekatere operacije prednost pred ostalimi, kar so vas učili že
v osnovni šoli. Pravimo, da imajo operacije **prioriteto**. Operacija z višjo prioriteto ima prednost pred operacijo z
nižjo prioriteto.

Primer:

* `×` ima višjo prioriteto kot `+` (to je *dogovor* in ne matematično dejstvo)
* `∧` ima višjo priroriteto kot `∨`

Poleg prioritete imajo nekatere operacije tudi **asociiranost**. Kako naj razumemo izraz `8 - 3 - 2`, kot `(8 - 3) - 2`
ali kot `8 - (3 - 2)`? V šoli so vas učili, da je

    A - B - C = (A - B) - C

Pravimo, da `-` veže na levo oziroma da ima **levo asociiranost**. Ker beremo z leve na desno, ima večina operacij levo
asociiranost. Velja na primer

    A + B + C = (A + B) + C
    A × B × C = (A × B) × C

Morda bo kdo pripomnil, da velja `(A + B) + C = A + (B + C)` in da zato ni pomembno, kako razumemo `A + B + C`. To je
res v preprostih primerih, ko vemo, da smo s `+` označili seštevanje števil. Kaj pa, če s `+` označimo kako drugo
preslikavo? Ali `(A + B) + C = A + (B + C)` velja tudi v programiskih jezikih, pri katerih lahko pride do prekoračitve
največjega možnega števila?

Primer operacije z desno asociiranostjo je implikacija: `P ⇒ Q ⇒ R` je enako `P ⇒ (Q ⇒ R)`.


\subsection{Izrazi predstavljajo drevesa}

Izrazi so zaporedja znakov, ki jih pišemo z leve na desno. A kje drugje bi jih morda pisali z desne na levo ali
navpično. Izrazi so le *predstavitve* tako imenovanih **sintaktičnih dreves**. Na primer `((3 + x) × y)²` predstavlja
sintaktično drevo (potenciranje predstavimo z znakom `^`)

           ^
          / \
         ×   2
        / \
       +   y
      / \
     3   x

O sintaktičnih drevesih ne bomo govorili, a jih omenimo, ker so pomembna iz dveh razlogov: sintaktična drevesa so
*podatkovni tip*, s katerim v programu dejansko obdelujemo izraze; s pomočjo sintaktičnih dreves lahko simbolni zapis
predstavimo kot posebno vrsto algebre, ki razkriva pomembno matematično strukturo izrazov.


\subsection{Ostala sintaktična pravila}

Sintaktičnih pravil je še več, od katerih omenimo le tri.

Argumente operacije ali funkcije včasih zapišemo v **podnapis** ali **nadnapis**. Na primer, če je `a : ℕ → ℝ`
preslikava, pogosto pišemo `aᵢ` namesto `a(i)`.

Argumente operacije lahko opustimo in od bralca pričakujemo, da bo pravilno uganil, kaj smo mislili. Pravimo, da so to
**implicitni argumenti**. Primer implicitnih argumetov smo že videli, ko smo zapisali prvo in drugo projekcijo `pr₁` in
`pr₂`:

    pr₁ : A × B → A
    pr₂ : A × B → A

Če bi bili zelo natančni, bi morali pri projekcijah zapisati tudi množici `A` in `B`, ki tvorita kartezični produkt, na
primer nekaj takega kot

    pr₁ᴬ,ᴮ : A × B → A

Ko torej vpeljemo novo zapis, lahko nekatere argumente razglasimo za *implicitne*, kar pomeni, da jih bomo opuščali,
kadar to ne pripelje do zmede.

**Naloga:** ali ima kompozicija preslikav `∘` implicitne argumente? Katere?

Simbol lahko tudi **preobtežimo**, da ima več pomenov, nato pa od bralca pričakujemo, da bo uganil, katerega smo
mislili. Na primer, `+` uporabljamo za:

* seštevanje naravnih števil
* seštevanje celih števil
* seštevanje racionalnih števil
* seštevanje realnih števil
* seštevanje kompleksnih števil
* seštevanje vektorjev
* seštevanje matrik
* itd.

S preobteževanjem ne gre pretiravati, ker lahko pripelje do zmede. Običajno z istim simbolom označimo različne
operacije, ki imajo kaj skupnega. Na primer, `+` uporabljamo za komutativno, asociativno operacijo z nevtralnim
elementom.


\section{Logične formule}

Izrazi, ki označujejo števila, se imenujejo **aritmetični izrazi**.

Irazi, ki označujejo matematične izjave, se imenujejo **logični izrazi** ali **logične formule**. Razumevanje, branje in pisanje le-teh zahteva kar nekaj treninga, zato se mu bomo posvetili tu in na vajah. Pravzaprav ne bomo vadili le razumevanja zapisa, ampak tudi, kako matematiki razmišljajo in razumejo drug drugega.

Danes o dokazih in pravilih dokazovanja še ne bomo govorili, bomo pa pojasnili intuitivni pomen logičnih operacij.

Logične formule in logika nasploh sestoji iz:

* **izjavni račun** zaobjema logične veznike `¬`, `∧`, `∨`, `⇒`, `⇔`
* **predikatni račun** zaobjema izjavni račun, ter kvantifikatorja `∀` in `∃`


\subsection{Izjavni račun}

**Izjavn vezniki** so naslednje operacije:

* **resničnostni konstanti** `⊥` in `⊤`: beremo ju "neresnica" in "resnica"

* **negacija** `¬`: izjavo `¬A` beremo "`A` ne velja" ali "ni res, da `A`"

* **konjunkcija** `∧`: izjavo `A ∧ B` beremo "`A` in `B`"

* **disjunkcija** `∨`: izjavo `A ∨ B` beremo "`A` ali `B`"

* **implikacija** `⇒`: izjavo `A ⇒ B` lahko beremo na več načinov:

   * "Iz `A` sledi `B`."
   * "Če `A`, potem `B`."
   * "`A` samo če `B`."
   * "`B` sledi iz `A`."
   * "`A` je zadosten pogoj za `B`."
   * "`B` je potreben pogoj za `A`."

* **ekvivalenca** `⇔`: izjavo `A ⇔ B` beremo

   * "`A` je ekvivalentno `B`."
   * "`A`, če in samo če `B`."
   * "`A` natanko tedaj, ko `B`."
   * "`A` je zadosten in potreben pogoj za `B`."

Malo bolj neobičajna je:

* **ekskluzivna disjunkcija** `⊕`: izjava `A ⊕ B` beremo "bodisi `A` bodisi `B`

Prioriteta veznikov, od najvišje do najnižje:

* `¬`
* `∧`
* `∨`, `⊕`
* `⇒`, `⇔`

Primer: `¬ A ∧ B ⇒ C ∨ D` beremo kot `((¬ A) ∧ B) ⇒ (C ∨ D)`.

Asociranost veznikov:

* leva asociiranost: `∧`, `∨`, `⊕`
* desna asociiranost: `⇒`

Ekvivalenca `⇔` nima asociiranost, zato je zapis `A ⇔ B ⇔ C` načeloma dvoumen, a v praksi pomeni `(A ⇔ B) ∧ (B ⇔ C)`

**Opomba:** Tudi zapis `x = y = z` pravzaprav ni smiselen, saj sta `(x = y) = z` in `x = (y = z)` oba nesmiselna. V
praksi `x = y = z` pomeni `(x = y) ∧ (y = z)`. Pa še to: koliko enačb je izraženih z `a = b = c = d`? Tri! Toliko kot je
enačajev.

**Opomba k opombi:** Zapis `x ≠ y ≠ z` je nejasen in se mu je bolje izogibati, saj zlahka pripelje do pomote, ker iz `x
≠ y` in `y ≠ z` ne sledi nujno `x ≠ z`.

Glede razumevanja veznikov, omenimo:

* disjunkcija je *inkluzivna*, kar pomeni, da je `A ∨ B` resnična izjava, če sta `A` in `B` resnični,
* v implikaciji `A ⇒ B` se `A` imenuje **antecedent** in `B` **konsekvent**. Implikacija je veljavna, če je antecedent neveljaven,
* ekvivalenco `A ⇔ B` lahko razumemo kot okrajšavo za `(A ⇒ B) ∧ (B ⇒ A)`.


\subsection{Kvantifikatorja}

Matematične izjave vsebujejo fraze, kot so "za vse", "za neki", "obstaja vsaj en", "za natanko enega" ipd. Le-te izrazimo s **kvantifiaktorji**. Osnovna kvantifikatorja sta **univerzalni** in **eksistenčni**.


\subsubsection{Univerzalni kvantifikator `∀`}

Formulo

    ∀ x ∈ A . ϕ

beremo:

* Za vsak `x` iz `A` velja `ϕ`
* Vsi `x` iz `A` zadoščajo `ϕ`
* `ϕ` za vse `x` iz `A`

Pika pri tem nima nobenega posebnega pomena, lahko bi pisali tudi

* `∀ x ∈ A , ϕ`
* `∀ x : A , ϕ`
* `(∀ x : A) ϕ`

Nekateri matematiki pišejo tudi `ϕ, ∀ x ∈ A`. Ta zapis odsvetujemo, ker ne deluje, ko kombiniramo več
kvantifikatorjev hkrati. (Dobro pa je razumeti, da tako pišejo, ker se lepo sliši "`ϕ` za vse `x` iz `A`".)

Omenili smo že, da `∀ x ∈ ∅ . ϕ` vedno velja. To bomo utemeljili naslednjič.

\subsubsection{Eksistenčni kvantifikator `∃`}

Formulo

    ∃ x ∈ A . ϕ

beremo:

* Obstaja `x` iz `A` velja `ϕ`
* Obstaja vsaj en `x` iz `A` velja `ϕ`
* Za neki `x` iz `A` velja `ϕ`
* `ϕ` za neki `x` iz `A`

S tem povemo, da obstaja *eden ali več* takih `x`. Na primer, izjava `∃ x ∈ ℕ . x
< 3` je veljavna, saj obstajo kar tri naravna števila, ki so manjša od `3`.


\subsubsection{Prioriteta `∀` in `∃`}

Prioriteta kvantifikatorjev `∀` in `∃` je nižja od prioritete veznkov. Na primer:

* `∀ x ∈ A . ϕ ∧ ψ` je enako `∀ x ∈ A . (ϕ ∧ ψ)`
* `∀ x ∈ ℝ . x > 0 ⇒ ϕ` je enako ``∀ x ∈ ℝ . (x > 0 ⇒ ϕ)`

Kvantifikator vedno zaobjame vse, kar zmore:

* `∀ x ∈ A . ϕ ∧ ∃ y ∈ B . ψ` je enako `∀ x ∈ A . (ϕ ∧ (∃ y ∈ B . ψ))` in *ni* enako `(∀ x ∈ A . ϕ) ∧ (∃ y ∈ B . ψ)`

* `(P ∧ ∀ x ∈ A . Q ⇒ R) ⇒ ∃ y ∈ B . S` je enako `(P ∧ ∀ x ∈ A . (Q ⇒ R)) ⇒ (∃ y ∈ B . S)` in *ni* enako 
  `(P ∧ (∀ x ∈ A . Q) ⇒ R) ⇒ (∃ y ∈ B . S)`


\subsubsection{Kombinacija `∀` in `∃`}

Pozor, vrstnega reda kvantifikatorjev ne smemo mešati!

Primer:

* `∀ x ∈ ℝ . ∃ y ∈ ℝ . x < y` pomeni "vsako realno število je manjše od nekega realnega števila" (kar je res),
* `∃ x ∈ ℝ . ∀ y ∈ ℝ . x < y` pomeni "obstaja najmanjše realno število" (kar ni res).

To dejstvo bomo utrjevali na vajah. Zapomnite se, da morate biti tudi pri ostalih predmetih posebej pozorni na vrstni red
"za vsak" in "obstaja". Je profesorica pri analizi rekla "za vsak `ε > 0` obstaja tak `δ > 0` da ..." ali je rekla
"ostaja tak `δ > 0` da za vsak `ε > 0` ...". Če boste zamešali ti dve izjavi na ustnem izpitu iz analize,
boste imeli pokvarjen dan, ali pa cele počitnice!


\subsubsection{Kvantifikator z dodatnim pogojem}

Pogosto kvantifikacijo kombiniramo z dodatnim pogojem, na primer

* "Obstaja liho naravno število, ki ni deljivo s 7."
* "Vsako sodo naravno število je deljivo s 3."

V prvem primeru je dodatni pogoj izražen z besedico "liho" in v drugem s "sodo". Kako zapiemo take izjave s formulo, kam
vtaknemo dodatni pogoj? Ijzavi pretvorimo po korakih:

* "Obstaja liho naravno število, ki ni deljivo s 7."
* "Obstaja naravno število, ki je liho in ki ni deljivo s 7."
* "Obstaja naravno število, ki je liho in deljivo s 7."
* "Obstaja `x` iz `ℕ`, da je `x` lih in `x` je deljiva s 7."
* `∃ x ∈ ℕ . (x je lih) ∧ (x je deljiv s 7)`

In še druga izjava:

* "Vsako sodo naravno število je deljivo s 3."
* "Vsako naravno število, ki je sodo, je deljivo s 3."
* "Za vsako naravno število velja, da če je sodo, potem je deljivo s 3."
* "Za vsak `x` iz `ℕ` velja, če je `x` sod, potem je `x` deljiv s `3`."
* `∀ x ∈ ℕ . x sod ⇒ x deljiv s 3`

Zapomnimo si: **dodatni pogoj pri `∃` izrazimo `∧`** in **dodatni pogoj pri `∀` izrazimo `⇒`**.

\subsubsection{Vezane in proste spremenljivke}

V nekaterih izrazih nastopajo spremenljivke, ki so **vezane**. To pomeni, da je njihovo območje veljavnosti imejeno,
oziroma da so neke vrste lokalne spremenljivke. Spremenljivka, ki ni vezana, je **prosta**. Primeri:

* V funkcijskem predpisu `x ↦ x² + y` je `x` vezan in `y` prost.
* V funkcijskem predpisu `(x,y) ↦ x² + y` sta `x` in `y` vezana.
* V integralu `∫ (x + a)² d x` je `x` vezan in `a` prost.
* V vsoti `∑_(i = 0)^n (i² + 1)` je `i` vezan in `n` prost.
* V formuli `∀ x ∈ ℝ . x³ + 3 x < 7` je `x` vezana spremenljivka.

Če vezano spremenljivko preimenujemo, se izraz ne spremeni:

**Pozor:** za novo ime vezane spremenljivke moramo izbrati spremenljivko, ki ni prosta. Na primer, v integralu

    ∫ (a + x)² d t

smemo `x` preimenovati v `t`:

    ∫ (a + t)² d t

Ta dva izraza štejemo za *enaka*. Ne bi pa smeli `x` preimenovati v `a`:

    ∫ (a + a)² d a

Pravimo, da se je prosta spremenljivka `a` *ujela* v integral.



\chapter{Definicije in dokazi}

\section{Enolični obstoj}

\subsection{Kvantifikator za enolični obstoj $\exists!$}

S kvantifikatorjema $\forall$ in $\exists$ lahko izrazimo tudi druge kvantifikatorje.
Na primer, ">obstajata vsaj dva elementa $x$ in $y$ iz $A$, da velja $\phi(x,y)$"< zapišemo
%
\begin{equation*}
    \some{x \in A} \some{y \in A} x \neq y \land \phi(x,y)
\end{equation*}
%
Kako pa izrazimo ">obstaja natanko en $x$ iz $A$, da velja $\phi(x)$"<? Takole:
%
\begin{equation*}
  (\some{x \in A} \phi(x)) \land \all{y z \in A} \phi(y) \land \phi(z) \lthen y = z
\end{equation*}
%
ali ekvivalentno
%
\begin{equation*}
    \some{x \in A} (\phi(x) \land \all{y \in A} \phi(y) \lthen x = y).
\end{equation*}
%
To okrajšamo $\exactlyone{x \in A} \phi(x)$ in beremo ">obstaja natanko en $x$ iz $A$, da velja $\phi(x)$"<.
%
Uporablja se tudi zapis $\exists^1 x \in A \,.\, \phi(x)$.

\subsection{Operator enoličnega opisa}

Če dokažemo, da obstaja natanko en $x \in A$, ki zadošča pogoju $\phi(x)$, potem se lahko nanj smiselno sklicujemo z ">tisti $x$ iz $A$, ki zadošča $\phi(x)$"<. Primeri:
%
\begin{itemize}
\item ">tisto realno število $x$, za katero je $x³ = 2$"<, namreč kubični koren 2,
\item ">tista množica $S$, ki nima nobenega elementa"<, namreč prazna množica.
\end{itemize}
%
Protiprimeri:
%
\begin{itemize}
\item ">tisto racionalno število $x$, za katero je $x² = 2$"<, saj takega števila ni,
\item ">tisto realno število $x$, za katero je $x² = 2$"<, ker sta dve taki števili,
\item ">tista množica $S$, ki ima natanko en element"<, ker je takih množic je zelo veliko.
\end{itemize}
%
To je lahko zelo koristen način za opredelitev matematičnih objektov, zato uvedemo zanj simbolni zapis. Če dokažemo
%
\begin{equation*}
    \exactlyone{x \in A} \phi(x)
\end{equation*}
%
potem lahko pišemo
%
\begin{equation*}
  \descr{x \in A} \phi(x),
  \qquad\qquad\text{">tisti $x \in A$, za katerega velja $\phi(x)$"<}
\end{equation*}
%
Torej velja
%
\begin{equation*}
  \phi(\descr{x \in A} \phi(x)).
\end{equation*}
%
Spremenljivka $x$ je \emph{vezana} v $\descr{x \in A} \phi(x)$.

\begin{primer}
  Denimo, da še ne bi poznali simbola $\sqrt{}$ za kvadatne korene. Tedaj bi
  lahko kvadratni koren iz $2$ zapisali kot
  %
  \begin{equation*}
    \descr{x \in R} (x > 0 \land x^2 = 2)
  \end{equation*}
  %
  Še več, preslikavo $\sqrt{} : \RR_{\geq 0} \to \RR_{\geq 0}$ lahko definiramo takole:
  %
  \begin{equation*}
    \sqrt{} : x \mapsto (\descr{y \in \RR} (y \geq 0 \land y^2 = x)).
  \end{equation*}
\end{primer}

\begin{naloga}
  Zapišite ">limita zaporedja $a : \NN \to \RR$""< z operatorjem $\iota$, pod predpostavko,
  da je $a$ konvergentno zaporedje. Najprej povejte z besedami ">limita zaporedja $a$ je
  tisti $x \in \RR$, ki \dots"<, nato pa zapišite še v obliki $\descr{x \in \RR} \dots$.
\end{naloga}


\begin{opomba}
  Ne pozabite: zapis $\descr{x \in A} \phi(x)$ je veljaven samo v primeru, da velja
  $\exactlyone{x \in A} \phi(x)$.
\end{opomba}


\section{Spremenljivke in definicije}

Preden v matematičnem ebsedilu uporabimo simbol ali spremenljivko, ga moramo \emph{vpeljati}. To pomeni, da moramo pojasniti, kakšen je pomen simbola. Poznamo dva osnovna načina za vpeljavo novih simbolov:
%
\begin{itemize}
\item \item Nov simbol $s$ lahko \textbf{definiramo} kot okrajšavo za neki drugi izraz ali logično formulo.
\item Nov simbol $s$ je (vezana ali prosta) \textbf{spremenljivka}, ki predstavlja neki (neznan, poljuben, nedoločen) element dane množice $A$.
\end{itemize}
%
V obeh primerih dodamo simbol~$s$ v \textbf{kontekst}, se pravi v spisek znanih simbolov. Če smo simbol uvedli le začasno (na primer v enem poglavju, ali v delu dokaza), ga iz konteksta odstranimo, ko ni več veljaven.

Matematiki zapisujejo definicije in vpeljujejo spremenljivke na razne načine.

\subsection{Vpeljava spremenljivke}

Če želimo vpeljati spremljivko $x$, ki predstavlja neki poljuben ali neznani element množice $A$, zapišemo
%
\begin{quote}
  Naj bo $x \in A$.
\end{quote}
%
S tem postane $x$ veljavna spremenljivka, ki jo lahko uporabljamo. O njen vemo le to, da je element množice $A$ -- pravimo, da je $x$ \textbf{prosta spremenljivka}. V matematičnih besedilih boste zasledili tudi naslednje fraze, ki uvedejo prosto spremenljivko:
%
\begin{itemize}
\item ">Naj bo $x \in A$ poljuben.""<
\item ">Obravnavajmo poljuben $x \in A$.""<
\item ">Izberimo poljuben $x \in A$.""<
\item ">Denimo, da imamo poljuben $x \in A$.""<
\end{itemize}
%
Pozor, beseda ">izberimo"< bi komu dala misliti, da si lahko izbere neki konkretni~$x$, a to preprečuje beseda ">poljuben", ki jo matematik uporabi, kadar želi povedati, da je~$x$ neznana ali nedoločena (poljubna) vrednost.

\begin{naloga}
  Denimo, da učitelj reče ">Naj bo $n$ (poljubno) naravno število""<, nato pa vas vpraša ">Ali je $n$ sodo število?""<, kako boste odgovorili?
\end{naloga}

\subsection{Definicija simbola}

Definicija je v prvi vrsti \textbf{okrajšava} za neki izraz. Z njo uvedemo nov simbol~$s$ in mu pripišemo neko vrednost. Simbol $s$ je enak vrednosti, ki smo mu jo pripisali. Simbolni zapis za definicijo je
%
\begin{equation*}
  s \defeq \ldots
\end{equation*}
%
Na primer, v besedilu bi lahko napisali ">Naj bo $s := \sqrt{\log_2 7 + \pi/6}$."< S tem smo v kontekst dodali simbol $s$ in predpostavko $s = \sqrt{\log_2 7 + \pi/6}$. V matematičnih besedili boste zasledili tudi naslednje načine za definicijo:
%
\begin{itemize}
\item $s = \sqrt{\log_2 7 + \pi/6}$ (namesto $\defeq$ uporabimo $=$)
\item $s \cong \sqrt{\log_2 7 + \pi/6}$ (namesto $\defeq$ uporabimo $\cong$)
\item $s \triangleq \sqrt{\log_2 7 + \pi/6}$ (namesto $\defeq$ uporabimo $\triangleq$)
\end{itemize}
%
Kadar definiramo simbol tako, da mu priredimo funkcijski predpis, recimo
%
\begin{equation*}
  f \defeq (x \mapsto x^2 + 7)
\end{equation*}
%
to raje zapišemo kot
%
\begin{equation*}
  f(x) \defeq x^2 + 7.
\end{equation*}
%
Kadar definiramo simbol s pomočjo enoličnega obstoja, recimo
%
\begin{equation*}
  r \defeq \descr{x \in \RR} x^3 = 2
\end{equation*}
%
to raje zapišemo z besedami:
%
\begin{equation*}
  \text{Naj bo $r$ tisto realno število, ki zadošča $r^3 = 2$.}
\end{equation*}
%
Poglejmo še, kako definiramo okrajšave za logične formule. Denimo, da želimo s $\phi(x)$ označiti izjavo $\some{y \in \RR} y^2 = x + 1$. Glede na zgornji dogovor, zapišemo
%
\begin{equation*}
  \phi \defeq (x \mapsto (\some{y \in \RR} y^2 = x + 1))
\end{equation*}
%
ali
%
\begin{equation*}
  \phi(x) \defeq (\some{y \in \RR} y^2 = x + 1).
\end{equation*}
%
Vendar takega zapisa v praksi ne boste videli. Dosti bolj pogost je zapis
%
\begin{equation*}
  \phi(x) \defiff \some{y \in \RR} y^2 = x + 1
\end{equation*}
%
ali pa kar $\phi(x) \liff \some{y \in \RR} y^2 = x + 1$.

\subsection{Definicije novih matematičnih pojmov}

Kaj pa definicije novih pojmov, ki jih srečujete pri predavanjih, denimo pri analizi?

\begin{definicija}
  Zaporedje števil $a : \NN \to \RR$ je \textbf{neomejeno}, če za vsak $x \in \RR$ obstaja $i \in \NN$, da je $a_i > x$.
\end{definicija}

\noindent
S stališča simbolnega zapisa, je to le uveba novega simbola $\mathsf{neomejeno}$:
%
\begin{equation*}
  \mathsf{neomejeno}(a) \defeq (\all{x \in \RR} \some{i \in \NN} a_i > x).
\end{equation*}
%
Seveda bistvo take definicije ni le krajši zapis izjave $\all{x \in \RR} \some{i \in \NN} a_i > x$, ampak uporabna vrednost pojma ">neomejeno zaporedje"<.


\section{Konstrukcije in dokazi}

zivedokMatematiki v sklopu svojih aktivnosti \emph{konstruiramo} matematične objekte:
%
\begin{itemize}
\item v geometriji so znane konstrukcije z ravnilom in šestilom,
\item računanje števk števila $\pi$ je konstrukcija približka,
\item reševanje enačbe, je konstrukcija števila z želeno lastnostjo,
\item konstruiramo lahko elemente množice, pogosto kar tako, da jih zapišemo, na primer $(2, \inl(3)) \in \NN \times (\ZZ + \ZZ)$.
\end{itemize}

Poleg tega \emph{dokazujemo} matematične izjave. Na dokaz lahko gledamo kot na konstrukcijo, saj je to le še ena zvrst matematičnega objekta. Ker pa so dokazi skoraj vedno zapisani v naravnem jeziku, jih matematiki pogosto dojemajo ločeno od ostalih matematičnih objekotv (števila, preslikave, množice, ploskve, \dots).

Kaj pravzaprav je dokaz? V prvi vrsti je dokaz utemeljitev matematične izjave. Zgrajen je po točno določenih \emph{pravilih sklepanja}, ki jih lahko podamo formalno in jih tudi implementiramo na računalniku.\footnote{Kogar to zanima, si lahko ogleda ">\href{https://youtu.be/Z500sma3h90}{The dawn of formalized mathematics}"< (\href{https://www.icloud.com/keynote/0Gkr1yM7XY-31aQleWf-fiW7A}{prosojnice}) in se nauči uporabljati kak  \href{https://ncatlab.org/nlab/show/proof+assistant}{dokazovalni pomočnik} (v zadnjem času hitro napreduje \href{https://leanprover.github.io}{Lean}).
}

V praksi ljudje ne pišejo vseh podrobnosti v dokazu, ker bi bil tak dokaz nečitljiv in nerazumljiv. Pogosto podajo samo glavno idejo, iz katere lahko izkušeni matematik sam rekonstruira dokaz. Iz dobro napisanega dokaza se lahko naučimo marsikaj novega, poleg goleda dejstva, da dokaza izjava velja.

Mi bomo vadili podrobno pisanje dokazov. Pri ostalih predmetih boste videli ">žive dokaze"<, ki imajo manj podrobnosti in so zapisani manj formalno. A vsi pravilni matematični dokazi se dajo zapisati na način, kot ga bomo predstavili mi (in celo zapisati povsem formalon z dokazovalnim pomočnikom).

\subsection{Kako pišemo dokaze}

Pravila sklepanja so kot pravila igre. Ne povedo, kako dobro igrati, samo kaj je dovoljeno. Seveda bomo hkrati s pravili sklepanja povedali nekaj namigov in nasvetov, kako dokaz poiščemo. A kot pri vsaki igri velja, da vaja dela mojstra.

Dokaz ima vgnezdeno strukturo: sestoji iz delov in pod-dokazov, ki sestojijo iz nadaljnih pod-dokazov itn., ki se zaključijo z osnovnimi dejstvi. Vsi ti kosi so s pomočjo pravil sklepanja zloženi v dokazno ">drevo"<.

Ko pišemo dokaz, moramo v vsakem trenutku poznati
%
\begin{itemize}
\item \textbf{cilj}: kaj trenutno dokazujemo in
\item \textbf{kontekst}: katere spremenljivke in predpostavke imamo trenutno na voljo.
\end{itemize}
%
Ko napravimo korak v dokazu, mora biti utemeljen z enim od pravil sklepanja. Dokaz je
popoln, ko smo utemeljili vse poddokaze, ki ga sestavljajo. Kot primer si poglejmo zelo podroben dokaz izjave
%
$(p \lor q) \land r \lthen (p \land r) \lor (q \land r)$.

\begin{center}
  \fbox{\parbox{0.6\textwidth}{
    Dokažimo $(p \lor q) \land r \lthen (p \land r) \lor (q \land r)$. \\
    \hbox{}\quad (1) Predpostavimo $(p \lor q) \land r.$ \\
    \hbox{}\quad (2) Zaradi (1) velja $p \lor q$. \\
    \hbox{}\quad (3) Zaradi (1) velja $r$. \\
    \hbox{}\quad Zaradi (2) lahko obravnavamo dva primera:\\
    \hbox{}\qquad \fbox{\parbox{0.5\textwidth}{
      (a) če velja $p$:\\
          \hbox{}\quad Dokažimo $(p \land r) \lor (p \land r)$.\\
          \hbox{}\quad Dokažimo levi disjunkt $p \land r$: \\
          \hbox{}\qquad (i) $p$ velja zaradi (a) \\
          \hbox{}\qquad (ii) $r$ velja zaradi (3).
    }} \\
    \hbox{}\qquad \fbox{\parbox{0.5\textwidth}{
      (b) če velja $q$:\\
          \hbox{}\quad Dokažimo $(p \land r) \lor (p \land r)$.\\
          \hbox{}\quad Dokažimo desni disjunkt $q \land r$: \\
          \hbox{}\qquad (i) $q$ velja zaradi (b) \\
          \hbox{}\qquad (ii) $r$ velja zaradi (3).
    }}
  }}
\end{center}
%
Dokaz bi bolj po človeško napisali takole:
%
\begin{quote}
  Predpostavimo $p \lor q$ in $r$. Če velja $p$, potem sledi $p \land r$ ter od tod $(p \land r) \lor (p \land r)$. Če pa velja $q$, sledi $q \land r$ ter spet $(p \land r) \lor (p \land r)$. $\Box$
\end{quote}
%
Ali pa kar takole:
%
\begin{quote}
  Očitno.
\end{quote}

Pravila sklepanja delimo na:
%
\begin{itemize}
\item \textbf{pravila vpeljave}, ki povedo, kako dokažemo izjavo, ter
\item \textbf{pravila uporabe}, ki povedo, kako lahko že znano izjavo uporabimo.
\end{itemize}
%
Poleg tega poznamo še pravila o zamenjavi:
%
\begin{itemize}
\item \textbf{zamenjava enakih izrazov}: izraz lahko vedno zamenjamo z njim enakim,
\item \textbf{zamenjava ekvivalentnih izjav}: izjavo vedno lahko zamenjamo z njej ekvivalentno.
\end{itemize}

Dokaz je skupek računskih korakov in sklepov, s katerimi utemeljimo izjavo. V vsakem
trenutku mora biti jasno, kaj dokazujemo, katere spremenljivke so veljavne in katere
predpostavke so na voljo.
%
Nekateri deli dokaza so samostojni poddokazi pomožnih izjav. Vse spremenljivke in
predpostavke, ki jih uvedemo v poddokazu, so na voljo izključno v poddokazu samem.

\subsection{Pravila vpeljave}
\label{sec:pravila-vpeljave}

S pravilom za vpeljavo \emph{neposredno} dokažemo izjavo. Za vsak veznik in kvantifikator ponazorimo, kako uporabimo
pripadajoče pravilo vpeljave.

\subsubsection{Konjunkcija}
%
\begin{quote}
  \sl
  Dokažimo $\phi \land \psi$.
  \begin{enumerate}
  \item Dokažimo $\phi$: \quad \brac{dokaz $\phi$}
  \item Dokažimo $\psi$: \quad \brac{dokaz $\psi$}
  \end{enumerate}
\end{quote}

\subsubsection{Disjunkcija}

Prvi način:
%
\begin{quote}
  \sl
  Dokažimo $\phi \lor \psi$.
  %
  \begin{itemize}
  \item[] Zadostuje dokazati levi disjunkt $\phi$: \quad \brac{dokaz $\phi$}
  \end{itemize}
\end{quote}
%
Drugi način:
%
\begin{quote}
  \sl
  Dokažimo $\phi \lor \psi$.
  %
  \begin{itemize}
  \item[] Zadostuje dokazati desni disjunkt $\psi$: \quad \brac{dokaz $\psi$}
  \end{itemize}
\end{quote}

\subsubsection{Implikacija}

\begin{quote}
  \sl
  Dokažimo $\phi \lthen \psi$:
  %
  \begin{itemize}
  \item[] Predpostavimo $\phi$. \\
        Dokažimo $\psi$: \quad \brac{dokaz $\psi$}
  \end{itemize}
\end{quote}

\subsubsection{Ekvivalenca}

\begin{quote}
  \sl
  Dokažimo $\phi \liff \psi$.
  \begin{enumerate}
  \item Dokažimo $\phi \lthen \psi$: \quad \brac{dokaz $\phi \lthen \psi$}
  \item Dokažimo $\psi \lthen \phi$: \quad \brac{dokaz $\psi \lthen \phi$}
  \end{enumerate}
\end{quote}

\subsubsection{Resnica}

Resnice $\top$ ni treba dokazovati, zapišemo ">očitno"<. \footnote{
V praksi $\top$ nastopi kot izjava, ki jo želimo dokazati, ko neko drugo izjavo poenostavimo. Primer: ko dokazujemo
$12^2 + 12^2 < 17^2$, najprej izračunamo, da je to ekvivalentno $288 < 289$, kar je ekvivalentno $\top$. S tem je dokaz
zaključen, saj smo dobili $\top$.}

\subsubsection{Neresnica}

Kadar dokazujemo $\bot$, pravimo, da ``iščemo protislovje''.

\begin{quote}
  \sl
  Poiščimo protislovje.
  \begin{enumerate}
  \item Dokažimo $\phi$: \quad \brac{dokaz $\phi$}
  \item Dokažimo $\lnot \phi$: \quad \brac{dokaz $\lnot\phi$}
  \end{enumerate}
\end{quote}

\subsubsection{Negacija}

\begin{quote}
  \sl
  Dokažimo $\lnot \psi$:
  %
  \begin{itemize}
  \item[] Predpostavimo $\psi$. \\
          Poiščimo protislovje: \quad \dots
  \end{itemize}
\end{quote}
%
Opomba: ni nujno, da poiščemo protislovje med $\psi$ in $\lnot\psi$, vsako protislovje je sprejemljivo.

\subsubsection{Univerzalna izjava}

\begin{quote}
  \sl
  Dokažimo $\all{x \in A} \phi(x)$.
  %
  \begin{enumerate}
  \item[] Naj bo $x \in A$. \\
          Dokažemo $\phi(x)$: \quad \brac{dokaz $\phi(x)$}
  \end{enumerate}
\end{quote}
%
\textbf{Pozor:} spremenljivka $x$ mora biti \emph{sveža}, se pravi, da je ne uporabljamo nikjer drugje. Če jo, najprej izberemo svežo spremenljivko~$y$ in $x$ preimenujemo v~$y$.

\subsubsection{Eksistenčna izjava}

\begin{quote}
  \sl
  Dokažimo $\some{x \in A} \phi(x)$:
  %
  \begin{itemize}
  \item[] Podamo $x \mathbin{{:}{=}} \langle\text{izraz}\rangle$. \\
          Dokažemo $\langle\text{izraz}\rangle \in A$: \quad \dots \\
          Dokažemo $\phi(\langle\text{izraz}\rangle)$: \quad \dots
  \end{itemize}
\end{quote}
%
Opomba: $\langle\text{izraz}\rangle$ sme vsebovati vse proste spremenljivke, ki so trenutno na voljo ($x$ \emph{ni} na
voljo).

\subsection{Pravila uporabe}

Pravila uporabe nam povedo, kako iz predpostavk in že znanih dejstev izpeljemo nova dejstva.

\subsubsection{Konjunkcija} 

\begin{quote}
  \sl
  Vemo, da velja $\phi \land \psi$.\\
  Torej velja $\phi$.\\
  Torej velja $\psi$.
\end{quote}
%
Opomba: v praksi tega koraka ne delamo, ampak namesto predpostavke $\phi \land \psi$ kar takoj vpeljemo ločeni
predpostavki $\phi$ in $\psi$.


\subsubsection{Disjunkcija}
%
\begin{quote}
  \sl
  Dokažimo $\rho$.\\
  Vemo, da velja $\phi \lor \psi$, torej obravnavamo primera:\\
  %
  \begin{enumerate}
  \item Če velja $\phi$: \\
        Dokažemo $\rho$: \quad \brac{dokaz $\rho$}
  \item Če velja $\psi$: \\
        Dokažemo $\rho$: \quad \brac{dokaz $\rho$}
  \end{enumerate}
\end{quote}

\subsubsection{Implikacija}

\begin{quote}
  \sl
  Vemo, da velja $\phi \lthen \psi$.
  %
  \begin{itemize}
  \item[] Dokažimo $\phi$: \quad \brac{dokaz $\phi$}
  \end{itemize}
  %
  Torej velja tudi $\psi$.
\end{quote}

\subsubsection{Resnica}

Resnica ni uporabna kot predpostavka in jo lahko zavržemo.

\subsubsection{Neresnica}

\begin{quote}
  \sl
  Dokažimo $\rho$:
  \begin{itemize}
  \item[]
    Ugotovimo, da velja $\bot$.\\
    Ker iz neresnice sledi karkoli, velja $\rho$.
  \end{itemize}
\end{quote}


\subsubsection{Negacija}

Negacijo $\lnot \phi$ uporabimo tako, da dokažemo $\phi$ in zaključimo dokaz.
%
\begin{quote}
  \sl
  Dokažimo $\rho$.\\
  Vemo, da velja $\lnot\phi$.
  %
  \begin{itemize}
  \item Dokažimo $\phi$: \quad \brac{dokaz $\phi$}
  \end{itemize}
  %
  Torej velja $\rho$.
\end{quote}

\subsubsection{Univerzalna izjava}

\begin{quote}
  \sl
  Vemo, da velja $\all{x \in A} \phi(x)$.\\
  Vemo, da je $\langle\text{izraz}\rangle \in A$.\\
  Torej velja $\phi(\langle\text{izraz}\rangle)$.
\end{quote}

\subsubsection{Eksistenčna izjava}

\begin{quote}
  \sl
  Dokažimo $\rho$.\\
  Vemo, da velja $\some{x \in A} \phi(x)$.
  %
  \begin{enumerate}
  \item[] Imamo $x \in A$, za katerega velja $\phi(x)$.\\
       Dokažemo $\rho$: \quad \brac{dokaz $\rho$}
  \end{enumerate}
\end{quote}
%
\textbf{Pozor:} spremenljivka $x$ mora biti sveža, se pravi, da se ne pojavlja v $\rho$ ali kjerkoli drugje. Če se~$x$ pojavi kje drugje, ga moramo najprej nadomestiti s svežo spremenljivko~$y$.

\subsubsection{Izključena tretja možnost in dokaz s protislovjem}

\textbf{Pravilo izključene tretje možnosti} pravi, da vedno velja $\phi \lor \lnot\phi$ in ga uporabimo takole:
%
\begin{quote}
  \sl
  Dokažimo $\rho$.\\
  Velja $\phi \lor \lnot \phi$:
  %
  \begin{enumerate}
  \item Če velja $\phi$:\\
        Dokažemo $\rho$: \quad \dots
  \item Če velja $\lnot\phi$. \\
        Dokažemo $\rho$: \quad \dots
  \end{enumerate}
\end{quote}
%
\textbf{Dokaz s protislovjem} poteka takole:
%
\begin{quote}
  \sl
  Dokažimo $\rho$. Dokazujemo s protislovjem:
  %
  \begin{itemize}
  \item[] Predpostavimo $\lnot\rho$.\\
        Poiščimo protislovje: \quad \dots
  \end{itemize}
\end{quote}
%
Opomba: dokaz s protislovjem in pravilo vpeljave za negacijo sta \emph{različni}
pravili!


\chapter{Logika}\label{poglavje:logika}





\section{Logični simboli}\label{razdelek:logicni-simboli}

Preproste izjave, kot na primer \nls{$n$ je sodo število.}, že znamo zapisati s simboli: $2 \divides n$. Povečini pa delamo z bolj kompleksnimi, sestavljenimi izjavami. Tudi za te obstaja simbolni zapis; na primer, izjavo \nls{Če je $n$ sodo število, je tudi kvadrat števila $n$ sod.}, zapišemo kot $2 \divides n \implies 2 \divides n^2$. Seveda ta izjava velja za vsa naravna števila (znaš to dokazati?). To zapišemo takole: $\all{n \in \NN} (2 \divides n \implies 2 \divides n^2)$. V tem razdelku si bomo ogledali, kako povezati preprostejše izjave v bolj sestavljene in kako to v splošnem simbolno zapisati.

Kot smo navajeni iz naravnih jezikov, posamične stavke povežemo v sestavljeno poved z \emph{vezniki}. Najpogosteje uporabljeni matematični vezniki so v tabeli~\ref{tabela:standardni-izjavni-vezniki}.

\begin{table}[!ht]
\centering
\begin{tabular}{|ccc|}
\hline
\textbf{Izjavni veznik} & \textbf{Oznaka} & \textbf{Kako preberemo} \\
\hline
negacija & $\lnot{p}$ & ne $p$ \\
konjunkcija & $p \land q$ & $p$ in $q$ \\
disjunkcija & $p \lor q$ & $p$ ali $q$ \\
implikacija & $p \impl q$ & če $p$, potem $q$ \\
ekvivalenca & $p \lequ q$ & $p$ natanko tedaj, ko $q$ \\
\hline
\end{tabular}
\caption{Standardni izjavni vezniki}\label{tabela:standardni-izjavni-vezniki}
\end{table}

\begin{opomba}
V matematiki se za izjavne veznike običajno uporabljajo zgoraj navedene tujke, ampak vsaka od njih seveda ima svoj pomen. Dobesedni prevodi teh tujk so:
\begin{itemize}
\item
negacija $\to$ zanikanje,
\item
konjunkcija $\to$ vezava,
\item
disjunkcija $\to$ ločitev,
\item
implikacija $\to$ vpletenost,
\item
ekvivalenca $\to$ enakovrednost.
\end{itemize}
Za primerjavo: spomnite se vezalnega in ločnega priredja iz slovenščine!
\end{opomba}

\begin{zgled}
Naj $p$ označuje stavek \nls{Zunaj dežuje.} in $q$ stavek \nls{Vzamem dežnik.}. Tedaj $\lnot{p}$ pomeni \nls{Zunaj ne dežuje.} in $p \impl q$ pomeni \nls{Če zunaj dežuje, potem vzamem dežnik.}.
\end{zgled}

Kose sestavljene izjave lahko veže več kot en veznik. V tem primeru se (tako kot pri računanju s števili) dogovorimo o prednosti veznikov. Po dogovoru je vrstni red veznikov tak, kot v tabeli~\ref{tabela:standardni-izjavni-vezniki}, tj.~najmočneje veže negacija, nato konjunkcija, nato disjunkcija, nato implikacija, nato ekvivalenca. Kadar želimo, da se najprej izvede veznik z nižjo prednostjo, uporabimo oklepaje.

\begin{zgled}
Označimo sledeče stavke:
\begin{quote}
$p$ \ \ldots\ldots\ \nls{Imam čas.} \\
$q$ \ \ldots\ldots\ \nls{Ostanem doma.}
\end{quote}
Tedaj $\lnot{p} \land q$ pomeni isto kot $(\lnot{p}) \land q$, to je \nls{Nimam časa in ostanem doma.}, medtem ko $\lnot(p \land q)$ pomeni \nls{Ni res, da imam čas in ostanem doma.}.
\end{zgled}
\davorin{Če komu pade na pamet primer boljših stavkov, je zaželjeno, da popravi\ldots}

Poleg zgoraj navedenih izjavnih veznikov se včasih uporabljajo še sledeči (tabela~\ref{tabela:nadaljnji-izjavni-vezniki}).

\begin{table}[!ht]
\centering
\begin{tabular}{|ccc|}
\hline
\textbf{Izjavni veznik} & \textbf{Oznaka} & \textbf{Kako preberemo} \\
\hline
stroga disjunkcija & $p \xor q$ & bodisi $p$ bodisi $q$ \\
Shefferjev\tablefootnote{Henry Maurice Sheffer (1882 -- 1964) je bil ameriški logik.} veznik & $p \shf q$ & ne hkrati $p$ in $q$ \\
Łukasiewiczev\tablefootnote{Jan Łukasiewicz (beri: \hill{u}ukaśj\^{e}vič) (1878 -- 1956) je bil poljski logik in filozof.} veznik & $p \luk q$ & niti $p$ niti $q$ \\
\hline
\end{tabular}
\caption{Nekateri nadaljnji izjavni vezniki}\label{tabela:nadaljnji-izjavni-vezniki}
\end{table}

Za strogo disjunkcijo (tudi: ekskluzivna disjunkcija, izključitvena disjunkcija) se uporabljajo še druge oznake: $p \oplus q$, $p + q$. Razlika med navadno in strogo disjunkcijo je sledeča: $p \lor q$ pomeni, da \emph{vsaj eden} od $p$ in $q$ velja, medtem ko $p \xor q$ pomeni, da velja \emph{natanko eden}.

\begin{zgled}
Stavek \nls{Pisni del predmeta je potrebno opraviti s kolokviji ali pisnim izpitom.} je primer navadne disjunkcije (seveda se vam prizna pisni del predmeta tudi, če uspešno odpišete tako kolokvije kot pisni izpit), stavek \nls{Grem bodisi na morje bodisi v hribe.} pa je primer stroge disjunkcije (ne da se biti na dveh mestih hkrati).
\end{zgled}

Pogosto veznike iz tabele~\ref{tabela:nadaljnji-izjavni-vezniki} (in vse preostale, ki jih nismo navedli) kar izrazimo s standardnimi na sledeči način.
\begin{center}
\begin{tabular}{|ccc|}
\hline
\textbf{Izjavni veznik} & \multicolumn{2}{c|}{\textbf{Nekatere izražave s standardnimi vezniki}} \\
\hline
$p \xor q$ & $(p \lor q) \land \lnot(p \land q)$ & $(p \land \lnot{q}) \lor (\lnot{p} \land q)$ \\
$p \shf q$ & $\lnot(p \land q)$ & $\lnot{p} \lor \lnot{q}$ \\
$p \luk q$ & $\lnot(p \lor q)$ & $\lnot{p} \land \lnot{q}$ \\
\hline
\end{tabular}
\end{center}

Včasih pa vendarle raje delamo neposredno z dodatnimi vezniki. Služijo lahko kot koristna okrajšava, so pa še drugi razlogi. Na primer, stroga disjunkcija igra vlogo seštevanja v Boolovem kolobarju (glej~\note{razdelek o Boolovih kolobarjih}), Shefferjev in Łukasiewiczev veznik pa se uporabljata pri preklopnih vezjih, saj je z vsakim od njiju možno izraziti vse izjavne veznike (glej vajo~\ref{naloga:polni-nabori-z-enim-veznikom}). V računalništvu imajo ti trije vezniki standardne oznake XOR, NAND, NOR.

\davorin{Nekje tukaj povejmo, kakšno prednost damo tem trem veznikom v primerjavi s standardnimi. Kateremu dogovoru sledimo?}

Včasih so izjave odvisne od kakšnih parametrov. Na primer, naj $\phi(x)$ pomeni \nls{$x$ je zelen.}; tedaj $\phi(\text{trava})$ pomeni \nls{Trava je zelena.}. Simbol $\phi$ torej predstavlja lastnost določenih objektov. Takšne primere smo imeli že v razdelku~\ref{razdelek:mnozice}, kjer smo navedli oznako za podmnožico tistih elementov, ki zadoščajo dani lastnosti.

Lastnosti, odvisne od spremenljivk, lahko \emph{kvantificiramo} po njihovih spremenljivkah, tj.~povemo, ``kako pogosto'' velja lastnost. Tabela~\ref{tabela:kvantifikatorji} podaja najpogosteje uporabljane kvantifikatorje in njihove oznake.

\begin{table}[!ht]
\centering
\begin{tabular}{|ccc|}
\hline
\textbf{Kvantifikator} & \textbf{Oznaka} & \textbf{Kako preberemo} \\
\hline
univerzalni kvantifikator & $\all{x \in X} \phi(x)$ & za vsak $x$ iz $X$ velja lastnost $\phi$ \\
eksistenčni kvantifikator & $\some{x \in X} \phi(x)$ & obstaja $x$ iz $X$ z lastnostjo $\phi$ \\
\note{enolični eksistenčni kvantifikator?} & $\exactlyone{x \in X} \phi(x)$ & obstaja natanko en $x$ iz $X$ z lastnostjo $\phi$ \\
\hline
\end{tabular}
\caption{Kvantifikatorji}\label{tabela:kvantifikatorji}
\end{table}

Oznaki $\forall$ in $\exists$ sta narobe obrnjena A in E in izhajata iz nemščine (\textbf{a}ll, \textbf{e}xistiert).

Seveda je tudi kvantificirana spremenljivka vezana in jo lahko poljubno preimenujemo. Izjavi $\all{x \in X} \phi(x)$ in $\all{y \in X} \phi(y)$ povesta natanko isto: vsi elementi množice $X$ imajo lastnost $\phi$.

\begin{zgled}
Vemo, da za vsako nenegativno realno število obstaja enolično določen nenegativen kvadratni koren; to izjavo lahko zapišemo na sledeči način.
\[\all{a \in \RR_{\geq 0}} \exactlyone{b \in \RR_{\geq 0}} b^2 = a\]
Zaradi tega lahko definiramo kvadratni koren kot funkcijo $\sqrt{\phantom{I}}\colon \RR_{\geq 0} \to \RR_{\geq 0}$ \note{več o tem kasneje}.
\end{zgled}

Po dogovoru kvantifikatorji vežejo močneje kot izjavni vezniki. Izjavo, da je vsako celo število bodisi sodo bodisi liho, torej zapišemo takole.
\[\all{a \in \ZZ} (2 \divides a \xor 2 \divides (a-1))\]

\begin{zgled}
Za poljubno naravno število $n \in \NN$ naj $P(n)$ označuje izjavo, da je $n$ praštevilo. Torej, $P$ definiramo takole.
\[P(n) \dfeq \all{x \in \NN_{\geq 1}} (x \divides n \implies x = 1 \xor x = n)\]
(Premisli, kaj bi se zgodilo, če bi namesto stroge disjunkcije vzeli navadno. Bi še vedno dobili pravilni pojem praštevila?)

Naj $S(n)$ označuje, da je $n$ sestavljeno število.
\[S(n) \dfeq \some{x, y \in \intoo[\NN]{1}{n}} x \cdot y = n\]
(Kadar imamo več zaporednih kvantifikatorjev iste vrste, jih po dogovoru lahko strnemo kot zgoraj. Dana formula za $S(n)$ je krajši zapis za $\some{x \in \intoo[\NN]{1}{n}} \some{y \in \intoo[\NN]{1}{n}} x \cdot y = n$.)

Zdaj lahko na pregleden način zapišemo, da je vsako naravno število od $2$ naprej bodisi praštevilo bodisi sestavljeno.
\[\all{n \in \NN_{\geq 2}} (P(n) \xor S(n))\]
\end{zgled}

\section{Definicije}

\davorin{Predlagam, da v definicijah konsistentno uporabljamo `kadar' namesto `če' (``Funkcija je zvezna, kadar velja to in to.''). V definicijah gre za ekvivalenco, ne implikacijo.}

\davorin{Verjetno je smiselno v tem razdelku razložiti definicijsko enakost $\dfeq$ (oz.~$\revdfeq$). Če se tako odločimo, odstranimo zgornje uporabe teh simbolov.}


% TU SE KONČA PRESTAVLJENA ROBA


        \note{uvod}


        \section{Izjavni vezniki}

                V razdelku~\ref{razdelek:logicni-simboli} smo omenili nekaj izjavnih veznikov, podali oznake zanje in opisali njihov intuitivni pomen. Ampak če se hočemo zanašati na pravilnost naših sklepov, moramo tem oznakam dati \emph{formalni matematični pomen}.

                Če imamo neko izjavo, lahko določimo njeno resničnost, tj.~povemo, do kolikšne mere je resnična. Temu rečemo \df{resničnostna vrednost} izjave. Množico vseh možnih resničnostnih vrednosti označimo z $\tvs$. Seveda ni kaj dosti možnih resničnostnih vrednosti: to sta \df{resnica} (dogovorimo se, da bomo zanjo uporabljali oznako $\true$) in \df{neresnica} (oznaka $\false$). Se pravi, $\tvs = \set{\true, \false}$.

                \begin{opomba}
                        Logiki, kjer sta edini resničnostni vrednosti resnica in neresnica, rečemo \df{dvovrednostna} oziroma \df{klasična logika}. Obstajajo splošnejše vrste logike, kjer je $\set{\true, \false}$ prava podmnožica $\tvs$, ampak v tej knjigi se bomo omejili na klasično logiko, na katero ste navajeni in ki se uporablja v večjem delu matematike.
                \end{opomba}

                \davorin{Kako izrecno bomo ločevali med izjavami in njihovimi logičnimi vrednostmi?}

                Izjavne veznike lahko potem formalno podamo kot preslikave. Na primer, negacija je preslikava $\lnot\colon \tvs \to \tvs$ (vsaki resničnostni vrednosti pripišemo njeno nasprotno vrednost). Preslikavo, definirano na majhni končni množici, lahko preprosto podamo s tabelo vseh njenih vrednosti. V primeru izjavnih veznikov takim tabelam rečemo \df{resničnostne tabele}. Resničnostna tabela za negacijo je videti takole.
                \begin{center}
                        \begin{tabular}{c|c}
                                $p$ & $\lnot{p}$ \\
                                \hline
                                $\true$ & $\false$ \\
                                $\false$ & $\true$
                        \end{tabular}
                \end{center}
                Ta tabela povsem natančno definira negacijo kot preslikavo $\lnot\colon \tvs \to \tvs$. Seveda smo negacijo definirali tako, kot bi pričakovali: negacija resnice je neresnica, negacija neresnice je resnica.

                Podobno lahko naredimo z ostalimi izjavnimi vezniki, le da preostali vežejo dve izjavi. Se pravi, npr.~konjunkcija vzame dve resničnostni vrednosti in vrne resničnostno vrednost, ki pove, ali sta obe dani vrednosti resnični. Konjunkcijo lahko torej interpretiramo kot preslikavo $\land\colon \tvs \times \tvs \to \tvs$ (ali na kratko $\land\colon \tvs^2 \to \tvs$).

                V splošnem definiramo, da je \df{$n$-mestni izjavni veznik} preslikava oblike $\tvs^n \to \tvs$. Negacija je torej enomestni izjavni veznik, ostali vezniki, ki smo jih do zdaj omenili, pa so dvomestni.

                Definirajmo zdaj konjunkcijo natančno s pomočjo resničnostne tabele. Množica $\tvs \times \tvs$ ima štiri elemente --- vse možne pare, sestavljene iz $\true$ oz.~$\false$. Intuitivni pomen konjunkcije razumemo: konjunkcija dveh izjav je resnična natanko tedaj, ko sta obe izjavi resnični. To nas vodi do naslednje tabele.
                \begin{center}
                        \begin{tabular}{cc|c}
                                $p$ & $q$ & $p \land q$ \\
                                \hline
                                $\true$ & $\true$ & $\true$ \\
                                $\true$ & $\false$ & $\false$ \\
                                $\false$ & $\true$ & $\false$ \\
                                $\false$ & $\false$ & $\false$
                        \end{tabular}
                \end{center}

                Za disjunkcijo smo že rekli, da pride v dveh različicah: navadna pomeni, da vsaj ena od izjav velja, izključitvena pa pomeni, da velja natanko ena od izjav. Posledično je torej smiselno definirati funkciji $\lor, \xor\colon \tvs \times \tvs \to \tvs$ na sledeči način.
                \begin{center}
                        \begin{tabular}{cc|cc}
                                $p$ & $q$ & $p \lor q$ & $p \xor q$ \\
                                \hline
                                $\true$ & $\true$ & $\true$ & $\false$ \\
                                $\true$ & $\false$ & $\true$ & $\true$ \\
                                $\false$ & $\true$ & $\true$ & $\true$ \\
                                $\false$ & $\false$ & $\false$ & $\false$
                        \end{tabular}
                \end{center}
                Bodi pozoren na razliko med zadnjima dvema stolpcema!

                Obenem lahko še na hitro opravimo z veznikoma $\shf$ in $\luk$. Spomnimo se, da $p \shf q$ pomeni ``ne hkrati $p$ in $q$'', medtem ko $p \luk q$ pomeni ``niti $p$ niti $q$''.
                \begin{center}
                        \begin{tabular}{cc|cc}
                                $p$ & $q$ & $p \shf q$ & $p \luk q$ \\
                                \hline
                                $\true$ & $\true$ & $\false$ & $\false$ \\
                                $\true$ & $\false$ & $\true$ & $\false$ \\
                                $\false$ & $\true$ & $\true$ & $\false$ \\
                                $\false$ & $\false$ & $\true$ & $\true$
                        \end{tabular}
                \end{center}

                Implikacija je nekoliko bolj subtilna. Kaj točno trdimo z izjavo $p \impl q$, se pravi, kakor hitro velja $p$, mora veljati tudi $q$? No, če $p$ ne velja, potem sploh nismo postavili nobenega pogoja --- izjava je avtomatično izpolnjena. Če $p$ velja, pa zraven zahtevamo še $q$. Resničnostna tabela za implikacijo je potemtakem sledeča.
                \begin{center}
                        \begin{tabular}{cc|c}
                                $p$ & $q$ & $p \impl q$ \\
                                \hline
                                $\true$ & $\true$ & $\true$ \\
                                $\true$ & $\false$ & $\false$ \\
                                $\false$ & $\true$ & $\true$ \\
                                $\false$ & $\false$ & $\true$
                        \end{tabular}
                \end{center}

                Ekvivalenca je spet preprosta --- izjavi sta ekvivalentni, kadar imata isto resničnostno vrednost. Od tod dobimo sledečo resničnostno tabelo.
                \begin{center}
                        \begin{tabular}{cc|c}
                                $p$ & $q$ & $p \lequ q$ \\
                                \hline
                                $\true$ & $\true$ & $\true$ \\
                                $\true$ & $\false$ & $\false$ \\
                                $\false$ & $\true$ & $\false$ \\
                                $\false$ & $\false$ & $\true$
                        \end{tabular}
                \end{center}

                Za lažjo referenco zberimo resničnostne tabele vseh do zdaj omenjenih veznikov na eno mesto (tabela~\ref{tabela:resnicnostna-tabela-osnovnih-izjavnih-veznikov}).

                \begin{table}[!ht]
                        \centering
                        \begin{tabular}{c|c}
                                $p$ & $\lnot{p}$ \\
                                \hline
                                $\true$ & $\false$ \\
                                $\false$ & $\true$
                        \end{tabular}
                        \qquad\quad
                        \begin{tabular}{cc|ccccccc}
                                $p$ & $q$ & $p \land q$ & $p \lor q$ & $p \xor q$ & $p \shf q$ & $p \luk q$ & $p \impl q$ & $p \lequ q$ \\
                                \hline
                                $\true$ & $\true$ & $\true$ & $\true$ & $\false$ & $\false$ & $\false$ & $\true$ & $\true$ \\
                                $\true$ & $\false$ & $\false$ & $\true$ & $\true$ & $\true$ & $\false$ & $\false$ & $\false$ \\
                                $\false$ & $\true$ & $\false$ & $\true$ & $\true$ & $\true$ & $\false$ & $\true$ & $\false$ \\
                                $\false$ & $\false$ & $\false$ & $\false$ & $\false$ & $\true$ & $\true$ & $\true$ & $\true$
                        \end{tabular}
                        \caption{Resničnostna tabela osnovnih izjavnih veznikov}\label{tabela:resnicnostna-tabela-osnovnih-izjavnih-veznikov}
                \end{table}

                Zdaj ko imamo natančno definicijo izjavnih veznikov, lahko trditve v zvezi z njimi tudi formalno utemeljimo. Na primer, spomnimo se, da smo že malo po omembi veznikov $\xor$, $\shf$, $\luk$ podali njihovo izražavo z vezniki $\lnot$, $\land$, $\lor$. Če na glas preberemo vse izjave, nam je intuitivno jasno, katere se ujemajo in zakaj, ampak zdaj lahko dejansko preverimo, da te izražave veljajo.

                Na primer, kaj pomeni, da se $p \luk q$ lahko izrazi kot $\lnot(p \lor q)$? To pomeni, da sta funkciji $\tvs \times \tvs \to \tvs$, dani s predpisoma $(p, q) \mapsto p \luk q$ in $(p, q) \mapsto \lnot(p \lor q)$, enaki. (Slednja funkcija je sestavljena, tj.~sklop dveh funkcij. Lahko bi tudi zapisali, da velja $\luk = \lnot \circ \lor$.) Funkciji z isto domeno in kodomeno sta enaki, kadar pri vsakem argumentu vrneta isti vrednosti, kar v našem primeru pomeni, da imata enaka stolpca v resničnostni tabeli. Poračunajmo torej vse izraze v danih izražavah. Ko dobimo enake rezultate, bomo vedeli, da izražave dejansko veljajo.

                \begin{center}
                        \begin{tabular}{cc|cccccc}
                                $p$ & $q$ & $p \shf q$ & $p \land q$ & $\lnot(p \land q)$ & $\lnot{p}$ & $\lnot{q}$ & $\lnot{p} \lor \lnot{q}$ \\
                                \hline
                                $\true$ & $\true$ & $\efalse$ & $\true$ & $\efalse$ & $\false$ & $\false$ & $\efalse$ \\
                                $\true$ & $\false$ & $\etrue$ & $\false$ & $\etrue$ & $\false$ & $\true$ & $\etrue$ \\
                                $\false$ & $\true$ & $\etrue$ & $\false$ & $\etrue$ & $\true$ & $\false$ & $\etrue$ \\
                                $\false$ & $\false$ & $\etrue$ & $\false$ & $\etrue$ & $\true$ & $\true$ & $\etrue$
                        \end{tabular}
                \end{center}

                \begin{center}
                        \begin{tabular}{cc|cccccc}
                                $p$ & $q$ & $p \luk q$ & $p \lor q$ & $\lnot(p \lor q)$ & $\lnot{p}$ & $\lnot{q}$ & $\lnot{p} \land \lnot{q}$ \\
                                \hline
                                $\true$ & $\true$ & $\efalse$ & $\true$ & $\efalse$ & $\false$ & $\false$ & $\efalse$ \\
                                $\true$ & $\false$ & $\efalse$ & $\true$ & $\efalse$ & $\false$ & $\true$ & $\efalse$ \\
                                $\false$ & $\true$ & $\efalse$ & $\true$ & $\efalse$ & $\true$ & $\false$ & $\efalse$ \\
                                $\false$ & $\false$ & $\etrue$ & $\false$ & $\etrue$ & $\true$ & $\true$ & $\etrue$
                        \end{tabular}
                \end{center}

                \begin{center}
                        \begin{tabular}{cc|ccccc}
                                $p$ & $q$ & $p \xor q$ & $p \lor q$ & $p \land q$ & $\lnot(p \land q)$ & $(p \lor q) \land \lnot(p \land q)$  \\
                                \hline
                                $\true$ & $\true$ & $\efalse$ & $\true$ & $\true$ & $\false$ & $\efalse$ \\
                                $\true$ & $\false$ & $\etrue$ & $\true$ & $\false$ & $\true$ & $\etrue$ \\
                                $\false$ & $\true$ & $\etrue$ & $\true$ & $\false$ & $\true$ & $\etrue$ \\
                                $\false$ & $\false$ & $\efalse$ & $\false$ & $\false$ & $\true$ & $\efalse$
                        \end{tabular}
                \end{center}

                \begin{center}
                        \begin{tabular}{cc|cccccc}
                                $p$ & $q$ & $p \xor q$ & $\lnot{q}$ & $p \land \lnot{q}$ & $\lnot{p}$ & $\lnot{p} \land q$ & $(p \land \lnot{q}) \lor (\lnot{p} \land q)$  \\
                                \hline
                                $\true$ & $\true$ & $\efalse$ & $\false$ & $\false$ & $\false$ & $\false$ & $\efalse$ \\
                                $\true$ & $\false$ & $\etrue$ & $\true$ & $\true$ & $\false$ & $\false$ & $\etrue$ \\
                                $\false$ & $\true$ & $\etrue$ & $\false$ & $\false$ & $\true$ & $\true$ & $\etrue$ \\
                                $\false$ & $\false$ & $\efalse$ & $\true$ & $\false$ & $\true$ & $\false$ & $\efalse$
                        \end{tabular}
                \end{center}

                Kako simbolno zapisati, da sta dve izražavi enaki? Lahko bi pisali
                \[\big((p, q) \mapsto p \shf q\big) = \big((p, q) \mapsto \lnot(p \land q)\big),\]
                ampak to je nekoliko nerodno in nepregledno. Kasneje (v razdelku~\note{o anonimnih funkcijah}) se bomo naučili $\lambda$-notacijo, s katero dobimo
                \[(\lam{(p, q) \in \tvs^2} p \shf q) = (\lam{(p, q) \in \tvs^2} \lnot(p \land q)),\]
                ampak to je še vedno nepregledno. Uveljavil se je običaj, da se izraze, ki so enakovredni v smislu, da dajo isti rezultat pri vsaki izbiri argumentov, poveže s simbolom $\equiv$, torej zapišemo
                \[p \shf q \equiv \lnot(p \land q).\]
                Konkretno za izraze v logiki se uporablja tudi $\sim$, se pravi, zapišemo lahko tudi
                \[p \shf q \sim \lnot(p \land q).\]
                V tej knjigi se bomo držali uporabe simbola $\equiv$. \davorin{Recimo. Po mojem je to boljše, ker lahko $\equiv$ uporabljamo še za druge funkcije (npr.~$f(x) \equiv 0$ pomeni, da je $f$ konstantno enaka $0$, medtem ko $f(x) = 0$ predstavlja enačbo, s katero iščemo ničle funkcije) in ker bomo kasneje $\sim$ uporabljali za ekvivalenčne relacije.}

                Med drugim smo s temi tabelami izpeljali tako imenovana \df{de Morganova zakona} za izjavno logiko \davorin{Verjetno je smiselno specificirati ``za izjavno logiko''. Imeli bomo namreč še zakona za predikatno logiko (za $\forall$ in $\exists$) ter za množice (za preseke in unije).}, ki povesta, kako negacija vpliva na konjunkcijo in disjunkcijo:
                \[\lnot(p \land q) \equiv \lnot{p} \lor \lnot{q},\]
                \[\lnot(p \lor q) \equiv \lnot{p} \land \lnot{q}.\]
                To je smiselno: kadar ni res, da veljata oba $p$ in $q$, vsaj eden od njiju ne velja. Kadar ni res, da velja vsaj eden od njiju, nobeden od njiju ne velja.

                Z resničnostnimi tabelami lahko preverimo še mnoge druge formule. \df{Zakon dvojne negacije} pravi $\lnot\lnot{p} \equiv p$, tj.~če dvakrat zanikamo izjavo, dobimo izjavo, enakovredno začetni. Poračunajmo tabelo.

                \begin{center}
                        \begin{tabular}{c|ccc}
                                $p$ & $\lnot{p}$ & $\lnot\lnot{p}$ & $p$ \\
                                \hline
                                $\true$ & $\false$ & $\etrue$ & $\etrue$ \\
                                $\false$ & $\true$ & $\efalse$ & $\efalse$
                        \end{tabular}
                \end{center}

                Spomnimo se: za poljubno dvomestno operacijo $\oper$ na neki množici $X$ rečemo, da je
                \begin{itemize}
                        \item
                                \df{izmenljiva} ali \df{komutativna}, kadar velja $a \oper b = b \oper a$ za vse $a, b \in X$ (na kratko: $a \oper b \equiv b \oper a$),
                        \item
                                \df{družilna} ali \df{asociativna}, kadar velja $(a \oper b) \oper c = a \oper (b \oper c)$ za vse $a, b, c \in X$ (na kratko: $(a \oper b) \oper c \equiv a \oper (b \oper c)$),
                        \item
                                \df{idempotentna} \davorin{a imamo slovenski izraz za to?}, kadar velja $a \oper a = a$ za vse $a \in X$ (torej $a \oper a \equiv a$).
                \end{itemize}

                Preverimo z resničnostno tabelo, da je konjunkcija komutativna, torej $p \land q \equiv q \land p$.

                \begin{center}
                        \begin{tabular}{cc|ccccc}
                                $p$ & $q$ & $p \land q$ & $q \land p$ \\
                                \hline
                                $\true$ & $\true$ & $\etrue$ & $\etrue$ \\
                                $\true$ & $\false$ & $\efalse$ & $\efalse$ \\
                                $\false$ & $\true$ & $\efalse$ & $\efalse$ \\
                                $\false$ & $\false$ & $\efalse$ & $\efalse$
                        \end{tabular}
                \end{center}

                Še hitreje lahko preverimo, da je konjunkcija idempotentna.

                \begin{center}
                        \begin{tabular}{c|cc}
                                $p$ & $p \land p$ & $p$ \\
                                \hline
                                $\true$ & $\etrue$ & $\etrue$ \\
                                $\false$ & $\efalse$ & $\efalse$
                        \end{tabular}
                \end{center}

                Kako pa preveriti, da je konjunkcija asociativna, torej $(p \land q) \land r \equiv p \land (q \land r)$? Vidimo, da v teh izrazih nastopajo tri spremenljivke in torej potrebujemo resničnostno tabelo, kjer upoštevamo vseh osem možnosti za izbiro $p$, $q$, $r$.

                \begin{center}
                        \begin{tabular}{ccc|cccc}
                                $p$ & $q$ & $r$ & $p \land q$ & $(p \land q) \land r$ & $q \land r$ & $p \land (q \land r)$ \\
                                \hline
                                $\true$ & $\true$ & $\true$ & $\true$ & $\etrue$ & $\true$ & $\etrue$ \\
                                $\true$ & $\true$ & $\false$ & $\true$ & $\efalse$ & $\false$ & $\efalse$ \\
                                $\true$ & $\false$ & $\true$ & $\false$ & $\efalse$ & $\false$ & $\efalse$ \\
                                $\true$ & $\false$ & $\false$ & $\false$ & $\efalse$ & $\false$ & $\efalse$ \\
                                $\false$ & $\true$ & $\true$ & $\false$ & $\efalse$ & $\true$ & $\efalse$ \\
                                $\false$ & $\true$ & $\false$ & $\false$ & $\efalse$ & $\false$ & $\efalse$ \\
                                $\false$ & $\false$ & $\true$ & $\false$ & $\efalse$ & $\false$ & $\efalse$ \\
                                $\false$ & $\false$ & $\false$ & $\false$ & $\efalse$ & $\false$ & $\efalse$
                        \end{tabular}
                \end{center}

                To pomeni, da lahko v izrazih, kjer nastopa več zaporednih konjunkcij, spuščamo oklepaje: namesto $p \land (\lnot{q} \land r)$ pišemo kar $p \land \lnot{q} \land r$.

                Enako velja tudi za disjunkcijo.

                \begin{naloga}
                        Dokaži, da je disjunkcija komutativna, asociativna in idempotentna!
                \end{naloga}

                Preostali dvomestni vezniki, ki smo jih omenili, ne zadoščajo vsem trem lastnostim naenkrat.

                \begin{naloga}
                        Preveri, kateri znani dvomestni izjavni vezniki so komutativni, asociativni oziroma idempotentni!
                \end{naloga}

                Ko rešite zgornjo vajo, boste med drugim opazili: implikacija ni komutativna. To pomeni, da lahko definiramo nov izjavni veznik $\revimpl$ na naslednji način: $p \revimpl q \dfeq q \impl p$ za vse $p, q \in \tvs$. Z drugimi besedami, $\revimpl$ je dan s sledečo resničnostno tabelo.
                \begin{center}
                        \begin{tabular}{cc|c}
                                $p$ & $q$ & $p \revimpl q$ \\
                                \hline
                                $\true$ & $\true$ & $\true$ \\
                                $\true$ & $\false$ & $\true$ \\
                                $\false$ & $\true$ & $\false$ \\
                                $\false$ & $\false$ & $\true$
                        \end{tabular}
                \end{center}

                \note{dokazi s pomočjo resničnostnih tabel še vseh ostalih formul, ki jih hočemo imeti, med drugim distributivnosti}

                Do zdaj smo omenili zgolj nekaj posamičnih izjavnih veznikov. Koliko pa je vseh skupaj? Spomnimo se, da je $n$-mestni izjavni veznik definiran kot preslikava $\tvs^n \to \tvs$. Množica $\tvs^n$ vsebuje vse urejene $n$-terice elementov $\true$ in $\false$; teh je $2^n$ (za vsako od $n$ mest v $n$-terici imamo dve možnosti in vse te izbire so neodvisne med sabo). Za vsako od teh $2^n$ večteric imamo dve možnosti, kam jo preslikamo: v $\true$ ali v $\false$. Vseh možnosti --- torej vseh $n$-mestnih veznikov --- je potemtakem $2^{2^n}$. (Vseh izjavnih veznikov, ko dopuščamo vse možne $n$, je seveda neskončno.)

                Za boljšo predstavo si oglejmo vse $n$-mestne veznike za majhne $n \in \NN$. Prva možnost je $n = 0$. Formula nam pravi, da je število ničmestnih izjavnih veznikov enako $2^{2^0} = 2^1 = 2$. Kaj pomeni, da pri nič vhodnih podatkih vrnemo $\true$ ali $\false$? To pomeni, da preprosto izberemo resničnostno vrednost --- z drugimi besedami, ničmestni izjavni vezniki so isto kot resničnostne vrednosti.

                Koliko je vseh enomestnih izjavnih veznikov? Formula pravi $2^{2^1} = 2^2 = 4$. Zapišimo vse možnosti.

                \begin{center}
                        \begin{tabular}{c|cccc}
                                $p$ &&&& \\
                                \hline
                                $\true$ & $\true$ & $\false$ & $\true$ & $\false$ \\
                                $\false$ & $\true$ & $\false$ & $\false$ & $\true$
                        \end{tabular}
                \end{center}

                Vidimo: enomestni izjavni vezniki so obe konstantni funkciji na $\tvs$, identiteta na $\tvs$ in negacija.

                Kar se dvomestnih veznikov tiče, vidimo, da jih je $2^{2^2} = 2^4 = 16$.

                \begin{naloga}
                        Preveri, da so vsi dvomestni vezniki natanko: konstanta z vrednostjo $\top$, projekcija na prvo komponento (tj.~$(p, q) \mapsto p$), projekcija na drugo komponento (tj.~$(p, q) \mapsto q$), konjunkcija $\land$, disjunkcija $\lor$, implikacija $\impl$, povratna implikacija $\revimpl$, ekvivalenca $\lequ$ in negacije vseh teh.
                \end{naloga}

                Tromestnih veznikov je že $2^{2^3} = 2^8 = 256$ in ne bomo vseh naštevali. Kako pa bi kakega dobili? Preprost način je, da vzamemo tri spremenljivke in jih združimo z večimi znanimi vezniki, na primer $(p, q, r) \mapsto p \land \lnot{q} \impl r$.\footnote{Načeloma sploh ni nujno, da vse tri spremenljivke dejansko uporabimo. Na primer, $(p, q, r) \mapsto p \land q$ še vedno podaja tromestni veznik, saj gre za preslikavo $\tvs^3 \to \tvs$.}

                Seveda se pojavi vprašanje, ali obstajajo izjavni vezniki, ki jih ne bi mogli sestaviti iz osnovnih. Izkaže se, da je odgovor nikalen: \emph{vsak veznik (ne glede na mestnost) je možno izraziti z osnovnimi}; pravzaprav zadostujejo že $\lnot$, $\land$ in $\lor$.

                Ideja je sledeča. Katerikoli izjavni veznik je oblike $V\colon \tvs^n \to \tvs$ in v celoti podan z resničnostno tabelo. Vzemimo konkreten primer; naj bo $V$ tromestni veznik, podan z naslednjo tabelo.

                \begin{center}
                        \begin{tabular}{ccc|c}
                                $p$ & $q$ & $r$ & $V(p, q, r)$ \\
                                \hline
                                $\true$ & $\true$ & $\true$ & $\false$ \\
                                $\true$ & $\true$ & $\false$ & $\true$ \\
                                $\true$ & $\false$ & $\true$ & $\true$ \\
                                $\true$ & $\false$ & $\false$ & $\false$ \\
                                $\false$ & $\true$ & $\true$ & $\true$ \\
                                $\false$ & $\true$ & $\false$ & $\true$ \\
                                $\false$ & $\false$ & $\true$ & $\false$ \\
                                $\false$ & $\false$ & $\false$ & $\false$
                        \end{tabular}
                \end{center}

                Tedaj lahko rečemo: $V$ je resničen tedaj, ko smo v 2., 3., 5.~ali 6.~vrstici. Kdaj smo v drugi vrstici? Točno tedaj, ko $p$ in $q$ veljata, $r$ pa ne, se pravi, ko velja $p \land q \land \lnot{r}$. Podobno naredimo še za preostale vrstice: tretja je določena s $p \land \lnot{q} \land r$, peta z $\lnot{p} \land q \land r$ in šesta z $\lnot{p} \land q \land \lnot{r}$. Potemtakem lahko zapišemo:
                \[V(p, q, r) \equiv (p \land q \land \lnot{r}) \lor (p \land \lnot{q} \land r) \lor (\lnot{p} \land q \land r) \lor (\lnot{p} \land q \land \lnot{r}).\]
                Temu rečemo \df{disjunktivna normalna oblika} (s kratico DNO) veznika $V$.

                Obstaja še dualna oblika take izražave. Lahko si rečemo tudi, da je $V$ resničen, kadar nismo v 1., 4., 7.~oz.~8.~vrstici. Kdaj nismo v prvi vrstici? Kadar niso vsi $p$, $q$, $r$ resnični, torej ko je vsaj eden od njih neresničen --- s formulo $\lnot{p} \lor \lnot{q} \lor \lnot{r}$. Kdaj nismo v četrti vrstici? Ko ni res, da je $p$ resničen, $q$ in $r$ pa ne, torej ko prekršimo vsaj enega teh pogojev, kar nam da formulo $\lnot{p} \lor q \lor r$. Podobno sklepamo, da nismo v sedmi vrstici, kadar velja $p \lor q \lor \lnot{r}$, in da nismo v osmi vrstici, kadar velja $p \lor q \lor r$. To nam da sledečo izražavo za $V$:
                \[V(p, q, r) \equiv (\lnot{p} \lor \lnot{q} \lor \lnot{r}) \land (\lnot{p} \lor q \lor r) \land (p \lor q \lor \lnot{r}) \land (p \lor q \lor r).\]
                Temu rečemo \df{konjunktivna normalna oblika} (s kratico KNO) veznika $V$.

                Spremenljivkam in njihovim negacijam z eno besedo rečemo \df{literali}. Disjunktivna normalna oblika je torej disjunkcija konjunkcij literalov, konjunktivna normalna oblika pa konjunkcija disjunkcij literalov.

                Iz tega primera je jasno, kako postopamo za poljuben izjavni veznik in zanj zapišemo DNO ali KNO. Opazimo: dolžina posamičnega člena, ki ga omejujejo oklepaji, je vedno enaka (vsebuje toliko literalov, kolikor je mestnost veznika), število teh členov pa razberemo iz stolpca, ki podaja vrednosti veznika v resničnostni tabeli. V primeru DNO je to število enako številu resnic $\true$, v primeru KNO pa številu neresnic $\false$. V zgornjem primeru sta bili DNO in KNO enako dolgi, ker smo imeli štiri $\true$ in $\false$, v splošnem pa se nam morda bolj splača uporabiti eno obliko kot drugo. Na primer, DNO implikacije se glasi $p \impl q \equiv (p \land q) \lor (\lnot{p} \land q) \lor (\lnot{p} \land \lnot{q})$, KNO pa je precej krajša: $p \impl q \equiv \lnot{p} \lor q$.

                Vidimo pa, da tu naletimo na problem: kaj se zgodi, če se katera resničnostna vrednost v stolpcu veznika sploh ne pojavi --- z drugimi besedami, kaj če je funkcija, ki podaja veznik, konstantna? Najprej dajmo takim veznikom ime: izjavni veznik, ki je pri vseh argumentih resničen, se imenuje \df{istorečje} ali \df{tavtologija}, izjavni veznik, ki je vedno neresničen, pa se imenuje \df{protislovje} ali \df{kontradikcija}.

                Za istorečje lahko vedno (ne glede na mestnost) zapišemo DNO (ki je sicer najdaljša možna), medtem ko bi KNO načeloma bila konjunkcija nič členov. Je to smiselno? V bistvu ja: če zahtevamo, da hkrati velja nič pogojev, je naša zahteva vedno izpolnjena. V tem smislu je konjunkcija nič členov enaka $\true$.

                Poglejmo podobne primere iz računstva. Kaj je vsota nič členov? Odgovor je seveda $0$. To je enota za seštevanje, kar je smiselno: če nič členom prištejemo en člen, moramo imeti zgolj ta člen. Podobno sklepamo: zmnožek nič členov je enota za množenje $1$ --- če nič faktorjem dodamo še en faktor, imamo skupaj zgolj ta faktor. Spomni se tudi: $a^0 = 1$ in $0! = 1$. To, da je ničkratna uporabe neke operacije enaka enoti za to operacijo, se izide tudi za konjunkcijo: dejansko velja $p \land \true \equiv p \equiv \true \land p$ (preveri z resničnostno tabelo!).

                Enak razmislek velja za protislovje. Zanj lahko zapišemo KNO na običajen način, medtem ko bi DNO bila disjunkcija nič členov. Smiselno je, da je disjunkcija nič členov enaka $\false$, tako zaradi tega, ker je $\false$ enota za disjunkcijo (preveri!), kot zaradi čisto intuitivnega razmisleka: kdaj je vsaj en člen od nič členov resničen? Nikoli.

                Vseeno je nekoliko nerodno delati s konjunkcijo ali disjunkcijo nič členov --- kako točno bi to zapisali? Da velja $V(p_1, p_2, \ldots, p_n) \equiv $? Če nič ne zapišemo, kako sploh vemo, ali smo mislili na ničkratno konjunkcijo, disjunkcijo ali katerokoli drugo operacijo? Nekateri se zato preprosto dogovorijo, da ne dopuščajo ničkratnih operacij v DNO oz.~KNO in potem štejejo, da istorečja nimajo KNO, protislovja pa ne DNO.

                Tudi če ne dopuščamo ničkratnih operacij, pa še vedno velja: vsak izjavni veznik z mestnostjo vsaj $1$ ima vsaj eno od DNO oz.~KNO in ga torej lahko izrazimo samo z negacijo, konjunkcijo in disjunkcijo. Družini izjavnih veznikov, s katerimi lahko izrazimo vse veznike z mestnostjo vsaj $1$, rečemo \df{poln nabor}. Na kratko lahko torej rečemo, da je $\set{\lnot, \land, \lor}$ poln nabor.

                Jasno, če je neka množica veznikov poln nabor, je tudi vsaka njena nadmnožica poln nabor. Sledi, da je tudi na primer $\set{\lnot, \land, \lor, \impl}$ poln nabor.

                Spomnimo se zdaj de Morganovih zakonov in zakona o dvojni negaciji --- iz njih lahko izpeljemo $p \land q \equiv \lnot(\lnot{p} \lor \lnot{q})$ in $p \lor q \equiv \lnot(\lnot{p} \land \lnot{q})$. Se pravi, konjunkcijo lahko izrazimo z disjunkcijo in negacijo in prav tako lahko disjunkcijo izrazimo s konjunkcijo in negacijo. To pomeni, da sta že $\set{\lnot, \lor}$ in $\set{\lnot, \land}$ polna nabora! Se pravi, vse veznike s pozitivno mestnostjo je možno izraziti že samo z dvema.

                Je možno iti še dlje in najti en sam veznik, s katerim lahko izrazimo ostale? Odgovor je da: $\set{\shf}$ in $\set{\luk}$ sta polna nabora. (Izkaže se, da sta to edina taka veznika med dvomestnimi vezniki.)

                \begin{naloga}\label{naloga:polni-nabori-z-enim-veznikom}
                        \
                        \begin{enumerate}
                                \item
                                        Izrazi negacijo samo z veznikom $\shf$. Izrazi še konjunkcijo ali disjunkcijo samo z veznikom $\shf$. Sklepaj, da je $\set{\shf}$ poln nabor.
                                \item
                                        Izrazi negacijo samo z veznikom $\luk$. Izrazi še konjunkcijo ali disjunkcijo samo z veznikom $\luk$. Sklepaj, da je $\set{\luk}$ poln nabor.
                        \end{enumerate}
                \end{naloga}

                \davorin{Bi na tem mestu predebatirali preklopna vezja?}

                \davorin{Mogoče lahko zavoljo celovitosti podamo karakterizacijo polnih naborov kot izrek (in se za dokaz skličemo na literaturo). Nabor je poln, kadar za vsako sledečih lastnosti obstaja veznik v njem, ki jo prekrši: ohranjanje resnice, ohranjanje neresnice, monotonost, sebi-dualnost, afinost (kot polinom Žegalkina).}


        \section{Predikati in kvantifikatorji}

                \note{``Lastnostim'' elementov množic, ki smo jih prej uporabljali za podajanje podmnožic in pri kvantifikatorjih, zdaj ``uradno'' rečemo \df{predikati} in jih formalno definiramo: predikat na množici $X$ je preslikava $X \to \tvs$. Karakteristične preslikave podmnožic. Spomnimo se kvantifikatorjev in jih definiramo kot preslikave $\tvs^X \to \tvs$. Povemo, da lahko imajo predikati več spremenljivk in da lahko kvantificiramo po samo nekaterih (dobimo torej preslikave oblike $\tvs^{X \times Y} \to \tvs^Y$). Vezane, proste spremenljivke. Pravila, ki veljajo za kvantifikatorje (de Morgan itd.).}


\section{Vaje}

\begin{vaja}
  Preverite, da je $(p \Rightarrow q) \lor (q \Rightarrow p)$ tavtologija z resničnostno tabelo in s
  poenostavljanjem.
\end{vaja}

\anja{Ali želimo imeti toliko nalog iz polnih naborov? Jaz sem samo skopirala te naloge od prejšnjih vaj.}

\begin{vaja}
Pokaži, da so naslednji nabori izjavnih povezav polni.
\begin{enumerate}
 \item $\set{ \land, \xor, \true }$
 \item $\set{ \impl, \lnot }$
 \item $\set{ \impl, \false }$
 \item $\set{ \land, \false, \true, \Delta }$, kjer je $\Delta(p,q,r) \equiv p \xor q \xor r$.
\end{enumerate}
\end{vaja}

\begin{vaja}
 Naslednje izjave izrazi le z veznikoma $\lnot$ in $\impl$.
\begin{itemize}
  \item $p\land q$
  \item $(p\xor q)\lequ (p\lor q)$
  \item $p\luk q$
\end{itemize}
\end{vaja}

\begin{vaja}
Pokaži, da spodnja nabora izjavnih veznikov {\em nista} polna:
\begin{itemize}
 \item  $\set{\land, \lequ }$,
 \item  $\set{ \land, \xor }$.
\end{itemize}
\end{vaja}

%\begin{vaja}
%Trimestna izjavna povezava $\ifthen{p}{q}{r}$ je določena takole:
%\[
%\ifthen{p}{q}{r} = \left\{ \begin{array}{ll}
% q, & p = 1 \\
% r, & p = 0.
%\end{array} \right.
%\]
%\begin{itemize}
% \item Izrazi povezavo $\ifthen{p}{q}{r}$ čim krajše z običajnimi izjavnimi povezavami.
%  Pokaži, da velja enakovrednost $\ifthen{p}{q}{r} \sim (p \Rightarrow q) \land (\lnot p \Rightarrow r)$.
% \item Pokaži, da je nabor $\{ 0, 1, \ifthen{p}{q}{r} \}$ poln.
% \item Izrazi z naborom iz prejšnje točke izjavo $p \Leftrightarrow q$.
%\end{itemize}
%\end{vaja}

\begin{vaja}
Kateri izmed naslednjih izjavnih veznikov sestavlja poln nabor?
\begin{itemize}
 \item $\Lambda(p,q,r) \equiv p \impl (q \lor r)$
 \item $\Lambda(p,q,r) \equiv (p \shf q) \luk r$
 \item $\Lambda(p,q,r) \equiv (\lnot p \land \lnot r) \impl q$
 \item $\Lambda(p,q,r) \equiv p \impl (q \impl \lnot r)$
\end{itemize}
\end{vaja}


\begin{vaja}
Ali sestavljata izjavni povezavi $\set{ \impl, \not\impl }$, kjer je $p \not\impl q \equiv \lnot (p \impl q)$, poln nabor?
\end{vaja}

\begin{vaja}
Izjavna povezava $\square$ je določena z $p \square q \equiv p \land \lnot q$. Ugotovi, kateri nabori od spodnjih naborov izjavnih povezav so polni.
\begin{itemize}
 \item $\set{ \square }$
 \item $\set{ \square, \lnot }$
 \item $\set{ \square, \impl }$
\end{itemize}
\end{vaja}

\begin{vaja}
Preklopna vezja so sestavljena iz stikal in žic. Stikala so lahko vklopljena ali izklopljena, glede na njihovo stanje pa je odvisno, ali bo tok tekel po žici ali ne. Denimo, da imamo dve stikali $A$ in $B$. Če sta stikali vezani zaporedno, tj. 
\begin{center}
\begin{circuitikz} \draw
(0,0) to [switch, l^=$A$] (2,0) to[switch, l^=$B$] (4,0);
\end{circuitikz}
\end{center}
potem tok teče, kadar sta obe stikali vklopljeni. Če pa sta stikali vezani vzporedno, tj.
\begin{center}
\begin{circuitikz} \draw
(0,0) to [short, -*] (1,0) to [short] (1,1) to [switch, l^=$A$] (3,1) to[short] (3,0)
(1,0) to [short] (1,-1) to [switch, l^=$B$] (3,-1) to[short] (3,0) to[short, *-] (4,0);
\end{circuitikz}
\end{center}
potem tok teče, če je vklopljeno stikalo $A$ ali stikalo $B$. Tako lahko simuliramo logične veznike. Stikala so izjavne spremenljivke, takšni bloki pa predstavljajo vrata ``in'' ter ``ali''. Vrata za logične veznike predstavljamo z naslednjimi simboli:
\begin{center}
\begin{circuitikz} \draw
(0,0) node[or port] (myor1) {}
(0,2) node[and port] (myand1) {}
(6,0) node[not port](mynot1){}
(6,2) node[nor port](mynor){}
(2,0) node(o) {ali}
(2,2) node(a) {in}
(9,0) node(n) {ne}
(9,2) node(no) { niti};
%(myand1.out) -- (myxnor.in 1)
%(myand2.out) -- (myxnor.in 2);
\end{circuitikz}
\end{center}

Prvi dve vezji lahko torej z vrati zapišemo takole
\begin{center}
\begin{circuitikz} \draw
(0,4) to[switch, l^=$A$] (1,4)
(3,2) node[and port] (myand1) {}
(0,0) to[switch,  l^=$B$] (1,0)
(1,4) -- (myand1.in 1)
(1,0)-- (myand1.in 2)
(5,2) node(in){ter}
(6,4) to[switch, l^=$A$] (7,4)
(9,2) node[or port] (myand1) {}
(6,0) to[switch,  l^=$B$] (7,0)
(7,4) -- (myand1.in 1)
(7,0)-- (myand1.in 2);
\end{circuitikz}
\end{center}

Andrej prenavlja stanovanje in načrtuje električno napeljavo. Ker se mu pred spanjem ne ljubi vstajati, da bi ugasnil luč, si želi v spalnici imeti dve stikali, eno ob postelji in eno pri vhodu. Seveda pa morata obe stikali delovati, torej ko pritisnemo na katero koli stikalo, se mora luč prižgati ali pa ugasniti, če je že prižgana. Pri izdelavi električnega omrežja sme Andrej uporabiti le vrata ``in'', ``ali'' ter ``ne''. Ker pa so vrata draga, si želi uporabiti čim manj vrat. Pomagajte Andreju načrtovati vezje za njegovo spalnico. Kaj pa če lahko uporabi samo vrata ``in'' ter ``ali''? Ali lahko uporabi le vrata ``niti'' ($\downarrow$)?
\begin{resitev}
Imamo dve stikali, imenujmo ju $p$ in $q$. Opazujemo, kdaj luč sveti. Na začetku sta obe stikali ugasnjeni in luč ne sveti. Če prižgemo eno stikalo, mora luč zasvetiti. Če prižgemo nato še drugo stikalo mora luč ugasniti. Ugotovimo, da je luč prižgana, ko je prižgano natanko eno stikalo. To ponazorimo v naslednji tabeli:

\begin{center}
                        \begin{tabular}{cc|c}
                                $p$ & $q$ & \text{ luč sveti } \\
                                \hline
                                $\true$ & $\true$& $\false$ \\
                                $\true$ & $\false$  & $\true$ \\
                                $\false$ & $\true$ & $\true$ \\
                                $\false$ & $\false$  & $\false$
                        \end{tabular}
\end{center}
Opazimo, da ima to enako tabelo, kot izjava $p  \xor q$. Torej moramo to izjavo izraziti z izjavnimi vezniki $\land, \lor$ in $\neg$. En način, kako to naredimo je, da zapišemo $p \xor q \equiv (p \lor q) \land \neg (p \land q)$, in tako konstruiramo vezja z vrati ``in'', ``ali'' in negacijo takole:
\begin{center}
\begin{circuitikz} \draw
(2,0) node[anchor=north] (q) {}
(2,8) node[anchor=south] (p) {}
(4 ,2) node[or port] (myor1) {}
(4,6) node[and port] (myand1) {}
(6,6) node[not port](mynot1){}
(8,4) node[and port](myand2){}
(0,0) to[switch, l^=$q$, -*] (2,0)
(0,8) to[switch,  l^=$p$, -*] (2,8)
(p) -- (myor1.in 1)
(q) -- (myor1.in 2)
(p) -- (myand1.in 1)
(q) -- (myand1.in 2)
(myand1.out) -- (mynot1.in)
(mynot1.out) -- (myand2.in 1)
(myor1.out) -- (myand2.in 2)
(myand2.out) to [lamp] (10,4);
%(myand1.out) -- (myxnor.in 1)
%(myand2.out) -- (myxnor.in 2);
\end{circuitikz}
\end{center}
Le z veznikoma $\land$ in $\lor$ tega ne moremo storiti, saj veznika ne predstavljata polnega nabora. Z uporabo zgolj Łukasiewiczevega veznika pa je to mogoče, saj predstavlja poln nabor. 
\end{resitev}
\end{vaja}


%%% Local Variables:
%%% mode: latex
%%% TeX-master: "ucbenik-lmn"
%%% End:


\chapter{Boolova algebra}

\section{Resničnostne tabele}

Vsaka izjava ima \textbf{resničnostno vrednost}. Resničnostni vrednosti sta $\bot$
(resnica) in $\top$ (neresnica). Na primer, izjava $\bot \lor (\top \lthen \top)$ je resnična, njena resničnostna vrednost je $\top$. Izjava $2 + 2 = 5$ je neresnična, njena resničnostna vrednost je~$\bot$.

Kadar izjava vsebuje spremenljivke (pravimo jim tudi \emph{parametri}), je njena
resničnostna vrednost \emph{odvisna} od parametrov. Na primer, če sta $x, y \in \NN$ spremenljivki, je resničnostna vrednost izjave $x + 2 y < 3$ odvisna
od $x$ in $y$, kar lahko prikažemo z \textbf{resničnostno tabelo}:
%
\begin{center}
  \begin{tabular}{ccc}
    \toprule
    $x$ & $y$ & $x + 2 y < 3$ \\ \midrule
    $0$ & $0$ & $\top$ \\
    $0$ & $1$ & $\top$ \\
    $1$ & $0$ & $\top$ \\
    $2$ & $0$ & $\top$ \\
    $1$ & $1$ & $\bot$ \\
    $0$ & $2$ & $\bot$ \\
    $\vdots$ & $\vdots$ & $\vdots$ \\
    \bottomrule
  \end{tabular}
\end{center}
% 
Kot vidimo, je lahko resničnostna tabela neskončna. Bolj uporabne so končne resničnostne tabele, v katerih parametri zavzemajo vrednosti iz končne množice.

V izjavi lahko nastopajo tudi \textbf{izjavne spremenljivke} ali \textbf{izjavni simboli}, to se spremenljivke, ki zavzamejo vrednosti $\bot$ in $\top$.
Na primer, naj bo $\two = \set{\bot, \top}$ in $p, q \in \two$. Tedaj je $\neg p \lor q$ izjava, katere resničnostna tabela je
%
\begin{center}
  \begin{tabular}{ccc}
    \toprule
    $p$ & $q$ & $\neg p \lor q$ \\ \midrule
    $\bot$ & $\bot$ & $\top$ \\
    $\bot$ & $\top$ & $\top$ \\
    $\top$ & $\bot$ & $\bot$ \\
    $\top$ & $\top$ & $\top$ \\
    \bottomrule
  \end{tabular}
\end{center}

Izjava $\phi(p_1, \ldots, p_n)$, v kateri nastopajo izjavne spremenljivke $p_1, \ldots, p_n$ (in nobeni drugi parametri) določa preslikavo
%
\begin{equation*}
  \two \times \cdots \times \two \to \two
\end{equation*}
%
s predpisom
%
\begin{equation*}
    (p_1, \ldots, p_n) \mapsto \phi(p_1, \ldots, p_n)
\end{equation*}
%
Preslikavi, ki slika iz produkta $\two \times \cdots \times \two$ v $\two$ pravimo \textbf{Boolova preslikava}. Prikažemo jo lahko z resničnostno tabelo. Če ima preslikava~$n$ argumentov, ima tabela $2^n$ vrstic.


\subsection{Tavtologije}

Izjava je \textbf{tavtologija}, če je njena resničnostna vrednost $\top$ ne glede na
vrednosti parametrov. Premisli: kako iz resničnostne tabele razberemo, ali je
izjava tavtologija?

\begin{izrek}
  Naj bo $\phi$ izjava, v kateri nastopajo le izjavni simboli
  $p_1,\ldots,p_n$. Tedaj je $\phi$ tavtologija, če in samo če ima dokaz.
\end{izrek}

\begin{dokaz}
  Dokaz najdete v \cite{prijatelj92:_osnov}.
\end{dokaz}

\noindent
%
Izrek je pomemben, ker nam pove, da lahko dokazovanje izjav nadomestimo s preverjanjem resničnostnih tabel.

\begin{opomba}
  Izrek velja samo za izjave, ki jih sestavimo iz izjavnih simbolov, $\bot$, $\top$ in
  logičnih veznikov $\neg$, $\land$, $\lor$, $\lthen$, $\liff$. Za splošne izjave, ki vsebujejo tudi $\forall$ in $\exists$ izrek \emph{ne} velja. Lahko se namreč zgodi, da ima izjava neskončno resničnostni tabelo, v kateri so vse resničnostne vrednosti~$\top$, a izjava nima dokaza.
\end{opomba}

\subsection{Polni nabori}

Vsaka formula v izjavnem računu ima resničnostno tabelo. Ali lahko vsako tabelo
dobimo kot resničnostno tabelo neke formule? Na primer, ali obstaja formula,
katere resničnostna tabela se glasi
%
\begin{center}
  \begin{tabular}{ccc}
    \toprule
    $p$ & $q$ & ? \\ \midrule
    $\bot$ & $\bot$ & $\bot$ \\
    $\bot$ & $\top$ & $\top$ \\
    $\top$ & $\bot$ & $\top$ \\
    $\bot$ & $\bot$ & $\bot$ \\
    \bottomrule
  \end{tabular}
\end{center}
%
Odgovor je pritrdilen. Podajmo dva načina, kako tako izjavo izračunamo iz tabele.

\subsubsection{Disjunktivna oblika}
\label{sec:disjunktivna-oblika}

Za vsako vrstico v tabeli, ki ima vrednost $\top$ zapišemo konjunkcijo simbolov in
njihovih negacij, pri čemer negiramo tiste simbole, ki imajo v dani vrstici vrednost
$\bot$. Na primer, v zgornji tabeli imata druga in tretja vrstica vrednost $\top$, zanju
zapišemo konjukciji:
%
\begin{itemize}
\item 2.~vrstica: $\neg p \land q$,
\item 3.~vrstica: $p \land \neg q$.
\end{itemize}
%
Nato tvorimo disjunkcijo tako dobljenih konjukcij:
%
\begin{equation*}
  (\neg p \land q) \land (p \land \neg q).
\end{equation*}
%
Dobljena formula ima želeno resničnostno tabelo.

\subsubsection{Konjuktivna oblika}
\label{sec:konjuktivna-oblika}

Za vsako vrstico v tabeli, ki ima vrednost $\bot$ zapišemo
disjunkcijo simbolov in njihovih negacij, pri čemer negiramo tiste simbole, ki
imajo v dani vrstici vrednost $\top$. Na primer, v zgornji tabeli imata prva in
čertrta vrstica vednost $\bot$, zanju zapišemo disjukciji:
%
\begin{itemize}
\item 1.~vrstica: $p \lor q$
\item 4.~vrstica: $\neg p \lor \neg q$
\end{itemize}
%
Nato tvorimo konjunkcijo tako dobljenih disjunkcij:
%
\begin{equation*}
  (p \lor q) \land (\neg p \lor \neg q).
\end{equation*}
%
Zgornjo tabelo bi lahko dobili tudi kot resničnostno tabelo formule
%
\begin{equation*}
    p \liff q
\end{equation*}

\subsection{Polni nabori}
\label{sec:polni-nabori}

Vidimo, da lahko vsako resničnostno tabelo dobimo z uporabo veznikov $\neg$, $\lor$ in
$\land$. \textbf{Polni nabor} je tak izbor veznikov, k katerim lahko dobimo vsako
resničnostno tabelo.

Torej je $\neg$, $\lor$, $\land$ poln nabor. Lahko bi ga še zmanjšali na $\neg$, $\land$, saj lahko $p \lor q$ izrazimo kot $\neg p \land \neg q$.

Nabor $\land$, $\lor$ pa \emph{ni} poln, saj ne moremo dobiti resničnostne tabele
%
\begin{center}
  \begin{tabular}{cc}
    \toprule
    $p$ & ? \\ \midrule
    $\bot$ & $\top$ \\
    $\top$ & $\bot$ \\
    \bottomrule
  \end{tabular}
\end{center}
%
Res, če iz $p$, $\land$ in $\lor$ sestavimo poljubno formulo $\phi(p)$, na primer $(p \land (p \lor p)) \land p$, bo ta ekvivalentna~$p$ in bo zato veljalo $\phi(\top) = \top$, zgornja tabela pa zahteva $\phi(\top) = \bot$.


\section{Boolova algebra}

Ekvivalentni izjavi imata enake resničnostne vrednosti, torej lahko ekvivalenco
$\liff$ obravnavamo kar kot enakost, saj to tudi je, kar se tiče resničnostnih
vrednosti. Zato lahko namesto $p \liff q$ pišemo tudi $p = q$, če imamo v mislih le
resničnostne vrednosti.

\begin{opomba}
  Ekvivalentni izjavi imata lahko različen \emph{pomena}. Na primer,
  $\all{x, y \in R} x + y = y + x$ in
  $\all{\alpha \in R} \sin(2 \alpha) = 2 \cdot \cos \alpha \cdot \sin \alpha$ sta
  ekvivalentni, saj sta obe resnični, a ne moremo reči, da je njun pomen enak. (Predstavljate si, da bi bi vas v srednji šoli profesorica matematike vprašala adicijski izrek za $\sin$, vi pa bi odgovorili ">vrstni red seštevanja realnih števil ne vpliva na vrednost vsote"<.)
\end{opomba}


Za logične veznike veljajo \emph{algebrajska pravila}, se pravi enačbe, kakršne poznamo v algebri. Ta pravila lahko uporabljamo kot računska pravila, s katerimi lahko izjavo poenostavmi v ekvivalentno obliko. Pogosto je tako računanje bolj prikladno kot dokazovanje. Spodaj našteta pravila lahko preverimo tako, da zapišemo resničnostne tabele izjav in jih primerjamo.

Pravilom, ki veljajo za logične veznike, pravimo \textbf{Boolova algebra}.
Razdelimo jih po sklopih.

Pravila za konjukcijo:
%
\begin{align}
  (p \land q) \land r &= p \land (q \land r) \tag{asociativnost $\land$} \\
  p \land q &= q \land p \tag{komutativnost $\land$} \\
  p \land p &= p \tag{idempotentnost $\land$} \\
  \top \land p &= p \tag{$\top$ je nevtralen za $\land$} \\
  \bot \land p &= \bot \tag{$\bot$ absorbira $\land$}
\end{align}
%
Pravila za disjunkcijo:
%
\begin{align}
  (p \lor q) \lor r &= p \lor (q \lor r) \tag{asociativnost $\lor$} \\
  p \lor q &= q \lor p \tag{komutativnost $\lor$} \\
  p \lor p &= p \tag{idempotentnost $\lor$} \\
  \bot \lor p &= p \tag{$\bot$ je nevtralen za $\lor$} \\
  \top \lor p &= \top \tag{$\top$ absorbira $\lor$}
\end{align}
%
Pravila za implikacijo:
%
\begin{align}
  (p \lthen q) &= (\neg q \lthen \neg p) \tag{kontrapozitivna oblika $\lthen$}\\
  (p \lthen q) &= \neg p \lor q \notag \\
  (\bot \lthen q) &= \top \notag \\
  (\top \lthen q) &= q \notag \\
  (p \lthen \bot) &= \neg p \notag \\
  (p \lthen \top) &= \top \notag
\end{align}
%
Kombinirana pravila:
%
\begin{align}
  \neg(p \land q) &= \neg p \lor \neg q \tag{de Morganovo pravilo za $\land$} \\
  \neg(p \lor q) &= \neg p \land \neg q \tag{de Morganovo pravilo za $\lor$} \\
  \neg(p \lthen q) &= \neg p \land q \notag \\
  p \land (p \lor q) &= p \tag{absorbcijsko pravilo za $\land$}\\
  p \lor (p \land q) &= p \tag{absorbcijsko pravilo za $\lor$} \\
  p \land (q \lor r) &= (p \land q) \lor (p \land r) \tag{distributivnost $\land$}\\
  p \lor (q \land r) &= (p \lor q) \land (p \lor r) \tag{distributivnost $\lor$}
\end{align}
%
Pravila za negacijo:
%
\begin{align}
  \neg \top  &= \bot \notag \\
  \neg \bot &= \top \notag \\
  \neg\neg p &= p \tag{negacija je involucija} \\
  p \lor \neg p &= \top \tag{izključena tretja možnost} \\
  p \land \neg p &= \bot \notag
\end{align}

Zapišimo še uporabna logična pravila za kvantifikatorje. Tokrat uporabimo $\liff$
namesto $=$, ker je to bolj običajno:
%
\begin{align*}
  (\all{x \in \emptyset} \phi(x))   &\iff   \top \\
  (\some{x \in \emptyset} \phi(x))   &\iff   \bot \\
  (\all{x \in \set{a}} \phi(x))   &\iff   \phi(a) \\
  (\some{x \in \set{a}} \phi(x))   &\iff   \phi(a) \\
  (\neg \all{x \in A} \phi(x))   &\iff   \some{x \in A} \neg \phi(x) \\
  (\neg \some{x \in A} \phi(x))   &\iff   \all{x \in A} \neg \phi(x) \\
  (\psi \lthen \all{x \in A} \phi(x))   &\iff   \all{x \in A} \psi \lthen \phi(x) \\
  (\psi \lor \all{x \in A} \phi(x))   &\iff   \all{x \in A} \psi \lor \phi(x) \\
  (\psi \land \some{x \in A} \phi(x))   &\iff   \some{x \in A} \psi \land \phi(x) \\
  (\all{u \in A \times B} \phi(u))   &\iff   \all{x \in A} \all{y \in B} \phi(x, y) \\
  (\some{u \in A \times B} \phi(u))   &\iff   \some{x \in A} \some{y \in B} \phi(x, y) \\
  (\all{u \in A + B} \phi(u))   &\iff   (\all{x \in A} \phi(\inl(x))) \land (\all{y \in B} \phi(\inr(y))) \\
  (\all{u \in A \cup B} \phi(u))   &\iff   (\all{x \in A} \phi(x)) \land (\all{y \in B} \phi(y)) \\
  (\some{u \in A + B} \phi(u))   &\iff   (\some{x \in A} \phi(\inl(x))) \lor (\some{y \in B} \phi(\inr(y))) \\
  (\some{u \in A \cup B} \phi(u))   &\iff   (\some{x \in A} \phi(x)) \lor (\some{y \in B} \phi(y)) \\
  (\all{u \in \set{x \in A \such \psi(x)}} \phi(u))   &\iff   \all{x \in A} \psi(x) \lthen \phi(x) \\
  (\some{u \in \set{x \in A \such \psi(x)}} \phi(u))   &\iff   \some{x \in A} \psi(x) \land \phi(x)
\end{align*}
%
Te ekvivalence je treba preveriti tako, da jih dokažemo.


\chapter{Podmnožice in potenčna množica}
\textbf{To poglavje še ni predelano v {\LaTeX}.}
%\chapter{Podmnožice in potenčne množice}

\subsection{Definicija relacije $\subseteq$}

Pravimo, da je množica $S$ \textbf{podmnožica} množice $T$, pišemo $S \subseteq T$, ko velja $\all{x \in S} x \in T$. Pravimo tudi, da je $S$ \textbf{vsebovana} v $T$ in da je $T$ \textbf{nadmnožica}~$S$.

Vedno velja $\emptyset \subseteq S$ in $S \subseteq S$.

Princip ekstenzionalnosti za množice pravi:
%
\begin{equation*}
  S = T \iff (\all{x \in S} S \in T) \land (\all{y \in T} y \in S)
\end{equation*}
%
kar lahko zapišemo s podmnožicami:
%
\begin{equation*}
  S = T \iff S \subseteq T \land T \subseteq S.
\end{equation*}
%
Vsaka podmnožica $S \subseteq A$ opredeljuje neko lastnost elementov iz $A$: tisti
elementi, ki imajo opredeljeno lasnost, so v $S$, ostali pa ne.

\begin{primer}
  Naj bo $P$ množica vseh praštevil, torej je $P \subseteq N$. Podmnožica $P$
  opredeljuje lasnost ">je praštevilo"<.
\end{primer}


\subsection{Kako tvorimo podmnožice}

Če je $\phi(x)$ logična formula, v kateri nastopa spremenljivka $x \in A$, lahko tvorimo množico
%
\begin{equation*}
    \set{ x \in A \such \phi(x) }.
\end{equation*}
%
Pri tem je $x$ vezana spremenljivka. Za to množico velja:
%
\begin{equation*}
    a \in \set{ x \in A \such \phi(x) } \iff a \in A \land \phi(a).
\end{equation*}
%
Povedano z besedami: elementi množice $\set{ x \in A \such \phi(x) }$ so tisti elementi iz $A$, ki zadoščajo pogoju $\phi$.
%
Velja $\set{ x \in A \such \phi(x) } \subseteq A$, prav tako pa
\begin{equation*}
  \set{x \in A \mid \phi(x)} \subseteq \set{x \in A \mid \psi(x)} \iff
  \all{x \in A} \phi(x) \lthen \psi(x).
\end{equation*}


\subsection{Kanonična inkluzija}

Za podmnožico $S \subseteq T$ definiriamo \textbf{kanonično inkluzijo} ali \textbf{kanonično vključitev} $i_S : S \to T$, s predpisom $i_S : x \mapsto x$. Pozor, to ni identiteta, razen v primeru $S = T$!.
Oznaka $i_S$ ni standardna, pravzaprav standardne oznake ni.

Če je $f : T \to U$ in $S \subseteq T$, pravimo kompozitumu $f \circ i_S$ \textbf{zožitev} preslikave $f$ na $S$, pišemo $\restrict{f}{S}$.


\section{Potenčna množica}

\subsection{Definicija potenčne množice}

Za vsako množico $A$ tvorimo množico $\pow{A}$, ki ji pravimo \textbf{potenčna množica}.
Elementi potenčne množice $\pow{A}$ so natanko podmnožice množice $A$:
%
\begin{equation*}
    S \in \pow{A} \iff S \subseteq A
\end{equation*}
%
Na priemr $\pow{\emptyset} = \set{\emptyset}$ in
%
\begin{equation*}
  P(\set{a,b,c}) = \set{\set{}, \set{a}, \set{b}, \set{c}, \set{a,b}, \set{a,c}, \set{b,c}, \set{a,b,c}}.
\end{equation*}


\subsection{Karakteristične funkcije}

\textbf{Karakteristična funkcija} na množici $A$ je fukcija z domeno $A$ in kodomeno $\two$. Tu je $\two = \set{\bot, \top}$ množica resničnostnih vrednosti.

Eksponentna množica $\two^A$ je torej množica vseh karakterističnih funkcij na $A$.

\begin{opomba}
  Karakteristične funkcije se uporabljajo tudi v analizi, kjer jih
  običajno razumemo kot preslikave $A \to \set{0,1}$ namesto $A \to \set{\bot, \top}$. Ker sta množici $\set{\bot,\top}$ in $\set{0,1}$ izomorfni, to ni bistvena razlika.
\end{opomba}

Karakteristično funkcijo si lahko predstavljamo kot preslikavo, ki opredeljuje
neko lastnost elementov~$A$: tisti elementi, ki imajo opredeljeno lastnost, se
slikajo v $\top$, ostali pa v $\bot$.

\begin{primer}
  Preslikava $p : \NN \to \two$, definirana s predpisom
  %
  \begin{equation*}
    p(n) = 
    \begin{cases}
      \top & \text{če je $n$ praštevilo}, \\
      \bot & \text{če $n$ ni praštevilo}.
    \end{cases}
  \end{equation*}
  %
  je karakteristična preslikava lastnosti ">je praštevilo"<. Lahko bi jo zapisali tudi takole:
  %
  \begin{equation*}
    p(n) = (\some{k, m \in \NN} k \geq 2 \land m \leq 2 \land n = k \cdot m).
  \end{equation*}
\end{primer}


\subsection{Izomorfizem $\pow{A} \cong 2^A$}

Videli smo, da lahko neko lastnost elementov množice $A$ predstavimo bodisi s
podmnožico bodisi s karakteristično preslikavo. To nam da idejo, da med
podmnožicami $A$ in karakterističnimi preslikavami na $A$ obstaja neka zveza.

\begin{izrek}
  $\pow{A} \cong 2^A$.
\end{izrek}

\begin{dokaz}
  Definirajmo preslikavi
  %
  \begin{align*}
    \chi &: \pow{A} \to 2^A &
    \xi &: 2^A \to \pow{A} \\
    \chi_S(x) &\defeq
      \begin{cases}
        \top & \text{če $x \in S$,} \\
        \bot & \text{če $x \not\in S$,}
      \end{cases}
    &
    \xi_f &\defeq \set{x \in A \such f(x) = \top}.
  \end{align*}
  %
  Ta predpisa bi lahko krajše zapisali tudi takole:
  %
  \begin{align*}
  \chi_S(x) &\defeq (x \in S), &
  \xi_f &\defeq \set{x \in A \such f(x) }.
  \end{align*}
  %
  Preslikavi $\chi_S$ pravimo \textbf{karakteristična funkcija podmnožice $S$}.
  %
  Trdimo, da sta $\chi$ in $\xi$ inverza:
  %
  \begin{enumerate}
  \item 
    Dokažimo $\chi \circ \xi = \id[2^A]$. Uporabimo princip ekstenzionalnosti za preslikave.
    Naj bo $f \in 2^A$. Dokažimo, da je $\chi_{\xi_f} = f$.
    Uporabimo princip ekstenzionalnosti za preslikave. Naj bo $x \in A$:
    %
    \begin{equation*}
      \chi_{\xi_f}(x) = (x \in \xi_f) = f(x).
    \end{equation*}

  \item
    Dokažimo $\xi \circ \chi = id_{\pow{A}}$. Uporabimo princip ekstenzionalnosti za preslikave. Naj bo $S \in \pow{A}$. Dokažimo, da je $\xi_{\chi_S} = S$:
    %
    \begin{equation*}
      \xi_{\chi_S} = \set{x \in A \such \chi_S(x)} = \set{x \in A \such x \in S} = S.
    \end{equation*}
  \end{enumerate}
\end{dokaz}

\subsection{Boolova algebra podmnožic}

Podmnožice množice $A$ tvorijo Boolovo algebro za operacije presek $\cap$, unija $\cup$ in relativni komplement. Nevtralni element za unijo je $\emptyset$ in nevtralni element za presek je $A$.

Definirajmo tudi operacijo \textbf{simetrična razlika $\oplus$}, ki podmnožicama $S, T \in A$ priredu podmnožico
%
\begin{equation*}
  S \oplus T \defeq (S \setminus T) \cup (T \setminus S) = (S \cup T) \setminus (S \cap T).
\end{equation*}
%
Potenčna množica $\pow{A}$ je za operacijo $\oplus$ Abelova grupa.


\chapter{Razredi in družine}
\textbf{To poglavje še ni predelano v {\LaTeX}.}
%\chapter{Razredi in družine}

\section{Russellov paradoks}

V prejšnji lekciji smo spoznali zapis podmnožice

    { x ∈ A | φ(x) }

ki tvori podmnožico `A` vseh elementov, ki zadoščajo pogoju `x`. Ko je bila
teorija množic še v povojih, se je sama po sebi ponujala ideja, da bi lahko
opisali množice kot "kakršnokoli zbirko stvari. Torej bi lahko pisali

    { x | φ(x) }

za množico vseh tistih stvari (objektov, matematičnih entitet), ki zadoščajo
pogoju `φ`. Se pravi, da bi veljalo

    a ∈ { x | φ(x) } ⇔ φ(a)

A izkaže se, da ne moremo kar tako tvoriti povsem poljubnih množic objektov. To
je odkril znameniti filozof, logik in matematik Betrand Russell. Razmislek se po
njem imenuje *Russellov paradoks*. Le-ta je v matematiko vnesel pravo "krizo
temeljev", iz katere se je v prvi polovici 20. stoletja razvila logika in
temelji matematike, kot jih poznamo danes.

Russellov paradoks gre takole. Denimo, da bi lahko tvorili poljubne množice
objektov. Tedaj bi lahko tvorili tudi množico vseh množic, ki niso element same
sebe:

    R := { S | S ∉ S }

Sedaj bomo izpeljali protislovje tako, da bomo dokazali `R ∈ R` in `R ∉ R`:

1. Dokažimo `R ∉ R`.

   Denimo, da bi veljalo `R ∈ R`. Potem po definiciji `R` velja `R ∉ R`, kar
   je v protislovju s predpostavko `R ∈ R`.

2. Dokažimo `R ∈ R`. V prvem koraku smo že dokazali `R ∉ R`, torej po
   definiciji `R` velja `R ∈ R`.

Kaj lahko storimo? Očitno je treba pazljivo nadzorovati dopustne konstrukcije
množic.

\section{Množice in razredi}

V sodobni teoriji množic Russellov paradoks razrešimo tako, da ločimo med dvema
različnima zvrstema zbirk ali skupkov elementov, namreč **množicami** in **razredi**.

Torej imamo opravka s tremi zvrstmi matematičnih objektov:

1. Elemnti, ki niso množice (na primer naravna števila), pravimo jim **urelementi**.
2. Zbirke elementov, ki se imenujejo **množice**.
3. Zbirke elementov, ki se umenujejo **razredi**.

Elementi množic so urelementi in množice. Enako velja za razrede.

V čem je torej razlika med množicami in razredi?

> Množica je lahko element (druge množice ali razreda)
> Razred ne more biti element (druge množice ali razreda).

S tem želimo povedati, da je zapis

    x ∈ Y

*neveljaven*, če je `x` razred. Se pravi, če je `x` razred, potem `x ∈ Y` sploh
ni veljaven izraz. Ne moremo govoriti o tem, da je resničen ali neresničen, saj
sploh ni smiselen.

Vsaka množica je hkrati razred. Ni pa vsak razred tudi množica.

Razred je množica, če ga lahko skonstruiramo še na kak drug način s pomočjo
pravil za konstrukcije množic (kartezični produkti, vsote, eksponenti, unije,
preseki, podmnožice in vse ostale konstrukcije množic, ki jih bomo še spoznali).

**Pravi razred** je tak razred, ki ni množica.

Z zapisom

    { x | φ(x) }

definiramo *razred* vseh objektov, ki zadoščajo pogoju `φ`. Se pravi, da velja

    a ∈ { x | φ(x) } ⇔ φ(a)

Poglejmo nekaj primerov.

**Russellov razred** `R := { S | S ∉ S }` vsebuje vse množice, ki niso element
same sebe. Paradoks smo razrešili, saj je nesmiselno zapisati `R ∈ R`.

**Razred vseh množic**

    V := { S | S je množica }

ki ga označimo tudi s `Set`. To je pravi razred. Res, če bi bil `V` množica,
potem bi lahko tvorili podmnožico

    { S ∈ V | S ∉ S }

ki ni nič drugega kot Russellov `R`. Tako bi spet dobili protislovje. Torej `V`
ni množica.

Ostali primeri, v katere se ne bomo poglabljali:

* razred vseh enojcev `{ S | ∃! x ∈ S . ⊤ }`
* razred vseh grup
* razred vseh vektorskih prostorov

Z razredi lahko delamo tako kot z množicami: tvorimo unije, preseke in produkte
razredov, govorimo po podrazredih). Pri tem uporabljamo enake oznake za
operacije kot pri množicah. Paziti moramo le, da razreda nikoli ne uporabimo kot
element kake množice ali razreda. Na primer, če je `C` razred, lahko tvorimo
"potenčni razred" `P(C)`, ki vsebuje vse *podmnožice* `C`:

    P(C) := { S | S ∈ Set ∧ S ⊆ C }

Ne smemo pa tvoriti `{ D | D ⊆ C }`, ker bi s tem `C` postal element razreda `{D | D ⊆ C}`.

\section{Družine množic}

Pogosto imamo opravka z zbirko množic. Če je zbirka končna, lahko množice preprosto
naštejemo in vsako od njih poimenujemo

    A = ...
    B = ...
    C = ...

Če je množic neskončno, jih morda lahko oštevilčimo:

    A_1 = ...
    A_2 = ...
    A_3 = ...
    A_4 = ...
    ...

A tu se zadeve še ne končajo, saj lahko v splošnem obravnavamo poljubno zbirko množic.
Takim zbirkam pravimo *družine množic*. Družina množic je *indeksirana* z elementi
neke množice `I`, ki ji pravimo *indeksna množica*. Za vsak `i ∈ I` imamo množico

    A_i = ...

To lahko izrazimo tudi takole:

> **Definicija:** **Družina množic** je preslikava `I → Set`. Množici `I` pravimo *indeksna*
> množica.

Primeri družin:

1. Končno zbirko množic lahko indeksiramo s končno množico. Denimo, da imamo
   množice `A`, `B`, `C`, `D`, `E`. Iz njih lahko tvorimo družino `S`

        I = {1, 2, 3, 4, 5}

        S_1 = A
        S_2 = B
        S_3 = C
        S_4 = D
        S_5 = E.

2. Množice v družini se lahko ponavljajo. V prejšnjem primeru bi lahko na primer
   imeli `A = C` in bi tako veljalo `S_1 = S_3`. Skrajni primer je *konstantna* družina,
   v kateri so vse množice med seboj enake.

3. *Prazna* družina je družina množic, ki je indeksirana `∅`.

4. Prazno družino moramo ločiti od družine praznih množic

        I → Set
        i ↦ ∅

5. Neprazna družina je družina indeksirana z neprazno množico.
   Družina nepraznih množic je družina, v kateri so vse množice neprazne:

      * Prazna družina je družina nepraznih množic.
      * Družina praznih množic je lahko prazna družina (ko je indeksna množica `∅`)
      * Družina praznih množic je lahko neprazna družina (ko je indeksna množica nerazna).

\section{Konstrukcije in operacije z družinami množic}

Operacije `×`, `+`, `∩` in `∪` lahko posplošimo tako, da namesto z dvema
množicama delujejo na poljubnem številu množic. V ta namen uporabimo družine
množic.

\subsection{Presek in unija družine}

Presek in unija družine `A : I → Set` je definirana takole:

    ⋃_(i ∈ I) A_i = { x | ∃ i ∈ I . x ∈ A_i }

    ⋂_(i ∈ I) A_i = { x | ∀ i ∈ I . x ∈ A_i }

Pozor! Na desni strani imamo razred! Res se lahko zgodi, da dobimo pravi razred, denimo
kot presek prazne družine:

    ⋂_(i ∈ ∅) A_i = { x | ∀ i ∈ ∅ . x ∈ A_i } = { x | ⊤ } = V

Kdaj pa dobimo množico? Presek neprazne družine je vedno množica. Res, če imamo
`k ∈ I`, potem velja

    ⋂_(i ∈ ∅) A_i = { x ∈ A_k | ∀ i ∈ ∅ . x ∈ A_i }

Sedaj na desni ne stoji več razred, ampak podmnožica množice `A_k`.

Kaj pa unija družine množic? Ali je množica, ali bi lahko dobili pravi razred, denimo `V`,
kot unijo družine množic? Izkaže se, da za to potrebujemo aksiom:

**Aksiom:** Unija družine množic je množica.

\subsection{Kartezični produkt}

Denimo, da imam družino množic `A : I → Set`.

**Funkcija izbire f** za `A` je prirejanje, ki vsakemu indeksu `i ∈ I` priredi neki element
`f(i) ∈ A_i` iz `A_i`.

Primer: funkcija izbire za družino

     A : N → Set
     A_n = { x ∈ R | 0 < x < 2^(-n) }

je na primer `f(n) = 2^(-n - 1) ali pa f(n) = 2^(-n) / 3`. To ni edina funkcija izbire za `A`.

**Kartezični produkt** družine `A : I → Set` je množica

    ∏_(i ∈ I) A_i

katere elementi so funkcije izbire za `A`. To je nova konstrukcija množice.

Za vsak `j ∈ I` imamo **`j`-to projekcijo**

    pr_j :  ∏_(i ∈ I) A_i → A_j
    pr_j :  f ↦ f(j)

Običajni kartezični produkt dveh množic je poseben primer produkta množic, namreč družine
množic, ki je indeksirana z `I = {1, 2}`. Natančneje, velja

  `A × B ≅ ∏_(i ∈ {1, 2}) C_i`

  kjer je `C_1 = A` in `C_2 = B`.

Tudi eksponentna množica je poseben primer produkta množic, saj velja

    B^A ≅ ∏_(a ∈ A) B

Na desni imamo produkt konstantne družine množic

    A → Set
    a ↦ B

\subsection{Koprodukt ali vsota množic}

Vsoto množic posplošimo na koprodukt družine. Za dano družino `A : I → Set` tvorimo množico

    ∑_(i ∈ I) A_i

Elementi koprodukta so oblike

    in_k(a)

kjer je `k ∈ I` in `a ∈ A_k`. Preslikavi

    in_k : A_k → ∑_(i ∈ I) A_i

pravimo **`k`-ta injekcija**.

Namesto `∑` se piše tudi `∐`.

TODO: prva in druga projekcija.

Poseben primer koprodukta je vsota `A + B`, saj velja

    A + B ≅ ∑_(k ∈ {1, 2}) C_k

kjer je

    C : {1, 2} → Set
    C_1 = A
    C_2 = B.

Kartezični produkt `A × B` je tudi poseben primer koprodukta, saj velja

    A × B ≅ ∑_{a ∈ A} B

Na desni imamo tokrat koprodukt konstantne družine množic

    A → Set
    a ↦ B




\chapter{Lastnosti preslikav}
\textbf{To poglavje še ni predelano v {\LaTeX}.}
%\chapter{Lastnosti preslikav}

Mnogi ste v srednji šoli že spoznali osnovne lastnosti preslikav, kot so injektivnost, surjektivnost in bijektivnost
preslikave. V tej lekciji ponovimo te pojme in jih povežemo še s pojmoma monomorfizem in epimorfizem, ki sta pomembna v
algebri

\section{Osnovne lastnosti preslikav}

\subsection{Injektivna, surjektivna, bijektivna preslikava}

**Definicija:** Preslikava `f : A → B` je

* **injektivna**, ko velja `∀ x y ∈ A . f(x) = f(y) ⇒ x = y`
* **surjektivna**, ko velja `∀ y ∈ B . ∃ x ∈ A . f(x) = y`
* **bijektivna**, ko je surjektivna in injektivna

*Opomba:* Pogosto vidimo definicijo injektivnosti, ki pravi, da `f` slika različne elemente v različne vrednosti, se
pravi `∀ x y ∈ A . x ≠ y ⇒ f(x) ≠ f(y)`. Ta definicija je ekvivalentna naši, a jo ne priporočamo, ker je manj uporabna.
Naša definicija namreč podaja recept, kako preverimo injektivnost: predpostavimo `f(x) = f(y)` in od tod izpeljemo
`x = y` tako, da predelamo *enačbo* `f(x) = f(y)` v enačbo `x = y`. To je v splošnem lažje kot predelava *neenačb* v
*neenačbe*.

**Naloga:** primerjaj definicijo injektivnosti z zahtevo, da mora biti prirejanje, ki določa preslikavo, enolično.

**Naloga:** primerjaj definicijo surjektivnost z zahtevo, da mora biti prirejanje, ki določa preslikavo, celovito.


\subsection{Monomorfizmi in epimorfizmi}

**Definicija:** Preslikava `f : A → B` je

* **monomorfizem (mono)**, ko jo lahko krajšamo na levi:
  `∀ C ∈ Set ∀ g, h : C → A . f ∘ g = f ∘ h ⇒ g = h`

* **epimorfizem* (epi)**, ko jo lahko krajšamo na desni:
  `∀ C ∈ Set ∀ g, h : B → C . g ∘ f = h ∘ f ⇒ g = h`

Pojma monomorfizem in epimorfizem sta uporabna, ker nam omogočata, da *krajšamo* funkcije, ki nastopajo v enačbah. Na
vajah boste reševali naloge, kjer to pride prav.

**Izrek 1:** Naj bosta `f : A → B` in `g : B → C` preslikavi.

1. Kompozicija monomorfizmov je monomorfizem.
2. Kompozicija epimorfizmom je epimorfizem.
3. Če je `g ∘ f` monomorfizem, je `f` monomorfizem.
4. Če je `g ∘ f` epimorfizem, je `g` epimorfizem.

*Dokaz:*

1. Naj bosta `f : A → B` in `g : B → C` monomorfizma. Dokazujemo, da je `g ∘ f` tudi monomorfizem.
   Naj bosta `h, k : D → A` preslikavi, za kateri velja `(g ∘ f) ∘ h  = (g ∘ f) ∘ k`. Dokazujemo `h = k`.
   Ker je kompozicija preslikav asociativna, velja `g ∘ (f ∘ h) = (g ∘ f) ∘ h  = (g ∘ f) ∘ k g ∘ (f ∘ k)`.
   Ker je `g` monomorfizem, ga smemo krajšati na levi, torej dobimo `f ∘ h = f ∘ k`. Ker je `f` monomorfizem,
   ga smemo krajšati in dobimo želeno enakost `h = k`.

2. Dokaz je podoben 1, le vloga leve in desne strani se spremeni (vaja).

3. Dokaz je podoben 4, le vloga leve in desne strani se spremeni (vaja).

4. Naj bosta `f : A → B` in `g : B → C` preslikavi in `g ∘ f` epimorfizem. Dokazujemo, da
   je `g` epimorfizem. Naj bosta `h, k : C → D` taki preslikavi, da velja `h ∘ g = k ∘ g`.
   Dokazujemo, da je `h = k`. Iz `h ∘ g = k ∘ h` sledi `(h ∘ g) ∘ f = (k ∘ g) ∘ f`. Če
   upoštevamo asociativnost kompozicije, dobimo `h ∘ (g ∘ f) = k ∘ (g ∘ f)`. Ker je `g ∘
   f` epimorfizem, ga smemo krajšati na desni, od koder dobimo želeno enakost `h = k`.
□

**Izrek 2:** Za preslikavo `f : A → B` velja

1. `f` je monomorfizem ⇔ `f` je injektivna
2. `f` je epimorfizem ⇔ `f` je surjektivna
3. `f` je izomorfizem ⇔ `f` je bijektivna

*Dokaz:*

1. Če je `f` monomorfizem in `f(x) = f(y)`, tedaj je
   `(f ∘ (u ↦ x)) () = f(x) = f(y) = (f ∘ (u ↦ y)) ()`, torej
   `(u ↦ x) = (u ↦ y) torej x = y`.

   Če je `f` injektivna in `f ∘ g = f ∘ h`, potem je za vsak `x`
   `f(g(x)) = f(h(x))`, torej `g(x) = h(x)` za vsak `x`, torej `g = h`.

2. Če je `f` epimorfizem: obravnavajmo množico

        S = { z ∈ B | ∃ x ∈ A . f(x) = z }

   ter preslikavi `χ_S : B → 2` in `(y ↦ ⊤) : B → 2`. Ker velja
   `χ_S ∘ f = (y ↦ ⊤) ∘ f`, sledi `χ_S = (y ↦ ⊤)`, torej `S = B`,
   kar je surjektivnost.

   Če je `f` surjektivna in `g ∘ f = h ∘ f`: naj bo `y ∈ B`. Obstaja
   `x ∈ A`, da je `f(x) = y`. Torej je

        g(y) = g(f(x)) = h(f(x)) = h(y).

   Torej je `g = h`.

3. Če je `f` izomorfizem, potem

    * `f` je epi, ker je `id_B = f ∘ f⁻¹` epi
    * `f` je mono, ker je `id_A = f⁻¹ ∘ f` mono

   Če je `f` bijektivna, potem je njen inverz `f⁻¹` definiran s predpisom

    `f(y) = ι x ∈ A . f(x) = y`      "tisti x, ki ga f slika v y"

   Dokazati je treba `∃! x . f(x) = y:`

   * `∃ x . f(x) = y` je surjektivnost `f`
   * `∀ x₁ x₂ . f(x₁) = y ∧ f(x₂) = y ⇒ x₁ = x₂` sledi iz injektivnosti `f`
□

\subsection{Retrakcija in prerez}

Spoznajmo še pojem retrakcije in prereza. Na predavanjih bomo s sliko pojasnili, zakaj se tako imenujeta.

**Definicija:** Če sta `f : A → B` in `g : B → A` taki preslikava, da velja `f ∘ g = id_B`, pravimo:

* `f` je **levi** inverz `g`
* `g` je **desni** inverz `f`
* `g` je **prerez** preslikave `f`
* `f` je **retrakcija** iz `B` na `A`

Opomba: retrakcija in prerez *ni* isto kot izomorfizem!

**Izrek 3:** Retrakcija je epimorfizem, prerez je monomorfizem.

*Dokaz:*

Denimo, da velja `f ∘ g = id`, torej je `f` retrakcija in `g` prerez. Ker je `id`
monomorfizem, je po izreku 1 tudi `g` monomorfizem. In ker je `id` epimorfizem, je po
istem izreku `f` monomorfizem. □


\section{Slike in praslike}

\subsection{Izpeljane množice}

Naj bo `f : A → B` preslikava. Tedaj definiramo **izpeljano množico**

    { f(x) | x ∈ A } := { y ∈ B | ∃ x ∈ A . y = f(x) }

ter **izpeljano množico s pogojem**

    { f(x) | x ∈ A | φ(x) } := { y ∈ B | ∃ x ∈ A . φ(x) ∧ y = f(x) }

Običajno se piše izpeljano množico s pogojem kar

    { f(x) | x ∈ A ∧ φ(x) }

*Primer:* Množica vseh kvadratov naravnih števil je izpeljana množica `{ n² | n ∈ N }`.

\subsection{Slike in praslike}

**Definicija:** Naj bo `f : A → B` preslikava:

1. **Praslika** podmnožice `S ⊆ B` je `f^*(S) := { x ∈ A | f(x) ∈ S }`.
2. **Slika** podmnožice `T ⊆ A` je `f_*(T) := { y ∈ B | ∃ x ∈ A . f(x) = y }`.

Kot vidimo, lahko sliko zapišemo tudi kot izpeljano množico

    f_*(T) := { f(x) | x ∈ T }

Običajni zapis za prasliko `f^*(S)` je tudi `f⁻¹(S)`, vendar tega zapisa mi ne bomo uporabljali, ker napačno namiguje, da ima `f` inverz. Boste pa ta zapis videli marsikje drugje, ker so matematiki pravzaprav precej konzervativni in ne marajo sprememb.

Običajni zapis za sliko `f_*(S)` je tudi `f(S)` ali `f[S]`. Predvsem `f(S)` se uporablja v praksi, a tudi tega odsvetujemo. Kako naj pri takem zapisu ločimo med `f(x)` in `f_*({x})`?

**Zaloga vrednosti** je slika domene, torej `f_*(B)`.

\subsection{Slike in praslike kot preslikave višjega reda}

Naj bo `f : A → B`. Tedaj sta tudi `f^*` in `f_*` preslikavi:

* `f^* : P(B) → P(A)` je določena s predpisom `S ↦ { x ∈ A | f(x) ∈ S }`
* `f_* : P(A) → P(B)` je določena s predpisom `T ↦ { f(x) | x ∈ T }`

Še več, tudi "zgoraj zvezdica `^*`" in "spodaj zvezdica `_*`" sta preslikavi

    ^* : B^A → P(A)^P(B)
    _* : B^A → P(B)^P(A)

Ker slikata preslikave v preslikave, pravimo, da sta to preslikavi *višjega reda*. Primer preslikave višjega reda je tudi odvod, ki funkciji priredi njen odvod.

\subsection{Lastnosti slike in praslike}

**Izrek 4:** Naj bo `f : A → B` preslikava:

* praslike so monotone: če je `S ⊆ T ⊆ A`, potem je `f_*(S) ⊆ f_*(T)`
* slike so monotone: če je `X ⊆ Y ⊆ B`, potem je `f^*(X) ⊆ f^*(Y)`.

*Dokaz:* Vaja.

**Izrek 5:** Prasike ohranjajo preseke in unije: za vse `f : A → B` in `S : I → P(B)` velja

* `f^* (⋃_{i ∈ I} S_i) = ⋃_{i ∈ I} f^*(S_i)`
* `f^* (⋂_{i ∈ I} S_i) = ⋂_{i ∈ I} f^*(S_i)`

*Dokaz:* Dokažimo prvo izjavo, druga je zelo podobna, le da `∃` zamenjamo z `∀`.

Dokazujemo `f^* (⋃_{i ∈ I} S_i) ⊆ ⋃_{i ∈ I} f^*(S_i)`.
Naj bo `x ∈ f^* (⋃_{i ∈ I} S_i)` in dokazujemo `x ∈ ⋃_{j ∈ I} f^*(S_j)`.
Ker je `f x ∈ ⋃_{i ∈ I} S_i` obstaja `k ∈ I`, da je `f x ∈ S_k`, torej je
`x ∈ f^* S_k ⊆ ⋃_{i ∈ I} f^*(S_i)`. □

**Izrek 6:** Naj bo `f : A → B` in `T : I → P(A)`. Tedaj je

* `f_* (⋃_{i ∈ I} T_i = ⋃_{i ∈ I} f_*(T_i)`.
* `f_* (∩_{i ∈ I} T_i) ⊆ ⋂_{i ∈ I} f_*(S_i)`.

*Dokaz:* Vaja.

**Naloga:** Iz zgornjih dveh izrekov izpeljite naslednja dejstva:

* `f^*(∅) = ∅`
* `f_*(∅) = ∅`
* `f^*(B) = A`
* `f^*(S ∪ T) = f^*(S) ∪ f^*(T)`
* `f^*(S ∩ T) = f^*(S) ∩ f^*(T)`

Poleg tega imamo za `S ⊆ B`

    f^*(Sᶜ) = (f^*(S))ᶜ.


\chapter{Relacije}
\textbf{To poglavje še ni predelano v {\LaTeX}.}
%\chapter{Relacije}

\section{Predikati}

\textbf{Predikat} na množici $A$ opredeljuje kako lastnost elementov množice $A$. Če
je $P$ predikat na $A$ in $x \in A$, potem se je smiselno vprašati, ali $x$
zadošča predikatu $P$. Odgovor je resničnostna vrednost, ki jo označimo s $P(x)$.

\begin{primer}
  Na množici naravnih števil $\NN$ lahko obravnavamo predikat ">je sodo
  število"<. Tako na primer $4$ zadošča predikatu ">je sodo število"<, $7$ pa mu zadošča.
\end{primer}

Predikat $P$ na množici $A$ lahko predstavimo na dva načina:
%
\begin{itemize}
\item kot preslikavo $P : A \to \two$, ki slika $x \in A$ v resničnostno vrednost $P(x)$,
\item kot podmnožico $P \subseteq A$ tistih $x \in A$, za katere velja $P(x)$.
\end{itemize}
%
Oba načina predstavitve sta uporabna, spoznali pa smo že izomorfizem med njima,
saj velja $P(A) \iso \two^A$.

\section{Relacije}

Relacije s večmestni predikati. Se pravi, relacija $R$ opredeljujejo kako
lastnost urejenih večteric kartezičnega produkta $A_1 \times A_2 \times \cdots \times A_n$. Pravimo, da je $R$ \textbf{$n$-člena} ali \textbf{$n$-mestna relacija} na množicah $A_1, …, A_n$.

\begin{primer}
  Na množici točk v ravnini lahko obravnavamo relacijo kolinearnosti.
  To je trimestna relacija: točke $A$, $B$ in $C$ so kolinearne, kadar obstaja
  premica, ki vsebuje vse tri točke.
\end{primer}

Relacijo $R$ na množicah $A_1, \ldots, A_n$ lahko predstavimo na dva načina, podobno
kot predikate:
\begin{itemize}
\item kot preslikavo $R : A_1 \times A_2 \times \cdots \times A_n \to \two$,
\item kot podmnožico $R \subseteq A_1 \times A_2 \times \cdots \times A_n$.
\end{itemize}
%
Bolj običajna je predstavitev s podmnožicami, zato bomo dejstvo, da je $R$
relacija na množicah $A_1, \ldots, A_n$ zapisali kar kot $R \subseteq A_1 \times A_2 \times \cdots \times A_n$.
Za elemente $x_1 \in A_1, \ldots, x_n \in A_n$ dejstvo, da so v relaciji $R$ zapišemo
$R(x_1, \ldots, x_n)$, včasih pa tudi $(x_1, \ldots, x_n) \in R$.

Na množicah $A_1, \ldots, A_n$ lahko vedno definiramo:
%
\begin{itemize}
\item \textbf{prazno relacijo $\emptyset$}: nobeni elementi niso v prazni relaciji,
\item \textbf{univerzalno relacijo $A_1 \times A_2 \times \cdots \times A_n$}: vsi elementi so v univerzalni relaciji.
\end{itemize}
%
Univerzalna relacija se imenuje tudi \textbf{polna relacija}.
%
V praksi so najbolj pogoste \textbf{dvomestna relacije}, se pravi relacije na dveh
množicah, $R \subseteq A \times B$.
V tem primeru pravimo množici $A$ \textbf{domena} in $B$ \textbf{kodomena} relacije $R$, relaciji $R$ pa relacije med $A$ in $B$.

Pomembna relacija na množici $A$ je \textbf{enakost} ali \textbf{diagonala} na $A$:
%
\begin{equation*}
    \diag[A] \defeq \set{ (x, y) \in A \times A \such x = y }
\end{equation*}
%
Zakaj ji pravimo diagonala?

Izmed dvočlenih relacij so najbolj pogoste relacije, pri katerih se domena in
kodomena ujemata, torej $R \subseteq A \times A$. V tem primeru pravimo, da je $R$ \textbf{relacija na množici $A$}.

Denimo, da je $R \subseteq A \times B$ relacija, $x \in A$ in $y \in B$. Dejstvo, da sta $x$ in $y$ v relaciji $R$ zapišemo na enega od načinov
%
\begin{equation*}
  (x, y) \in R
  \qquad
  R(x, y)
  \qquad
  x \rel{R} y
\end{equation*}
%
Prvi zapis se uporablja, kadar je $R$ podana kot podmnožica $A \times B$, drugi kadar
podamo~$R$ z logično formulo. Tretji način je tudi pogost, še posebej kadar je
relacija označena s simbolom kot je $=$, $\neq$, $<$, $>$, $\sqsubseteq$, $\sim$ ipd.

Relacijo lahko predstavimo na več načinov, na primer z logično formulo, z resničnostno tabelo, ali z usmerjenim grafom.
%
Z grafom predstavimo $R \subseteq A \times A$ tako, da za vozlišča grafa vzamemo
elemente množice $A$, nato pa narišemo puščico od $x$ do $y$, kadar velja $x \rel{R} y$.


\section{Osnovne lastnosti relacij}

Relacije, ki so pomembne v matematični praksi imajo pogosto lastnosti, ki jih poimenujemo. Za relacijo $R \subseteq A \times A$ pravimo da je:
%
\begin{itemize}
  \item \textbf{refleksivna:} $\all{x \in A} x \rel{R} x$,
  \item \textbf{simetrična:} $\all{x, y \in A} x \rel{R} y \lthen y \rel{R} x$,
  \item \textbf{antisimetrična:} $\all{x, y \in A} x \rel{R} y \land y \rel{R} x \lthen x = y$,
  \item \textbf{tranzitivna:} $\all{x, y, z \in A} x \rel{R} y \land y \rel{R} z \lthen x \rel{R} z$,
  \item \textbf{irefleksivna:} $\all{x \in A} \lnot (x \rel{R} x)$,
  \item \textbf{asimetrična:} $\all{x, y \in A} x \rel{R} y \lthen \lnot (y \rel{R} x)$,
  \item \textbf{sovisna:} $\all{x, y \in A} x \neq y \lthen x \rel{R} y \lor y \rel{R} x$,
  \item \textbf{strogo sovisna:} $\all{x, y \in A} x \rel{R} y \lor y \rel{R} x$.
\end{itemize}
%

\begin{naloga}
  Kako iz usmerjenega grafa relacije razberemo refleksivnost in simetričnost? Kaj pa ostale lastnosti?
\end{naloga}

\section{Operacije na relacijah}

\subsection{Unija, presek in komplement relacij}

Ker so relacije pravzaprav podmnožice, lahko na njih uporabljamo operacije unija $\cup$,
presek $\cap$ in komplement $\compl{\Box}$. Denimo, da sta $R, S \subseteq A \times B$ relaciji. Tedaj velja:
%
\begin{align*}
  x (R \cup S) y &\iff x \rel{R} y \lor x S y, \\
  x (R \cap S) y &\iff x \rel{R} y \land x S y, \\
  x \compl{R} y &\iff \lnot (x \rel{R} y).
\end{align*}

\begin{primer}
  Za relacije enakosti in urejenost na realnih številih velja:
  %
  \begin{itemize}
  \item Komplement relacije enakosti $=$ je relacija neenakosti $\neq$.
  \item Unija relacij $<$ in $>$ na realnih številih je relacija $\neq$.
  \item Presek relacij $\leq$ in $\geq$ na realnih številih je relacija $=$.
  \end{itemize}
\end{primer}


\subsection{Transponirana relacija}

Dvojiške relacije lahko tudi \textbf{transponiramo}. Transponiranka relacije $R \subseteq A \times B$ je relacija $\transpose{R} \subseteq B \times A$, definirana s predpisom
%
\begin{equation*}
    y \transpose{R} x \defiff x \rel{R} y
\end{equation*}
%
ali ekvivalentno
%
\begin{equation*}
  \transpose{R} \defeq \set{ (y, x) \in B \times A \such x \rel{R} y }.
\end{equation*}
%
Očitno velja $\transpose{(\transpose{R})} = R$, torej je transponiranje \emph{involucija}.

\begin{primer}
  Transpozicija relacije $<$ na realnih številih $\RR$ je relacija $>$ na $\RR$.
  Komplement relacije $<$ na $\RR$ je relacija $\geq$ na $\RR$.
\end{primer}

\subsection{Kompozitum relacij}

Nadalje definiramo \textbf{kompozitum} relacij $R \subseteq A \times B$ in $S \subseteq B \times C$ kot relacijo $S \circ R \subseteq A \times C$, s predpisom
%
\begin{equation*}
    x (S \circ R) z \defiff \some{y \in B} x \rel{R} y \land y S z
\end{equation*}
%a
ali ekvivalentno
%
\begin{equation*}
  S \circ R \defeq
  \set{ (x, z) \in A \times C \such \some{y \in B} (x,y) \in R \land (y,z) \in S }.
\end{equation*}
%
Se pravi, da sta $x \in A$ in $z \in C$ v relaciji $S \circ R$, če sta preko $S$ in $R$
povezana s kakim elementom $y \in B$.

\begin{primer}
  Kompozitum relacij ">$x$ je otrok od $y$"< in ">$z$ je mati od $y$"< je relacija
  ">$z$ je babica od $x$"<.
\end{primer}

\begin{izrek}
  Komponiranje relacij je asociativno in diagonala je enota.
\end{izrek}

\begin{naloga}
  Zgornji izrek zapiši bolj natančno, da bo razvidno, kaj so domene in kodomene relacij.
\end{naloga}

\begin{dokaz}
  Najprej dokažimo asociativnost kompozicije.
  %
  Naj bo $R \subseteq A \times B$, $S \subseteq B \times C$ in $T \subseteq C \times D$ ter $a \in A$ in $d \in D$. Tedaj velja
  %
  \begin{align}
    a (T \circ (S \circ R)) d &\iff  \notag \\
    \some{c \in C} a (S \circ R) c \land c T d &\iff \notag \\
    \some{c \in C} (\some{b \in B} a \rel{R} b \land b \rel{R} c) \land c T d \label{eq:comp-1}
  \end{align}
  %
  in
  %
  \begin{align}
    a ((T \circ S) \circ R) d &\iff \notag \\
    \some{b \in B} a \rel{R} b \land b (T \circ S) d &\iff \notag \\
    \some{b \in B} a \rel{R} b \land (\some{c \in C} b S c \land c T d) \label{eq:comp-2}
  \end{align}
  %
  Torej je treba dokazati ekvivalenco izjav~\eqref{eq:comp-1} in~\eqref{eq:comp-2}, kar prepuščamo za vajo. Naj namignemo, da je treba pri dokazovanju ekvivalence uporabiti \emph{Frobeniuseva pravilo}
  %
  \begin{equation*}
    (\some{x \in X} p \land q(x)) \liff p \land \some{x \in X} q(x).
  \end{equation*}
  %
  V pravilu je $p$ formula, v kateri $x$ ne nastopa kot prosta spremenljivka.

  Dokažimo še, da je diagonala enota za kompozicijo: naj bo $R \subseteq A \times B$ ter $x \in A$ in $y \in B$. Tedaj velja
  %
  \begin{align*}
    x (\diag[B] \circ R) y &\iff \\
    \some{z \in B} x \rel{R} y \land y \diag[B] z  &\iff \\
    \some{z \in B} x \rel{R} y \land y = z &\iff \\
    x \rel{R} y
  \end{align*}
  %
  V zadnjem koraku smo uporabili ekvivalenco $(\some{u \in U} u = v \land P(v)) \liff P(v)$. Podobno dokažemo, da je diagonala desna enota.
\end{dokaz}

Kompozitum relacij ima torej podobne lastnosti kot kompozitum funkcij.

\subsection{Potenca relacije}

Za $n \in \NN$ definiramo \textbf{$n$-to potenco} relacije $R \subseteq A \times A$ kot relacijo $R^n \subseteq A \times A$ takole:
%
\begin{equation*}
    x R^n y \defiff
    \some{z_0, \ldots, z_n \in A}
    z_0 = x \land z_n = y \land \all{i \in {0, \ldots, n-1}} z_i \rel{R} z_{i+1}.
\end{equation*}
%
To je precej nečitljiva formula. Bolj razumljiva definicija je potenca kot $n$-kratni kompozitum relacije $R$ same s sabo:
%
\begin{equation*}
    R^n \defeq \underbrace{R \circ \cdots \circ R}_n
\end{equation*}
%
kjer se desni $R$ ponovi $n$-krat. Kaj dobimo, ko za $n$ vstavimo $0$? Enoto za kompozitum:
%
\begin{equation*}
    R^0 = \diag[A].
\end{equation*}

\section{Funkcijske relacije}

Funkcijo $f : A \to B$ smo definirali kot \emph{prirejanje} med elementi $A$ in $B$. A
kaj pravzaprav je ">prirejanje"<? Je to funkcijski predpis, program, kaj drugega?
Sedaj lahko povemo natančno: prirejanje, s katerim je podana funkcija, je
\emph{relacija} med elementi domene in kodomene.

\begin{definicija}
  Naj bo $f : A \to B$ funkcija. \textbf{Graf} funkcije $f$ je relacija
  $\Gamma_f \subseteq A \times B$, definirana s predpisom
  %
  \begin{equation*}
    x \,\Gamma_{\!f}\, y \liff f(x) = y
  \end{equation*}
  %
  ali ekvivalentno
  %
  \begin{equation*}
    \Gamma_{\!f} \defeq \set{ (x, y) \in A \times B \such f(x) = y }.
  \end{equation*}
\end{definicija}

Skratka, graf funkcije ni nič drugega kot njeno prirejanje.
%
Sedaj pa se vprašajmo: kakšnim pogojem mora zadoščati relacija $R \subseteq A \times B$, da je prirejanje za neko funkcijo? Odgovor poznamo: biti mora enolična in celovita.

\begin{definicija}
  Relacija $R \subseteq A \times B$ je \textbf{funkcijska relacija}, če je
  %
  \begin{itemize}
  \item \textbf{celovita:} $\all{x \in A} \some{y \in B} x \rel{R} y$ in
  \item \textbf{enolična:} $\all{x \in A} \all{y_1, y_2 \in B} x \rel{R} y_1 \land x \rel{R} y_2 \lthen y_1 = y_2$.
  \end{itemize}
  %
  Ekvivalentno oba pogoja skupaj zapišemo: $\all{x \in A} \exactlyone{y \in B} x \rel{R} y$.
\end{definicija}

Graf $\Gamma_{\!f} \subseteq A \times B$ funkcije $f : A \to B$ je vedno funkcijska relacija.
%
Funkcijska relacija $R \subseteq A \times B$ določa preslikavo $\phi_R : A \to B$ definirano s predpisom
%
\begin{equation*}
  \phi_R : x \mapsto \descr{y \in B} x \rel{R} y.
\end{equation*}
%
Če iz funkcije $f : A \to B$ tvorimo njen graf $\Gamma_{\!f}$, nato pa iz njega funkcijo
$\phi_{\Gamma_{\!f}} : A \to B$ dobimo nazaj prvotno funkcijo $f$. Obratno, če je $R$ funkcijska relacija, tedaj je $\Gamma_{\phi_R}$ enaka $R$. Torej imamo izomorfizem
%
\begin{equation*}
  B^A \iso \set{ R \in \pow{A \times B} \such \all{x \in A} \exactlyone{y \in B} x \rel{R} y }.
\end{equation*}
%

\begin{izjava}
  Kompozitum funkcij se ujema s kompozitumom relacij:
  $\Gamma_{g \circ f} = \Gamma_g \circ \Gamma_{\!f}$.
\end{izjava}

\begin{dokaz}
  Dokaz prepustimo za vajo, še prej pa morate izjavo zapisati bolj natančno: od
  kod in kam slikata preslikavi $f$ in $g$, kaj pomeni kompozitum na levi in kaj
  na desni?
\end{dokaz}


\section{Ovojnice relacij}

Pogosto imamo opravka z relacijo $R$, ki nima želene lastnosti (na primer ni
tranzitivna) mi pa želimo relacijo, ki to lastnost ima. Ali lahko $R$ kako
spremenimo, da bo imela želeno lastnost? Če to lahko naredimo na več načinov,
ali se eden od njih odlikuje?

\begin{definicija}
  Naj bo $R \subseteq A \times A$ relacija. Tedaj pravimo, da je relacija $T \subseteq A \times A$ \textbf{tranzitivna ovojnica} relacije $R$, če velja:
  %
  \begin{enumerate}
  \item $T$ je tranzitivna,
  \item $R \subseteq T$ in
  \item če je $S \subseteq A \times A$ tranzitivna in velja $R \subseteq S$, tedaj je $T \subseteq S$.
  \end{enumerate}
\end{definicija}

Povedano drugače: tranzitivna ovojnica relacije $R$ je \textsf{najmanjša} tranzitivna
relacija, ki vsebuje $R$. Zaenkrat ne vemo, ali ima vsaka relacija tranzitivno
ovojnico.

Izraz ">ovojnica"< uporabljamo, ker si lahko mislimo, da smo relacijo ovili
s tranzitivno relacijo tako, da se ji slednja čim bolj prilega. Namesto ">ovojnica"<
rečemo tudi \textbf{ogrinjača} ali \textbf{zaprtje}.

Poleg tranzitivne ovojnice lahko definiramo tudi druge ovojnice:
%
\begin{itemize}
  \item \textbf{Refleksivna ovojnica} relacije $R \subseteq A \times A$ je najmanjša refleksivna relacija, ki vsebuje $R$.
  \item \textbf{Simetrična ovojnica} relacije $R \subseteq A \times A$ je najmanjša simetrična relacija, ki vsebuje $R$.
  \item \textbf{Refleksivna tranzitivna ovojnica} relacije $R \subseteq A \times A$ je najmanjša refleksivna in tranzitivna relacija, ki vsebuje $R$.
\end{itemize}
%
Ali take ovojnice sploh obstajajo? Obravnavajmo le tranzitivne ovojnice, saj so
ostali dokazi zelo podobni. Ključno pri dokazu obstoja tranzitivne ovojnice je
naslednje dejstvo.

\begin{lema}
  Naj bo $A$ množica in $R : I \to P(A \times A)$ družina relacij na $A$. Če za
  vsak $i \in I$ velja, da je $R_i$ tranzitivna relacija, potem je tudi presek $\bigcap R$ tranzitivna relacija.
\end{lema}

\begin{dokaz}
  Iz definicije preseka družine množic (relacije so le posebne množice) sledi
  %
  \begin{equation*}
  x (\textstyle\bigcap R) y \liff \all{i \in I} x R_i y.
  \end{equation*}
  %
  Dokažimo, da je $\textstyle\bigcap R$ tranzitivna.
  Naj bodo $x, y, z \in A$ in denimo, da velja
  $x (\textstyle\bigcap R) y$ in $y (\textstyle\bigcap R) z$, kar je ekvivalentno
  %
  \begin{equation*}
    \all{i \in I} x R_i y
    \iinn
    \all{j \in I} y R_j z.
  \end{equation*}
  %
  Dokazati moramo $x (\textstyle\bigcap R) z$, kar je ekvivalentno
  %
  $\all{k \in I} x R_k z$.
  %
  Naj bo torej $k \in I$, dokazujemo $x R_k z$. Uporabimo $\all{i \in I} x R_i y$ pri $i = k$ in dobimo $x R_k y$.
  %
  Uporabimo $\all{j \in I} y R_j z$ pri $j = k$ in dobimo $y R_k z$.
  %
  Po predpostavki je $R_k$ tranzitivna relacija, torej velja $x R_k z$.
\end{dokaz}

\begin{izrek}
  Vsaka relacija ima enolično tranzitivno ovojnico.
\end{izrek}

\begin{dokaz}
  Najprej premislimo, da ima $R$ največ eno tranzitivno ovojnico: če sta
  $S$ in $T$ obe tranzitivni ovojnici $R$, potem iz definicije tranzitivne ovojnice
  sledi $S \subseteq T$ in $T \subseteq S$, torej velja $S = T$.

  Sedaj pokažimo, da $R$ ima tranzitivno ovojnico. Naj bo $R \subseteq A \times A$. Definirajmo množico relacij
  %
  \begin{equation*}
    D \defeq \set{ S \subseteq A \times A \such \text{$R \subseteq S$ in $S$ je tranzitivna} }.
  \end{equation*}
  %
  Trdimo, da je $\textstyle\bigcap D$ tranzitivna ovojnica relacije $R$.
  %
  Iz prejšnje leme sledi, da je $\textstyle\bigcap D$ tranzitivna.
  %
  Ker velja $R \subseteq S$ za vsak $S \in D$, seveda sledi $R \subseteq \textstyle\bigcap D$.
  %
  Če je $R \subseteq T$ in $T \subseteq A \times A$ tranzitivna relacija, tedaj velja $T \in D$, torej je $\bigcap D \subseteq T$.
\end{dokaz}


Po istem kopitu pokažemo, da ima vsaka relacija $R \subseteq A \times A$ tudi ostale
ovojnice. Je pa zgornji izrek neroden, ker nam dokaz ne poda uporabnega opisa
tranzitivne ovojnice. Povejmo, kako lahko razne ovojnice opišemo bolj
eksplicitno:
%
\begin{enumerate}
\item Refleksivna ovojnica relacije $R$ je relacija $R \cup \diag[A]$, se pravi, da
  relaciji $R$ dodamo še diagonalo.
\item 
  Simetrična ovojnica relacije $R$ je relacija $R \cup \transpose{R}$.
\item
  Tranzitivna ovojnica relacije $R$ je relacija $R^{+} \defeq \bigcup_{n \geq 1} R^n$, se pravi
  %
  \begin{equation*}
    R^{+} \defeq R \cup (R \circ R) \cup (R \circ R \circ R) \cup \cdots
  \end{equation*}
\item
  Refleksivna tranzitivna ovojnica relacije $R$ je relacija $R^{*} \defeq \bigcup_{n \geq 0} R^n$, se pravi
  %
  \begin{equation*}
  R^{*} \defeq \diag[A] \cup R \cup (R \circ R) \cup (R \circ R \circ R) \cup \cdots
  \end{equation*}
\end{enumerate}


\chapter{Ekvivalenčne relacije}
\textbf{To poglavje še ni predelano v {\LaTeX}.}
%\chapter{Ekvivalenčne relacije}

\section{Ekvivalenčne relacije}

\begin{definicija}
  Relacija $R \subseteq A \times A$ je \textbf{ekvivalenčna relacija}, če je refleksivna, tranzitivna in simetrična. Kadar velja $x \rel{R} y$, pravimo, da sta $x$ in $y$ \textbf{ekvivalentna} glede na~$R$.
\end{definicija}

\begin{opomba}
  Kdor reče ">ekvivalentna relacija"<, je noob. Kdor reče, da sta ">$x$ in $y$
  ekvivalenčna"<, je rookie.
\end{opomba}

Ekvivalenčne relacije se običajno označuje s simboli, ki so podobni znaku za enakost:
$\equiv$, $\sim$, $\simeq$, $\cong$.

\begin{primer}
  Primeri ekvivalenčnih relacij:
  \begin{enumerate}
    \item Relacija ">vzporednost"< med premicami v ravnini.
    \item Relacija ">skladnost"< med trikotniki v ravnini.
    \item Relacija ">podobnost"< med trikotniki v ravnini.
    \item Relacija ">isti ostanek pri deljenju s 7"< na množici $\NN$.
    \item Prazna relacija $\emptyset \subseteq A \times A$ je ekvivalenčna le v primeru, da je $A = \emptyset$.
    \item Polna relacija $A \times A$ je ekvivalenčna.
    \item Diagonala (enakost) je ekvivalenčna relacija.
  \end{enumerate}
\end{primer}

\subsection{Ekvivalenčna relacija porojena s preslikavo}

Posebej pomemben je primer ekvivalenčne relacije \textbf{porojene (ali inducirane) s preslikavo}:
naj bo $f : A \to B$ preslikava in definirajmo relacijo $\sim_f$ na $A$ s predpisom
%
\begin{equation*}
  x \sim_f y \liff f(x) = f(y)
\end{equation*}
%
Tedaj je $\sim_f$ ekvivalenčna relacija:
%
\begin{itemize}
\item refleksivnost: $x ~_f x$ velja, ker velja $f(x) = f(x)$,
\item tranzitivnost: če je $x ~_f y$ in $y ~_f z$, potem je $f(x) = f(y)$ in $f(y) = f(z)$, torej $f(x) = f(z)$ in $x ~_f z$,
\item simetričnost: če je $x ~_f y$, potem je $f(x) = f(y)$, torej $f(y) = f(x)$ in $y ~_f x$.
\end{itemize}
%
Ali je vsaka ekvivalenčna relacija porojena z neko preslikavo?

\begin{primer}
  Premici sta vzporedni natanko tedaj, ko imata enaka smerna vektorja. Če je
  torej $P$ množica vseh premic, $\RR^2$ množica vektorjev v ravnini, in $s : P \to \RR^2$
  preslikava, ki premici $P$ priredi njen enotski smerni vektor, ki leži v zgornji polravnini ali
  na pozitivnem delu osi $x$, tedaj velja
  \begin{equation*}
    p \parallel q \liff s(p) = s(q).
  \end{equation*}
  %
  Torej je vzporednost porojena s preslikavo $s$.
\end{primer}

\section{Ekvivalenčni razredi in kvocientne množice}

\begin{definicija}
  Naj bo $E \subseteq A \times A$ ekvivalenčna relacija. \textbf{Ekvivalenčni razred} elementa $x \in A$ je množica
  $[x]_A \defeq \set{ y \in A \such x \rel{E} y }$. Z besedami: ekvivalenčni razred~$x$ je množica vseh elementov, ki so mu
  ekvivalentni.
\end{definicija}

\begin{opomba}
  Kdor reče ">ekvivalentni razred"<, je newbie.
  Če pustimo šalo ob strani: ekvivalenčni razredi se tako imenujejo zaradi zgodovinskih razlogov. Beseda ">razred"< nakazuje dejstvo, da so imajo elementi ekvivalenčnega razredi vsi nekaj skupnega (">delavski razred"<, ">Tina Maze je razred zase"<) in ne, da niso množice (saj očitno so).
\end{opomba}

\begin{definicija}
  Naj bo $E \subseteq A \times A$ ekvivalenčna relacija. \textbf{Kvocientna ali faktorska množica} ali \textbf{kvocient} $A/E$ je množica vseh ekvivalenčnih razredov:
  %
  \begin{equation*}
    A/E \defeq \set{ \xi \in \pow{A} \such \some{x \in A} \xi = [x]_A }.
  \end{equation*}
  %
  Z izpeljanimi množicami lahko to zapišemo bolj razumljivo
  % 
  \begin{equation*}
    A/E = \set{ [x]_E \such x \in A }.
  \end{equation*}
  %
  \textbf{Kanonična kvocientna preslikava} $q_E : A \to A/E$ je preslikava, ki vsakemu elementu
  priredi njegov ekvivalenčni razred: $q_E(x) \defeq [x]_A$.
\end{definicija}

\begin{izrek}
  Vsaka ekvivalenčna relacija je porojena z neko preslikavo.
\end{izrek}

\begin{dokaz}
  Dokažimo, da je ekvivalenčna relacija porojena s svojo kvocientno preslikavo.

  Naj bo $E$ ekvivalenčna relacija na $A$. Najprej ugotovimo naslednje: za vse $x,
  y \in A$ velja
  %
  \begin{equation*}
    x \rel{E} y \liff [x]_E = [y]_E.
  \end{equation*}

  ($\lthen$) Če je $x \rel{E} y$ potem je $[x]_E \subseteq [y]_E$, ker iz $z \rel{E} x$ in $x \rel{E} y$ sledi $z \rel{E} y$.
  Podobno dokažemo $[y]_E \subseteq [x]_E$.

  ($\Leftarrow$) Če je $[x]_E = [y]_E$ potem je $y \in [y]_E = [x]_E$, torej po definiciji $[x]_E$
  dobimo $x \rel{E} y$.

  Sedaj izrek sledi zlahka: $q_E(x) = q_E(y) \liff [x]_E = [y]_E \liff x \rel{E} y$.
\end{dokaz}


\subsection{Razdelitev množice}

\begin{definicija}
  \textbf{Razdelitev} ali \textbf{particija} množice $A$ je množica nepraznih, paroma
  disjunktnih množic, ki tvorijo pokritje $A$ (kar pomeni, da je $A$ enaka njihovi uniji). Se
  pravi, to je množica $S \subseteq \pow{A}$, za katero velja:
  %
  \begin{enumerate}
  \item Elementi razdelitve so neprazni: $\all{B \in S} B \neq \emptyset$.
  \item Vsaka dva elementa razdelitve sta bodisi enaka bodisi disjunktna:
    %
    \begin{equation*}
      \all{B, C \in S} B = C \lor B \cap C = \emptyset.
    \end{equation*}
  \item Elementi razdelitve tvorijo pokritje $A$, se pravi $A = \bigcup S$.
  \end{enumerate}
\end{definicija}

\begin{primer}
  Primeri razdelitev:
  %
  \begin{enumerate}
  \item Navpične premice tvorijo razdelitev ravnine.
  \item Množici sodih in lihih števil tvorita razdelitev naravnih števil.
  \item Množica $\set{\set{1,2}, \set{3,5}, \set{4,6,7}}$ tvori razdelitev $\set{1,2,3,4,5,6,7}$.
  \item Množica $\set{\set{1,2,3,4,5,6,7}}$ tvori razdelitev $\set{1,2,3,4,5,6,7}$.
  \end{enumerate}
\end{primer}

\begin{izrek}
  Naj bo $E \subseteq A \times A$ ekvivalenčna relacija. Njeni ekvivalenčni razredi tvorijo
  razdelitev množice $A$.
\end{izrek}

\begin{dokaz}
  Dokažimo, da so ekvivalenčni razredi neprazni, paroma disjunktni in da tvorijo poktritje.

  Naj bo $\xi \in \pow{A}$ ekvivalenčni razred za $E$. Tedaj obstaja $x \in A$, da je $\xi = [x]_E$,
  torej je $x \in \xi$ in zato $\xi \neq \emptyset$.

  Naj bosta $\zeta, \xi \in \pow{A}$. Dokazali bomo $\zeta \cap \xi \neq \emptyset \lthen \zeta = \xi$. Če je $x \in \zeta \cap \xi$, potem velja $\zeta \subseteq \xi$ ker: naj bo $y \in \zeta$, tedaj je $y \rel{E} x$ in ker je $x \in \xi$ velja $y \in \xi$. Simetrično dokažemo $\xi \subseteq \zeta$.

  Očitno je unija vseh ekvivalenčnih razredov podmnožica $A$, saj je vsak ekvivalenčni razred podmnožica $A$. Zagotovo
  pa je vsak $x \in A$ v kakem ekvivalenčnem razredu, namreč $x \in [x]_E$.
\end{dokaz}

Torej vsaka ekvivalenčna relacija na $A$ določa razdelitev mnnožice $A$, namreč na
ekvivalenčne razrede. Velja pa tudi obrat: vsaka razdelitev $S \subseteq \pow{A}$ določa ekvivalenčno
relacijo na $A$, namreč $\simeq_S$ definiran s predpisom
\begin{equation*}
    x \simeq_S y \defiff \some{B \in S} x \in B \land y \in B.
\end{equation*}
%
Z besedami: $x$ in $y$ sta ekvivalentna, kadar sta v istem elementu razdelitve. Prazvzaprav
smo ugotovili, da imamo izomorfizem množic
%
\begin{equation*}
  \set{ E \subseteq A \times A \such \text{$E$ je ekvivalenčna relacija na $A$} } \iso
  \set{ S \subseteq \pow{A} \such \text{$S$ je razdelitev $A$} }.
\end{equation*}
%
V eno smer izomorfizem ekvivalenčni relaciji $E$ priredi njeno razdelitev, v drugo pa razdelitvi priredimo ekvivalenčno
relacijo, kakor smo to opisali zgoraj. (Premislite, da sta ti preslikavi inverza.)


\subsection{Prerezi kvocientne preslikave in aksiom izbire}

Ekvivalenčni razred je natanko določen že z enim od svojih elementov, zato pogosto želimo
namesto ekvivalenčnih razredov navesti le njihove predstavnike.

\begin{definicija}
  Naj bo $E$ ekvivalenčna relacija na $A$. Množico $C \subseteq A$, ki vsak
  ekvivalenčni razred relacije $E$ seka natanko enkrat, imenujemo \textbf{izbor predstavnikov}
  (ekvivalenčnih razredov) za relacijo $E$.
\end{definicija}

Izbor predstavnikov $C \subseteq A$ za $E$ določa preslikavo $c : A/E \to A$, ki priredi
ekvivalenčnemu razredu $\xi$ tisti $x \in \xi$, ki je element $C$:
%
\begin{align*}
  c &: A/E \to A \\
  c &: \xi \mapsto \descr{x \in \xi} x \in C
\end{align*}
%
Preslikava $c : A/E \to A$ je \emph{prerez} kvocientne preslikave $q_E : A \to A/E$.

\begin{izjava}
  Če je $s : A/E \to A$ prerez kvocientne preslikave $q_E : A \to A/E$, potem je
  njegova slika $\img{s}(A/E) = \set{ c(\xi) \such \xi \in A/E }$ izbor predstavnikov za $E$.
\end{izjava}

\begin{dokaz}
  Vaja.
\end{dokaz}

Ker izbor predstavnikov in prerez kvocientne preslikave določata drug drugega, včasih tudi
prerez imenujemo ">izbor predstavnikov"<.

\begin{primer}
  Definirajmo $\sim$ na množici celih števil $Z$ s predpisom
  %
  \begin{equation*}
    a \sim b \defiff 7 \mathrel{|} a - b.
  \end{equation*}
  %
  Torej sta števili $a$ in $b$ ekvivalentni, če dasta enak ostanek pri deljenju s~$7$,
  na primer $13 \sim 20$ in $\lnot (13 \sim 15)$.
  %
  Ekvivalenčni razred števila $a$ dobimo tako, da $a$ prištejemo vse večkratnike števila $7$:
  %
  \begin{equation*}
    [a]_{\sim} = \set{ a + 7 \cdot k \such k \in \ZZ }.
  \end{equation*}
  %
  Na primer,
  \begin{equation*}
    [13]_\sim = \set{ 7 \cdot k + 13 \such k \in \ZZ }
           = \set{ \ldots, -22, -15, -8, -1, 6, 13, 20, 27, 34, 41, \ldots}.
  \end{equation*}
  %
  Koliko pa je ekvivalenčnih razredov? Toliko, kot je ostankov pri deljenju s~$7$, torej sedem. Množica
  $\set{0, 1, 2, 3, 4, 5, 6}$ je izbor predstavnikov za $\sim$, saj je vsako celo število ekvivalentno natanko enemu od
  teh števil mo modulu $7$.
  %
  Ni pa to edini izbor! Tudi $\set{0, 1, 2, 3, 4, 5, 6, 13}$ je izbor in prav tako $\set{-7, -6, -5, -4, -3, -2, -1}$.
\end{primer}

Ali ima vsaka ekvivalenčna relacija izbor predstavnikov? Da to vprašanje ni tako
enostavno, kot se zdi na prvi pogled, doma premislite o nalslednji nalogi.

\begin{naloga}
  Na množici realnih števil $\RR$ definiramo relacijo $E$ s predpisom
  %
  \begin{equation*}
    x \rel{E} y  \defiff  x - y \in \QQ.
  \end{equation*}
  %
  Se pravi, da sta števili ekvivalentni, če je njuna razlika racionalno število. Podajte kak
  izbor predstavnikov za $E$.
\end{naloga}

\begin{izrek}
  Naslednje izjave so ekvivalentne:
  %
  \begin{enumerate}
  \item Vsaka surjektivna preslikava ima desni inverz (prerez).
  \item Vsaka ekvivalenčna relacija ima izbor predstavnikov.
  \item Vsaka družina nepraznih množic ima funkcijo izbire.
  \item Produkt družine nepraznih množic je neprazen.
  \end{enumerate}
\end{izrek}

\begin{dokaz}
  ($1 \lthen 2$):
  %
  Naj bo $E \subseteq A \times A$ ekvivalenčna relacija na $A$. Tedaj je $q_E : A \to A/E$
  surjektivna, zato ima po predpostavki (1) prerez, ki določa izbor predstavnikov.

  ($2 \lthen 3$):
  %
  Naj bo $A : I \to \Set$ družina nepraznih množic. Naj bo $\sim$ ekvivalenčna relacija
  na koproduktu $K \defeq \sum_{i \in I} A_i$, porojena s prvo projekcijo $\fst : S \to I$, t.j.,
  %
  \begin{equation*}
    \inj[i](x) \sim \inj[j](y) \liff i = j.
  \end{equation*}
  %
  Po predpostavki (2) obstaja izbor predstavnikov za $\sim$, se pravi taka množica $C \subseteq K$, da
  za vsak $u \in K$ obstaja natanko en $v \in C$, da je $\fst(u) = \fst(v)$. Definirajmo $f : I \to
  \bigcup A$ s predpisom
  %
  \begin{equation*}
    f(i) \defeq \descr{x \in A_i} \inj[i](x) \in C
  \end{equation*}
  %
  Očitno je $f$ funkcija izbire za družino $A$, če je izraz na desni veljaven:
  %
  \begin{itemize}
  \item Enoličnost: iz $\inj[i](x) \in C$ in $\inj[i](y) \in C$ sledi $\inj[i](x) = \inj[j](y)$.
  \item Celovitost: ker je $A_i$ neprazna, obstaja $z \in A_i$, torej obstaja $v \in C$, da je
    $i = \fst(\inj[i](z)) = \fst(v)$, in je potemtakem $\snd(v) \in A_i$ element, za katerega velja
    $\inj[i](\snd(v)) \in C$.
  \end{itemize}

  ($3 \lthen 4$):
  %
  Elementi produkta so funkcije izbire, zato je produkt res neprazen, če obstaja
  kaka funkcija izbire.

  ($4 \lthen 1$):
  %
  Naj bo $f : X \to Y$ surjektivna. Definirajmo družino $A : Y \to \Set$ s
  predpisom $A_y = \invimg{f}(\set{y})$. Ker je $f$ surjektivna, je $A$ družina nepraznih
  množic. Po predpostavki (4) je produkt te družine neprazen, torej vsebuje neko
  funkcijo izbire $c : Y \to \bigcup A$, se pravi, da je $f(c(y)) = y$ za vsak $y \in Y$.
  Opazimo še, da je $\bigcup A = Y$, torej je $c$ prerez $f$.
\end{dokaz}

Izbor prestavnikov je torej ekvivalenten še nekaterim drugim trditvam. Pa te veljajo? Za
to potrebujemo aksiom.

\begin{aksiom}[Aksiom izbire]
  Vsaka družina nepraznih množic ima funkcijo izbire.
\end{aksiom}

Se pravi, če je $A : I \to \Set$ taka družina množica, da za vsak $i \in I$ velja $A_i \neq \emptyset$,
tedaj obstaja $f : I \to \bigcup A$, za katerega je $f(i) \in A_i$ za vse $i \in I$.
%
O aksiomu izbire bomo še govorili.


\subsection{Univerzalna lastnost kvocientne množice}

Naj bo $E$ ekvivalenčna relacija na $A$ in $B$ množica. Pogosto želimo definirati
preslikavo
%
\begin{equation*}
    f : A/E \to B
\end{equation*}
%
s pomočjo preslikave $A \to B$. Kdaj lahko to naredimo?

\begin{izrek}
  Naj bo $E$ ekvivalenčna relacija na $A$ in $g : A \to B$ preslikava, ki je \emph{skladna} z $E$, kar pomeni da $g$
  slika ekvivalentne elemente v enake: $\all{x, y \in A} x \rel{E} y \lthen g(x) = g(y)$. Tedaj obstaja natanko ena
  preslikava $f : A/E \to B$, da je $f([x]_E) = g(x)$ za vse $x \in A$, ali drugače povedano, $f \circ q_E = g$.
\end{izrek}

\begin{dokaz}
  Dokažimo najprej, da imamo največ eno tako preslikavo. Denimo da za $f_1 : A/E \to B$ in
  $f_2 : A/E \to B$ velja $f_1 \circ q_E = f_2 \circ q_E$. Ker je $q_E$ surjektivna, je epi in jo smemo
  krajšati na desni, od koder res sledi $f_1 = f_2$.

  Sedaj dokažimo, da $f$ obstaja. V ta namen naj bo $\phi \subseteq A/E \times B$ relacija
  %
  \begin{equation*}
    \phi(\xi, y) \defiff \some{x \in A} x \in \xi \land g(x) = y.
  \end{equation*}
  %
  Trdimo, da je $\phi$ funkcijska relacija:
  %
  \begin{itemize}
  \item
    Enoličnost: če je $\phi(\xi, y_1)$ in $\phi(\xi, y_2)$, potem obstajata $x_1, x_2 \in \xi$, da je $g(x_1) = y_1$
    in $g(x_2) = y_2$. Ker pa velja $x_1 \rel{E} x_2$ in je $g$ skladna z $E$, sledi $y_1 = g(x1) = g(x_2) = y_2$.

  \item  Celovitost: naj bo $\xi \in A/E$. Tedaj obstaja $x \in \xi$. Očitno velja $g(\xi, g(x))$.
  \end{itemize}
  %
  Naj bo $f : A/E \to B$ preslikava, ki je določena s funkcijsko relacijo $\phi$. Za $x \in A$
  velja $\phi([x]_E, f([x]_E))$, od tod pa iz definicije $\phi$ sledi tudi $g(x) = f([x]_E)$.
\end{dokaz}

\begin{opomba}
  Profesorja prosite, da pojasni ali sem zapiše, zakaj se reče ">univerzalna lastnost"< kvocientne množice.
\end{opomba}


\section{Kanonična razčlenitev preslikave}

Naj bo $f : A \to B$ preslikava. Naj bo $\sim_f$ ekvivalenčna relacija na $A$, ki jo porodi
$f$, in $q_f : A \to A/E$ kanonična kvocientna preslikava (morali bi jo pisati $q_{\sim_f}$,
kar je nečitljivo). Naj bo $i : \img{f}(A) \to B$ kanonična inkluzija slike $f$ v kodomeno.
Preslikava $f : A \to \img{f}(A)$ je skladna s $\sim_f$, zato obstaja (natanko ena) preslikava
$b_f : A/f \to \img{f}(A)$, da velja $b_f([x]_\sim) = f(x)$. Trdimo:
%
\begin{enumerate}
\item $f = i_f \circ b_f \circ q_f$ in
\item $q_f$ je surjektivna, $b_f$ je bijektivna in $i_f$ je injektivna.
\end{enumerate}
%
Računajmo: $f(x) = b_f([x]_\sim) = i_f(b_f([x]_\sim)) = i_f(b_f(q_f(x)))$, za vse $x \in A$, od
koder sledi prva trditev.

Vemo že, da je kanonična kvocientna preslikava surjektivna in kanonična inkluzija
injektivna. Ostane nam še bijektivnost preslikave $b_f$:
%
\begin{itemize}
\item $b_f$ je injektivna: naj bosta $\xi, \zeta \in A/(\sim_f)$ in denimo, da velja $b_f(\xi) = b_f(\zeta)$.
  Obstajata $x, y \in A$, da je $\xi = [x]_\sim$ in $\zeta = [y]_\sim$. Velja
  %
  \begin{equation*}
    f(x) = i_f(b_f(q_f(x))) = i_f(b_f(\xi)) = i_f(b_f(\zeta)) = i_f(b_f(q_f(y))) = f(y),
  \end{equation*}
  %
  torej je $x \sim_f y$ in zato $\xi = [x]_\sim = [y]_\sim = \zeta$.

  \item $b_f$ je surjektivna: naj bo $u \in \img{f}(A)$. Tedaj obstaja $x \in A$, da je $u = f(x)$.
  Vzemimo $\xi = [x]_E$ in preverimo: $b_f(\xi) = b_f([x]_\sim) =f(x) = u$.
\end{itemize}

\chapter{Delne urejenosti}
\textbf{To poglavje še ni predelano v {\LaTeX}.}
%\chapter{Relacije urejenosti}

\section{Relacije urejenosti}
\begin{definicija}
  Relacija $R \subseteq A \times A$ je:
  %
  \begin{enumerate}
  \item \textbf{šibka urejenost}, ko je refleksivna in tranzitivna,
  \item \textbf{delna urejenost}, ko je refleksivna, tranzitivna in antisimetrična,
  \item \textbf{linearna urejenost}, ko je delna urejenost in je strogo sovisna ($\all{x, y \in A} x \rel{R} y \lor y \rel{R} x$).
  \end{enumerate}
\end{definicija}

Za relacije urejenosti ponavadi uporabljamo simbole, ki spominjajo na znak $\leq$, kot so $\preceq$, $\subseteq$, $\sqsubseteq$ ipd.

\begin{primer}
  Primeri urejenosti:
  \begin{enumerate}
    \item Relacija deljivosti na naravnih številih je delna urejenost.
    \item Relacija deljivosti na celih številih je šibka urejenost, ni pa delna urejenost.
    \item Relacija $\leq$ na realnih številih je linearna urejenost.
    \item Relacija $\subseteq$ na $\pow{A}$ je delna urejenost. Za katere množice $A$ je linearna?
    \item Relacija $=$ je delna urejenost. Imenuje se tudi \textbf{diskretna urejenost}.
  \end{enumerate}
\end{primer}


\begin{definicija}
  V delni ureditvi $(P, {\leq})$ je \textbf{veriga} taka podmnožica $V \subseteq P$, ki je linearno urejena z relacijo~$\leq$, se pravi $\all{x, y \in V} x \leq y \lor y \leq x$. \textbf{Antiveriga} je taka podmnožica $A \subseteq P$, ki je diskretno urejena z relacijo~$\leq$, se pravi $\all{x, y \in A} x \leq y \lthen x = y$.
\end{definicija}

\begin{primer}
\end{primer}

\begin{primer}
  Primeri verig in antiverig:
  %
  \begin{itemize}
  \item Če je $(P, {\leq})$ linearno urejena, je vsaka njena podmnožica veriga. Na primer, vsaka podmnožica $\NN$ je veriga glede na~$\leq$.
  \item Potence števila $2$ tvorijo verigo v $\NN$ glede na relacijo deljivosti.
  \item Praštevila tvorijo antiverigo v $\NN$ glede na relacijo deljivosti.
  \item V $(\pow{\QQ}, {\subseteq})$ imamo neštevno verigo
    %
    $V = \set{S \in \pow(\QQ) \mid \text{$S$ je doljna množica}}$.
    %
    Množica $S \subseteq \QQ$ je \textbf{doljna}, če velja
    $\all{x y \in \QQ} x \leq y \land y \in \QQ \lthen x \in \QQ$.
    Res, vsak Dedekindov rez je doljna množica, le-teh pa je neštevno mnogo.
  \end{itemize}
\end{primer}



\subsection{Hassejev diagram}

Končno delno ureditev $(A, \leq)$ lahko predstavimo s \textbf{Hassejevim diagramom}: elemente
množice $A$ narišemo tako, da je $x$ pod $y$, kadar velja $x \leq y$. Nato povežemo vozlišči $x$ in $y$, če je $y$ neposredni naslednik $x$, se pravi, da velja $x \neq y$, $x \leq y$ in iz $x \leq z \leq y$ sledi $x = z \lor z = y$.

\begin{naloga}
  Narišite Hassejev diagram relacije deljivosti na množici $\set{0, 1, \dots, 10}$ ter
  Hassejev diagram relacije $\subseteq$ na množici $\pow(\{a,b,c\})$.
\end{naloga}

\begin{naloga}
  Kako v Hassejevem diagramu prepoznamo verigo? In kako prepoznamo antiverigo?
\end{naloga}


\subsection{Operacije na urejenostih}

\subsubsection{Obratna urejenost}

Če je $\leq$ delna urejenost na $P$ potem je tudi transponirana relacija $\geq$, definirana z
%
\begin{equation*}
    x \geq y \liff x \leq y,
\end{equation*}
%
delna urejenost na $P$. Če je $\leq$ linearna, je $\geq$ linearna.

\subsubsection{Produktna in leksikografska urejenost}

Naj bosta $(P, {\leq_P})$ in $(Q, {\leq_Q})$ delni urejenosti. Na kartezičnem produktu $P \times Q$ lahko definiramo dve urejenosti.

Prva je \textbf{produktna} urejenost
%
\begin{equation*}
  (x_1,y_1) \leq_{\times} (x_2,y_2) \defiff x_1 \leq_P x_2 \land y_1 \leq_Q y_2
\end{equation*}
%
in druga \textbf{leksikografska} urejenost
%
\begin{equation*}
  (x_1,y_1) \preceq_\mathrm{lex} (x_2,y_2)
  \defiff (x_1 \neq x_2 \land x_1 \leq_P x_2) \lor (x_1 = x_2 \land y_1 \leq_Q y_2).
\end{equation*}


\begin{naloga}
  Kako si predstavljamo produktno in leksikografsko ureditev na $[0,1] \times [0,1]$, če $[0,1]$ uredimo z običajno relacijo $\leq$? Na sliki označite območji
  %
  \begin{equation*}
    \set{(x,y) \in [0,1] \times [0, 1] \such (1/2,1/3) \leq_\times (x,y)}
  \end{equation*}
  %
  in
  %
  \begin{equation*}
    \set{(x,y) \in [0,1] \times [0, 1] \such (1/2,1/3) \leq_\mathrm{lex} (x,y)}.
  \end{equation*}
\end{naloga}

\begin{izjava}
  Produktna in leksikografska urejenosti sta delni urejenosti. Leksikografska urejenost linearnih urejenosti je linearna.
\end{izjava}

\begin{dokaz}
  Dejstvo, da je produktna urejenost refleksivna, tranzitivna in antisimetrična, pustimo za vajo. Preverimo, da je leksikografska urejenost $\leq_\mathrm{lex}$ delna urejenost.

  Dokaz, da je $\leq_\mathrm{lex}$ je refleksivna: za vsak $(x, y) \in P \times Q$ velja $x = x \land y \sqsubseteq y$, torej velja $(x, y) \sqsubseteq (x, y)$.

  Dokaz, da je $\leq_\mathrm{lex}$ je antisimetrična: naj bosta $(x_1,y_1), (x_2,y_2) \in P \times Q$ in denimo, da velja
  %
  \begin{equation*}
    (x_1, y_1) \leq_\mathrm{lex} (x_2, y_2) \land (x_2, y_2) \leq_\mathrm{lex} (x_1, y_1)
  \end{equation*}
  %
  To je ekvivalentno
  %
  \begin{align*}
  & (x_1 \neq x_2 \land x_1 \leq_P x_2 \land x_2 \neq x_1 \land x_2 \leq_P x_1) \lor {}\\
  & (x_1 \neq x_2 \land x_1 \leq_P x_2 \land x_2 = x_1 \land y_2 \leq_Q y_1) \lor {}\\
  & (x_1 = x_2 \land y_1 \leq_Q y_2 \land x_2 \neq x_1 \land x_2 \leq_P x_1) \lor {}\\
  & (x_1 = x_2 \land y_1 \leq_Q y_2 \land x_2 = x_1 \land y_2 \leq_Q y_1).
  \end{align*}
  %
  Če v zgornji formuli upoštevamo, da je $x_1 \neq x_2 \land x_1 = x_2$, vidimo, da sta drugi in tretji disjunkt ekvivalentna $\bot$, zato
  je izjava ekvivalentna:
  \begin{align*}
  &(x_1 \neq x_2 \land x_1 \leq_P x_2 \land x_2 \neq x_1 \land x_2 \leq_P x_1) \lor {}\\
  &(x_1 = x_2 \land y_1 \leq_Q y_2 \land x_2 = x_1 \land y_2 \leq_Q y_1).
  \end{align*}
  %
  A tudi prvi disjunkt je ekvivalenten $\bot$, ker iz $x_1 \leq_P x_2 \land x_2 \leq_P x_1$ sledi $x_1 = x_2$, saj je $\leq_P$ po predpostavki antisimetrična. Torej ostane samo zadnji disjunkt, ki je ekvivalenten
  \begin{equation*}
    x_1 = x_2 \land y_1 \leq_Q y_2 \land y_2 \leq_Q y_1.
  \end{equation*}
  %
  Ker je $\leq_Q$ antisimetrična, sledi $x_1 = x_2$ in $y_1 = y_2$, kar smo želeli dokazati.

  Dokaz, da je $\leq_\mathrm{lex}$ tranzitivna: naj bodo $(x_1,y_1), (x_2,y_2), (x_3, y_3) \in P \times Q$ in denimo, da velja
  %
  \begin{equation*}
    (x_1, y_1) \leq_\mathrm{lex} (x_2, y_2) \land (x_2, y_2) \leq_\mathrm{lex} (x_3, y_3).
  \end{equation*}
  %
  To je ekvivalentno
  %
  \begin{align*}
  & (x_1 \neq x_2 \land x_1 \leq_P x_2 \land x_2 \neq x_3 \land x_2 \leq_P x_3) \lor  {} \\
  & (x_1 \neq x_2 \land x_1 \leq_P x_2 \land x_2 = x_3 \land y_2 \leq_Q y_3) \lor {} \\
  & (x_1 = x_2 \land y_1 \leq_Q y_2 \land x_2 \neq x_3 \land x_2 \leq_P x_3) \lor {} \\
  & (x_1 = x_2 \land y_1 \leq_Q y_2 \land x_2 = x_3 \land y_2 \leq_Q y_3)
  \end{align*}
  %
  Obravnavajmo štiri primere in v vsakem od njih dokažimo $(x_1, y_1) \leq_\mathrm{lex} (x_3, y_3)$, se pravi
  $(x_1 \neq x_3 \land x_1 \leq_P x_3) \lor (x_1 = x_3 \land y_1 \leq_Q y_3)$:
  %
  \begin{enumerate}
  \item Če velja $x_1 \neq x_2 \land x_1 \leq_P x_2 \land x_2 \neq x_3 \land x_2 \leq_P x_3$: ker je $\leq$ tranzitivna sledi $x_1 \leq_P x_3$, poleg tega pa velja $x_1 \neq
    x_3$: če bi veljalo $x_1 = x_3$, bi iz predpostavk dobili $x_3 \leq_P x_2 \land x_2 \leq_P x_3$, od koder bi sledilo $x_2 = x_3$, kar je v
    protislovju s predpostavko $x_2 \neq x_3$.

  \item Če velja $x_1 \neq x_2 \land x_1 \leq_P x_2 \land x_2 = x_3 \land y_2 \leq_Q y_3$: ker je $x_2 = x_3$ iz prvih dveh predpostavk sledi $x_1 \neq x_3 \land x_1 \leq_P x_3$.

  \item Če velja $x_1 = x_2 \land y_1 \leq_Q y_2 \land x_2 \neq x_3 \land x_2 \leq_P x_3$: ker je $x_1 = x_2$ iz zadnjih dveh predpostavk sledi $x_1 \neq x_3 \land x_1 \leq_P x_3$.

  \item Če velja $x_1 = x_2 \land y_1 \leq_Q y_2 \land x_2 = x_3 \land y_2 \leq_Q y_3$: torej je $x_1 = x_3$ ker je $=$ tranzitivna in $y_1 \leq_Q y_3$ ker je $\leq_Q$ tranzitivna.
  \end{enumerate}
  %
  Nazadnje preverimo še, da je $\leq_\mathrm{lex}$ linearna, če sta $\leq$ in $\leq_Q$ linearni. Naj bosta $(x_1,y_1), (x_2,y_2) \in P \times Q$. Dokazati želimo
  %
  \begin{equation*}
    (x_1, y_1) \preceq (x_2, y_2) \lor (x_2, y_2) \preceq (x_1, y_1).
  \end{equation*}
  %
  To je ekvivalentno disjunkciji
  % 
  \begin{align*}
    & (x_1 \neq x_2 \land x_1 \leq_P x_2) \lor {} \\
    & (x_1 = x_2 \land y_1 \leq_Q y_2) \lor {} \\
    & (x_2 \neq x_1 \land x_2 \leq_P x_1) \lor {} \\
    & (x_2 = x_1 \land y_2 \leq_Q y_1),
  \end{align*}
  %
  kar je ekvivalentno
  %
  \begin{align*}
    & (x_1 \neq x_2 \land (x_1 \leq_P x_2 \lor x_2 \leq_P x_1)) \lor {} \\
    &(x_1 = x_2 \land (y_1 \leq_Q y_2 \lor y_2 \leq_Q y_1)).
  \end{align*}
  %
  Ker sta $\leq_P$ in $\leq_Q$ linearni, je to ekvivalentno
  %
  \begin{equation*}
    (x_1 \neq x_2 \land \top) \lor (x_1 = x_2 \land \top),
  \end{equation*}
  %
  kar je ekvivalentno
  \begin{equation*}
    (x_1 \neq x_2) \lor (x_1 = x_2).
  \end{equation*}
  %
  To pa drži po zakonu o izključeni tretji možnosti. S tem je linearnost $\leq_\mathrm{lex}$, dokazana.
\end{dokaz}

\subsubsection{Vsota urejenosti}

Naj bosta $(P, \leq_P)$ in $(Q, \leq_Q)$ delni urejenosti. Na vsoti $P + Q$ lahko
definiramo urejenost $\leq_{+}$ s predpisom:
%
\begin{equation*}
  u \leq_{+} v \defiff
  \begin{aligned}[t]
    & (\some{x, y \in P} u = \inl(x) \land v = \inl(y) \land x \leq_P y) \lor {} \\
    & (\some{s, t \in Q} u = \inr(s) \land v = \inr(t) \land s \leq_Q t).
  \end{aligned}
\end{equation*}

\subsubsection{Zaporedna vsota urejenosti}

Naj bosta $(P, \leq_P)$ in $(Q, \leq_Q)$ delni urejenosti. Na vsoti $P + Q$ lahko definiramo urejenost $\leq_{\to}$ s predpisom:
%
\begin{equation*}
  u \leq_{\to} v \defiff
  \begin{aligned}[t]
    &(\some{x, y \in P} u = \inl(x) \land v = \inl(y) \land x \leq_P y) \lor {} \\
    &(\some{x \in P} \some{s \in Q} u = \inl(x) \land v = \inr(s)) \lor {} \\
    &(\some{s, t \in Q} u = \inr(s) \land v = \inr(t) \land s \leq_Q t).
  \end{aligned}
\end{equation*}
%
Torej so vsi elementi $P$ pred vsemi elementi $Q$. Zaporedna vsota linearnih urejenosti je linearna.


\subsubsection{Potenca urejenosti}

Naj bo $(P, \leq)$ delna urejenost in $A$ množica. Na eksponentni množici $P^A$ lahko definiramo urejenost $\preceq$ s predpisom:
%
\begin{equation*}
  f \preceq g \defiff \all{x \in A} f(x) \leq g(x).
\end{equation*}

\begin{naloga}
  Ali je $\preceq$ linearna, kadar je $\leq$ linearna?
\end{naloga}


\subsubsection{Delna urejenost, inducirana s šibko ureditvijo}

Naj bo $(P, \leq)$ šibka ureditev. Relacija $\sim$ na $P$, definirana s predpisom
%
\begin{equation*}
  x \sim y \defiff x \leq y \land y \leq x,
\end{equation*}
%
je ekvivalenčna relacija. Na kvocientu $P/{\sim}$ lahko definiramo relacijo $\preceq$ s
predpisom
%
\begin{equation*}
  [x] \preceq [y] \defiff x \leq y.
\end{equation*}
%
Treba je preveriti, da je relacija dobro definirana, saj smo uporabili predstavnike ekvivalenčnih razredov. Se pravi, ali velja
\begin{equation*}
  x \sim x' \land y \sim y' \lthen (x \leq y \liff x' \leq y') ?
\end{equation*}
%
Pa preverimo. Denimo, da velja $x, y, x', y' \in P$ in $x \sim x'$ in $y \sim y'$.
Torej velja
\begin{equation*}
  x \leq x' \land x' \leq x \land y \leq y' \land y' \land x.
\end{equation*}
%
Sedaj dokažimo $x \leq y \liff x' \leq y'$:
%
\begin{enumerate}
\item Če velja $x \leq y$ potem $x' \leq x \leq y \leq y'$.
\item Če velja $x' \leq y'$, potem $x \leq x' \leq y' \leq y$.
\end{enumerate}
%
Torej je $\preceq$ dobro definirana.

\begin{izjava}
  Relacija, ki je inducirana s šibko ureditvijo, je delna ureditev.
\end{izjava}

\begin{dokaz}
  Refleksivnost in tranzitivnost $\preceq$ sledita iz refleksivnosti in tranzitivnosti~$\leq$. Preverimo antisimetričnost: denimo, da velja $[x] \leq [y]$ in $[y] \leq [x]$. Tedaj velja $x \leq y$ in $y \leq x$, torej velja $x \sim y$ in $[x] = [y]$.
\end{dokaz}

\begin{primer}
  Obravnavajmo cela števila $\ZZ$ in deljivost $\mid$, ki je šibka
  ureditev. Za vse $k, m \in \ZZ$ velja
  \begin{equation*}
    k \sim m \liff k \mid m \land m \mid k \liff |k| = |m|.
  \end{equation*}
  %
  Torej je $\ZZ/{\sim} \cong \NN$, kjer izomorfizem preslika $[k] \mapsto |k|$. Delna ureditev na $\ZZ/{\sim}$ inducirana z deljivostjo je spet deljivost (ko jo prenesemo iz $\ZZ/{\sim}$ na $\NN$ s pomočjo izomorfizma).
\end{primer}


\subsection{Monotone preslikave}

\begin{definicija}
  Preslikava $f : P \to Q$ med delnima urejenostma $(P, {\leq_P})$ in $(Q, {\leq_Q})$ je
  \textbf{monotona} (ali \textbf{naraščajoča}), ko velja $\all{x, y \in P} x \leq_P y \lthen f(x) \leq_Q f(y)$.
\end{definicija}

\begin{definicija}
  Preslikava $f : P \to Q$ med delnima urejenostma $(P, \leq_P)$ in $(Q, \leq_Q)$ je
  \textbf{antitona} (ali \textbf{padajoča}), ko velja $\all{x, y \in P} x \leq_P y \lthen f(y) \leq_Q f(x)$.
\end{definicija}

\begin{opomba}
  V analizi ">monotona"< pomeni ">monotona ali antitona"<. To ni nič
  čudnega, ker ">dan"< tudi pomeni ">dan in noč">.
\end{opomba}

\begin{izrek}
  Kompozicija monotonih preslikav je monotona. Identiteta je monotona.
\end{izrek}

\begin{dokaz}
  Naj bosta $f : P \to Q$ in $g : Q \to R$ monotoni preslikavi med delnimi
  urejenostmi $(P, {\leq_P})$, $(Q, {\leq_Q})$ in $(R, {\leq_R})$. Če je $x \leq_P y$, potem je zaradi monotonosti $f$ tudi $f(x) \leq_Q f(y)$, nato pa je zaradi monotonosti $g$ spet $g(f(x)) \leq_R g(f(y))$. Identiteta je očitno monotona.
\end{dokaz}

\begin{primer}
  Primeri monotonih preslikav:
  \begin{enumerate}
    \item Konstantna preslikava je monotona.
    \item Seštevanje ${+} : \RR \times \RR \to \RR$ je monotona operacija glede na produktno ureditev na $\RR \times \RR$.
    \item Množenje ${\times} : \RR \times \RR \to \RR$ ni monotona operacija.
  \end{enumerate}
\end{primer}


\subsection{Meje}

\begin{definicija}
  Naj bo $(P, {\leq})$ delna urejenost, $S \subseteq P$ in $x \in P$:
  \begin{itemize}

  \item $x$ je \textbf{spodnja meja} podmnožice $S$, ko velja $\all{y \in S} x \leq y$,

  \item $x$ je \textbf{zgornja meja} podmnožice $S$, ko velja $\all{y \in S} y \leq x$,

  \item $x$ je \textbf{infimum} ali \textbf{največja spodnja meja} ali \textbf{natančna spodnja meja} podmnožice $S$, ko je spodnja meja $S$ in velja: za vse $y \in P$, če je $y$ spodnja meja
    $S$, potem je $y \leq x$,

  \item $x$ je \textbf{supremum} ali \textbf{najmanjša zgornja meja} ali \textbf{natančna zgornja meja} podmnožice $S$, ko je zgornja meja $S$ in velja: za vse $y \in P$, če je $y$ zgornja meja $S$, potem je $x \leq y$,

  \item $x$ je \textbf{minimalni element} podmnožice $S$, ko velja $x \in S$ in $\all{y \in S} y \leq x \lthen x = y$,

  \item $x$ je \textbf{maksimalni element} podmnožice $S$, ko velja $x \in s$ in
      $\all{x \in S} x \leq y \lthen x = y$,

  \item $x$ je \textbf{najmanjši} ali \textbf{prvi} element ali \textbf{minimum} podmnožice $S$, ko velja $x \in S$ in $\all{y \in S} x \leq y$,

  \item $x$ je \textbf{največji} ali \textbf{zadnji} element ali \textbf{maksimum} podmnožice $S$, ko velja $x \in S$ in $\all{y \in S} y \leq x$.
\end{itemize}
\end{definicija}

\begin{opomba}
  Minimalni element ni isto kot minimum (in maksimalni element ni isto kot maksimum).
\end{opomba}

Kadar govorimo o ">prvem elementu"< ali ">maksimalnem elementu"< in ne povemo, na
katero podmnožico se nanaša element, imamo običajno v mislih kar celotno delno
ureditev.

\begin{izrek}
  Naj bo $(P, {\leq})$ delna urejenost in $S \subseteq P$. Tedaj ima $S$ največ en
  infimum in največ en supremum, ki ju zapišemo $\inf S$ ter $\sup S$, kadar obstajata.
\end{izrek}

\begin{dokaz}
  Denimo, da sta $x$ in $y$ oba infimum $S$. Ker je $y$ spodnja meja za
  $S$ in $x$ njen infimum, velja $y \leq x$. Podobno velja $x \leq y$, torej $x = y$. Za
  supremum je dokaz podoben.
\end{dokaz}

\begin{primer}
  Supremum končne neprazne množice $S \subseteq \NN$ za relacijo deljivosti $\mid$
  je najmanjši skupni večkratnik elementov iz $S$. Infimum je največji skupni
  delitelj. Kaj pa, če je $S$ prazna ali neskončna?
\end{primer}

\subsection{Mreže}

\begin{definicija}
  Naj bo $(P, {\leq})$ delna urejenost:
  %
  \begin{enumerate}
  \item $(P, \leq)$ je \textbf{mreža}, ko imata vsaka dva elementa $x, y \in P$ infimum in supremum.

  \item $(P, \leq)$ je \textbf{omejena mreža}, ko ima vsaka končna podmnožica $P$ infimum in supremum.

  \item $(P, \leq)$ je \textbf{polna mreža}, ko ima vsaka podmnožica $P$ infimum in supremum.
  \end{enumerate}
  %
  Infimum in supremum elementov $x$ in $y$ pišemo $x \land y$ in $x \lor y$.
\end{definicija}

\begin{izrek}
  Delna urejenost $(P, {\leq})$ je omejena mreža natanko tedaj, ko ima
  najmanjši element in največji element, ter imata vsaka sva elementa infimum in supremum.
\end{izrek}

\begin{dokaz}
  Denimo, da je $(P, \leq)$ omejena mreža. Tedaj $P$ ima najmanjši element, namreč
  $\sup \emptyset$, in največji element, namreč $\inf \emptyset$. Infimum in supremum $x$ in $y$ sta seveda $\inf \set{x, y}$ in $\sup \set{x, y}$.

  Denimo, da ima $P$ najmanjši element $\bot_P$ in največji element $\top_P$, vsaka dva
  elementa pa imata infimum in supremum. Naj bo $S \subseteq P$ končna množica:
  %
  \begin{enumerate}
  \item če je $S = \emptyset$, potem je $\inf S = \top_P$ in $\sup S = \bot_P$,
  \item če je $S = \set{x_1, \ldots, x_n}$ za $n > 0$, potem je $\inf S = \inf \set{x_1, \ldots, x_{n-1}} \lor x_n$ in $\sup S = \sup \set{x_1, \ldots, x_{n-1}} \lor x_n$.
  \end{enumerate}
\end{dokaz}

\begin{primer}
  Primeri mrež:
  %
  \begin{enumerate}
  \item Množica $\two = \set{\bot, \top}$ je omejena mreža za relacijo $\lthen$.
  \item Relacija deljivosti na množici pozitivnih naravnih števil je omejena mreža.
  \item Potenčna množica $\pow{A}$, urejena z $\subseteq$, je polna mreža.
  \item Zaprti interval $[a,b]$, urejen z $\leq$, je polna mreža.
  \item Realna števila $R$, urejena z $\leq$,
  \end{enumerate}
\end{primer}


\chapter{Indukcija in dobra osnovanost}
\textbf{To poglavje še ni predelano v {\LaTeX}.}
%\chapter{Indukcija in dobra osnovanost}

\section{Dobra osnovanost}

\subsection{Indukcija na naravnih številih}

Poznamo že indukcijo na naravnih številih. Zapišemo jo lahko na dva načina,
kjer naslednika števila $n$ označimo $\suc{n}$:
%
\begin{enumerate}
\item Kot aksiom o predikatih na naravnih številih:
  %
  \begin{equation*}
  \phi(0) \land (\all{n \in \NN} \phi(n) \lthen \phi(\suc{n})) \lthen \all{m \in \NN} \phi(m)
  \end{equation*}

\item Kot lastnost podmnožic naravnih števil:
  %
  \begin{equation*}
    \all{S \in \pow{\NN}} 0 \in S \land (\all{k \in \NN} k \in S \lthen \suc{k} \in S) \lthen S = \NN
  \end{equation*}
\end{enumerate}
%
Uporabljali bomo verzijo s podmnožicami. Najprej jo predelajmo v ekvivalentno obliko:
%
\begin{align}
  &\all{S \in \pow{\NN}} 0 \in S \land (\all{k \in \NN} k \in S \lthen \suc{k} \in S) \lthen S = \NN \tag{$\liff$} \\
  &\all{S \in \pow{\NN}} 0 \in S \land (\all{m \in \NN} (\all{k \in \NN} \suc{k} = m \lthen k \in S) \lthen m \in S) \lthen S = \NN \tag{$\liff$} \\
  &\all{S \in \pow{\NN}} (\all{m \in \NN} (\all{k \in \NN} \suc{k} = m \lthen k \in S) \lthen m \in S) \lthen S = \NN. \notag
\end{align}
%
Kaj smo dosegli? Bazo indukcije in indukcijski korak smo združili v eno samo predpostavko
%
\begin{equation}
  \label{eq:ind-N}
  \all{m \in \NN} (\all{k \in \NN} \suc{k} = m \lthen k \in S) \lthen m \in S
\end{equation}
%
Če vstavimo $m \defeq 0$, dobimo:
%
\begin{align}
  &(\all{k \in \NN} \suc{k} = 0 \lthen k \in S) \lthen 0 \in S \tag{$\liff$} \\
  &(\all{k \in \NN} \bot \lthen k \in S) \lthen 0 \in S \tag{$\liff$} \\
  &(\all{k \in \NN} \top) \lthen 0 \in S \tag{$\liff$} \\
  &\top \lthen 0 \in S \tag{$\liff$} \\
  &0 \in S \notag
\end{align}
%
Če vstavimo $m \defeq \suc{n}$ dobimo:
%
\begin{align}
  &(\all{k \in \NN} \suc{k} = \suc{n} \lthen k \in S) \lthen \suc{n} \in S \tag{$\liff$} \\
  &(\all{k \in \NN} k = n \lthen k \in S) \lthen \suc{n} \in S \tag{$\liff$} \\
  &n \in S \lthen \suc{n} \in S \notag
\end{align}
%
To pa sta ravno običajna pogoja za indukcijo.

Ali lahko izrazimo indukcijo na naravnih številih tudi brez operacije naslednik?
Da, s pomočjo relacije $<$:
%
\begin{equation*}
    \all{S \in \pow{\NN}} (\all{m \in \NN} (\all{k \in \NN} k < m \lthen k \in S) \lthen m \in S) \lthen S = \NN
\end{equation*}
%
Temu principu pravimo tudi \textbf{krepka indukcija}, z besedami jo povemo takole: za podmnožico $S \subseteq \NN$ velja
$S = \NN$, če za vse $m \in \NN$ velja ">če so vsa števila manjša od $m$ v $S$, potem je tudi $m$ v $S$"<.

Denimo, da $S$ res ima dano lastnost. Ali je $0 \in S$? Da, ker za vse predhodnike $0$ velja, da
so $S$ (saj jih ni). Ali je $1 \in S$? Da, saj za vse predhodnike $1$ velja, da so v $S$. Ali je $2 \in
S$? Da, saj za vse predhodnike $2$ velja, da so v $S$. In tako naprej.


\subsection{Dobra osnovanost}

Princip indukcije na naravnih številih posplošimo, pri čemer izhajamo iz principa indukcije, izraženega s pomočjo lastnosti~\eqref{eq:ind-N}, v kateri relacijo ">neposredni predhodnik"< nadomestimo s splošno relacijo.

\begin{definicija}
  Relacija $R \subseteq A \times A$ je \textbf{dobro osnovana}, kadar velja
  %
  \begin{equation}
    \label{eq:ind-wf}%
    \all{S \in P(A)} (\all{y \in A} (\all{x \in A} x \rel{R} y \lthen x \in S) \lthen y \in S) \lthen S = A.
  \end{equation}
  %
  Množici $S \subseteq A$, ki zadošča pogoju
  %
  \begin{equation*}
  \all{y \in A} (\all{x \in A} x \rel{R} y \lthen x \in S) \lthen y \in S
  \end{equation*}
  %
  pravimo \textbf{$R$-progresivna} množica ali, da je $S$ \textbf{progresivna za $R$}.
\end{definicija}

Pogoj \eqref{eq:ind-wf} je \emph{indukcijski predpis} za dobro osnovano relacijo~$R$. Nekatere relacije temu predpisu zadoščajo in druge ne. Na primer, relacija ">neposredni predhodnik"< na $\NN$ mu zadošča, saj v tem primeru dobimo običajno indukcijo na~$\NN$.

\begin{primer}
  Preverimo, da je relacija ">neposredni predhodnik"< $P$ na množici $A = \set{0, 1, \ldots, 42}$ dobro osnovana.
  Natančneje, govorimo o relaciji
  %
  \begin{equation*}
    m \rel{P} n \defiff m + 1 = n.
  \end{equation*}
  %
  Naj bo $S \subseteq A$ progresivna množica, torej zadošča
  %
  \begin{equation*}
    \all{y \in A} (\all{x \in A} x + 1 = y \lthen x \in S) \lthen y \in S.
  \end{equation*}
  %
  Če vstavimo $y = 0$, dobimo
  %
  \begin{equation*}
    (\all{x \in A} x + 1 = 0 \lthen x \in S) \lthen 0 \in S,
  \end{equation*}
  %
  kar je ekvivalentno $0 \in S$. Torej je $0 \in S$. Nato vstavimo $y = 1$ in dobimo
  %
  \begin{equation*}
    (\all{x \in A} x + 1 = 1 \lthen x \in S) \lthen 1 \in S,
  \end{equation*}
  %
  kar se poenostavi v $0 \in S \lthen 1 \in S$. Ker smo že dokazali $0 \in S$, sledi tudi $1 \in S$. V naslednjem koraku vstavimo $y = 2$, poenostavimo in dobimo $1 \in S \lthen 2 \in S$, torej $2 \in S$. Tako nadaljujemo do $y = 42$ in ugotovimo, da res velja $S = A$. S tem smo pokazali, da je $P$ dobro osnovana. Seveda ni bistveno, da smo uporabili $42$.
\end{primer}

\subsection{Dvojiška drevesa}

Naravna števila $\NN$ so \textbf{induktivno definirana množica}. To pomeni, da elemente $\NN$
opredelimo s pravili, ki povedo, kako se gradi naravna števila:
%
\begin{itemize}
\item $0 \in \NN$,
\item če je $n \in \NN$, potem je $\suc{n} \in \NN$.
\end{itemize}
%
Množica $\NN$ vsebuje natanko tiste elemente, ki jih lahko zgradimo s pomočjo teh pravil:
%
\begin{equation*}
    0, 0^{+}, 0^{++}, 0^{+++}, 0^{++++}, \ldots
\end{equation*}
%
Tu sta $0$ in $\suc{{}}$ mišljena kot simbolni oznaki, podobno kot $\inl$ in $\inr$ v definiciji vsote množic. Dejstvo,
da $\NN$ vsebuje natanko tiste elemente, ki jih lahko zgradimo s pomočjo $0$ in $\suc{{}}$ ni nič drugega kot indukcija
na~$\NN$.

Podobno lahko definiramo tudi druge induktivne množice, ki tudi zadoščajo principu indukcije.
%
Na primer, \textbf{dvojiška drevesa} so induktivno definirana množica $\Tree$ s predpisoma:
%
\begin{itemize}
\item $\emptyTree \in \Tree$,
\item če je $t_1 \in \Tree$ in $t_2 \in \Tree$, potem je $\tree{t_1, t_2} \in \Tree$
\end{itemize}
%
Z besedami: drevo je bodisi prazno, bodisi je sestavljeno iz dveh \textbf{poddreves}. Ali znamo
našteti vsa drevesa, ali še bolje, jih narisati?
%
\begin{align*}
    & \emptyTree, \\
    & \tree{\emptyTree, \emptyTree} \\
    & \tree{\emptyTree, \tree{\emptyTree, \emptyTree}}, \\
    & \tree{\tree{\emptyTree, \emptyTree}, \emptyTree}, \\
    & \tree{\tree{\emptyTree, \emptyTree}, \tree{\emptyTree, \emptyTree}}, \\
    & \vdots
\end{align*}
%
Definirajmo relacijo $R \subseteq \Tree \times \Tree$ s predpisom:
%
\begin{equation*}
  t \rel{R} s \defiff \some{u \in \Tree} s = \tree{t, u} \lor s = \tree{u, t}.
\end{equation*}
%
To je relacija ">neposredno poddrevo"<. Je dobro osnovana, česar ne bomo dokazali, porodi pa naslednji princip indukcije za dvojiška drevesa.

\begin{izjava}[Indukcija za dvojiška drevesa]
  Naj bo $S \subseteq \Tree$ podmnožica dreves, za katero velja:
  %
  \begin{itemize}
  \item prazno drevo je v $S$,
  \item za vsa drevesa $t_1$ in $t_2$ velja: če je $t_1 \in S$ in $t_2 \in S$, potem je $\tree{t_1, t_2} \in S$.
  \end{itemize}
  %
  Tedaj je $S = \Tree$.
\end{izjava}

Princip povejmo še s pomočjo predikatov.

\begin{izjava}[Indukcija za dvojiška drevesa]
  Naj bo $\phi$ predikat na dvojiških drevesih, za katerega velja:
  %
  \begin{itemize}
  \item baza indukcije: $\phi(\emptyTree)$
  \item indukcijski korak: za vsa drevesa $t_1$ in $t_2$, če velja $\phi(t_1)$ in $\phi(t_2)$, potem
    $\phi(\tree{t_1, t_2})$.
  \end{itemize}
  %
  Tedaj $\all{t \in \Tree} \phi(t)$.
\end{izjava}

Kot vidimo, imamo v indukcijskem koraku \emph{dve} indukcijski predpostavki, ker ima vsako
sestavljeno drevo dve poddrevesi.


\subsubsection{Dobra osnovanost in padajoče verige}

Kako pa bi dobili kak proti-primer, se pravi, relacijo, ki ni dobra osnovanost? Poiskati
moramo kako lastnost, ki jo imajo vse dobre osnovanosti, nato pa relacijo, ki te lastnosti nima.

\begin{definicija}
  Naj bo $R \subseteq A \times A$ relacija na $A$. \textbf{Padajoča veriga} za relacijo $R$
  je zaporedje $a : \NN \to A$, za katerega velja $\all{i \in \NN} a(i+1) \rel{R} a(i)$.
\end{definicija}

Se pravi, da je padajoča veriga zaporedje, za katerega velja
%
\begin{equation*}
  \cdots a_4 \rel{R} a_3 \rel{R} a_2 \rel{R} a_1 \rel{R} a_0
\end{equation*}
%
\textbf{Cikel} za relacijo~$R$ je končna podmnožica $\set{a_0, \ldots, a_n} \subseteq A$ da velja
%
\begin{equation*}
  a_0 \rel{R} a_1 \rel{R} \cdots \rel{R} a_n \rel{R} a_0.
\end{equation*}
%
Iz cikla dobimo padajočo verigo, tako da cikel ponavljamo v nedogled:
%
\begin{equation*}
  \cdots \rel{R} a_0 \rel{R} \cdots \rel{R} a_n
         \rel{R} a_0 \rel{R} \cdots \rel{R} a_n \rel{R} a_0.
\end{equation*}

\begin{lema}
  V dobri osnovanosti ni ciklov in ni padajočih verig.
\end{lema}

\begin{dokaz}
  Dovolj je pokazati, da ni padajočih verig, saj iz cikla dobimo padajočo verigo.
  Denimo, da je $a : \NN \to A$ padajoča veriga za $R \subseteq A \times A$. Dokazali bomo, da $R$ ni dobro
  osnovana. Se pravi, da moramo poiskati $R$-progresivno podmnožico $S \subseteq A$, za katero velja
  $S \neq A$. Vzemimo $S \defeq A \setminus \set{ a(i) \mid i \in \NN}$. Očitno velja $S \neq A$, saj 
  $a(0) \not\in S$. Preverimo, da je $S$ progresivna, se pravi, da je
  %
  \begin{equation*}
    \all{y \in A} (\all{x \in A} x \rel{R} y \lthen x \in S) \lthen y \in S.
  \end{equation*}
  %
  Naj bo $y \in A$ in denimo, da velja
  \begin{equation}
    \label{eq:verige}
    \all{x \in A} x \rel{R} y \lthen x \in S
  \end{equation}
  %
  Dokazati moramo $y \in S$. Obravnavamo dve možnosti:
  %
  \begin{itemize}
  \item če $y \in S$, potem seveda sledi $y \in S$.
  \item če $y \not\in S$, potem obstaja $i \in \NN$, da je $y = a(i)$. Ker je $a(i+1) \rel{R} a(i)$, iz
    predpostavke~\eqref{eq:verige} sledi $y = a(i) \in S$.
  \end{itemize}
  Torej v vsakem primeru velja $y \in S$.
\end{dokaz}

\begin{primer}
  Sedaj lahko zlahka priskrbimo kak proti-primer. Na primer, cela števila $\ZZ$ z relacijo $R \subseteq \ZZ \times \ZZ$
  %
  \begin{equation*}
    a \rel{R} b \defiff a + 1 = b
  \end{equation*}
  %
  niso dobro osnovana, ker imajo padajočo verigo
  %
  \begin{equation*}
    \cdots \rel{R} (-3) \rel{R} (-2) \rel{R} (-1) \rel{R} 0
  \end{equation*}
  %
  Prav tako ni dobro osnovana relacija $<$ na intervalu $[0,1]$, ker imamo padajočo verigo
  $n \mapsto 2^{-n}$.
\end{primer}

\section{Dobra urejenost}

Posplošimo sedaj še krepko indukcijo na naravnih številih. Tokrat bomo najprej posplošili
strogo urejenost $<$.

\subsection{Stroge urejenosti}

\begin{definicija}
  Relacija $R \subseteq A \times A$ je \textbf{stroga urejenost}, če je
  %
  \begin{itemize}
  \item irefleksivna: $\all{x \in A} \lnot (x \rel{R} x)$ in
  \item tranzitivna: $\all{x, y, z \in A} x \rel{R} y \land y \rel{R} z \lthen x \rel{R} z$.
  \end{itemize}
  %
  Stroga urejenost je \textbf{linearna}, če je še
  %
  \begin{itemize}
  \item sovisna: $\all{x, y \in A} x \rel{R} y \lor x = y \lor y \rel{R} x$.
  \end{itemize}
  %
  Za stroge urejenosti uporabljamo simbole $<$, $\subset$, $\prec$, $\sqsubset$ ipd.
\end{definicija}

Relaciji $<$ in $\leq$ na številih sta med seboj povezani, saj denimo za realna števila velja
%
\begin{equation*}
  x < y \iff x \leq y \land x \neq y
\end{equation*}
%
in
%
\begin{equation}
  \label{eq:leq-iff-lteq}
  %
  x \leq y \iff x < y \lor x = y
\end{equation}
%
To velja v splošnem. Stroga urejenost $<$ na množici $A$ porodi delno urejenost $\leq$ na $A$,
definirano s predpisom:
%
\begin{equation*}
    x \leq y  \defiff x = y \lor x \leq y.
\end{equation*}
%
V obratno smer, delna urejenost $\sqsubseteq$ določa strogo urejenost $\sqsubset$, definirano s predpisom
%
\begin{equation}
  \label{eq:leq-iff-neqlt}
  a \sqsubset b  \defiff  a \neq b \land a \sqsubseteq b.
\end{equation}
%
Seveda je treba preveriti naslednja dejstva, ki jih postimo za vajo:
%
\begin{itemize}
\item če je $<$ stroga urejenost, potem je $\leq$ definirana s \eqref{eq:leq-iff-lteq} delna urejenost
\item če je $\sqsubseteq$ delna urejenost, potem je $\sqsubset$ definirana s \eqref{eq:leq-iff-neqlt} stroga urejenost.
\end{itemize}
%
Tako lahko prehajamo med delno in strogo urejenostjo.

\subsection{Dobra ureditev}

\begin{definicija}
  Relacija je \textbf{dobra ureditev}, če je dobro osnovana in stroga linearna ureditev.
\end{definicija}

\begin{izrek}
  Relacija je dobra ureditev natanko tedaj, ko je dobro osnovana in sovisna.
\end{izrek}

\begin{dokaz}
  V eno smer je ekvivalenca očitna, zato dokažimo samo obratno smer. Denimo, da je
  $R \subseteq A \times A$ dobro osnovana in sovisna relacija. Dokazujemo, da je dobra ureditev, se pravi,
  da potrebujemo še irefleksivnost in tranzitivnost $R$.

  Relacija $R$ je irefleksivna: če bi veljalo $x \rel{R} x$ za $x \in A$, potem $R$ ne bi bila dobro
  osnovana, ker bi vsebovala padajočo verigo $\cdots x \rel{R} x \rel{R} x$.

  Relacija $R$ je tranzitivna: denimo, da velja $x \rel{R} y$ in $y \rel{R} z$. Dokazujemo $x \rel{R} z$. Ker je $R$
  sovisna, velja $x \rel{R} z$ ali $x = z$ ali $z \rel{R} x$. Pokažimo, da $x = z$ in $z \rel{R} x$ nista
  možna:
  %
  \begin{itemize}
  \item Če je $x = z$, potem velja $x \rel{R} y$ in $y \rel{R} x$, torej $x$ in $y$ tvorita cikel, a
    $R$ je dobro osnovana, zato to ni možno.
  \item Če velja $z \rel{R} x$, potem dobimo cikel $x \rel{R} y \rel{R} z \rel{R} x$, kar spet ni možno.
  \end{itemize}
\end{dokaz}

\begin{lema}
  \label{lem:nepr-min-veriga}%
  Denimo, da je $<$ stroga urejenost na neprazni množici~$B$. Če $B$ nima
  $\leq$-minimalnega elementa, potem ima padajočo verigo.
\end{lema}

\begin{dokaz}
  Denimo, da $B$ nima minimalnega elementa, torej
  %a
  \begin{equation*}
    \lnot \some{x \in B} \all{y \in B} y \leq x \lthen y = x.
  \end{equation*}
  %
  To je ekvivalentno
  %
  \begin{equation*}
    \all{x \in B} \some{y \in B} y \leq x \land y \neq x
  \end{equation*}
  %
  kar je ekvivalentno
  \begin{equation}
    \label{eq:lema-min-elem}
    \all{x \in B} \some{y \in B} y < x.
  \end{equation}
  %
  Padajočo verigo $b : \NN \to B$ definiramo z zaporedjem izbir: ker je $B$ neprazna, lahko izberemo
  neki element $b(0) \in B$. Denimo, da smo za neki $i \in \NN$ že izbrali elemente $b(0), \ldots, b(i)$
  tako, da velja
  %
  \begin{equation*}
    b(i) < b(i-1) < \ldots < b(1) < b(0).
  \end{equation*}
  %
  Ker $B$ nima minimalnega elementa, $b(i)$ ni minimalni, torej po \eqref{eq:lema-min-elem} obstaja tak $y \in B$, da je $y < b(i)$. Torej lahko izberemo $b(i+1) \in B$, da velja $b(i+1) < b(i)$.
\end{dokaz}

\begin{opomba}
  V zgornjem dokazu smo uporabili \emph{aksiom odvisne izbire}, ki je poseben primer
  aksioma izbire in o katerem bomo še govorili.
\end{opomba}

\begin{izrek}
  \label{izr:dobr-osn-iff}
  Naj bo $\sqsubset$ relacija na $A$. Tedaj so ekvivalentne naslednje izjave:
  %
  \begin{enumerate}
  \item \label{it:dobr-osn-1}%
    $\sqsubset$ je dobro osnovana,
  \item \label{it:dobr-osn-2}%
    vsaka \emph{neprazna} $S \subseteq A$ ima $\sqsubseteq$-minimalni element,
  \item \label{it:dobr-osn-3}%
    $\sqsubset$ nima padajoče verige.
  \end{enumerate}
\end{izrek}

\begin{dokaz}
  ($1 \lthen 2$)
  %
  Denimo, da je $S \subseteq A$ neprazna. Če uporabimo \eqref{it:dobr-osn-1} na $A \setminus S$ dobimo
  %
  \begin{equation*}
    (\all{y \in A} (\all{x \in A} x \sqsubset y \lthen x \in A \setminus S) \lthen y \in A \setminus S) \lthen A \setminus S = A.
  \end{equation*}
  %
  Ker je $S$ neprazna, dobimo zaporedje ekvivalentnih izjav:
  \begin{align*}
    &(\all{y \in A} (\all{x \in A} x \sqsubset y \lthen x \in A \setminus S) \lthen y \in A \setminus S) \lthen \bot
    \tag{$\liff$} \\
    &\lnot (\all{y \in A} (\all{x \in A} x \sqsubset y \lthen x \in A \setminus S) \lthen y \in A \setminus S)
    \tag{$\liff$} \\
    &\some{y \in A} (\all{x \in A} x \sqsubset y \lthen x \in A \setminus S) \land y \not\in A \setminus S
    \tag{$\liff$} \\
    &\some{y \in A} (\all{x \in A} x \sqsubset y \lthen x \not\in S) \land y \in S
    \tag{$\liff$} \\
    &\some{y \in S} \all{x \in A} x \sqsubset y \lthen x \not\in S
    \tag{$\liff$} \\
    &\some{y \in S} (\all{x \in A} x \sqsubset y \lthen x \not\in S) \notag
  \end{align*}
  %
  Torej obstaja element $y \in S$ z lastnostjo, da pod njim ni nobenega elementa iz
  $S$, kar pa pomeni, da je $y$ iskani minimalni element.

  ($2 \lthen 3$) Denimo, da je $a : \NN \to A$ padajoča veriga. Tedaj slika $\set{ a(n) \mid n \in \NN }$ ne bi imela
  minimalnega elementa, v nasprotju z \eqref{it:dobr-osn-2}.

  ($3 \lthen 1$) Denimo, da je $S \subseteq A$ progresivna. Trdimo, da množica $C \defeq A \setminus S$ nima
  minimalnega elementa. Če bi bil $c \in C$ minimalni v $C$, bi to pomenilo
  %
  \begin{equation*}
    \all{x \in A} x \sqsubset c \lthen x \not\in C,
  \end{equation*}
  %
  kar je ekvivalentno
  %
  \begin{equation*}
    \all{x \in A} x \sqsubset c \lthen x \in S.
  \end{equation*}
  %
  Ker je $S$ progresivna, od tod sledi $c \in S$, kar ni mogoče.
  %
  Dokazati moramo, da je $C$ prazna. Če ne bi bila, bi lahko uporabili lemo~\ref{lem:nepr-min-veriga} in dobili padajočo verigo v $A$, kar je v nasprotju s \eqref{it:dobr-osn-3}.
\end{dokaz}

\begin{izrek}
  \label{izr:dobra-urejenost-karakterizacija}
  Naj bo $\sqsubset$ stroga urejenost na $A$. Tedaj so ekvivalentne naslednje izjave:
  %
  \begin{enumerate}
  \item[(1)] $\sqsubset$ je dobro urejena,
  \item[(2)] vsaka \emph{neprazna} množica $S \subseteq A$ ima $\sqsubset$-prvi element: to je tak $x \in S$, da velja
    $\all{y \in S} x \neq y \lthen x \sqsubset y$.
  \item[(3)] $A$ nima $\sqsubset$-padajoče verige in $\sqsubset$ je sovisna.
  \end{enumerate}
\end{izrek}

\begin{dokaz}
  Za nalogo predelajte dokaz prejšnjega izreka v dokaz tega izreka.
\end{dokaz}

\begin{primer}
  Primeri dobro urejenih množic:
  %
  \begin{enumerate}
  \item Končna množica $\set{0, \ldots, n}$ urejena z relacijo $<$.

  \item Naravna števila $\NN$ urejena z relacijo $<$.

  \item Če sta $(P, \leq_P)$ in $(Q, \leq_Q)$ dobri urejenosti, potem je dobro urejena tudi $P + Q$ z relacijo
    $\sqsubseteq$, ki~$P$ postavi pred~$Q$:
    %
    \begin{equation*}
      u \sqsubseteq v \defiff
      \begin{aligned}[t]
        &(\some{x \in P} \some{y \in Q} u = \inl(x) \land v = \inr(y)) \lor {}\\
        &(\some{x \in P} \some{y \in P} u = \inl(x) \land v = \inl(y) \lor x \leq_P y) \lor {}\\
        &(\some{x \in Q} \some{y \in Q} u = \inr(x) \land v = \inr(y) \lor x \leq_Q y).
      \end{aligned}
    \end{equation*}

  \item
    S prejšnjim primerom lahko seštevamo dobre urejenosti, na primer $\NN + \set{0, 1, 2}$ je dobra
    urejenost
    %
    \begin{equation*}
      \inl{0} < \inl{1} < \inl{2} < \cdots < \inr{0} < \inr{1} < \inr{2}.
    \end{equation*}
  \end{enumerate}
\end{primer}

\section{Ordinalna števila}
\label{sec:ordinalna-tevila}

Dobra urejenost na množici~$A$ postavi njene elemente v vrsto (strogo linearno urejenost), ki nima padajočih verig.
Končno množico lahko dobro uredimo na več načinov, na primer elemente $\set{0, 1, 2, \ldots, n-1}$ lahko postavimo v vrsto na $n!$ načinov. Množico vseh naravnih števil lahko postavimo v vrsto brez padajočih verig vsaj na tri načine,
%
\begin{equation*}
  0, 1, 2, 3, 4, 5, \ldots, n, n + 1, \ldots
\end{equation*}
%
in
%
\begin{equation*}
  1, 0, 3, 2, 5, 4, \ldots, 2 n + 1, 2 n, \ldots
\end{equation*}
%
in
%
\begin{equation*}
  0, 2, 4, 6, 8, \ldots, 1, 3, 4, 5, \ldots
\end{equation*}
%
Zdi se, da sta prvi in drugi način ">isti tip"< urejenosti in se razlikujeta od tretjega. Res, v tretji vrsti ima $1$ neskončno predhodnikov, v prvi in drugi pa takega elementa ni. Govorimo o naslednjem pojmu.

\begin{definicija}
  Dobri ureditvi $(P, {\leq_P})$ in $(Q, {\leq_Q})$ \textbf{izomorfni}, če obstajata monotoni preslikavi $f : P \to Q$ in $Q \to P$, da velja $f \circ g = \id[Q]$ in $g \circ f = \id[P]$.
\end{definicija}

Seveda je izomorfnost ekvivalenčna relacija, ki je definirana na pravem razredu vseh dobrih urejenosti. 
Koristno bi bilo imeti kak izbor predstavnikov zanjo, saj bi lahko z njimi merili ">dolžino"< dobre urejenosti. Takim predstavnikom pravimo \textbf{ordinalna števila}. A kako bi jih dobili? Pri 19.~letih je \href{https://en.wikipedia.org/wiki/John_von_Neumann}{John von Neumann} predlagal:
%
\begin{quote}
  \emph{">Ordinalno število je množica svojih predhodnikov, urejeno z relacijo $\in$."<}
\end{quote}
%
Poglejmo, kako deluje njegova ideja:
%
\begin{itemize}

\item Končna ordinalna števila sovpadajo z naravnimi števili:
  %
  \begin{align*}
    0 &\defeq \emptyset \\
    1 &\defeq \set{0} = \set{\emptyset} \\
    2 &\defeq \set{0, 1} = \set{\emptyset, \set{\emptyset}} \\
    3 &\defeq \set{0, 1, 2} = \set{\emptyset, \set{\emptyset}, \set{\emptyset, \set{\emptyset}}} \\
      &\vdots
  \end{align*}

\item Množica vseh končnih ordinalnih števil je prvo neskončno ordinalno število
  %
  \begin{equation*}
    \omega = \set{0, 1, 2, 3, \ldots}.
  \end{equation*}

\item Številu $\omega$ sledijo
  %
  \begin{align*}
    \omega + 1 &\defeq \set{0, 1, 2, \ldots, \omega} \\
    \omega + 2 &\defeq \set{0, 1, 2, \ldots, \omega, \omega + 1} \\
    \omega + 3 &\defeq \set{0, 1, 2, \ldots, \omega, \omega + 1, \omega + 2} \\
               &\vdots \\
    \omega + \omega &\defeq \set{0, 1, 2, \ldots, \omega, \omega + 1, \omega + 2, \ldots} \\
    \omega + \omega + 1 &\defeq \set{0, 1, 2, \ldots, \omega, \omega + 1, \omega + 2, \ldots, \omega + \omega} \\
               &\vdots
  \end{align*}
\end{itemize}

\begin{naloga}
  Kako bi si predstavljali naslednje ordinale: $\omega + \omega + \omega$, $\omega \cdot \omega$, $\omega^3$, $\omega^\omega$?
\end{naloga}

Von Neumann je imel pravo idejo, a pušča kanček dvoma, ker je definicija ordinalnega števila \emph{rekurzivna} (se nanaša sama nase). Če se malce potrudimo, da lahko von Neumannove ordinale opredelimo neposredno.

\begin{definicija}
  Množica $z$ je \textbf{tranzitivna}, če iz $x \in y$ in $y \in z$ sledi $x \in z$.
\end{definicija}

\noindent
%
Poimenovanje je smiselno, saj je pogoj v definiciji ravno tranzitivnost relacije~$\in$.
Ekvivalentno lahko pogoj izrazimo takole: množica $z$ je tranzitivna, če iz $y \in z$ sledi $y \subseteq z$.

\begin{primer}
  Množica $\set{\emptyset, \set{\emptyset}, \set{\set{\emptyset}}}$ je tranzitivna, niso pa vsi njeni elementi tranzitivne množice, saj $\set{\set{\emptyset}}$ ni tranzitivna, ker $\emptyset \in \set{\emptyset} \in \set{\set{\emptyset}}$ vendar $\emptyset \not\in \set{\set{\emptyset}}$.
\end{primer}

\begin{naloga}
  Dokažite, da so ekvivalentni pogoji:
  %
  \begin{enumerate}
  \item $A$ je tranzitivna množica,
  \item $\bigcup A \subseteq A$,
  \item $A \subseteq \pow{A}$.
  \end{enumerate}
\end{naloga}

Sedaj lahko zapišemo definicijo von Neumannovih ordinalov, ki ni rekurzivna.

\begin{definicija}
  \label{def:von-neuman-ordinal}
  \textbf{(Von Neumannov) ordinal} je tranzitivna množica, ki je z relacijo $\in$ dobro urejena.
\end{definicija}

Razred vseh von Neumannovih ordinalov označimo z $\On$ (v angleščini ">ordinal number"<). To je pravi razred, česar ne bomo dokazali. Kogar zanima dokaz, naj poišče ">Burali-Fortijev paradoks"<, ki je celo starejši od Russellovega paradoksa.

\begin{naloga}
  Poiščite množico, ki \emph{ni} tranzitivna in je dobro urejena z relacijo $\in$.
\end{naloga}

Ali definicija~\ref{def:von-neuman-ordinal} res sovpada z idejo, da je ordinal množica svojih prednikov? To potrjuje naslednja izjava.

\begin{izjava}
  Če je $\alpha$ ordinal in $\beta \in \alpha$, potem je $\beta$ ordinal.
\end{izjava}

\begin{dokaz}
  Ker je $\alpha$ tranzitivna množica, je $\beta \subseteq \alpha$, zato je $\beta$ z relacijo~$\in$ dobro urejen. Dokazati moramo še, da je $\beta$ tranzitivna množica. Denimo, da je $\gamma \in \beta$.
  Tedaj je $\gamma \in \alpha$ in ker je $\alpha$ z $\in$ linearno urejen, velja bodisi $\gamma \in \beta$ bodisi $\gamma = \beta$ bodisi $\beta \in \gamma$. A ker druga in tretja možnost ne prideta v poštev, saj bi dobili cikel $\gamma \in \beta \in \gamma$, velja prva, kar smo želeli dokazati.
\end{dokaz}

\begin{naloga}
  V zgornjem dokazu smo uporabili naslednje dejstvo: če je $(P, {<})$ dobra ureditev in $Q \subseteq P$, tedaj je $Q$ z relacijo $<$ zoženo na~$Q$ tudi dobra ureditev. Zapišite dokaz.
\end{naloga}

Brez dokaza navedimo, da so von Neumannovi ordinali izbor predstavnikov za dobre urejenosti.

\begin{izrek}
  Vsaka dobra ureditev je izomorfna natanko enemu von Neumannovemu ordinalu.
\end{izrek}



\chapter{Moč množic}

V tej lekciji bomo govorili o velikosti množic, končnih množicah in neskončnih množicah.

\section{Aksiom odvisne izbire}

Kasneje bom potrebovali inačico aksioma izbire, ki se glasi:

\begin{aksiom}[Odvisna izbira]
  Naj bo $A$ neprazna množica in $R \subseteq A \times A$ celovita relacija, se pravi
  $\all{x \in A} \some{y \in A} x \, R \, y$.
  %
  Tedaj obstaja tako zaporedje $a : \NN \to A$, da za vse $n \in N$ velja $a_n \, R \,  a_{n+1}$.
\end{aksiom}

Aksiom odvisne izbire sledi iz aksioma izbire, česar tu ne bomo dokazali.

Aksiom odvisne izbire se v praksi uporabi, kadar želimo konstruirati zaporedje $a : \NN \to A$, pri čemer sta izpolnjena dva pogoja:
%
\begin{enumerate}
\item za vsak člen zaporedja $a_n$ imamo na voljo eno ali več izbir,
\item izbire za člen $a_{n+1}$ so odvisne od tega, kaj smo izbrali za $a_n$.
\end{enumerate}
%
Primer uporabe bomo videli v nadaljevanju.

\section{Končne množice}

Kako bi definirali pojem ">končna množica"<?

\begin{definicija}
  Za vsako naravno število $n \in \NN$, naj bo
  \textbf{standardna končna množica} $[n] = \set{k \in \NN \such k < n}$.
\end{definicija}

\noindent
Torej velja
%
\begin{align*}
  [0] &= \{\} \\
  [1] &= \{0\} \\
  [2] &= \{0, 1\} \\
  [3] &= \{0, 1, 2\} \\
      &\vdots
\end{align*}

\begin{definicija}
  Množica je \textbf{končna}, če je izomorfna kaki standardni končni množici.
\end{definicija}


Velja naslednje (ne bomo dokazali): če je $A \iso [m]$ in $A \iso [n]$, potem je $m = n$. Torej za končno
množico~$A$ obstaja natanko en $n \in N$, da velja $A \iso [n]$. Temu $n$ pravimo \textbf{moč} množice $A$,
saj nam pove, koliko elementov ima $A$. Moč končne množice $A$ označimo z $|A|$.

Za moči končnih množic velja
%
\begin{align*}
  |[n]| &= n, \\
  |A \times B| &= |A| \times |B|, \\
  |A + B| &= |A| + |B|, \\
  |B^A| &= |B|^{|A|}.
\end{align*}
%
Zgornje enačbe je treba razumeti pravilno: na levi nastopajo $\times$, $+$ in potenciranje kot operacije na množicah, na desni pa kot operacije na naravnih številih.

Za unijo velja \textbf{pravilo vključitve in izključitve}:
%
\begin{equation*}
 |A \cup B| = |A| + |B| - |A \cap B|.
\end{equation*}
%
Pravilo se tako imenuje, ker smo pri štetju elementov $A \cup B$ \emph{vključili} elemente $A$ in $B$, nato pa \emph{izključili} elemente preseka $A \cap B$, da jih ne bi šteli dvakrat.
%
Pravilo vključitve in izključitve za tri množice se glasi
%
\begin{equation*}
  |A \cup B \cup C| = |A| + |B| + |C| - |A \cap B| - |B \cap C| - |C \cap A| + |A \cap B \cap C|.
\end{equation*}

\begin{naloga}
  Zapišite pravilo vključitve in izključitv za unijo $A_1 \cup A_2 \cup \cdots \cup A_n$.
\end{naloga}


\section{Neskončne množice}

\begin{definicija}
  Množica je \textbf{neskončna}, če ni končna.
\end{definicija}

\begin{izrek}
  Množica $A$ je neskončna natanko tedaj, ko obstaja injektivna preslikava $\NN \to A$.
\end{izrek}

\begin{dokaz}

($\lthen$)
%
Denimo, da $A$ ni končna.
Injektivno preslikavo $e : \NN \to A$ definiramo s pomočjo akisoma odvisne izbire.
Ker $A$ ni izomorfna $[0]$, ni prazna, torej obstaja $e(0) \in A$.
Denimo, da smo že definirali $e$ kot injektivno preslikavo $[n] \to A$.
Tedaj jo lahko razširimo na injektivno preslikavo $e : [n+1] \to A$ takole: ker $e$ ni surjektivna (če bi bila, bi veljalo $A \iso [n]$ in $A$ bi bila končna), obstaja $x \in A$, ki ni v sliki $e$.
Sedaj \emph{izberemo} $e(n) \in A$, ki ni v sliki.
Tako dobimo $e : \NN \to A$, ki je injektivna.

($\Leftarrow$)
%
Denimo, da obstaja injektivna preslikava $e : \NN \to A$.
Če bi za neki $n$ veljalo $A \iso [n]$, bi imeli izomorfizem $f : A \to [n]$.
Tedaj bi bil kompozitum $f \circ e : \NN \to [n]$ injektivna preslikava, ta pa ne obstaja (dokaz opustimo).
\end{dokaz}

\subsection{Moč množic}

Tudi neskončnim množicam želimo prirediti moč, se pravi, neko mero velikosti. Preden pa nam bo to uspelo, se najprej naučimo primerjati velikost množic, ne da bi pri tem govorili o ">številu elementov"<.

\begin{definicija}
  Množici $A$ in $B$ imata enako moč, sta \textbf{ekvipolentni}, kadar sta izomorfni.
\end{definicija}

Ekvipolentnost in izomorfnost sta torej sinonima, ki pa se uporabljata v različnih situacijah. O ekvipolentnosti govorimo, ko imamo v mislih velikost množic ali število elementov. Izomorfnost je širši pojem, ki se uporablja tudi v algebri, topologiji in povsod, kjer imamo opravka z matematičnimi strukturami, in pomeni ">enakovredna struktura"<.

Spomnimo se, da je izomorfnost in torej tudi ekvipolentnost ekvivalenčna relacija.
Torej lahko tvorimo ekvivalenčne razrede glede na ekvipolentnost: vsaki množici $A$ priredimo razred vseh množic, ki so jih ekvipolentne:
%
\begin{itemize}
\item $[\emptyset]_{\iso} = \{ \emptyset \}$,
\item $[\set{ \unit }]_{\iso}$ je \emph{pravi razred} vseh enojcev,
\item $[\set{0, 1}]_{\iso}$ je \emph{pravi razred} vseh množic z dvema elementoma,
\item itd.
\end{itemize}
%
Dejstvo, da so razredi glede na izomorfnost pravi razredi in ne množice, je precej nerodna reč, saj z njimi ne moremo udobno delati (potrebovali bi ">super razrede"<, katerih elementi so razredi).
Izognemo se jim tako, da namesto z razredi delamo z izborom predstavnikov.

Pravzaprav smo ta trik že uporabili, ko smo govorili o moči končnih množic, ko smo za predstavnike ekvipolnenčnih razredov končnih množic izbrali standardne končne množice. Le-te nam lahko služijo kot ">števila"<, s katerimi opišemo moči končnih množic, saj med standardno končno množico $[n]$ in številom $n$ ni bistvene razlike. (Še več, kasneje bomo videli, da lahko naravna števila obravnavamo tako, da dejansko so standardne končne množice!)

Kako bi torej izbrali predstavnike razredov za ekvipolentnost za vse množice?
Če bi nam to uspelo, bi take predstavnike lahko uporabili kot števila, imenujejo se \textbf{kardinalna števila}, s katerimi bi merili moč množic.
Na to vprašanje bomo odgovorili v poglavju o kumulativnih hierarhiji, tu bomo le uporabili dejstvo, da je to možno narediti.

Vsaki množici $A$ torej priredimo nekega predstavnika razreda $[A]_{\iso}$, ki ga označimo $|A|$ in ga imenujemo \textbf{moč} množice~$A$. Za končne množice so to kar naravna števila, za splošne množice pa so to kardinalna števila.

Moči množic lahko primerjamo med seboj, čeprav ne vemo, kaj točno naravna števila so!

\begin{definicija}
  Naj bosta $A$ in $B$ poljubni množici. Pravimo:
  %
  \begin{enumerate}
  \item $A$ ima enako moč kot $B$, pišemo $|A| = |B|$, ko obstaja bijektivna preslikava $A \to B$.
  \item $A$ ima moč manjšo ali enako $B$, pišemo $|A| \leq |B|$, ko obstaja injektivna preslikava $A \to B$.
  \item $A$ ima moč manjšo kot $B$, pišemo $|A| < |B|$, če velja $|A| \leq |B|$ in $|A| \neq |B|$.
  \end{enumerate}
\end{definicija}

\begin{izrek}
  \label{izr:leq-iff-empty-or-onto}
  $|A| \leq |B|$ natanko tedaj, ko je $A = \emptyset$ ali obstaja surjekcija $B \to A$.
\end{izrek}

\begin{dokaz}
  Denimo, da je $f : A \to B$ injektivna in $A \neq \emptyset$. Torej obstaja neki $a \in A$.
  Definiramo preslikavo $g : B \to A$ takole:
  %
  \begin{equation*}
    g(y) = x  \defiff f(x) = y \lor (y \not\in f_{*}(A) \land x = a).
  \end{equation*}
  %
  Povedano malo drugače:
  %
  \begin{equation*}
    g(y) =
    \begin{cases}
      f^{-1}(y) & \text{če $y \in f_{*}(A)$,} \\
      a         & \text{če $y \not\in f_{*}(A)$.}
    \end{cases}
  \end{equation*}
  %
  Ker velja $g \circ f = \id[A]$, je $g$ retrakcija in zato surjektivna.

  Obratno, denimo, da je $A$ prazna ali obstaja surjekcija $f : B \to A$. Če je $A$
  prazna, je edina preslikava $\emptyset \to B$ injektivna. Če je $f : B \to A$ surjektivna,
  ima prerez (zakaj?), ki je injektivna preslikava.
\end{dokaz}


\subsection{Cantorjev izrek}

\begin{izrek}[Cantor]
  $|A| < |\pow{A}|$.
\end{izrek}

\begin{dokaz}
  Najprej dokažimo $|A| \leq |\pow{A}|$. Iščemo injektivno preslikava $f : A \to \pow{A}$. Vzemimo $f(x) = \{x\}$. Zlahka preverimo, da je $f$ res injektivna.

  Sedaj dokazujemo, da ne obstaja bijekcija $A \to \pow{A}$. Dokazali bomo, da ne obstaja surjekcija $A \to \pow{A}$, kar zadostuje. Denimo, da je $g : A \to \pow{A}$ poljbuna preslikava. Trdimo, da $g$ ni surjekcija. Res, podmnožica
  %
  \begin{equation*}
    S = \set{x \in A \mid x \not\in g(x) }
  \end{equation*}
  %
  ni v sliki $g$. Če bi bila, bi za neki $y \in A$ veljalo $g(y) = S$, a to bi vodilo v protislovje:
  % 
  \begin{itemize}
  \item velja $y \not\in S$: če $y \in S$ potem $y \not\in g(y) = S$ po definiciji $S$,
  \item velja $\lnot (y \not\in S)$: če $y \not\in S$ potem $y \not\in g(y) = S$.
  \end{itemize}
\end{dokaz}


\subsection{Števne in neštevne množice}

Moč množice $\NN$ označimo z $\aleph_0$.

\begin{definicija}
  Množica $A$ je \textbf{števna}, če velja velja $|A| \leq \aleph_0$.
\end{definicija}

\begin{definicija}
  Množica $A$ je \textbf{neštevna}, če ni števna.
\end{definicija}

\begin{izrek}
  Za vsako množico $A$ so ekvivalentne naslednje izjave:
  %
  \begin{enumerate}
  \item $A$ je števna,
  \item obstaja injektivna preslikava $A \to \NN$,
  \item $A$ je prazna ali obstaja surjektivna preslikava $\NN \to A$,
  \item obstaja surjektivna preslikava $\NN \to \one + A$,
  \item $A$ je končna ali izmoforna $\NN$.
  \end{enumerate}
\end{izrek}

\begin{dokaz}
$(1 \lthen 2)$
%
Če je $A$ števna, velja $|A| \leq \aleph_0 = |\NN|$, torej obstaja injektivna $A \to
\NN$ po definiciji relacije $\leq$.

$(2 \lthen 3)$
%
To sledi neposredno iz Izreka~\ref{izr:leq-iff-empty-or-onto}.

$(3 \lthen 4)$
%
Denimo, da je $A$ prazna ali obstaja surjektivna preslikava $\NN \to A$:
%
\begin{enumerate}
\item
  Če je $A = \emptyset$, potem seveda obstaja surjektivna preslikava $\NN \to \one + A$, in sicer
  $n \mapsto \inl \unit$.
\item 
  Če obstaja surjektivna preslikava $f : \NN \to A$, potem lahko definiramo surjektivno
  preslikavo $g : \NN \to \one + A$ s predpisom
  %
  \begin{equation*}
    g(n) =
    \begin{cases}
      \inl \unit      &\text{če $n = 0$,}\\
      \inr (f(n-1))   &\text{če $n > 0$.}
    \end{cases}
  \end{equation*}
\end{enumerate}

$(4 \lthen 5)$
%
Denimo, da obstaja surjektivna preslikava $r : \NN \to \one + A$.
Dokazali bomo, da je $A$ izomorfna $\NN$, če ni končna.
Predpostavimo torej, da $A$ ni končna.
Preslikva $r$ ima prerez $s : \one + A \to \NN$, ki je seveda injektivna preslikava.
Preslikva $s \circ \inr : A \to \NN$ je kompozitum injektivnih preslikav, zato je injektivna.
Ker $A$ ni končna, obstaja tudi injektivna preslikava $\NN \to A$.
Po izreku Cantor-Schröder-Bernstein, ki ga bomo dokazali spodaj, je torej $A$ izomorfna $\NN$.

$(5 \lthen 1)$
%
Če je $A$ končna, je števna, ker očitno velja $A = |[n]| \leq |\NN| = \aleph_0$.
Če je $A$ izomorfna $\NN$, potem velja $|A| = |\NN| \leq |\NN| = \aleph_0$.
\end{dokaz}

\begin{izrek}
  $\NN \times \NN \iso \NN$.
\end{izrek}

\begin{dokaz}
  Za vajo, poiščite dokaz v zapiskih iz analize ali na internetu.
\end{dokaz}

\textbf{Števna družina} je družina $A : I \to \Set$, katere indeksna množica~$I$ je števna.

\begin{izrek}
  Unija števne družine števnih množic je števna.
\end{izrek}

\begin{dokaz}
  Izrek bomo dokazali le za primer, ko je indeksna množica~$\NN$.
  %
  Najprej obravnavajmo unijo družine $A : \NN \to \Set$, kjer je $A_n$ števna za vse $n \in \NN$.
  Za vsak $n \in \NN$ obstaja surjektivna preslikava $\NN \to A_n + \one$. Po aksiomu izbire obstaja funkcija izbire
  %
  \begin{equation*}
    e \in \prod_{n \in \NN} \set{f : \NN \to A_n + \one \such \text{$f$ surjekcija}}.
  \end{equation*}
  %
  Definiramo $e' : \NN \times \NN \to \one + \bigcup_{n \in \NN} A_n$ s predpisom
  %
  \begin{equation*}
    e'(n, k) = e(n)(k).
  \end{equation*}
  %
  Trdimo, da je $e'$ surjekcija iz $\NN \times \NN$ na $\one + \bigcup_{n \in \NN} A_n$.
\end{dokaz}


\subsection{Cantor-Schröder-Bernsteinov izrek in zakon trihotomije}

\begin{izrek}[Cantor-Schröder-Bernstein]
  Če obstajata injektivni preslikava $A \to B$ in $B \to A$, potem obstaja bijektivna preslikava $A \to B$.
\end{izrek}

\begin{dokaz}
  Definirajmo družino $C : \NN \to \mathsf{Set}$ takole:
  %
  \begin{align*}
    C_0 &= A \setminus g_{*}(B), \\
    C_{n+1} &= g_{*}(f_{*}(C_n).
  \end{align*}
  %
  Naj bo $D = \bigcup_{n \in \NN} C_n$. Očitno je $C_n \subseteq A$ za vse
  $n \in \NN$, zato velja tudi $D \subseteq A$.

  Ker je $g$ injektivna, je bijekcija kot preslikava $g : B \to g_{*}(B)$, zato
  obstaja inverz $g^{-1} : g_{*}(B) \to B$. Trdimo, da velja
  $A \setminus D \subseteq g_{*}(B)$. Res, če velja $x \in A \setminus D$, tedaj
  $x \not\in D$ in zato $x \not\in C_0 = A \setminus g_{*}(B)$, od koder sledi
  $x \in g_{*}(B)$. Od tod sledi, da lahko $g^{-1}$ uporabimo na
  $x \in A \setminus D$.

  Definirajmo $h : A \to B$ s predpisom
  % 
  \begin{equation*}
    h(x) =
    \begin{cases}
      f(x), & \text{če $x \in D$,} \\
      g^{-1}(x) &\text{če $x \in A \setminus D$.}
    \end{cases}
  \end{equation*}
  %
  Dokažimo, da je $h$ injektivna preslikava.
  Denimo, da za $x, y \in A$ velja $h(x) = h(y)$. Obravnavamo štiri primere:
  %
  \begin{enumerate}
  \item Če je $x \in D$ in $y \in D$, potem je $f(x) = h(x) = h(y) = f(y)$ in
    zato $x = y$, saj je~$f$ injektivna.
  \item Če je $x \in A \setminus D$ in $y \in A \setminus D$, potem je
    $g^{-1}(x) = h(x) = h(y) = g^{-1}(y)$ in zato $x = y$, saj je $g^{-1}$
    injektivna.
  \item Če je $x \in D$ in $y \in A \setminus D$, potem je
    $f(x) = h(x) = h(y) = g^{-1}(y)$, zato je $y = g(g^{-1}(y)) = g(f(x))$.
    Obstaja $n \in \NN$, da je $x \in C_n$, od tod pa sledi
    $y = g(f(x)) \in C_{n+1} \subseteq D$, kar je v protislovju z
    $y \in A \setminus D$. Torej se ta primer sploh ne more zgoditi.
  \item Če je $x \in A \setminus D$ in $y \in D$, je razmislek kot v prejšnjem
    primru, le da zamenjamo vlogi~$x$ in~$y$.
  \end{enumerate}

  Preveriti moramo še, da je $h$ surjektivna preslikava. Naj bo $z \in B$.
  Poiskati moramo tak $x \in A$, da velja $h(x) = z$. Obravnavamo dva primera:
  %
  \begin{enumerate}
  \item Če $z \in f_{*}(D)$, potem obstaja $x \in D$, da je $f(x) = y$, s tem pa
    velja tudi $h(x) = f(x) = z$.
  \item Če velja $z \not\in f_{*}(D)$, potem vzamemo $x = g(z)$. Preverimo, da
    velja $h(x) = z$.

    Najprej dokažimo $x \not\in D$. Če bi namreč veljalo $x \in D$, potem bi
    obstajal $n \in \NN$, da je $x \in C_n$. Poleg tega
    $x = g(z) \not\in A \setminus g_{*}(B) = C_0$, zato velja $n > 0$. Se pravi,
    da obstaja $y \in C_{n-1}$, da je $g(z) = x = g(f(y))$. Ker je $g$
    injektivna, sledi $z = f(y)$, kar je v nasprotju z predpostavko
    $z \not\in f_{*}(D)$. Torej res velja $x \not\in D$.

    Ker $x \not\in D$, velja $h(x) = g^{-1}(x) = g^{-1}(g(z)) = z$, kar smo
    želeli dokazati.
  \end{enumerate}
\end{dokaz}

\begin{posledica}
  Če $|A| \leq |B|$ in $|B| \leq |A|$, potem $|A| = |B|$.
\end{posledica}

\begin{dokaz}
  To sledi neposredno iz izreka CSB in definicije $\leq$.
\end{dokaz}

Brez dokaza omenimo še, da velja \textbf{zakon trihotomije}: za vsaki množici $A$ in $B$
velja
%
\begin{equation*}
  |A| < |B| \lor |A| = |B| \lor |B| < |A|.
\end{equation*}
%
Relacija $\leq$ torej uredi moči množic linearno.



\subsection{Moč kontinuuma in Cantorjeva hipoteza}

Na vajah boste spoznali, da ima množica realnih števil $\RR$ enako moč kot potenčna množica $\pow(\NN)$. Moči $\RR$ in $\pow{\NN}$ pravimo \textbf{moč kontinuuma} (ker je ">kontinuum"< tudi staro ime za $\RR$). Že Georg Cantor, utemelitelj teorije množic, je postavil naslednji domnevo:
%
\begin{quote}
  \emph{\textbf{Cantorjeva hipoteza} Vsaka neštevna podmnožica realnih števil je izomorfna $\RR$.}
\end{quote}
%
Povedano, z drugimi besedami, po moči ni nobene množice strogo med $\NN$ in $\RR$. Cantorjeva hipoteza je ostala odprta dlje časa. Dokončno je Cohen pred dobrega pol stoletja dokazal naslednje:

\begin{izrek}[Cohen]
  Iz Zermelo-Fraenkelovih aksiomov teorije množic Cantorjeve hipoteze ne moremo niti dokazati niti ovreči.
\end{izrek}

Pravimo, da je Cantorjeva hipoteza \emph{neodvisna} od aksiomov teorije množic. Poznamo še posplošeno Cantorjevo hipotezo, ki se glasi:
%
\begin{quote}
  \textbf{Posplošena Cantorjeva hipoteza:}
  %
  Če je $|A| \leq |B| \leq |\pow{A}|$, potem je $|B| = |A|$ ali $|B| = |\pow{A}|$.
\end{quote}
%
Tudi posplošena Cantorjeva hipoteza je nedovisna od aksiomov teorije množic.
Danes vemo zelo veliko o tej hipotezi in poznamo še mnoge druge izjave o množicah, ki so neodvisne od Zermelo-Fraenkelovih aksiomov teorije množic.
Ti veljajo za nekakšno uradno različičo teorije množic in jih bomo obravnavali na naslednjih predavanjih.



\chapter{Zermelo-Fraenkelovi aksiomi}
\textbf{To poglavje še ni predelano v {\LaTeX}.}
%\chapter{Aksiomatska teorija množic}

\section{Kodiranje matematičnih objektov z množicami}

Z množicami smo izrazili številne matematične objekte, na primer:
%
\begin{itemize}
\item ordinalna števila smo predstavili kot množice svojih predhodnikov,
\item preslikavo $f : A → B$ lahko izrazimo kot funkcijsko relacijo med $A$ in $B$, torej kot
  podmnožico $A \times B$,
\item kvocientna množica $A/R$ je množica ekvivalenčnih razredov, ekvivalenčni razredi so spet
  množice,
\end{itemize}
%
Ali je možno vse matematične objekte predstaviti z množicami? Da!

\subsection{Urejeni pari}

Par $(x, y)$ lahko predstavimo z množico $\set{\set{x}, \set{x,y}}$. Tako dobimo
%
\begin{equation*}
  A \times B \defeq \set{ \set{\set{x}, \set{x,y}} \mid x \in A \land y \in B }.
\end{equation*}


\subsection{Vsota}

Elemente vsote $A + B$ kodiramo takole:
%
\begin{align*}
  \inl(x) &\defeq (x, 0) = \set{\set{x}, \set{x, \emptyset}}, \\
  \inr(x) &\defeq (x, 1) = \set{\set{x}, \set{x, \set{\emptyset}}}.
\end{align*}


\subsection{Naravna števila}

Kot smo že videli, lahko ordinalna števila kodiramo kot množice svojih predhodnikov, poseben primer pa so naravna števila, ki so končni ordinali.

Kako pa kodiramo operacijo naslednik? Definirajmo preslikavo \textbf{naslednik} ${}^{+} : \Set \to Set$,
%
\begin{equation*}
  \suc{x} \defeq x \cup \set{x}.
\end{equation*}
%
Če si predstavljamo, da je $x$ število, tedaj je $\suc{x}$ vsebuje predhodnike~$x$ in $x$, kar je ravno naslednik~$x$.
Naravna števila res dobimo tako, da na $\emptyset$ uporabljamo naslednik $suc{{}}$:
%
\begin{align*}
    0 &= \emptyset \\
    1 &= \suc{0} = \set{ 0 } = \set{\emptyset} \\
    2 &= \suc{1} = \set{0, 1} = \set{\emptyset, \set{\emptyset}} \\
    3 &= \suc{2} = \set{0, 1, 2} = \set{\emptyset, \set{\emptyset}, \set{\emptyset, \set{\emptyset}}} \\
    4 &= \suc{3} = \set{0, 1, 2, 3} =
       \set{\emptyset, \set{\emptyset}, \set{\emptyset, \set{\emptyset}},
            \set{\emptyset, \set{\emptyset}, \set{\emptyset, \set{\emptyset}}}} \\
      &\vdots
\end{align*}

\subsection{Cela števila}

Cela števila so kvocient $\NN \times \NN$:
%
\begin{equation*}
    \ZZ \defeq (\NN \times \NN)/{\sim},
\end{equation*}
%
kjer je
%
\begin{equation*}
  (a,b) \sim (c,d) \defiff a + d = c + b.
\end{equation*}
%
Urejeni par $(a, b)$ predstavlja razliko števil $a$ in $b$.


\subsection{Racionalna števila}

Racionalna števila so kvocient:
%
\begin{equation*}
  \QQ = (\ZZ \times \set{n \in \NN \such n > 0})/{\approx},
\end{equation*}
%
kjer je
%
\begin{equation*}
    (a,m) \approx (b,n) \defiff a n = b m.
\end{equation*}
%
Urejeni par $(a, n)$ predstavlja kvocient števil $a$ in $n$.

\subsection{Realna števila}

Realno število je Dedekindov rez, torej podmnožica $\QQ$. Reze ste obravnavali pri Analizi, tako da jih na tem mestu ne
bomo obnavljali.

In tako naprej. Ne pravimo, da je kodiranje vseh matematičnih objektov z množicami vedno
dobra ideja, vendar pa je dejstvo, da je to možno, pomembno spoznanje osnov matematike. Iz
njega na primer sledi tole: če je teorija množic neprotislovna, potem je neprotislovna
tudi vsa matematika, ki jo lahko kodiramo z množicami (torej več ali manj vsa običajna
matematika).



\section{Zermelo-Fraenkelovi aksiomi}

Aksiomi opredeljujejo množice brez urelementov (">\emph{Vse} je množica"<). Za aksiomatizacijo razredov bi morali zapisati drugačne aksiome, kot so na primer von Neumann-Bernays-Gödelovi aksiomi.

\begin{description}

\item[Ekstenzionalnost:] množici $A$ in $B$, ki imata iste elemente, sta enaki.

\item[Neurejeni par]: za vsak $x$ in $y$ je $\set{x, y}$ množica, ki vsebuje natanko $x$ in $y$:
  %
  \begin{equation*}
    \all{x y z} z \in \set{x, y} \liff z = x \lor z = y
  \end{equation*}
  %
  Okrajšava: $\set{x} = \set{x, x}$.

\item[Unija:] za vsako množico $A$ je $\bigcup A$ množica, ki vsebuje natanko vse
  elemente množic iz $A$:
  %
  \begin{equation*}
    \all{A x} x \in \bigcup A \liff \some{B \in A} x \in B.
  \end{equation*}

\item[Prazna množica:] množica $\emptyset$ nima elementa:
  %
  \begin{equation*}
  \all{x} x \not\in \emptyset.
  \end{equation*}

\item[Neskončna množica] obstaja množica, ki vsebuje $\emptyset$ in je zaprta za operacijo naslednik
  ($\suc{x} = x \cup \set{x}$):
  %
  \begin{equation*}
    \some{A} \emptyset \in A \land \all{x \in A} \suc{x} \in A.
  \end{equation*}

\item[Podmnožica:] za vsako množico $A$ in formulo $\phi$ je $\set{x \in A \mid| \phi(x)}$
  množica, ki vsebuje natanko vse element iz $A$, ki zadoščajo $\phi$:
  %
  \begin{equation*}
    \all{y} y \in \{x \in A | \phi(x)\} \liff \phi(y).
  \end{equation*}

\item[Potenčna množica:] za vsako množico $A$ je $\pow{A}$ množica, ki vsebuje
  natanko vse njene podmnožice:
  %
  \begin{equation*}
    \all{S} S \in \pow{A} \liff S \subseteq A.
  \end{equation*}

\item[Zamenjava] če je $A$ množica in $f : A \to \Set$ preslikava, je
  %
  $
    \img{f}(A) = \set{ y \mid \some{x \in A} y = f(x) }
  $
  %
  množica.

\item[Dobra osnovanost:] relacija ${\in} \subseteq \Set \times \Set$ je dobro osnovana.

\item[Aksiom izbire:] vsaka družina nepraznih množic ima funkcijo izbire.
\end{description}


\section{Kumulativna hierarhija}

Če lahko vse matematične objekte kodiramo z množicami, potem lahko na razred
vseh množic $\Set$ gledamo kot na celotni matematični svet. Razred $\Set$ ima
zanimivo strukturo, ki ji pravimo \textbf{kumulativna hierarhija}. Namreč, s pomočjo
Zermelo-Fraenkelovih aksiomov lahko tvorimo vse množice iz $\emptyset$ z
operacijama potenčna množica in unija. Postopek je \textbf{transfiniten} (neskončen), ima pa toliko korakov, kot je ordinalnih števil:
%
\begin{align*}
  V_0 &= \emptyset \\
  V_1 &= \pow{V_0} = \set{\emptyset} \\
  V_2 &= \pow{V_1} = \set{\emptyset, \set{\emptyset}} \\
  V_3 &= \pow{V_2} = \set{\emptyset, \set{\emptyset}, \set{\set{\emptyset}}, \set{\emptyset, \set{\emptyset}}} \\
      &\vdots \\
  V_\omega &= \textstyle\bigcup_{k < \omega} V_k \\
  V_{\omega+1} &= \pow{V_\omega} \\
  V_{\omega+2} = &\pow{V_{\omega+1}} \\
  &\vdots \\
  V_{\omega + \omega} &= \textstyle\bigcup_{\alpha < \omega + \omega} V_\alpha \\
  &\vdots
\end{align*}
%
Splošna formula se glasi $V_\alpha = \textstyle\bigcup_{\beta < \alpha} \pow{V_\beta}$.


\begin{naloga}
  Koliko elementov ima $V_5$?
\end{naloga}

Bistvo kumulativne hierarhije je, da zaobjame vse množice.

\begin{izrek}[Kumulativna hierarhija]
  $\Set = \bigcup_{\alpha \in \On} V_\alpha$.
\end{izrek}

\begin{dokaz}
  Dokaz opustimo, povejmo le, da je za izrek bistven aksiom o dobro osnovanosti. Le ta nam zagotavlja, da se vsaka padajoča $\in$-veriga konča z~$\emptyset$.
\end{dokaz}


% \subsection{Zakon trihotomije}

% V tem razadelku podamo še oris dokaza, da je aksiom izbire ekvivalenten zakonu trihotomije.

% \textbf{Definicija:} Naj bo $(P, <)$ dobra urejenost. Podmnožica $I ⊆ P$ je **začetni
% segment**, če je doljna množica: iz $x < y$ in $y ∈ I$ sledi $x ∈ I$.

% \textbf{Definicija:} Naj bosta $(P, <_P)$ in $(Q, <_Q)$ dobri urejenosti. Pravimo, da
% je preslikava $e : P \to Q$ **vložitev**, kadar velja:

% 1. $e$ je strogo monotona in
% 2. slika $e_{P}$ je začetni segment v $Q$.

% Vložitev je injektivna preslikava.

% **Lemma 1:** Naj bosta $(P, <_P)$ in $(Q, <_Q)$ dobri urejenosti. Če obstaja
% injektivna preslikava $P \to Q$, potem obstaja tudi vložitev $P \to Q$.

% Dokaz: opuščen.

% **Lemma 2:** Naj bosta $(P, <_P)$ in $(Q, <_Q)$ dobri urejenosti. Tedaj bodisi
% obstaja vložitev $P \to Q$ ali vložitev $Q \to P$.

% Dokaz: opuščen.

% \textbf{Izrek:} Aksiom izbire je ekvivalenten zakonu trihotomije: za vse množice $X$ in $Y$ velja
% $|X| ≤ |Y|$ ali $|Y| ≤ |X|$.

% Dokaz:

% Najprej predpostavimo, da velja aksiom izbire. Naj bosta $X$ in $Y$ množici. Ker
% velja aksiom izbire, lahko $X$ in $Y$ dobro uredimo, denimo z relacijama $<_X$
% in $<_Y$. Iz zgornje leme sledi, da obstaja vložitev $X \to Y$ ali $Y \to X$.
% Ker so vložitve injektivne, torej velja $|X| ≤ |Y|$ ali $|Y| ≤ |X|$.

% Predpostavimo zdaj, da za vse množice $X$ in $Y$ velja $|X| ≤ |Y|$ ali $|Y| ≤
% |X|$. Dokazali bomo, da lahko vsako množico dobro uredimo, iz česar sledi aksiom izbire.

\section{Aksiom izbire}

Za konec povejmo še nekaj več o aksiomu izbire in Zornovi lemi, ki mu je ekvivalentna. Le-ta se uporablja v algebri.

\begin{definicija}
  \textbf{Veriga} v delni urejenosti $(P, {\leq})$ je taka podmnožica $V \subseteq
  P$, ki je z $\leq$ linearno urejena, kar pomeni $\all{x y \in V} x \leq y \leq y \leq x$.
\end{definicija}

\begin{primer}
  Primeri verig:
  %
  \begin{itemize}
  \item Če je $(P, {\leq})$ linearno urejena, je vsaka njena podmnožica veriga.
  \item V $(\pow{\QQ}, {\subseteq})$ imamo neštevno verigo
    %
    $V = \set{S \in P(Q) \mid \text{$S$ je doljna množica}}$.
    %
    Množica $S \subseteq \QQ$ je \textbf{doljna}, če velja
    $\all{x y \in \QQ} x \leq y \land y \in \QQ \lthen x \in \QQ$.
  \end{itemize}
\end{primer}

\begin{lema}[Zornova lema]
  Če ima v delni urejenosti $(P, {\leq})$ vsaka veriga zgornjo mejo,
  potem ima $P$ maksimalni element.
\end{lema}

\begin{dokaz}
  Dokaz se naslanja na aksiom izbire in Bourbaki-Wittov izrek o negibnih točkah (glej
  spodaj). Naj bo $C$ množica vseh verig v $P$. Uredimo jo z $\subseteq$. Na njej definiramo preslikavo
  $f : C \to C$, ki razširi verigo, če ni maksimalna, sicer je ne spremeni (tu uporabimo
  izbiro):
  %
  \begin{itemize}
  \item Če je $V \in C$ maksimalna veriga v $P$ (glede na $\subseteq$), definiramo $f(V) \defeq V$.
  \item Če $V \in C$ ni maksimalna veriga v $P$, potem obstaja tak $x \in P \setminus V$, da je $V
    \cup \set{x}$ spet veriga. V tem primeru \emph{izberemo} tak $x$ in definiramo $f(V) \defeq V
    \cup \set{x}$.
  \end{itemize}
  %
  Po izreku Bourbaki-Witt ima $f$ negibno vrednost $V \in C$. Ta $V$ je maksimalna
  veriga $V$, saj bi sicer veljalo, da je $V = f(V) = V \cup \set{x}$ za neki $x \not\in V$,
  kar ni možno. Naj bo $m$ zgornja meja za verigo $V$. Trdimo, da je $m$
  maksimalni element v $P$: denimo, da velja $m \leq y$ za $m \in P$. Ker je $V \cup \set{y}$
  veriga, ki vsebuje maksimalno verigo $V$, sledi $V = V \cup \set{y}$, od tod pa $y \in V$
  ter $y \leq m$. Torej je $m = y$ in $m$ je res maksimalni element.
\end{dokaz}

\begin{definicija}
  Naj bo $(P, \leq)$ delna ureditev. Preslikava $f : P \to P$ je \textbf{progresivna}, ko
  velja $x \leq f(x)$ za vsak $x \in P$.
\end{definicija}

\begin{opomba}
  Progresivna preslikav ni nujno monotona. (Poiščite proti-primer!)
\end{opomba}

\begin{izrek}[Bourbaki-Witt]
  Naj bo $(P, {\leq})$ neprazna delna ureditev, v kateri ima vsaka veriga zgornjo mejo in $f : P \to P$ progresivna
  preslikava. Tedaj ima $f$ negibno točko: to je tak $x \in P$, da velja $f(x) = x$.
\end{izrek}

\begin{dokaz}
  Dokaz opustimo.
\end{dokaz}

\begin{izrek}
  V teoriji množic \emph{brez} aksioma izbire so naslednje izjave ekvivalentne:
  %
  \begin{enumerate}
  \item Aksiom izbire
  \item Zornova lema
  \item Princip dobre urejenosti: vsaka množica ima dobro ureditev.
  \end{enumerate}
\end{izrek}

\begin{dokaz}
  ($1 \lthen 2$) Glej Zornovo lemo.

  ($2 \lthen 3$) Skica dokaza: naj bo $A$ poljubna množica, ki jo želimo dobro urediti.
  %
  Definirajmo \emph{delne} dobre ureditev množice $A$: to so pari $(B,R)$, kjer je $B \subseteq A$
  in $R \subseteq B \times B$ dobra ureditev na $B$. Za delni dobri ureditvi $(B,R)$ in
  $(C,Q)$ pravimo, da je $(C,Q)$ \emph{razširitev} $(B,R)$, kadar velja $B \subseteq C$, $R \subseteq Q$ in
  še, da je $B$ začetni segment v $C$, kar pomeni
  %
  \begin{equation*}
    \all{x y \in C} x \rel{Q} y \land y \in B \lthen x \in B.
  \end{equation*}
  %
  Kadar je $(C,Q)$ razširitev $(B,R)$, pišemo $(B,R) \preceq (C,Q)$. Naj bo $P$ množica vseh delnih
  dobrih ureditev množice $A$,
  %
  \begin{equation*}
    P \defeq \set{ (B, R) \mid \text{$B \subseteq A$ in $R \subseteq B \times B$ in $R$ je dobra ureditev $B$}},
  \end{equation*}
  %
  urejena z relacijo $\preceq$. Očitno je $\preceq$ delna ureditev. Trdimo, da imajo verige v
  $P$ zgornje meje glede na $\preceq$: če je $V \subseteq P$ veriga dobro urejenih podmnožic
  $A$, je njena zgornja meja $(D,S)$ kar unija po komponentah:
  %
  \begin{align*}
    D &\defeq \bigcup \set{B \mid (B, R) \in V} \\
    S &= \bigcup \set{R \mid (B, R) \in V}.
  \end{align*}
  %
  Preverimo, da velja $(D,S) \in P$. Očitno je $(D,S)$ stroga linearna ureditev
  (vaja). Denimo, da bi v $(D,S)$ imeli neskončno padajočo verigo
  %
  \begin{equation*}
    \cdots \rel{S} x_3 \rel{S} x_2 \rel{S} x_1 \rel{S} x_0.
  \end{equation*}
  %
  Obstaja $(B,R) \in V$, da je $x_0 \in B$. Potem bi bila $x_0, x_1, x_2, x_3, \ldots$
  padajoča veriga v $(B,R)$, kar ni možno, saj je $(B,R)$ dobro urejena. Res, ker
  je $x_i \in V$, obstaja $(C,Q)$, da je $x_i \in C$. Če velja $(B,R) \preceq (C,Q)$, potem
  $x_i \in B$ po definicijo $\preceq$. Če velja $(C,Q) \preceq (B,R)$, potem $x_i \in B$, ker velja
  $C \subseteq B$. Torej je $(D,S)$ res delna ureditev $P$.

  Preverimo še, da velja $(B,R) \preceq (D,S)$ za vsak $(B,R) \in V$. Denimo, da je $y \in D$,
  $x \in B$ in $y \rel{S} x$. Obstaja $(C,Q) \in V$, da je $y \in C$. Če velja $(C,Q) \preceq (B,R)$,
  potem $y \in C \subseteq B$. Če pa velja $(B,R) \preceq (C,Q)$, potem je $y \in B$ po definiciji $\preceq$.

  Po Zornovi lemi obstaja maksimalni element $(B,R)$ v $P$. Trdimo, da je $B = A$. Če bi namreč
  obstajal $x \in B \setminus A$, bi lahko razširili $(B,R)$ na večjo dobro ureditev tako, da bi dodali $x$
  na konec $B$:
  %
  \begin{align*}
    & (B \cup \set{x}, R') \\
    & y \rel{R'} z \defiff z = x \land y \rel{R} z.
  \end{align*}
  %
  To ni možno, ker je $(B,R)$ maksimalna delna ureditev. Torej je res $A = B$ in
  našli so dobro ureditev $A$.

  $(3 \lthen 1)$ Naj bo $A : I \to Set$ družina nepraznih množic. Naj bo $\prec$ dobra ureditev
  na uniji $\bigcup A$. Ker so vse množice $A_i$ neprazne, ima vsaka od njih prvi element
  glede na $\prec$. Torej lahko definiramo funkcijo izbire $f$ s predpisom
  $f(i) \defeq \text{">prvi element $A_i$"<}$.
\end{dokaz}

\begin{izrek}
  Vsak vektorski prostor ima bazo.
\end{izrek}

\begin{dokaz}
  Dokaz: Naj bo $L$ vektorski prostor. Definiramo množico
  %
  \begin{equation*}
    P \defeq \set{ B \subseteq L \mid \text{$B$ je linearno neodvisna} }.
  \end{equation*}
  %
  Množico $P$ delno uredimo z relacijo $\subseteq$. Trdimo, da imajo verige v $P$ zgornje
  meje: zgornja meja verige $V \subseteq P$, je kar njena unija $\bigcup_{B \in V} B$. Seveda je
  treba preveriti, da je unija verige linearno neodvisnih množic spet linearno
  neodvisna (vaja). Po Zornovi lemi obstaja maksimalni element v $P$, torej
  maksimalna linearno neodvisna množica $B$ v $L$. To pa je seveda vektorska baza
  za $L$.
\end{dokaz}

\chapter{Kumulativna hierarhija}
\textbf{To poglavje še ni predelano v {\LaTeX}.}
%\section{Kumulativna hierahija}

Če lahko vse matematične objekte kodiramo z množicami, potem lahko na razred
vseh množic `Set` gledamo kot na celotni matematični svet. Razred `Set` ima
zanimivo strukturo, ki ji pravimo **kumulativna hierarhija**. Namreč, s pomočjo
aksiomov, ki jih bomo spoznali kasneje, lahko tvorimo vse množice iz `∅` z
oparacijama potenčna množica in unija. Postopek je **transfiniten**, kar pomeni,
da se nikoli ne konča in da po svoje številu presega moč vsake množice.

    V₀ = ∅
    V₁ = P(V₀) = {∅}
    V₂ = P(V₁} = {∅, {∅}}
    V₃ = P(V₂) = {∅, {∅}, {{∅}}, {∅, {∅}}}
    ...
    V_ω = ⋃ {Vᵢ | i < ω}
    V_(ω+1) = P(V_ω)
    V_(ω+2) = P(V_(ω+1))
    ...
    V_(ω + ω) = ⋃ {Vᵢ | i < ω + ω)}
    ...

Stopnje konstrukcija indeksiramo s t.i. **ordinalnimi števili**, ki jih bomo spoznali.

\section{Ordinalna števila}

**Definicija:** Množica $x$ je **tranzitivna**, če za vsak `y ∈ x` velja `y ⊆ x`.

Izraz *tranzitivna* je smiselen, ker govori o tranzitivnosti relacije `∈`, saj lahko pogoj `y ∈ x ⇒ y ⊆ x` zapišemo kot `z ∈ y ∧ y ∈ x ⇒ z ∈ x`.

Primeri tranzitivnih množic: `∅`, `{∅}`, `{∅, {∅}}`, `{{{∅}}, {∅}, ∅}`

**Definicija:** Množica je **hereditarno tranzitivna**, če so vsi njeni elementi tranzitivne množice.

(V splošnem se izraz "hereditarno" uporablja, kadar se lastnost nanaša na elemente, pomdnožice, ali podstrukture, se pravi na "potomce".)

**Definicija:** **Ordinalno število** je tranzitivna in hereditarno tranzitivna množica. Razred vseh ordinalnih števil označimo z `On`.

Definirajmo relacijo *naslednik* na množicah: `x^+ = x ∪ {x}`.

Preverimo lahko tole:

1. `∅ ∈ On`
2. če je `α ∈ On`, potem je tudi `α^+ ∈ On`.
3. `On` je zaprt za unije: če je `S ⊆ On` množica, potem je `U S ∈ On`.

Sedaj lahko gradimo `On` iterativno:

* `0 = ∅`
* z opreacijo naslednik dobimo naravna števila `n = {0, ..., n-1}`
* unija vseh naravnih števil je `ω`.
* z naslednik gradimo `ω`, `ω + 1`, `ω + 2`, ...
* unija teh je `ω + ω`
* in tako naprej

\section{Kardinalna števila}

**Definicija:** Ordinalno število `α` je **kardinalno**, če za vsak `β < α`
velja, da ne obstaja injektivna preslikava `α → β`.

\subsection{Zakon trihotomije}

V tem razadelku podamo še oris dokaza, da je aksiom izbire ekvivalenten zakonu trihotomije.

**Definicija:** Naj bo `(P, <)` dobra urejenost. Podmnožica `I ⊆ P` je **začetni
segment**, če je doljna množica: iz `x < y` in `y ∈ I` sledi `x ∈ I`.

**Definicija:** Naj bosta `(P, <_P)` in `(Q, <_Q)` dobri urejenosti. Pravimo, da
je preslikava `e : P → Q` **vložitev**, kadar velja:

1. `e` je strogo monotona in
2. slika `e_(P)` je začetni segment v `Q`.

Vložitev je injektivna preslikava.

**Lemma 1:** Naj bosta `(P, <_P)` in `(Q, <_Q)` dobri urejenosti. Če obstaja
injektivna preslikava `P → Q`, potem obstaja tudi vložitev `P → Q`.

Dokaz: opuščen.

**Lemma 2:** Naj bosta `(P, <_P)` in `(Q, <_Q)` dobri urejenosti. Tedaj bodisi
obstaja vložitev `P → Q` ali vložitev `Q → P`.

Dokaz: opuščen.

**Izrek:** Aksiom izbire je ekvivalenten zakonu trihotomije: za vse množice `X` in `Y` velja
`|X| ≤ |Y|` ali `|Y| ≤ |X|`.

Dokaz:

Najprej predpostavimo, da velja aksiom izbire. Naj bosta `X` in `Y` množici. Ker
velja aksiom izbire, lahko `X` in `Y` dobro uredimo, denimo z relacijama `<_X`
in `<_Y`. Iz zgornje leme sledi, da obstaja vložitev `X → Y` ali `Y → X`.
Ker so vložitve injektivne, torej velja `|X| ≤ |Y|` ali `|Y| ≤ |X|`.

Predpostavimo zdaj, da za vse množice `X` in `Y` velja `|X| ≤ |Y|` ali `|Y| ≤
|X|`. Dokazali bomo, da lahko vsako množico dobro uredimo, iz česar sledi aksiom izbire.



% \chapter{Množice}
\label{chap:mnozice}

V drugem delu predmeta bomo spoznali osnove teorije množic. Najprej pa
se bomo posvetili še naravnim številom in Peanovim aksiomom.

%%%%%%%%%%%%%%%%%%%%%%%%%%%%%%%%%%%%%%%%%%%%%%%%%%%%%%%%%%%%%%%%%%%%%%
\section{Naravna števila}
\label{sec:naravna-stevila}

Naravna števila
%
\begin{equation*}
  0, 1, 2, 3, 4, 5, 6, 7, 8, 9, 10, 11, 12, \ldots
\end{equation*}
%
vsi že dobro poznamo iz osnovne šole.\footnote{V teh zapiskih in v
  logiki nasploh vzamemo za prvo naravno število $0$. V osnovni šoli
  in drugje pa ponavadi za prvo naravno število jemljemo~$1$.} V tem
razdelku pokažimo, kako uvedemo naravna števila kot formalno teorijo v
logiki. V splošnem \emph{formalna teorija} opisuje neko matematično
strukturo ali družino struktur in je podana z osnovnimi simboli
(konstantami in operacijami), aksiomi in pravili sklepanja.

Teorijo naravnih števil, ki jo imenujemo tudi \emph{Peanova
  aritmetika}, sestoji iz konstante $0$, enočlene operacije
naslednik~$\suc{n}$ ter dvočlenih operacij seštevanje~$m + n$ in
množenje~$m \cdot n$. Množenje ia prednost pred seštevanjem, se pravi,
da je $k + m \cdot n = k + (m \cdot n)$ in ne $(k + m) \cdot n$.
Aksiomi in pravila sklepanja se glasijo:
%
\begin{enumerate}
  \item Nič ni naslednik:
  %
  \begin{equation*}
    \inferrule{ }{\suc{n} \neq 0}    
  \end{equation*}
  %
  \item Če sta naslednika enaka, sta števili enaki:
  %
  \begin{equation*}
    \inferrule{\suc{m} = \suc{n}}{m = n}
  \end{equation*}
  %
  \item Pravili za seštevanje:
  %
  \begin{mathpar}
    \inferrule{ }{0 + n = n}
    \and
    \inferrule{ }{\suc{m} + n = (m + n\suc{)}}    
  \end{mathpar}
  %
  \item Pravili za množenje:
  %
  \begin{mathpar}
    \inferrule{ }{0 \cdot n = 0}
    \and
    \inferrule{ }{\suc{m} \cdot n = m \cdot n + n}
  \end{mathpar}
  %
  \item Princip indukcije:
  %
  \begin{equation*}        
    \inferrule{\phi(0) \\ \xall{m}{\NN}{\phi(m) \lthen \phi(\suc{m})}}{\phi(n)}
  \end{equation*}
\end{enumerate}
%
Pri običajnem računanju z naravnimi števili uporabljamo vse znanje, ki
smo ga pridobili v šoli. Ko pa obravnavamo naravna števila kot
formalno teorijo, smemo uporabljati \emph{samo} konstante in simbole,
ki jih vpeljemo v teoriji, in se sklicevati \emph{samo} na Peanove
aksiome. Denimo, ker teorija ne vpelje simbolov $1$ in $2$, ju ne
smemo uporabljati, razen če ju prej definiramo kot okrajšavi za
$\suc{0}$ in $\suc{(\suc{0})}$. Prav tako ne smemo omenjati odštevanja
števil, ker to ni ena od operacij $+$ in $\cdot$, ne smemo govoriti o
parnosti števil, ne da bi prej ta pojem definirali, itn. Tudi osnovne
lastnosti seštevanja in množenja, kot sta komutativnost in
asociativnost, ne smemo uporabiti, če ju prej ne dokažemo. Matematiki
so seveda preverili, da vse običajne lastnosti števil dejansko sledijo
iz Peanovih aksiomov.

Glavno orodje pri dokazovanju lastnosti naravnih števil je princip
indukcije. V besedilu ga uporabimo takole:
%
\begin{quote}
  \em
  %
  Dokazujemo $\phi(n)$ z indukcijo po~$n$:
  %
  \begin{enumerate}
    \item Baza indukcije: (Dokaz, da velja $\phi(0)$.)
    \item Indukcijski korak: denimo, da za naravno število $m$ velja
      $\phi(m)$. (Dokaz, da velja $\phi(\suc{m})$.)
  \end{enumerate}
\end{quote}
%
Za zgled dokažimo, da je seštevanje komutativno. To naredimo v nekaj
korakih.

\begin{izjava}
  \label{izjava:peano-n-plus-0}
  Za vsako naravno število $m$ velja $m + 0 = m$.
\end{izjava}

\begin{dokaz}
  Dokazujemo z indukcijo. Baza indukcije: $0 + 0 = 0$ po enem od
  Peanovih aksiomov.
  %
  Indukcijski korak: denimo, da za naravno število $k$ velja $k + 0 =
  k$. Tedaj je $\suc{k} + 0 = \suc{(k + 0)} = \suc{k}$, kjer smo v
  prvem koraku uporabili enega od Peanovih aksiomov in v drugem
  indukcijsko predpostavko.
\end{dokaz}


\begin{izjava}
  \label{izjava:peano-m-plus-suc-n}
  Za vsaki naravni števili $m$ in $n$ velja $m + \suc{n} = \suc{(m + n)}$.
\end{izjava}

\begin{dokaz}
  Izjavo dokažemo z indukcijo po $m$.
  Baza indukcije: $0 + \suc{n} = \suc{n} = \suc{(0 + n)}$.
  %
  Indkucijski korak: denimo, da za naravno število $k$ velja $k +
  \suc{n} = \suc{(k + n)}$. Tedaj je
  %
  \begin{equation*}
    \suc{k} + \suc{n} = 
    \suc{(k + \suc{n})} =
    \suc{{\suc{(k + n)}}} =
    \suc{(\suc{k} + n)}.
  \end{equation*}
  %
\end{dokaz}

\begin{izjava}
  Za vsaki naravni števili $m$ in $n$ velja $m + n = n + m$.
\end{izjava}

\begin{dokaz}
  Izjavo dokažemo z indukcijo po $m$.
  Baza indukcije: $0 + n = n = n + 0$, kjer smo v prvem koraku uporabili Peanov aksiom in v drugem Izjavo~\ref{izjava:peano-n-plus-0}.
  %
  Indukcijski korak: denimo, da za naravno število $k$ velja $k + n = n + k$. Tedaj je
  %
  \begin{equation*}
    \suc{k} + n =
    \suc{(k + n)} =
    \suc{(n + k)} =
    n + \suc{k}.
  \end{equation*}
  %
  V prvem koraku smo uporabili Peanov aksiom, v drugem indukcijsko predpostavko, v tretjem pa Izjavo~\ref{izjava:peano-m-plus-suc-n}.
\end{dokaz}

%%%%%%%%%%%%%%%%%%%%%%%%%%%%%%%%%%%%%%%%%%%%%%%%%%%%%%%%%%%%%%%%%%%%%%
\section{Množice}
\label{sec:naivne-mnozice}

Množice so osnovni gradniki matematičnih objektov in struktur. V tem razdelku obravnavamo množice \emph{naivno}, se pravi s pomočjo neformalnih razlag. Samo formalno teorijo množic in aksiome bomo obravnavali v razdelku~\ref{sec:zfc}.

Množico si predstavljamo kot skupek ali zbirko poljubnih objektov, jim pravimo \emph{elementi} množice. Dejstvo, da je $x$ element množice $A$ zapišemo $x \in A$. Če $x$ ni element $S$, pišemo $x \not\in S$ kot okrajšavo za $\lnot (x \in S)$. Množica ni odvisna od tega, kako jo opišemo ali skonstruiramo, ampak le od tega, kateri elementi so v njej. To dejstvo izraža \emph{aksiomom o ekstenzionalnosti}, ki pravi, da sta množici $A$ in $B$ enaki natanko tedaj, ko vsebujeta iste elemente, kar zapišemo s formulo kot
%
\begin{equation*}
  A = B \liff \uall{x}{x \in A \liff x \in B}.
\end{equation*}
%
Množice gradimo iz osnovnih množic s pomočjo operacij.

\subsection{Osnovne množice}
\label{sec:osnovne-mnozice}

Najpreprostejša osnovna množica je \emph{prazna množica}, ki jo označimo z $\emptyset$. Dejstvo, da prazna množica ne vsebuje nobenih elementov izrazimo z aksiomom o prazni množici,
%
\begin{equation*}
  \xuall{x}{x \not\in \emptyset}.
\end{equation*}
%
V zvezi s prazno množico omenimo, da za vsako izjavo $\phi$ velja
%
\begin{equation*}
  \xall{x}{\emptyset}{\phi},
\end{equation*}
%
kar dokažemo takole: naj bo $x \in \emptyset$ poljuben. Ker velja $x \not\in \emptyset$, je to protislovje, od koder smemo sklepati $\phi$. Podobno za vsako izjavo $\phi$ velja
%
\begin{equation*}
  \lnot\xsome{x}{\emptyset}{\phi}.
\end{equation*}

\begin{naloga}
  Za katere množice $S$ velja $(\xall{x}{S}{\phi(x)}) \lthen   \xsome{x}{S}{\phi(x)}$?
\end{naloga}

Za osnovno množico vzamemo tudi množico naravnih števil~$\NN$, ki smo jo že spoznali v razdelku~\ref{sec:naravna-stevila}.

\subsection{Konstrukcije množic}
\label{sec:konstrukcije-mnozic}

Iz osnovnih množic lahko konstruiramo nove s pomočjo naslednjih operacij.

\subsubsection{Končne množice}
\label{sec:koncne-mnozice}

Naj bodo $a_1, \ldots, a_n$ poljubni objekti. Tedaj lahko tvorimo množico
%
\begin{equation*}
  \set{a_1, a_2, \ldots, a_n}
\end{equation*}
%
ki sestoji iz naštetih elementov, to je
%
\begin{equation*}
  \uall{x}{x \in \set{a_1, \ldots, a_n} \liff
    x = a_1 \lor \cdots x = a_n}.
\end{equation*}
%
Poseben primer take množice je \emph{enojec} $\set{a}$, za katerega velja
%
\begin{equation*}
  \uall{x}{x \in \set{a} \liff x = a}.
\end{equation*}


\begin{naloga}
  Ali je $\set{a, b} = \set{b, a}$? Ali je $\set{a, a, b} = \set{a, b}$? Uporabi aksiom o ekstenzionalnosti.
\end{naloga}

\subsubsection{Unija in presek}
\label{sec:unija-presek}

Družina množic.

Unije, preseki.

\subsubsection{Podmnožica}
\label{sec:podmnozica}

Podmnožica (separacija).

\subsubsection{Potenčna množica}
\label{sec:potencna-mnozica}

Potenčna množica.

\subsubsection{Kartezični produkt}
\label{sec:kartezicni-produkt}

Kartezični produkt. Produkt s prazno.

\subsubsection{Eksponentna množica}
\label{sec:eksponentna-mnozica}

Eksponentna množica. Eksponent s prazno.

\subsubsection{Vsota}
\label{sec:vsota-mnozic}

Disjunktna unija.

\subsubsection{Razlika in komplement}
\label{sec:vsota-mnozic}


\section{Funkcije}
\label{sec:funkcije}


Funkcija, neformalna definicija.

Kompozitum, asociativnost kompozituma. Identiteta.

Inverz funkcije. Inverz je enoličen, če obstaja.

Slika in inverzna slika.

Kdaj obstaja inverz? Surjektivna, injektivna, bijektivna funckija.

Epi in mono.

Sekcija in retrakcija.

Sekcija je mono, retrakcija je epi.

Standardne bijekcije za vsoto, produkt in eksponent.

\section{Relacije}
\label{sec:relacije}

Definicija relacije.

Nasprotna relacija. Komplement, unija.

\subsection{Funkcijske relacije}
\label{sub:funkcijske_relacije}


\subsection{Ekvivalenčne relacije}
\label{sub:ekvivalencne_relacije}


%Definirali smo pojem ekvivalenčne relacije in kvocienta množice po
%ekvivalenčni relaciji. Pokazali smo razne primere. Dokazali smo, da
%smemo definirati $f : A/{\sim} \to B$ na kvocientu tako, da definiramo
%$g : A \to B$, ki je skladen s~$\sim$.




Defincije. Primeri.

Ekvivalenčna relacija, generirana z relacijo.

Faktorska množica. Kako definiramo preslikavo na faktorski množici.

Kanonični razcep funkcije.

\subsection{Delna ureditev}
\label{sub:delna_ureditev}

Definicija delne ureditve. Primeri.

Linearna ureditev. Stroga linearna ureditev. Veriga.

Zgornja meja, spodna meja, infimum, supremum, maksimum, minimum, minimalni element, maksimalni element.



%%% Local Variables: 
%%% mode: latex
%%% TeX-master: "lmn"
%%% End: 



%--------------------------------------------------------------------
% BIBLIOGRAFIJA

\bibliographystyle{alpha}
\addcontentsline{toc}{chapter}{\numberline{}Literatura}
\markboth{}{Literatura}

{
\raggedright
\renewcommand{\markboth}[2]{}
\bibliography{literatura}
}


\end{document}

%%% Local Variables: 
%%% mode: latex
%%% TeX-master: t
%%% End: 
