\chapter{Relacije}

\section{Predikati}

**Predikat** na množici `A` opredeljuje kako lastnost elementov množice `A`. Če
je `P` predikat na `A` in `x ∈ A`, potem se je smiselno vprašati, ali `x`
zadošča predikatu `P`. Odgovor je resničnostna vrednost, ki jo označimo s `P(x)`.

**Primer:** Na množici naravnih števil `N` lahko obravnavamo predikat "je sodo
število". Tako na primer `4` zadošča predikatu "je sodo število", `7` pa mu zadošča.

Predikat `P` na množici `A` lahko predstavimo na dva načina:

* kot preslikavo `P : A → 2`, ki slika `x ∈ A` v resničnostno vrednost `P(x)`,
* kot podmnožico `P ⊆ A` tistih `x ∈ A`, za katere velja `P(x)`.

Oba načina predstavitve sta uporabna, spoznali pa smo že izomorfizem med njima,
saj velja `P(A) ≅ 2ᴬ`.

\section{Relacije}

Relacije s večmestni predikati. Se pravi, relacija `R` opredeljujejo kako
lastnost urejenih večteric kartezičnega produkta `A₁ × A₂ × ⋯ × A_n`. Pravimo, da
je `R` **`n`-člena** ali **`n`-mestna relacija** na množicah `A_1, …, A_n`.

**Primer:** Na množici točk v ravnini lahko obravnavamo relacijo kolinearnosti.
To je trimestna relacija: točke `A`, `B` in `C` so kolinearne, kadar obstaja
premica, ki vsebuje vse tri točke.

Relacijo `R` na množicah `A_1, …, A_n` lahko predstavimo na dva načina, podobno
kot predikate:

* kot preslikavo `R : A₁ × A₂ × ⋯ × A_n → 2`
* kot podmnožico `R ⊆ A₁ × A₂ × ⋯ × A_n`

Bolj običajna je predstavitev s podmnožicami, zato bomo dejstvo, da je `R`
relacija na množicah `A_1, …, A_n` zapisali kar kot `R ⊆ A₁ × A₂ × ⋯ × A_n`. Za
elemente `x₁ ∈ A₁, …, x_n ∈ A_n` dejstvo, da so v relaciji `R` zapišemo
`R(x₁, …, x_n)`, včasih pa tudi `(x₁, …, x_n) ∈ R`.

Na množicah `A_1, …, A_n` lahko vedno definiramo:

* **prazno relacijo `∅`**: nobeni elementi niso v prazni relaciji
* **univerzalno relacijo `A₁ × A₂ × ⋯ × A_n`**: vsi elementi so v univerzalni relaciji

Univerzalna relacija se imenuje tudi **polna relacija**.

V praksi so najbolj pogoste **dvomestna relacije**, se pravi relacije na dveh
množicah, `R ⊆ A × B`. V tem primeru pravimo množici `A` **domena** in `B`
**kodomena** relacije `R`.

Pomembna relacija na množici `A` je **enakost** ali **diagonala** na `A`:

    Δ_A := { (x, y) ∈ A × A | x = y }

Zakaj ji pravimo diagonala?

Izmed dvočelnih relacij so najbolj pogoste relacije, pri katerih se domena in
kodomena ujemata, torej `R ⊆ A × A`. V tem primeru pravimo, da je `R` **relacija
na množici `A`**.

Denimo, da je `R ⊆ A × B` relacija, `x ∈ A` in `y ∈ B`. Dejstvo, da sta `x` in
`y` v relaciji `R` zapišemo na enega od načinov

1. `(x, y) ∈ R`
2. `R(x, y)`
3. `x R y`

Prvi zapis se uporablja, kadar je `R` podana kot podmnožica `A × B`, drugi kadar
podamo `R` z logično formulo. Tretji način je tudi pogost, še posebej kadar je
relacija označena s simbolom kot je `=`, `≠`, `<`, `>`, `⊑`, `∼` ipd.

Relacijo lahko opišemo na več načinov:

* s formulo
* z resničnostno tabelo
* z usmerjenim grafom

Predstavitev relacije `R ⊆ A × A` z usmerjenim grafom: za vozlišča grafa vzamemo
elemente množice `A`, nato pa narišemo puščico od `x` do `y`, kadar velja `x R y`.

\section{Osnovne lastnosti relacij}

Relacije, ki so pomembne v matematični praksi imajo pogosto lastnosti, ki jih poimenujemo.

Za relacijo `R ⊆ A × A` pravimo da je:

* **refleksivna:** `∀ x ∈ A . x R x`
* **simetrična:** `∀ x, y ∈ A . x R y ⇒ y R x`
* **antisimetrična:** `∀ x, y ∈ A . x R y ∧ y R x ⇒ x = y`
* **tranzitivna:** `∀ x, y, z ∈ A . x R y ∧ y R z ⇒ x R z`
* **irefleksivna:** `∀ x ∈ A . ¬ (x R x)`
* **asimetrična:** `∀ x y ∈ A . x R y ⇒ ¬ (y R x)`
* **sovisna:** `∀ x y ∈ A . x ≠ y ⇒ x R y ∨ y R x`
* **strogo sovisna:** `∀ x y ∈ A . x R y ∨ y R x`

Vprašanje: kako iz usmerjenga grafa relacije razberemo refleksivnost in
simetričnost? Kaj pa ostale lastnosti?

\section{Operacije na relacijah}

\subsection{Unija, presek in komplement relacij}

Ker so relacije pravzaprav podmnožice, lahko na njih uporabljamo operacije unija `∩`,
presek `∪` in komplement `□ᶜ`. Denimo, da sta `R, S ⊆ A × B` relaciji. Tedaj velja:

* `x (R ∪ S) y ⇔ x R y ∨ x S y`
* `x (R ∩ S) y ⇔ x R y ∧ x S y`
* `x (Rᶜ) y ⇔ ¬ (x R y)`

**Primeri:**

* komplement relacije enakosti `=` je relacija neenakosti `≠`
* unija relacij `<` in `>` na realnih številih je relacija `≠`
* presek relacij `≤` in `≥` na realnih številih je relacija `=`

\subsection{Transponirana relacija}

Dvojiške relacije lahko tudi **transponiramo**. Transponiranka relacije `R ⊆ A × B`
je relacija `Rᵀ ⊆ B × A`, definirana s predpisom

    y Rᵀ x ⇔ x R y

ali ekvivalentno

    Rᵀ := { (y, x) ∈ B × A | x R y }

Očitno velja `(Rᵀ)ᵀ = R`, torej je transponiranje *involucija*.

**Primera:**

* transpozicija relacije `<` na realnih številih `ℝ` je relacija `>` na `ℝ`,
* komplement relacije `<` na `ℝ` je relacija `≥` na `ℝ`.


\subsection{Kompozitum relacij}

Nadalje definiramo **kompozitum** relacij `R ⊆ A × B` in `S ⊆ B × C` kot relacijo
`S ∘ R ⊆ A × C`, s predpisom

    x (S ∘ R) z ⇔ ∃ y ∈ B . x R y ∧ y S z

ali ekvivalentno

    S ∘ R := { (x, z) ∈ A × C | ∃ ∈ B . (x,y) ∈ R ∧ (y,z) ∈ S }

Se pravi, da sta `x ∈ A` in `z ∈ C` v relaciji `S ∘ R`, če sta preko `S` in `R`
povezana s kakim elementom `y ∈ B`.

**Primer:**

* kompozitum relacij "`x` je otrok od `y`" in "`z` je mati od `y`" je relacija
  "`z` je babica od `x`".

**Izrek:**

1. Kompozitum relacij je asociativen: `T ∘ (S ∘ R) = (T ∘ S) ∘ R`
2. Diagonala je enota za kompozicijo relacij: `Δ_B ∘ R = R = R ∘ Δ_A`

**Naloga:** Zgornji izrek zapiši bolj natančno, da bo razvidno, kaj so domene in kodomene relacij.

*Dokaz.*

Asociativnost kompozicije: Naj bo `R ⊆ A × B`, `S ⊆ B × C` in `T ⊆ C × D` ter `a ∈ A` in `d ∈ D`. Tedaj velja

    a (T ∘ (S ∘ R)) d                            ⇔
    ∃ c ∈ C . a (S ∘ R) c ∧ c T d                ⇔
    ∃ c ∈ C . (∃ b ∈ B . a R b ∧ b R c) ∧ c T d  (1)

in

    a ((T ∘ S) ∘ R) d                            ⇔
    ∃ b ∈ B . a R b ∧ b (T ∘ S) d                ⇔
    ∃ b ∈ B . a R b ∧ (∃ c ∈ C . b S c ∧ c T d)  (2)

Torej je treba dokazati ekvivalenco izjav, ki sledi iz *Frobeniuseva pravila*
(kjer je `p` formula, v kateri `x` ne nastopa kot prosta spremenljivka):

    (∃ x ∈ X . p ∧ q(x)) ⇔ p ∧ ∃ x ∈ X . q(x)

Dokaz ekvivalence prepuščamo za vajo (najlažje je narediti verigo ekvivalenc med obema izjavama).

Diagonala je enota za kompozicijo: naj bo `R ⊆ A × B` ter `x ∈ A` in `y ∈ B`. Tedaj velja

    x (Δ_B ∘ R) y                ⇔
    ∃ z ∈ B . x R y ∧ y Δ_B z    ⇔
    ∃ z ∈ B . x R y ∧ y = z      ⇔
    x R y

V zadnjem koraku smo uporabili ekvivalenco `(∃ u ∈ U . u = v ∧ P(v)) ⇔ P(v)`.
Podobno dokažemo, da je diagonala desna enota. □

Kompozitum relacij ima torej podobne lastnosti kot kompozitum funkcij.

Za `n ∈ N` definiramo **`n`-to potenco** relacije `R ⊆ A × A` kot relacijo `Rⁿ
⊆ A × A` takole:

    x Rⁿ y ⇔ ∃ z_0, …, z_n ∈ A . z_0 = x ∧ z_n = y ∧ ∀ i ∈ {0, …, n-1} . z_i R z_{i+1}

To je precej nečitljiva formula. Bolj razumljiva definicija je potenca kot `n`-kratni kompozitum relacije `R` same s
sabo:

    Rⁿ := R ∘ ⋯ ∘ R

kjer se desni `R` ponovi `n`-krat. Kaj dobimo, ko za `n` vstavimo `0`? Enoto za kompozitum:

    R⁰ = Δ_A


\subsection{Funkcijske relacije}

Funkcijo `f : A → B` smo definirali kot *prirejanje* med elementi `A` in `B`. A
kaj pravzaprav je "prierjanje"? Je to funkcijski predpis, program, kaj drugega?
Sedaj lahko povemo natančno: prirejanje, s katerim je podana funkcija, je
*relacija* med elementi domene in kodomene.

**Definicija:** Naj bo `f : A → B` funkcija. **Graf** funkcije `f` je relacija
`Γ_f ⊆ A × B`, definirana s predpisom

    x Γ_f y ⇔ f(x) = y

ali ekvivalentno

    Γ_f := { (x, y) ∈ A × B | f(x) = y }

Skratka, graf funkcije ni nič drugega kot njeno prirejanje.

Sedaj pa se vprašajmo: kakšnim pogojem mora zadoščati relacija `R ⊆ A × B`, da
je prirejanje za neko funkcijo? Odgovor poznamo: biti mora enolična in celovita.

**Definicija:** Relacija `R ⊆ A × B` je **funkcijska relacija**, če je

1. **celovita:** `∀ x ∈ A . ∃ y ∈ B . x R y`
2. **enolična:** `∀ x ∈ A . ∀ y₁, y₂ ∈ B . x R y₁ ∧ x R y₂ ⇒ y₁ = y₂`

Ekvivalentno oba pogoja skupaj zapišemo: `∀ x ∈ A . ∃! y ∈ B . x R y`.

Graf `Γ_f ⊆ A × B` funkcije `f : A → B` je vedno funkcijska relacija.

Funkcijska relacija `R ⊆ A × B` določa preslikavo `φ_R : A → B` definirano s predpisom

    φ_R : x ↦ (ι y ∈ B . x R y)

Če vzamemo funkcijo `f` in tvorimo njen graf `Γ_f` nato pa iz njega funckijo
`φ_(Γ_f)` dobimo nazaj prvotno funkcijo `f`. Obratno, če je `R` funkcijska
relacija, tedeaj je `Γ_(φ_R)` enaka `R`. Torej imamo izomorfizem

    Bᴬ ≅ { R ⊆ A × B | ∀ x ∈ A . ∃! y ∈ B . x R y }

**Izjava:** Kompozitum funkcij se ujema s kompozitumom relacij:

    Γ_(g ∘ f) = Γ_g ∘ Γ_f

Dokaz prepustimo za vajo, še prej pa morate izjavo zapisati bolj natančno: od
kod in kam slikata preslikavi `f` in `g`, kaj pomeni kompozitum na levi in kaj
na desni?

\section{Ovojnice relacij}

Pogosto imamo opravka z relacijo `R`, ki nima želene lastnosti (na primer ni
tranzitivna) mi pa želimo relacijo, ki to lastnost ima. Ali lahko `R` kako
spremenimo, da bo imela želeno lastnost? Če to lahko naredimo na več načinov,
ali se eden od njih odlikuje?

Naj bo `R ⊆ A × A` relacija. Tedaj pravimo, da je relacija `T ⊆ A × A`
**tranzitivna ovojnica** relacije `R`, če velja:

1. `T` je tranzitivna
2. `R ⊆ T`
3. če je `S ⊆ A × A` tranzitivna in velja `R ⊆ S`, tedaj je `T ⊆ S`.

Povedano drugače: tranzitivna ovojnica relacije `R` je najmanjša tranzitivna
relacija, ki vsebuje `R`. Zaenkrat ne vemo, ali ima vsaka relacija tranzitivno
ovojnico.

Izraz "ovojnica" uporabljamo, ker si lahko mislimo, da smo relacijo "ovili"
tranzitivno relacijo tako, da se ji slednja čim bolj prilega. Namesto "ovojnica"
rečemo tudi **ogrinjača** ali **zaprtje**.

Poleg tranzitivne ovojnice lahko definiramo tudi druge ovojnice:

* **refleksivna ovojnica** relacije `R ⊆ A × A` je najmanjša refleksivna relacija, ki vsebuje `R`
* **simetrična ovojnica** relacije `R ⊆ A × A` je najmanjša simetrična relacija, ki vsebuje `R`
* **refleksivna tranzitivna ovojnica** relacije `R ⊆ A × A` je najmanjša
  refleksivna in tranzitivna relacija, ki vsebuje `R`

Ali take ovojnice sploh obstajajo? Obravnavajmo le tranzitivne ovojnice, saj so
ostali dokazi zelo podobni. Ključno pri dokazu obstoja tranzitivne ovojnice je
naslednje dejstvo.

**Lema:** Naj bo `A` množica in `R : I → P(A × A)` družina relacij na `A`. Če za
vsak `i ∈ I` velja, da je `Rᵢ` tranzitivna relacija, potem je tudi presek `⋂ R`
tranzitivna relacija.

*Dokaz.* Iz definicije preseka družine množic (relacije so le posebne množice) sledi

    x (∩ R) y ⇔ ∀ i ∈ I . x Rᵢ y

Dokažimo, da je `∩ R` tranzitivna. Naj bodo `x, y, z ∈ A` in denimo, da velja
`x (∩ R) y` in `y (∩ R) z`, kar je ekvivalentno

    ∀ i ∈ I . x Rᵢ y            (1)

in

    ∀ j ∈ I . y Rⱼ z            (2)

Dokazati moramo `x (∩ R) z`, kar je ekvivalentno

    ∀ k ∈ I . x R_k z

Naj bo torej `k ∈ I`, dokazujemo `x R_k z`. Uporabimo (1) pri `i = k` in dobimo

    x R_k y

Uporabimo (2) pri `j = k` in dobimo

    y R_k z

Po predpostavki je `R_k` tranzitivna relacija, torej velja `x R_k z`. □

**Izrek:** Vsaka relacija `R ⊆ A × A` ima enolično tranzitivno ovojnico.

*Dokaz.* Najprej premislimo, da ima `R` največ eno tranzitivno ovojnico: če sta
`S` in `T` obe tranzitvni ovojnici `R`, potem iz definicije tranzitivne ovojnice
sledi `S ⊆ T` in `T ⊆ S`, torej velja `S = T`.

Sedaj pokažimo, da `R` ima tranzitivno ovojnico. Naj bo `R ⊆ A × A`. Definirajmo
množico relacij

    D := { S ⊆ A × A | R ⊆ S in S je tranzitivna }

Trdimo, da je `∩ D` tranzitivna ovojnica relacije `R`:

1. Iz prejšnje leme sledi, da je `∩ D` tranzitivna.

2. Ker velja `R ⊆ S` za vsak `S ∈ D`, seveda sledi `R ⊆ ∩ D`.

3. Denimo, da je `R ⊆ T` in `T ⊆ A × A` tranzitivna relacija. Tedaj velja `T ∈
   D`, torej je `T ⊆ ∩ D`. □

Po istem kopitu pokažemo, da ima vsaka relacija `R ⊆ A × A` tudi ostale
ovojnice. Je pa zgornji izrek neroden, ker nam dokaz ne poda uporabnega opisa
tranzitivne ovojnice. Povejmo, kako lahko razne ovojnice opišemo bolj
eksplicitno:

1. Refleksivna ovojnica relacije `R` je relacija `R ∪ Δ_A`, se pravi, da
   relaciji `R` dodamo še diagonalo.

2. Simetrična ovojnica relacije `R` je relacija `R ∪ Rᵀ`

3. Tranzitivna ovojnica relacije `R` je relacija `R⁺ := ∪_{n ≥ 1} Rⁿ`, se pravi

        R⁺ := R ∪ (R ∘ R) ∪ (R ∘ R ∘ R) ∪ ⋯

4. Refleksivna tranzitivna ovojnica relacije `R` je relacija `R* := ∪_{n ≥ 0} Rⁿ`, se pravi

        R* := Δ_A ∪ R ∪ (R ∘ R) ∪ (R ∘ R ∘ R) ∪ ⋯
