%% TO NI GLAVNA DATOTEKA
\documentclass[11pt,a4paper,twoside]{book}

%%%%%%%%%%%%%%%%%%%%%%%%%%%%%%%%%%%%%%%%%%%%%%%%%%%%%%%%%
%%%  Imported Packages
%%%%%%%%%%%%%%%%%%%%%%%%%%%%%%%%%%%%%%%%%%%%%%%%%%%%%%%%%
\usepackage[slovene]{babel}
\usepackage[utf8]{inputenc}
\usepackage[T1]{fontenc}

\usepackage{ifthen}
\usepackage{amssymb}
\usepackage{amsmath}
\usepackage{amsthm} % Must come after amsmath!
\usepackage{textgreek}
\usepackage{phonetic}
\usepackage{tablefootnote}
\usepackage{datetime2}

\usepackage{xcolor}
\definecolor{andrejcolor}{rgb}{0.7,0,0.7}
\definecolor{davorincolor}{rgb}{0,0.45,1}
\definecolor{anjacolor}{rgb}{1.0,0.5,0}

\newcommand{\andrej}[1]{{\small\textcolor{andrejcolor}{(#1 \ \mbox{--Andrej})}}}
\newcommand{\davorin}[1]{{\small\textcolor{davorincolor}{(#1 \ \mbox{--Davorin})}}}
\newcommand{\anja}[1]{{\small\textcolor{anjacolor}{(#1 \ \mbox{--Anja})}}}

\definecolor{notecolor}{rgb}{0.6,0.5,0.7}
\newcommand{\note}[1]{{\small\textcolor{notecolor}{(#1)}}}
\newcommand{\alert}[1]{{\small\textcolor{red}{\textbf{#1}}}}

\usepackage{tikz}
\usepackage{tkz-graph}
\usepackage{circuitikz}
\usepackage{xparse}
\usepackage{mathrsfs}
\usepackage{mathabx}
\usepackage{answers}

\usepackage{xypic}

%%% Trenutna verzija, če jo imamo
\IfFileExists{./verzija.tex}{\input{./verzija.tex}}{\newcommand{\OPTversion}{unknown}}

%%% Fonti kot v HoTT book.
\usepackage{mathpazo}
\usepackage[scaled=0.95]{helvet}
\usepackage{courier}
\linespread{1.05} % Palatino looks better with this

%%%%%%%%%%%%%%%%%%%%%%%%%%%%%%%%%%%%%%%%%%%%%%%%%%
%%% Globalne nastavitve
\def\OPTtitle{Logika in množice}

%%%%%%%%%%%%%%%%%%%%%%%%%%%%%%%%%%%%%%%%%%%%%%%%%%%%%%%%%%%%%
%%  Page Style & Margins (A4 page = 210mm x 297mm)

% PAGE GEOMETRY
\usepackage[papersize={210mm,297mm}, % A4
            twoside,
            includehead,
            top=1in, % margina na vrhu strani
            bottom=1in, % margina na dnu strani
            inner=1in, % margina na notranji strani strani
            outer=1in, % margina na zunanji strani strani
            bindingoffset=0pt % dodatna margina na notranji strani
           ]{geometry}

%%%%%%%%%%%%%%%%%%%%%%%%%%%%%%%%%%%%%%%%%%%%%%%%%%%%%%%%%%%%%
% HYPERLINKING AND PDF METADATA
\usepackage[backref=page,
            colorlinks,
            citecolor=linkcolor,
            linkcolor=linkcolor,
            urlcolor=linkcolor,
            unicode,
            pdfauthor={Andrej Bauer, Davorin Lešnik},
            pdftitle={\OPTtitle},
            pdfsubject={matematika},
            pdfkeywords={logika,množice,osnove matematike}]{hyperref}
\renewcommand{\backref}[1]{}
\renewcommand{\backrefalt}[4]{%
   \ifcase #1 %
   (Ni citirano.)
   \or
   (Citirano na strani\ #2.)
   \else
   (Citirano na straneh\ #2.)
   \fi}

\definecolor{linkcolor}{rgb}{0,0,0} % Barva hiperpovezav

%%%%%%%%%%%%%%%%%%%%%%%%%%%%%%%%%%%%%%%%%%%%%%%%%%%%%%%%%%%%%
%%%% Header and footers
%%%%%%%%%%%%%%%%%%%%%%%%%%%%%%%%%%%%%%%%%%%%%%%%%%%%%%%%%%%%%

\usepackage{fancyhdr} % To set headers and footers
\pagestyle{fancyplain}
\setlength{\headheight}{15pt}
\renewcommand{\chaptermark}[1]{\markboth{\textsc{Poglavje \thechapter. #1}}{}}
\renewcommand{\sectionmark}[1]{\markright{\textsc{\thesection\ #1}}}

\lhead[\fancyplain{}{{\thepage}}]%
      {\fancyplain{}{\nouppercase{\rightmark}}}
\rhead[\fancyplain{}{\nouppercase{\leftmark}}]%
      {\fancyplain{}{\thepage}}
\cfoot[\texttt{\footnotesize [delovna verzija {\OPTversion}- \DTMnow]}]{\texttt{\footnotesize [delovna verzija {\OPTversion}- \DTMnow]}}
\lfoot[]{}
\rfoot[]{}

%%%%%%%%%%%%%%%%%%%%%%%%%%%%%%%%%%%%%%%%%%%%%%%%%%%%%%%%%%%%%
%%%%%% Macros
%%%%%%%%%%%%%%%%%%%%%%%%%%%%%%%%%%%%%%%%%%%%%%%%%%%%%%%%%%%%%%%%%%%%%%%%%%%%%%%%%%%%%%%%%%%%%%%%%%%%%%%%%%%%%%%%%%%%%%
%%%  Commands
%%%%%%%%%%%%%%%%%%%%%%%%%%%%%%%%%%%%%%%%%%%%%%%%%%%%%%%%%%%%%%%%%%%%%%%%%%%%%%%%%%%%%%%%%%%%%%%%%%%%%%%%%%%%%%%%%%%%%%


%%%%%%%%%%%%%%%%%%%%%%%%%%%%%%%%%%%%%%%%%%%%%%%%%%%%%%%%%%%%%
%%%  Theorems etc.
%%%%%%%%%%%%%%%%%%%%%%%%%%%%%%%%%%%%%%%%%%%%%%%%%%%%%%%%%%%%%
{
\theoremstyle{theorem}
\newtheorem{izrek}{Izrek}[chapter]
\newtheorem{lema}[izrek]{Lema}
\newtheorem{trditev}[izrek]{Trditev}
\newtheorem{posledica}[izrek]{Posledica}
\newtheorem{pravilo}[izrek]{Pravilo}
}

{
\theoremstyle{definition}
\newtheorem{definicija}[izrek]{Definicija}
\newtheorem{opomba}[izrek]{Opomba}
\newtheorem{primer}[izrek]{Primer}
\newtheorem{zgled}[izrek]{Zgled}
\newtheorem{naloga}[izrek]{Naloga}
}


%%%%%%  Proofs
%%%%%%%%%%%%%%%%%%%%%%%%%%%%%%%%%%%%%%%%%%%%%%%%%%%%%%%%%%%%%
% Za dokaze uporabimo amsmath proof, sicer ne deluje \qedhere.


%%%%%%  Auxiliary
%%%%%%%%%%%%%%%%%%%%%%%%%%%%%%%%%%%%%%%%%%%%%%%%%%%%%%%%%%%%%
\newcommand{\sizedescriptor}[2]
{
\ifthenelse{\equal{#1}{0}}{}{
\ifthenelse{\equal{#1}{1}}{\big}{
\ifthenelse{\equal{#1}{2}}{\Big}{
\ifthenelse{\equal{#1}{3}}{\bigg}{
\ifthenelse{\equal{#1}{4}}{\Bigg}{
#2}}}}}
}

\newcommand{\someref}{{\small\textcolor{blue}{[\textbf{ref.}]}}}
\newcommand{\intermission}{\bigskip\medskip}
\newcommand{\ltc}[1]{$\backslash$\texttt{#1}}  % LaTeX command
\newcommand{\nls}[1]{``\textit{#1}''}  % sentence in a natural language

%%%%%%  Logical Quantifiers, λ- and ι-Expressions
%%%%%%%%%%%%%%%%%%%%%%%%%%%%%%%%%%%%%%%%%%%%%%%%%%%%%%%%%%%%%

\newcommand{\all}[1]{\forall #1 .\,}
\newcommand{\some}[1]{\exists #1 .\,}
\newcommand{\exactlyone}[1]{\exists{!} #1 .\,}
\newcommand{\lam}[1]{\lambda #1 .\,}
\newcommand{\that}[1]{\iota #1 .\,}

%%%%%%  Logic
%%%%%%%%%%%%%%%%%%%%%%%%%%%%%%%%%%%%%%%%%%%%%%%%%%%%%%%%%%%%%
\newcommand{\tvs}{\Omega}  % set of truth values
\newcommand{\true}{\top}  % truth
\newcommand{\false}{\bot}  % falsehood
\newcommand{\etrue}{\boldsymbol{\top}}  % emphasized truth
\newcommand{\efalse}{\boldsymbol{\bot}}  % emphasized falsehood
\newcommand{\impl}{\Rightarrow}  % implication sign
\newcommand{\revimpl}{\Leftarrow}  % reverse implication sign
\newcommand{\lequ}{\Leftrightarrow}  % equivalence sign
\newcommand{\xor}{\mathbin{\veebar}}  % exclusive disjunction sign
\newcommand{\shf}{\mathbin{\uparrow}}  % Sheffer connective
\newcommand{\luk}{\mathbin{\downarrow}}  % Łukasiewicz connective


%%%%%%  Sets
%%%%%%%%%%%%%%%%%%%%%%%%%%%%%%%%%%%%%%%%%%%%%%%%%%%%%%%%%%%%%
%  \set{1, 2, 3}         ->  {1, 2, 3}
%  \set{a \in X}{a < 1}  ->  {a ∈ X | a < 1}
\NewDocumentCommand{\set}
{O{auto} m G{\empty}}
{\sizedescriptor{#1}{\left}\{ {#2} \ifthenelse{\equal{#3}{}}{}{ \; \sizedescriptor{#1}{\middle}| \; {#3}} \sizedescriptor{#1}{\right}\}}
%\newcommand{\vsubset}{\Mapstochar\cap}
%\newcommand{\finseq}[1]{{#1}^*}
\newcommand{\pst}{\mathcal{P}}
\renewcommand{\complement}[1]{{#1}^C}


%%%%%%  Number Sets, Intervals
%%%%%%%%%%%%%%%%%%%%%%%%%%%%%%%%%%%%%%%%%%%%%%%%%%%%%%%%%%%%%
\newcommand{\NN}{\mathbb{N}}
\newcommand{\ZZ}{\mathbb{Z}}
\newcommand{\QQ}{\mathbb{Q}}
\newcommand{\RR}{\mathbb{R}}
\newcommand{\CC}{\mathbb{C}}
\newcommand{\HH}{\mathbb{H}}
\newcommand{\OO}{\mathbb{O}}
\newcommand{\intoo}[3][\RR]{{#1}_{(#2, #3)}}
\newcommand{\intcc}[3][\RR]{{#1}_{[#2, #3]}}
\newcommand{\intoc}[3][\RR]{{#1}_{(#2, #3]}}
\newcommand{\intco}[3][\RR]{{#1}_{[#2, #3)}}


%%%%%%  Maps and Relations
%%%%%%%%%%%%%%%%%%%%%%%%%%%%%%%%%%%%%%%%%%%%%%%%%%%%%%%%%%%%%
\newcommand{\id}[1][]{\mathrm{id}_{#1}}  % identity map
\newcommand{\argbox}{{\;\!\fbox{\phantom{M}}\;\!}}  % box for a function argument
\newcommand{\konst}[1]{\mathrm{k}_{#1}} % constant map
\newcommand{\rstr}[1]{\left.{#1}\right|}  % map restriction
\newcommand{\im}{\mathrm{im}}  % map image
\newcommand{\parto}{\mathrel{\rightharpoonup}}  % partial mapping sign
\NewDocumentCommand{\rel}
{O{\empty} O{\empty}}
{\ifthenelse{\equal{#1}{}}{\mathscr{R}}{{#1} \mathrel{\mathscr{R}} {#2}}}  % a relation
\NewDocumentCommand{\srel}
{O{\empty} O{\empty}}
{\ifthenelse{\equal{#1}{}}{\mathscr{S}}{{#1} \mathrel{\mathscr{S}} {#2}}}  % a second relation
\newcommand{\dom}{\mathrm{dom}}  % domain
\newcommand{\cod}{\mathrm{cod}}  % codomain
\newcommand{\dd}[1]{D_{#1}}  % domain of definition
\newcommand{\rn}[1]{Z_{#1}}  % range
\newcommand{\graph}[1]{\Gamma_{#1}}  % graph of a (partial) function
\NewDocumentCommand{\img}  % image
{O{\empty} m G{\empty}}
{{#2}_*\ifthenelse{\equal{#3}{}}{}{\!\sizedescriptor{#1}{\left}( {#3} \sizedescriptor{#1}{\right})}}
\NewDocumentCommand{\pim}  % preimage
{O{\empty} m G{\empty}}
{{#2}^*\ifthenelse{\equal{#3}{}}{}{\!\sizedescriptor{#1}{\left}( {#3} \sizedescriptor{#1}{\right})}}
\newcommand{\ec}[2][]{[\:\!{#2}\:\!]_{#1}}  % equivalence class
\newcommand{\transposed}[1]{\widehat{#1}}


%%%%%%  Projections and Injections
%%%%%%%%%%%%%%%%%%%%%%%%%%%%%%%%%%%%%%%%%%%%%%%%%%%%%%%%%%%%%
\NewDocumentCommand{\fst}
{O{\empty} O{\empty}}
{\pi_1^{{#1}\ifthenelse{\equal{#2}{}}{}{,}{#2}}}
\NewDocumentCommand{\snd}
{O{\empty} O{\empty}}
{\pi_2^{{#1}\ifthenelse{\equal{#2}{}}{}{,}{#2}}}
\NewDocumentCommand{\inl}
{O{\empty} O{\empty}}
{\iota_1^{{#1}\ifthenelse{\equal{#2}{}}{}{,}{#2}}}
\NewDocumentCommand{\inr}
{O{\empty} O{\empty}}
{\iota_2^{{#1}\ifthenelse{\equal{#2}{}}{}{,}{#2}}}


%%%%%%  Categories
%%%%%%%%%%%%%%%%%%%%%%%%%%%%%%%%%%%%%%%%%%%%%%%%%%%%%%%%%%%%%
\newcommand{\ct}[1]{\mathbf{#1}}
\newcommand{\mnoz}{\ct{Mno\check{z}}}
\newcommand{\pkol}{\ct{PKol}}  % category of semirings
\newcommand{\upkol}{\pkol_1}  % category of unital semirings
\newcommand{\kol}{\ct{Kol}}  % category of rings
\newcommand{\ukol}{\kol_1}  % category of unital rings


%%%%%%  Exercises and Solutions
%%%%%%%%%%%%%%%%%%%%%%%%%%%%%%%%%%%%%%%%%%%%%%%%%%%%%%%%%%%%%
\Newassociation{resitev}{Resitev}{resitve}
\renewcommand{\Resitevlabel}[1]{\emph{Re\v{s}itev~#1}}
{
\theoremstyle{definition}
\newtheorem{vaja}{Vaja}[chapter]
}


%%%%%%  Misc.
%%%%%%%%%%%%%%%%%%%%%%%%%%%%%%%%%%%%%%%%%%%%%%%%%%%%%%%%%%%%%
\renewcommand{\divides}{\,|\,}
% Načeloma bi morala biti navpična črta v \divides obdana z \mathrel, ampak to vodi do prevelikih presledkov.
\newcommand{\df}[1]{\emph{\textbf{#1}}}  % defined notion
\newcommand{\oper}{\mathop{\circledast}\nolimits}  % symbol for a generic operation
\newcommand{\soper}{\mathop{\boxasterisk}\nolimits}  % symbol for a second generic operation
\newcommand{\qo}[1]{\;\!\widetilde{#1}\;\!}  % quotient operation
\newcommand{\tconc}{\mathop{\bullet}\nolimits}  % symbol for binary tree concatenation
\newcommand{\conc}{\mathop{::}\nolimits}  % symbol for string concatenation
\newcommand{\ism}{\cong}  % isomorphic
\newcommand{\inv}[1]{#1^{-1}} % inverz preslikave
\newcommand{\equ}{\sim}  % equivalent
\newcommand{\dfeq}{\mathrel{\mathop:}=}  % definitional equality
\newcommand{\revdfeq}{=\mathrel{\mathop:}}  % reverse definitional equality
\newcommand{\isdefined}[1]{{#1}\!\downarrow}  % given value is defined
\newcommand{\kleq}{\simeq}  % Kleene equality
\newcommand{\claim}[3]{{#1} \;\colon\; \frac{#2}{#3}}  % claim, divided on context, assumptions, conclusions
\newcommand{\one}{\mathtt{\mathbf{1}}}  % generic singleton
\newcommand{\unit}{\mathord{()}}  % element in a generic singleton
\newcommand{\nul}{\mathtt{N}}  % null map
\newcommand{\suc}{\mathtt{S}}  % successor
\newcommand{\prd}{\mathtt{P}}  % predecessor
\newcommand{\tprd}{\tilde{\prd}}  % predecessor as a total function
\newcommand{\monus}{\mathbin{\vphantom{+}\text{\mathsurround=0pt \ooalign{\noalign{\kern-.35ex}\hidewidth$\smash{\cdot}$\hidewidth\cr\noalign{\kern.35ex}$-$\cr}}}}
% Definicija za monus pobrana s TeX Stack Exchange
\newcommand{\wf}{\prec}  % well-founded order
\NewDocumentEnvironment{implproof}  % proof of an implication
{O{\empty} G{\empty} O{=>} G{\empty}}
{
\begin{description}
\item[\quad$\sizedescriptor{#1}{\left}({#2}
\ifthenelse{\equal{#3}{=>}}{\impl}{
\ifthenelse{\equal{#3}{<=}}{\revimpl}{
\ifthenelse{\equal{#3}{->}}{\rightarrow}{
\ifthenelse{\equal{#3}{<-}}{\leftarrow}{
#3
}}}} {#4}\sizedescriptor{#1}{\right})$]\ \vspace{0.3em}\\
}
{
\end{description}
}
\NewDocumentCommand{\pres}  % presentation of an algebraic structure with generators and relations
{O{auto} m G{\empty}}
{\sizedescriptor{#1}{\left}\langle {#2} \ifthenelse{\equal{#3}{}}{}{ \; \sizedescriptor{#1}{\middle}| \; {#3}} \sizedescriptor{#1}{\right}\rangle}


%%%%%%%%%%%%%%%%%%%%%%%%%%%%%%%%%%%%%%%%%%%%%%%%%%%%%%%%%%%%%%%%%%%%%%%%%%%%%%%%%%%%%%%%%%%%%%%%%%%%%%%%%%%%%%%%%%%%%%

%%% Local Variables:
%%% mode: latex
%%% TeX-master: "ucbenik-lmn"
%%% End:


\begin{document}

%--------------------------------------------------------------------
%--------------------------------------------------------------------
% TITLE PAGE

% če se spremeni naslov, je treba spremeniti tudi zgoraj v paketu hyperref
\title{\OPTtitle\\\texttt{\OPTversion}}
\author{Andrej Bauer \and Davorin Lešnik}
\maketitle

%--------------------------------------------------------------------
%--------------------------------------------------------------------
% Foreword

\chapter*{Predgovor}%\addcontentsline{toc}{chapter}{\numberline{}Predgovor}

%--------------------------------------------------------------------
%--------------------------------------------------------------------
% TOC

\tableofcontents
% \listoftables % To se meni zdi nepotrebno, zakaj se to daje v knjige? (Andrej)

%--------------------------------------------------------------------
%--------------------------------------------------------------------
% BODY

\Opensolutionfile{resitve}

\input{matematicno-izrazanje.tex}
\chapter{Množice in preslikave}
\label{cha:mnozice-in-preslikave}


Temeljni gradniki sodobne matematike so \df{množice}, ki so skupki ali zbirke matematičnih
objektov, lahko spet množice. Vsaka množica sestoji iz \df{elementov} in je z njimi
natančno določena. Kadar je $a$ element množice $M$, to zapišemo $a \in M$.

Ideja množice kot poljubne zbirke elementov je zavajajoče preprosta, kar so na lastni koži
izkusili matematiki na prelomu iz 19.~v 20.~stoletje. Takrat so že vedeli, da so množice zelo
uporabne in da lahko iz njih tvorimo razne vrste matematičnih objektov. A znameniti
matematik in filozof Bertrand Russell je odkril paradoks, ki se imenuje po njem, in gre
takole. Naj bo~$R$ množica vseh množic, ki niso element same sebe. Ali $R$ je element~$R$?
Če je $R$ element $R$, potem iz definicije $R$ sledi, da $R$ ni element $R$. In če $R$ ni
element $R$, spet iz definicije $R$ sledi, da $R$ je element $R$. Torej $R$ hkrati je in
ni svoj element, kar je protislovje! Russellov paradoks ste morda že spoznali v
priljubljeni različici, ki govori o vaškem brivcu, ki brije vse vaščane, ki ne brijejo
samih sebe.

Russellov paradoks je povzročil pravo krizo v temeljih matematike. Ker so bile množice
nepogrešljivo orodje, jih niso hoteli kar zavreči, po drugi strani pa je bilo treba
preprečiti Russellov in druge paradokse, ki so jih še odkrili. Bertrand Russell je
predlagal rešitev, ki jo je poimenoval \df{teorija tipov}. Russellova teorija tipov je
pomembno vplivala na nadaljni razvoj temeljev matematike, sodobna teorija tipov pa je
pomembno orodje v računalništvu. Tako kot množice so bili tipi skupki elementov, a so
tvorili neskončno hierarhijo, v kateri so bili elementi tipa vedno iz nižjega nivoja
hierarhije kot tip, ki so mu pripadali. Za potrebe večine matematike zadostuje že
preprostejša dvoslojna hierarhija množic in \df{razredov}. Množice smejo biti elementi
množic in razredov, razredi pa ne. Russellov paradoks izgine, ker je $R$ razred vseh
tistih množic, ki niso same svoj element. Vprašanje, ali je $R$ element samega sebe, tako
postane nesmiselno, saj $R$ ni množica. A zaenkrat odložimo podrobnejšo obravnavo razredov
in se raje posvetimo osnovnima pojmoma, množica in preslikava.

V splošni razpravi o množicah, ki bi presegala meje matematične vede, bi se opirali na
zgodovinski in družbeni kontekst, jezikovni izvor in rabo besed `množica', `skupek' in
`zbirka', kognitivno analizo, eksperimente, filozofijo itn. Vsi ti vidiki so za matematike
izjemo koristni, saj iz takih ``pred-matematičnih'' obravnav črpamo sveže zamisli in
matematiko naredimo zares uporabno. Ko pa delujemo znotraj matematike, zunanje vplive
odmislimo in se zanašamo le še na pravila logičnega sklepanja in matematične zakone, da ne
prihaja do nejasnosti in dvomljivih sklepov.

Kot matematiki lahko ustvarimo takšen ali drugačen pojem množice in pri tem imamo popolno
svobodo. Se množica lahko spreminja ali vedno vsebuje iste elemente? Je pomemben vrsti red
elementov v množici? Sme množica biti element same sebe? Ali morajo biti elementi množice
izračunljivi? To so vprašanja, ki nimajo enoznačnega odgovora. In res je znanih več med
seboj nezdružljivih zvrsti teorije množic, ki matematično opredeljujejo različne vidike
običajnega razumevanja besede `množica'. Mi bomo spoznali ``standardno'' teorijo množic,
ki jo uporablja velika večina matematikov.


\section{Načelo ekstenzionalnosti}
\label{sec:nacelo-ekstenzionalnosti}

Zamisel, da je množica natančno določena s svojimi elementi, izrazimo z matematičnim
zakonom, ki mu pravimo \df{načelo ekstenzionalnosti}:

\begin{pravilo}[Ekstenzionalnost množic]
  Množici sta enaki, če vsebujeta iste elemente.
\end{pravilo}

Kaj pravzaprav pomeni, da je to ``pravilo'', ``matematični zakon'' ali ``načelo''? So ga
razglasili v parlementu, je to zakon narave, ali morda dogma, ki jo je razglasil profesor
na predavanjih? Bodo tisti, ki načela ekstenzionalnosti ne spoštujejo, deležni Lešnikove
masti? Ne. Matematični zakoni so \emph{dogovori}, nekakšna pravila matematične igre. V
zgodovinskem razvoju matematike so se uveljavili tisti dogovori, ki so bili uporabni v
naravoslovju in tehniki, ali pa so v njih matematiki videli notranjo lepoto in lastno
uporabno vrednost.

Pravkar smo se dogovorili, da bomo obravnavali matematične objekte množice, ki vsebujejo
elemente in da zanje velja načelo ekstenzionalnosti. Namesto besed `množica' in `element'
bi lahko izbrali tudi kaki drugi besedi, denimo `zbor' in `član', ali celo `morje' in
`riba', s čimer se matematična vsebina pojmov ne bi čisto nič spremenila, čeprav ne gre
preveč izzivati svojih stanovskih kolegic in kolegov. Strukturo, lastnosti in povezave med
matematičnimi objekti namreč določajo dogovorjeni matematični zakoni in ne besede, s
katerimi jih poimenujemo.

Še enkrat poudarimo, da ima vsakdo, še posebej pa mladi um, popolno svobodo matematičnega
ustvarjanja. Želite razmišljati o drugačnih množicah, ki ne zadoščajo načelou
ekstenzionalnsti? Ali pa o številih, ki zadoščajo zakonu $x + x = 0$? O geometriji, v
kateri skozi točko lahko potegnemo dve vzporednici k dani premici? Kar dajte! Pri tem vas
le prosimo, celo zahtevamo, da razmišljate temeljito, vztrajno in globoko, da ste iskreni
do sebe in ostalih ter da svoje zamisli in spoznanja predstavite na matematikom razumljiv
način.

Vrnimo se k našim množicam. Načelo ekstenzionalnosti nam pove, da lahko množico podamo
tako, da natančno opredelimo njene elemente. A to ne pomeni, da množica obstaja, brž ko jo
lahko natančno opredelimo! To je pot, ki vodi naravnost do Russelovega paradoksa, saj so
elementi paradoksalne množice~$R$ natančno opredeljeni. Potrebujemo dodatna pravila, ki
določajo dopustne \df{konstrukcije množic}. Izbrati jih moramo previdno, da se izognemo
težavam.

\section{Končne množice}
\label{sec:koncne-mnozice}

Posebej preprosta konstrukcija množic združi končen nabor matematičnih objektov v množico.
Na primer, če so $a$, $b$ in $c$ matematični objekti, potem lahko tvorimo množico
%
\begin{equation*}
  \set{a, b, c}
\end{equation*}
%
katere objekti so natanko $a$, $b$ in $c$. To pomeni, da za vsak matematični objekt~$x$
velja
%
\begin{equation*}
  \text{$x \in \set{a, b, c}$, če in samo če $x = a$ ali $x = b$ ali $x = c$.}
\end{equation*}
%
Fraza ``če in samo če'' tu pomeni, da velja dvoje:
%
\begin{enumerate}
\item Če $x = a$ ali $x = b$ ali $x = c$, potem $x \in \set{a, b, c}$.
\item Če $x \in \set{a, b, c}$, potem $x = a$ ali $x = b$ ali $x = c$.
\end{enumerate}
%
Tako nam na primer prva trditev zagotavlja $1+1 \in \set{1, 2, 3}$, ker velja
vsaj ena od možnosti: $1 + 1 = 1$ ali $1 + 1 = 2$ ali $1 + 1 = 3$. Iz druge trditve sledi, da
$5 \in \set{1, 2, 3}$ ne velja, ker ne velja nobena od možnosti: $5 = 1$ ali $5 = 2$ ali
$5 = 3$.

Splošna konstrukcija končnih množic poteka takole.

\begin{pravilo}
  \label{pravilo:koncna-mnozica}
  Za vse objekte $a$, $b$, \dots, $z$ je $\set{a, b, \ldots, z}$ množica, katere elementi
  so natanko objekti $a$, $b$, \dots, $z$.
\end{pravilo}

Za trenutek ustavimo tok misli in opozorimo, da zapis s tropičjem `$\ldots$' ni dovolj
natančen, saj dopušča dvoumnosti. Denimo, so elementi množice
%
\begin{equation*}
  \set{3, 5, 7, \ldots, 31},
\end{equation*}
%
liha števila med $3$ in $31$, ali samo praštevila? Zapis res ni dovolj natančen. Kljub
temu tak zapis v praksi uporabljamo, ker v praksi bralec večinoma pravilno ugane, kaj je
bilo mišljeno, saj imamo ljudje zelo podobne sposobnosti prepoznavanja vzorcev. Z
matematičnega vidika pa to ni dopustno, saj lahko tropičje \emph{vedno} razumemo na več
načinov. (Ne verjamete? Naslednji člen v zaporedju $1, 2, 3, \ldots$ je seveda~$5$, ker je
naslednji člen vsota prejšnjih dveh, kot v Fibonaccijevem zaporedju.)

Kot smo že omenili, želimo pojem množice, pri kateri vrstni red elementov ni pomemben.
Torej bi morali biti množici $\set{1, 2}$ in $\set{2, 1}$ enaki. Pa je to res? Velja ena
od treh možnosti:
%
\begin{enumerate}
\item Iz načela ekstenzionalnosti in konstrukcije množic $\set{1, 2}$ in $\set{2, 1}$ sledi, da sta enaki.
\item Iz načela ekstenzionalnosti in konstrukcije množic $\set{1, 2}$ in $\set{2, 1}$ sledi, da nista enaki.
\item Načelo ekstenzionalnosti in konstrukcije množic $\set{1, 2}$ in $\set{2, 1}$ ne določajo, ali sta enaki.
\end{enumerate}
%
V prvem primeru bi želeli dokazati enakost. V drugem primeru smo v zagati, saj smo se
dogovorili za matematična pravila, ki imajo neželene posledice. V tretjem primeru moramo
dodati še kakšne nove zakone o množicah. Na srečo obvelja prva možnost.

\begin{trditev}
  Množici $\set{1, 2}$ in $\set{2, 1}$ sta enaki.
\end{trditev}

\begin{proof}
  Dokaz, ki ga bomo zapisali je izjemno podroben in ga v praksi matematik ne bi zapisal,
  saj je z njegovim branjem več dela, kot če bi naredili sami. Ker pa želimo pokazati, da
  tudi najbolj trivialna dejstva lahko dokažemo, ga zapišimo.

  Izhajati smemo izključno iz naslednji dejstev:
  %
  \begin{itemize}
  \item načelo ekstenzionalnosti,
  \item $x \in \set{1, 2}$, če in samo če $x = 1$ ali $x = 2$,
  \item $x \in \set{2, 1}$, če in samo če $x = 2$ ali $x = 1$.
  \end{itemize}
  %
  Najprej uporabimo načelo ekstenzionalnosti, ki zagotavlja, da sta $\set{1, 2}$ in
  $\set{2, 1}$ enaki, če imata iste elemente. Dokažimo torej, da imata iste elemente. To
  naredimo v dveh korakih:
  %
  \begin{enumerate}
  \item Za vsak element $\set{1, 2}$ dokažemo, da je element $\set{2, 1}$.
    Naj bo $x \in \set{1, 2}$. Iz definicije množice $\set{1, 2}$
    sledi, da je $x = 1$ ali $x = 2$. Obravnavamo dva podprimera:
    %
    \begin{enumerate}
    \item Primer $x = 1$: iz $x = 1$ sledi, da je $x = 2$ ali $x = 1$, zato je $x \in \set{2, 1}$.
    \item Primer $x = 2$: iz $x = 2$ sledi, da je $x = 2$ ali $x = 1$, zato je $x \in \set{2, 1}$.
    \end{enumerate}
    %
  \item Za vsak element $\set{2, 1}$ dokažemo, da je element $\set{1, 2}$.

    Ta korak je povsem podoben prvemu, le da je treba povsod zamenjati~$1$ in~$2$.
    Matematik bi zato na tem mestu zapisal, da je drugi korak podoben prevemu in dokaz
    zaključil. A tega tokrat ne bomo storili in bomo zapisali popoln dokaz.

    Naj bo $x \in \set{2, 1}$. Iz definicije množice $\set{2, 1}$ sledi, da je $x = 2$ ali
    $x = 1$. Obravnavamo dva primera:
    %
    \begin{enumerate}
    \item Primer $x = 2$: iz $x = 2$ sledi, da je $x = 1$ ali $x = 2$, zato je $x \in \set{1, 2}$.
    \item Primer $x = 1$: iz $x = 1$ sledi, da je $x = 1$ ali $x = 2$, zato je $x \in \set{1, 2}$. \qedhere
    \end{enumerate}
    %
  \end{enumerate}
\end{proof}

Mimogrede, črn kvadratek označuje konec dokaza. Imenuje se tudi ``Halmos'' po matematiku
Paulu Halmosu, ki ga je prvi uporabljal. S podobnim razmislekom, ki ga prepuščamo za vajo,
lahko dokažemo, da ni pomembno, ali se element pojavi enkrat ali večkrat.

\begin{naloga}
  Podrobno dokažite, da sta množici $\set{1, 1, 2}$ in $\set{1, 2}$ enaki.
\end{naloga}

V prejšnji nalogi smo zapisali $\set{1, 1, 2}$. Pa je to sploh dovoljeno?
Pravilo~\ref{pravilo:koncna-mnozica} pravi, da lahko iz objektov $a, b, c, \ldots, z$
tvorimo končno množico $\set{a, b, \ldots, z}$. Nikjer ne piše, da smeta biti $a$ in $b$
enaka, zato je upravičeno vprašanje, ali je dovoljeno za $a$ in $b$ vzeti~$1$. V
matematiki vse razumemo dobesedno. V pravilu~\ref{pravilo:koncna-mnozica} piše ``Za vse
objekte'', torej imamo povsem proste roke. Povedano z drugimi besedami, množico
$\set{1, 1, 2}$ smemo tvoriti, ker nikjer ne piše, da morajo biti elementi različni.

V zvezi s pravilom~\ref{pravilo:koncna-mnozica} se pojavljajo še drugi dvomi. Ali smemo
tvoriti množico, ki ima več elementov, kot je črk abecede? Ali bi bilo pravilo še vedno
isto, če bi namesto ``$a, b, \ldots, z$'' zapisali ``$a, b, \ldots, j$''? Ali smemo
tvoriti množico z nič elementi? Če namreč vstavimo nič elementov, se pravilo glasi ``Za
vse objekte je $\set{\,}$ množica, katere elementi so natanko objekti,'' kar je vsaj
nenavadno. Iz nesrečnega tropičja se res ne vidi, kaj je in kaj ni dovoljeno. Če poškilite
v razdelek~\ref{sec:aksiomi-teorije-mnozic}, kjer so našteti ``uradni'' aksiomih teorije
množic, tam pravila o končnih množicah ne boste našli, saj sledi iz treh bolj osnovnih
pravil.

\begin{pravilo}
  \label{pravilo:prazna-mnozica}
  \df{Prazna množica} $\emptyset$ je množica, ki nima elementov.
\end{pravilo}

\begin{pravilo}
  \label{pravilo:neurejeni-dvojec}
  Za vsak $x$ in $y$ je \df{(neurejeni) par} ali \df{dvojec} $\set{x, y}$ množica, katere
  elementa sta natanko $x$ in $y$.
\end{pravilo}

\begin{pravilo}
  \label{pravilo:unija}
  Za vsaki množici $A$ in $B$ je \df{unija $A \cup B$} množica, ki ima za elemente
  natanko vse objekte, ki so element $A$ ali element $B$.
\end{pravilo}

V pravilu~\ref{pravilo:neurejeni-dvojec} smo besedo ``neurejeni'' zapisali v oklepaju, kar
pomeni, da beseda pravzaprav ni pombembna in bi jo lahko tudi izpustili. Se pravi, da
``neurejeni dvojec'' in ``dvojec'' pomenita isto. V primeru nejasnosti raje uporabimo
daljšo obliko.

Tri nova pravila skupaj nadomestijo pravilo~\ref{pravilo:koncna-mnozica} in odstranijo
marsikateri dvom o uporabi. Prvo pravilo pojasni, da lahko tvorimo množico brez elementov.
Poleg oznake $\emptyset$ je za prazno množico smiselno uporabiti tudi zapis $\set{\,}$.

Drugo pravilo pove, kako lahko tvorimo množico z dvema elementoma, pa tudi z enim.
Spomnimo se, pravila je treba brati dobesedno: za $x$ in $y$ bi lahko vzeli dvakrat isti
objekt~$z$ in tvorili množico $\set{z, z}$, ki ima natanko elementa $z$ in $z$. To je
pravzaprav množica z enim samim elementom $z$, zato ji pravimo tudi \df{enojec} in jo
zapišemo~$\set{z}$.

Tretje pravilo nam omogoča, da tvorimo večje množice. Denimo, množico z elementi $a$, $b$,
$c$ lahko tvorimo kot unijo
%
\begin{equation*}
  \set{a, b} \cup \set{c}.
\end{equation*}
%
To ni edini način, enako množico lahko dobimo na več načinov:
%
\begin{equation*}
  (\set{a} \cup \set{b}) \cup \set{c}
  \quad\text{ali}\quad
  \set{b} \cup \set{c, a}
  \quad\text{ali}\quad
  \set{a,c,a} \cup \set{b,c}
  \quad\text{itn.}
\end{equation*}
%
Seveda bi morali dokazati, da so vse te množice enake, a tega ne bomo storili.

Pogosto nam bo prišlo prav, da bomo imeli pri roki množico z enim elementom, pri čemer nam
bo vseeno, kaj ta element je. V ta namen postavimo pravilo, ki zagotavlja obstoj množice z
enim elementom.

\begin{pravilo}
  \label{pravilo:enojec}
  \df{Standardni enojec} je množica~$\one$, katere edini element je~$\unit$.
\end{pravilo}

Morda se zdi nenavadno, da množico označimo s številom, a ta občutek bo hitro izginil, ko
bomo računali z množicami. Pravaprav bi lahko prazno množico označili z nič $\mathbf{0}$,
in nekateri matematiki to dejansko počnejo.

Edini element množice $\one$ smo označili z nenavadnim zapisom $\unit$. Na tem mestu ne
bomo pojasnili, zakaj pišemo tako, radovedneži pa lahko pogledajo v
razdelek~\ref{sec:algebra-mnozic}. Mimogrede, seveda velja $\one = \set{\unit}$.

Pravilo~\ref{pravilo:enojec} ni nujno potrebno, saj lahko tvorimo veliko različnih enojcev
kar sami $\set{\emptyset}$, $\set{42}$, $\set{\set{\emptyset}}$ itn. Ali je kateri od njih
``prvi med enakimi'' in bi ga lahko uporabljali kot ``standardni'' enojec? Ker je odgovor
v veliki meri stvar osebnega mnenja, je bolje, da razglasimo pravilo, ki ustoliči
standardni enojec. S prazno množico nimamo podobnih težav, saj je ena sama.

% \subsection{Druge množice}

% \andrej{To ne paše sem, ker bi bilo tu dosti bolj naravno nadaljevati s preslikavami.
%  To bomo prestavili na mesto, kjer bo dejansko prišlo prav.}

% Množice, s katerimi v matematiki delamo, tipično vsebujejo števila, ali pa so vsaj na tak ali drugačen način izpeljane iz številskih množic. Spomnimo se standardnih oznak najpogosteje uporabljanih številskih množic.
% \begin{center}
% \begin{tabular}{|cc|}
% \hline
% \textbf{Množica} & \textbf{Oznaka} \\
% \hline
% množica naravnih števil & $\NN$ \\
% množica celih števil & $\ZZ$ \\
% množica racionalnih števil & $\QQ$ \\
% množica realnih števil & $\RR$ \\
% množica kompleksnih števil & $\CC$ \\
% \hline
% \end{tabular}
% \end{center}

% Nekateri $0$ vzamejo za naravno število, nekateri ne. To je v celoti stvar dogovora, kaj pomeni pojem ``naravno število''. Za nas bo prišlo bolj prav, če ničlo štejemo kot element množice naravnih števil, torej $\NN = \set{0, 1, 2, 3, \ldots}$.

\section{Preslikave}

Temelj matematike ne tvorijo le množice, ampak tudi drugi matematični pojmi. Prvi izmed
njih je \df{preslikava}, oziroma s tujko \df{funkcija}.\footnote{Nekateri uporabljajo
  izraz ``funkcija'' samo za tiste preslikave, ki slikajo v realna ali kompleksna števila,
  vendar to navado izpodriva računalništvo, saj funkcije v programskih jezikih nimajo
  omejitev. Dandanes večina matematikov besedo ``funkcija'' obravnava kot sopomenko besede
  ``preslikava'' in tako jo bomo uporabljali tudi mi.} V srednji šoli ste že spoznali
nekatere preslikave, kot so na primer linearne preslikave, trigonometrijske funkcije,
logaritem itd. Nas pa ne bodo zanimale posamezne preslikave, ali posebne lastnosti
preslikav, ampak preslikave na splošno.

Vsaka preslikava ima tri sestavne dele: \df{domeno} ali \df{začetno množico},
\df{kodomeno} ali \df{ciljno množico} in \df{predpis}. Domeni se pogosto reče tudi
\df{definicijsko območje}. Če govorimo o preslikavi, ki ima domeno~$X$ in kodomeno~$Y$, to
ponazorimo s puščico med $X$ in $Y$, takole
%
\begin{equation*}
  \xymatrix{
    {X} \ar[r] &
    {Y}
  }
\end{equation*}
%
Če želimo preslikavo poimenovati, na primer $f$, zapišemo
%
\begin{equation*}
  \xymatrix{
   {f : X} \ar[r] &
    {Y}
  }
  \qquad\text{ali}\qquad
  \xymatrix{
   {X} \ar[r]^{f} &
   {Y}
  }
\end{equation*}
%
Pravimo, da je \df{$f$ preslikava iz $X$ v $Y$}. Zapis nad puščico je prikladen, kadar
imamo opravka z večimi preslikavami, ki jih predstavimo z diagramom. Na primer,
%
\begin{equation*}
  \xymatrix{
    {X} \ar[r] &
    {Y} \ar[r]^{f} &
    {Z}  &
    {W} \ar[l]_{g}
  }
\end{equation*}
%
nam pove, da imamo opravka z (neimenovano) preslikavo iz $X$ v $Y$, s preslikavo $f$ iz
$Y$ v $Z$ in s preslikavo $g$ is $W$ v $Z$. Diagrami so lahko še precej bolj zapleteni.

Tretji del preslikave je predpis, ki določa, kako elemente domene preslikamo v elemente
kodomene. Kaj pravzaprav to pomeni? Možnih je več odgovorov. V srednji šoli predpis
enačimo z matematično formulo, ki spremenljivko preslika v vrednost, na primer $x$ slika v
$2 \sin(x + \pi/4)$. S simboli to zapišemo
%
\begin{equation*}
  x \mapsto 2 \sin(x + \pi/4).
\end{equation*}
%
in preberemo ``$x$ se slika v dvakrat sinus od $x$ plus pi četrtin.''
%
Matematiki smo natančni, zato ne mešamo uporabe puščic $\to$ in $\mapsto$. Navadna puščica
se uporablja pri oznaki domene in kodomene, repata pa v predpisu. V računalništvu besedo
`predpis' razumemo kot `programska koda' in o preslikavah razmišljajo kar kot o
algoritmih --- tudi to je eden od možnih pogledov na preslikave.

V teoriji množic razumemo besedo `predpis' kot kakršnokoli prirejanje med elementi množic
domene~$X$ in kodomene~$Y$, mora pa veljati:
%
\begin{itemize}
\item \df{celovitost}: vsakemu elementu iz $X$ je prirejen vsaj en element iz $Y$,
\item \df{enoličnost}: če sta elementu $x$ prirejena $y \in Y$ in $z \in Y$, potem $y = z$.
\end{itemize}

Za vsako množico~$A$ je \df{identiteta} na~$A$ preslikava
%
\begin{equation*}
  \id[A] : A \to A
\end{equation*}
%
ki poljubnemu elementu $x \in A$ priredi~$x$. To je celovito prirejanje, saj vsak
$x \in A$ ima prirejeni element, namreč kar $x$, je pa tudi enolično: če sta $y_1$ in
$y_2$ prirejena $x \in A$, potem sta oba enaka~$x$ in zato enaka drug drugemu.

Za vsaki množici $A$ in $B$ ter $b \in B$ \df{konstantna preslikava}
%
\begin{equation*}
  \konst{b} : A \to B
\end{equation*}
%
priredi vsakemu elementu iz~$A$ element~$b$. Sami premislite, da je tako prirejanje
celovito in enolično.

\subsection{Funkcijski predpisi}
\label{sec:funkcijski-predpisi}

Predpise lahko podamo na različne načine, najbolj pogost pa je \df{funkcijski predpis}, ki
se mu še posebej posvetimo in se ob njem naučimo nekaj natančnosti. Funkcijski predpis ima
obliko
%
\begin{equation*}
  x \mapsto \cdots,
\end{equation*}
%
ki smo jo že videli maloprej. Na desni, lahko namesto $\cdots$ zapišemo izraz, v katerem
se sme pojaviti simbol~$x$, denimo
%
\begin{equation*}
  x \mapsto 1 + x^2.
\end{equation*}
%
S funkcijskim predpisom zapišemo identiteto in konstantno preslikavo takole:
%
\begin{align*}
  \id[A] &: A \to A
  &
  \konst{b} &: A \to B
  \\
  \id[A] &: x \mapsto x
  &
  \konst{b} &: x \mapsto b.
\end{align*}

Ni nujno, da se~$x$ pojavi, denimo $x \mapsto 42$ vsakemu elementu iz domene priredi
število $42$. V funkcijskem predpisu se smejo pojaviti tudi drugi simboli, ki jim
pravimo \df{parametri}. Tako je
%
\begin{equation*}
  x \mapsto a \cdot x + b
\end{equation*}
%
funkcijski predpis s parametroma $a$ in $b$, ki elementu $x$ priredi element $a \cdot x + b$.

Spremenljivka $x$ nima v naprej določene vrednosti, pač pa kaže, kam lahko vstavimo
elemente domene. Pravimo, da je $x$ \df{vezana spremenljivka}, kar pomeni, da je veljavna
le v funkcijskem predpisu, nanj je vezana, in da ni pomembno, s katerim simbolom jo
označimo. Tako sta funkcijska predpisa
%
\begin{equation*}
  x \mapsto 1 + x^2
  \qquad\text{in}\qquad
  a \mapsto 1 + a^2
\end{equation*}
%
enaka in lahko bi celo pisali $\Box \mapsto 1 + \Box^2$ ali
$\heartsuit \mapsto 1 + \heartsuit^2$.

V funkcijskem predpisu mora na levi stati en sam simbol, ki na desni kaže, kam je treba
vstaviti element iz domene. Tako
%
\begin{equation*}
  \sin(x) \mapsto \cos(2 x),
  \qquad
  3 + 2 \mapsto 5
  \qquad\text{in}\qquad
  \sin(x) \mapsto 2 \cdot \sin(x)
\end{equation*}
%
\emph{niso} veljavni funkcijski predpisi.

Seveda dopuščamo možnost, da se vezana spremenljivka pojavi enkrat, večkrat ali sploh ne.
Funkcijska predpisa
%
%
\begin{equation*}
  x \mapsto 42
  \qquad\text{in}\qquad
  x \mapsto x \cdot \sin(x)
\end{equation*}
%
sta torej veljavna.

Če želimo preslikavo z danim funkcijskim predpisom poimenovati, na primer $f$, zapišemo
%
\begin{equation*}
  f : x \mapsto 1 + x^2.
\end{equation*}
%
To preberemo ``$f$ slika $x$ v ena plus $x$ na kvadrat.'' Običajna sta tudi zapisa
%
\begin{equation*}
  f(x) = 1 + x^2
  \qquad\text{in}\qquad
  f(x) \dfeq 1 + x^2.
\end{equation*}
%
Funkcijske predpise je podrobno prvi preučeval Alonzo Church,\footnote{Alonzo Church
  (1903--1995) je bil ameriški matematik in logik, ki je pomembno prispeval k razvoju
  logike in teoretičnega računalništva. Njegov študent, Dana Stewarta Scott, je imel
  študenta Marka Petkovška in Andreja Bauerja, slednji pa je imel študenta Davorina
  Lešnika.} ki je uporabljal zapis
%
\begin{equation*}
  \lambda x \,.\, 1 + x^2
\end{equation*}
%
in teorijo funkcijskih predpisov poimenoval \df{$\lambda$-račun}. V logiki se je njegov
zapis obdržal in se uveljavil tudi v programski jezikih:
%
\begin{itemize}
\item v Pythonu pišemo \verb|lambda x : 1+x**2|,
\item v Haskellu pišemo \verb|\x -> 1+x**2| in
\item v OCamlu pišemo \verb|fun x => 1+x*x|.
\end{itemize}
%
Predvsem v programiranju funkcijskim predpisom pravijo tudi \df{anonimne} ali \df{brezimne
  preslikave}.

Nekateri starejši zapisi funkcijskih predpisov so slabi, a jih ljudje vztrajno
uporabljajo. Opozorimo le na en slab zapis, ki povzroča precej preglavic, ne da bi se
matematiki tega zares zavedali. Funkcijski predpis mora določati vezano spremenljivko,
sicer ne vemo, kako vstaviti vrednosti, a na žalost jo matematiki pogosto izpustijo skupaj
z $\mapsto$, da ostane samo izraz na desni.
%
Težava je v tem, da se lahko v funkcijskem predpisu pojavi več kot en simbol. Če vam na primer povem, da imam v mislih funkcijski predpis
%
\begin{equation*}
  a \cdot x + b
\end{equation*}
%
boste vsi mislili, da je mišljeno $x \mapsto a \cdot x + b$. A pravzprav bi lahko bilo
tudi $a \mapsto a \cdot x + b$ ali $b \mapsto a \cdot x + b$ ali celo
$t \mapsto a \cdot x + b$! Namreč, nič ni narobe s funkcijskim predpisom, v katerem se
pojavijo dodatni simboli.

Morda pa lahko vezano spremenljivko in $\mapsto$ brez škode izpustimo, če v izrazu nastopa
samo en simbol, denimo $1 + x^2?$
%
A spet bi zabredli v težave. Je $42$ število ali funkcijski predpis $x \mapsto 42$? Je
$1 + x^2$ funkcijski predpis $x \mapsto 1 + x^2$ ali $a \mapsto 1 + x^2$?

Velikokrat površno rečemo, da funkcijski predpis podaja preslikavo. To ni res, saj smo že
prej povedali, da ima vsaka preslikava tri sestavne dele: domeno, kodomeno in prirejanje.
Res, če ne poznamo domene, ne moremo preveriti, ali je funkcijski predpis celovit. Denimo,
funkcijski predpis
%
\begin{equation*}
  x \mapsto \frac{x}{x^2 - 2}
\end{equation*}
%
ni celovit, če je domena množica realnih števil, in je celovit, če je domena množica
racionalnih števil. Tudi kodomeno moramo poznati, sicer ne moremo določiti nekaterih
lastnosti preslikave, kot je na primer surjektivnost, glej
razdelek~\ref{razdelek:injektivnost-in-surjektivnost}.



\subsection{Ostali načini podajanja preslikav}
\label{sec:ostali-predpisi}

Funkcijski predpisi niso edini način za podajanje prirejanja, zato omenimo še nekatere
druge.

Preslikavo s končno domeno lahko podamo s tabelo, na primer:
%
\begin{center}
  $f : \set{1, 2, 3, 5} \to \set{10, 20, 30}$

  \medskip

  \begin{tabular}{|c|c|} \hline
    1 & 10 \\ \hline
    2 & 10 \\ \hline
    3 & 20 \\ \hline
    5 & 10 \\ \hline
  \end{tabular}
\end{center}
%
To seveda pomeni, da $f$ elementu $1$ priredi $10$, $2$ priredi $10$, $3$ priredi $20$ in $5$
priredi $10$. Tabelo lahko predstavimo na različne načine, lahko kar naštejemo vsa prirejanja:
%
\begin{align*}
  f(1) &= 10 \\
  f(2) &= 10 \\
  f(3) &= 20 \\
  f(5) &= 10.
\end{align*}
%
Tudi
%
\begin{align*}
  1 &\mapsto 10 \\
  2 &\mapsto 10 \\
  3 &\mapsto 20 \\
  5 &\mapsto 10.
\end{align*}
%
je še vedno le tabela, ki prikazuje prirejanje. Ne sme nas motiti dejstvo, da smo
$\mapsto$ uporabili za naštevanje prirejanj, namesto za funkcijski predpis.

Preslikava je lahko določena tudi z opisom računskega postopka, pravimo mu \df{algoritem},
s pomočjo katerega izračunamo vrednost preslikave pri danem argumentu. Paziti moramo, da je
opis postopka res natančen in nedvoumen, lahko ga kar zapišemo kot program. Teoretični
računalničar bi pripomnil, da je treba pri tem izbrati programski jezik, ki ima ustrezno
matematično definicijo.

Preslikave lahko podamo tudi tako, da opišemo pogoje, pri katerih je element kodomene
prirejen elementu domene. Na primer, preslikavo $f : \NN \to \ZZ$ bi lahko definirali z
zahtevo, da naravnemu številu $n \in \NN$ priredimo celo število $k \in \ZZ$, kadar velja
%
\begin{equation*}
  k^2 \leq n < (k+1)^2.
\end{equation*}
%
To prirejanje je veljavno, če je celovito in enolično, česar ne bomo preverjali, lahko pa
poskusite sami. Nekaj prirejanj $f$ prikazuje naslednja razpredelnica:
%
\begin{align*}
0 &\mapsto 0   &   4 &\mapsto 2   &    8  &\mapsto 2   &   12 &\mapsto 3 \\
1 &\mapsto 1   &   5 &\mapsto 2   &    9  &\mapsto 3   &   13 &\mapsto 3 \\
2 &\mapsto 1   &   6 &\mapsto 2   &    10 &\mapsto 3   &   14 &\mapsto 3 \\
3 &\mapsto 1   &   7 &\mapsto 2   &    11 &\mapsto 3   &   15 &\mapsto 3
\end{align*}
%
Ali znate z besedami opisati preslikavo~$f$?

V splošnem je lahko preslikava podana s precej zapleteno konstrukcijo, ki zahteva veliko
preverjanja in dokazovanja. Osnovne načine podajanja preslikav bomo spoznali skupaj s
konstrukcijami množic.


\subsection{Aplikacija in substitucija}
\label{sec:aplikacija-in-subsitucija}

Do sedaj smo se ukvarjali s tem, kako preslikavo podamo, zdaj pa se vprašajmo, kako lahko
preslikavo uporabimo. Če je $f : X \to Y$ preslikava iz $X$ v $Y$ in je $x \in X$, potem
lahko \df{$f$ uporabimo na $x$} in dobimo \df{vrednost} preslikave~$f$ pri
\df{argumentu}~$x$, to je tisti edini element $Y$, ki ga~$f$ priredi~$x$. Vrednost $f$
pri~$x$ zapišemo
%
\begin{equation*}
  f(x)
  \qquad\text{ali}\qquad
  f\,x
\end{equation*}
%
in preberemo ``$f$ od $x$'' ali ``$f$ pri $x$''. Izraz $f(x)$, oziroma $f\,x$, se imenuje
\df{aplikacija}. Večinoma se uporablja zapis z oklepaji, a ne vedno: navajeni smo pisati
$\ln 2$ in $\sin \alpha$ namesto $\ln(2)$ in $\sin(\alpha)$. Oklepaje izpuščamo tudi v
nekaterih programskih jezikih in občasno v algebri.

V analizi je uveljavljen še en zapis za aplikacijo, ki se uporablja za zaporedja. Namreč,
zaporedje ni nič drugega kot preslikava $a : \NN \to \RR$ iz naravnih v realna števila.
Aplikacijo $a(n)$, ki označuje $n$-ti člen zaporedja, ponavadi pišemo~$a_n$, torej
argument podpišemo.

Preslikavo lahko uporabimo na argumentu tudi, če je nismo poimenovali. Na primer,
preslikavo $\RR \to \RR$, podano s funkcijskim predpisom
%
\begin{equation*}
  x \mapsto 1 + x^2
\end{equation*}
%
uporabimo na argumentu~$3$:
%
\begin{equation*}
  (x \mapsto 1 + x^2)(3).
\end{equation*}
%
Se vam zdi tak zapis nenavaden? Verjetno, a pomislite, zakaj je tako: ker običajno
preslikave poimenujemo in se nanje vedno sklicujemo z njihovim imenom. Prav nobenega
razloga ni, da ne bi s funkcijskimi predpisi delali tako, kot s števili, vektorji in
ostalimi matematičnimi objekti, na katere smo že navajeni. Računalničarji radi rečejo, da
je treba tudi preslikave obravnavati kot ``enakopravne državljane''. Prav imajo, zato
bomo vadili uporabo funkcijskih predpisov ter z njimi delali, kot da niso nič posebnega,
saj niso!

Kako pravzaprav določimo vrednost funkcije pri danem argumentu? To je odvisno od tega,
kako je podano prirejanje. Če imamo tabelarični prikaz, poiščemo argument v levem stolpcu
in pogledamo v desni stolpec. Če je preslikava podana s funkcijskim predpisom, argument
vstavimo v predpis. Na primer, če je $f : \RR \to \RR$ podana s funkcijskim predpisom
%
\begin{equation*}
  f(x) = 1 + x^2,
\end{equation*}
%
potem je vrednost $f(3)$ enaka $1 + 3^2$, kar je seveda enako~$10$, a to zahteva dodaten
račun, ki nas v tem trenutku ne zanima. Pravimo, da smo simbol~$x$ \df{zamenjali} ali
\df{substituirali} s~$3$, oziroma da smo~$3$ \df{vstavili} v~$f$ namesto~$x$. Seveda lahko
vstavimo argument neposredno v funkcijski predpis, zato je aplikacija
%
\begin{equation*}
  (x \mapsto 1 + x^2)(3)
\end{equation*}
%
seveda spet enaka $1 + 3^2$.

Preslikavo smemo uporabiti na poljubnem elementu domene, ki je lahko zapisan na bolj ali
manj zapleten način, pri čemer gre še vedno samo za zamenjavo. Na primer, v zgornjo
preslikavo~$f$ lahko vstavimo $3 + 4$ in dobimo $1 + (3 + 4)^2$ ali pa za neki $u \in \RR$
vstavimo $u + 2$ in dobimo $1 + (u + 2)^2$. V razdelku~\ref{sec:eksponent} bomo spoznali
še dodatna pravila za vstavljanje izrazov, ki se vrtijo okoli vezanih spremenljivk.


\subsection{Načelo ekstenzionalnosti preslikav}

Kot smo že omenili, je možih več pogledov na preslikave. Ali je pomembno, kako učinkovito
računamo vrednosti preslikave? Vsekakor, ampak ali naj to pomeni, da sta preslikavi
različni, če imata enake vrednosti, a je ena podana z učinkovitim pravilom in druga z
neučinkovitim? V matematiki je odgovor nikalen.

\begin{pravilo}[Ekstenzionalnost preslikav]
  Preslikavi sta enaki, če imata enaki domeni in kodomeni ter imata za vse argumente
  enaki vrednosti.
\end{pravilo}

Natančneje, če sta $f : A \to B$ in $g : C \to D$ preslikavi in velja $A = C$, $B = D$ ter
za vsak $x \in A$ velja $f(x) = g(x)$, tedaj velja $f = g$.

Takoj opozorimo na razliko med
%
\begin{equation*}
  f(x) = g(x)
  \qquad\text{in}\qquad
  f = g
\end{equation*}
%
saj bi marsikdo trdil, da med njima ni razlike. Levi izraz pravi, da sta $f(x)$ in $g(x)$
enaka elementa množice $C$, desni pa da sta $f$ in~$g$ enaki preslikavi iz $A$ v $B$. Na
sploh je treba razlikovati med $f$ in $f(x)$, saj to nikakor nista enaka objekta: prvi je
preslikava, drugi pa vrednost te preslikave pri~$x$. Verjetno nihče ne bi trdil, da je
preslikava $\cos$ isto kot $\cos \frac{\pi}{4}$, ali ne? Isti razmislek veleva, da
$\cos x$ ni isto kot $\cos$, če tudi si mislimo, da je $x$ poljuben. Zmeda izhaja iz
neprimernega zapisa preslikav. Če bi že od malih nog pravilno uporabljali funkcijske
predpise, bi seveda vedeli, da načelo ekstenzionalnosti za preslikave zagotavlja enakost
~$\cos$ in $x \mapsto \cos x$, oba pa sta različna od $\cos x$, ki sploh ni preslikava,
ampak neko realno število. Čeprav je število $\cos x$ odvisno od parametra~$x$, je še
vedno le število.

V bran tradicionalnemu zapisu pa moramo vseeno povedati, da se lahko \emph{dogovorimo} za
nekoliko napačen zapis, če to ne povzroča zmede. S tem se izognemu preveč birokratskemu
pisanju nebistvenih podrobnosti in lahko bistveno izboljšamo komunikacijo in razumevanje
med izkušenimi matematiki. A začetnikom priporočamo, da v dobrobit boljšega razumevanja
snovi vsaj na začetku študija raje vztrajajo pri doslednem zapisu.

Vrnimo se še k načelu ekstenzionalnosti preslikav. Ali ni pravzaprav očitno, da sta
preslikavi enaki, če imata enaki domeni, kodomeni in vrednosti? Morda res, a to ni razlog,
da tega ne bi eksplicitno zapisali. Vsak matematik vam ve povedati kako zgodbo o tem,
kako se je v dokazu skrivala napako ravno tam, kjer je bilo nekaj ``očitno''. Poleg tega
pa si lahko predstavljamo razmere, v katerih je smiselno razlikovati med dvema
preslikavama, ki imata vedno enake vrednosti, denimo v programiranju, kjer je učinkovitost
zelo pomembna.

%% STAR MATERIAL OD DAVORINA. Preveriti, kaj od tega je treba dati v besedilo, in kam.

% Množice ne obstajajo ločene ena od druge pač pa so med sabo povezane s
% \df{preslikavami} oziroma s tujko \df{funkcijami}.  Posamična preslikava slika elemente ene
% množice po določenem predpisu v elemente druge množice.

% Če je $f$ preslikava, ki slika iz množice $X$ v množico $Y$, to zapišemo
% %
% \begin{equation*}
%   f : X \to Y.
% \end{equation*}
% %
% Rečemo, da je množica~$X$ \df{začetna množica} ali \df{domena} preslikave~$f$, množica~$Y$
% pa je \df{ciljna množica} ali \df{kodomena} preslikave $f$.


% Običaj je, da predpis preslikave podamo s pomočjo spremenljivke, tipično z oznako $x$. Na primer, če je $f$ preslikava kvadriranja, njen predpis zapišemo kot
% \[f(x) = x^2.\]
% Na tem mestu je potrebno poudariti več reči.
% \begin{itemize}
% \item
% Velikokrat površno rečemo, da zgornji predpis podaja preslikavo. To ni povsem res --- to je zgolj predpis preslikave. Za to, da preslikavo v celoti podamo, je potrebno navesti tri stvari: poleg predpisa še domeno in kodomeno. Vse to je del informacije o preslikavi.

% To se jasno pokaže, če začnemo razmišljati o lastnostih preslikav. Se še spomnite iz srednje šole, kaj pomeni, da je preslikava surjektivna? (Bomo ponovili v razdelku~\ref{razdelek:injektivnost-in-surjektivnost}.) Če vzamemo, da preslikava $f$ zadošča zgornjemu predpisu in jo obravnavamo kot preslikavo $f\colon \RR \to \RR$, ni surjektivna, če jo obravnavamo recimo kot preslikavo $f\colon \RR_{\geq 0} \to \RR_{\geq 0}$, pa je.
% \item
% Za spremenljivko $x$ velja isto, kot smo razpravljali že v prejšnjem razdelku pri lastnostih elementov množic: spremenljivka $x$ nima vnaprej določene vrednosti, pač pa predstavlja mesto, kamor lahko vstavimo poljubno vrednost. Seveda je potem vseeno, če vzamemo kakšno drugo črko ali čisto drug simbol: $f(y) = y^2$ določa isti predpis kot $f(x) = x^2$; prav tako $f(\heartsuit) = \heartsuit^2$. Se pravi, tudi v tem primeru gre za nemo spremenljivko. Če si torej izberemo neko vrednost, jo lahko vstavimo na mesto spremenljivke in izračunamo vrednost dobljenega izraza, npr.~$f(3) = 3^2 = 9$ oziroma $f(2\pi) = (2\pi)^2 = 4\pi^2$. Predstavljajte si, da je spremenljivka pravzaprav škatlica, kamor lahko vstavite vrednost, torej
% \[f(\argbox) = \argbox^2.\]
% \item
% Alternativen način zapisa $f(x) = x^2$ je
% \[f\colon x \mapsto x^2.\]
% Pazimo: navadna puščica $\to$ podaja domeno in kodomeno, kot razloženo zgoraj. Repata puščica $\mapsto$ pa za posamičen element domene pove, v kateri element kodomene se preslika.

% Zapis z repato puščico je še posebej uporaben, kadar želimo podati preslikavo, ne da bi nam bilo potrebno izbrati ime zanjo. Na primer, realno funkcijo kvadriranja lahko v celoti podamo takole:
% \begin{align*}
% \RR &\to \RR \\
% x &\mapsto x^2
% \end{align*}
% (prva vrstica pove domeno in kodomeno, druga pa predpis). Tako podanim preslikavam potem rečemo \df{brezimne preslikave} (s tujko \df{anonimne funkcije}). Kasneje (v razdelku~\ref{razdelek:brezimne-preslikave}) bomo spoznali bolj strnjen zapis takih preslikav, ki je še posebej primeren za izvajanje operacij med preslikavami; takrat bomo takšno funkcijo zapisali kot $\lam{x \in \RR} x^2$.
% \end{itemize}

% \note{Sklop (kompozicija, kompozitum) preslikav. Identiteta kot enota za sklapljanje. Razčlenitev (dekompozicija, faktorizacija) preslikav.}

% \davorin{Definirati moramo tudi oznako $\set{f(x)}{x \in X}$, kar je druge vrste oznaka kot prej definirana $\set{x \in X}{\phi(x)}$. Se gremo primerjavo s Pythonom (razlika med \texttt{\{f(x) for x in X\}} in \texttt{\{x if phi(x)\}})? Smo matematični hipsterji in uvedemo oznako $\{f(x) \,|\, x \in X \,|\, \phi(x)\}$, ki ustreza \texttt{\{f(x) for x in X if phi(x)\}}, kar bi tudi prišlo prav?}

% Zaenkrat smo imeli primere, ko je bil prepis preslikave dan z eno samo spremenljivko, npr.~$f(x) = x^2$. Zelo pogoste so pa tudi \df{preslikave več spremenljivk}, npr.~$f(x, y) = x^2 + y^2$. Že osnovne računske operacije so take --- na primer, pri seštevanju vzamemo \emph{dva} podatka in vrnemo rezultat (vsoto).

% V takem primeru je smiselno reči: domena preslikave sestoji iz \df{dvojic} ali \df{parov} števil. Pri seštevanju je to, katero število je prvo, katero pa drugo, sicer nepomembno, pri kakšni drugi operaciji (npr.~že odštevanju), pa je, zato posebej zahtevajmo: gre za \df{urejene dvojice} (\df{pare}). Urejeno dvojico elementov $a$ in $b$ (v tem vrstem redu) po dogovoru zapišemo kot $(a, b)$. Vrednosti $a$ in $b$ imenujemo \df{komponenti} tega para; natančneje, $a$ je \df{prva komponenta}, $b$ pa \df{druga komponenta}.

% Če imamo dve množici $A$ in $B$, tedaj množico vseh urejenih dvojic, katerih prva komponenta je element iz $A$, druga komponenta pa element iz $B$, označimo $A \times B$ in imenujemo \df{zmnožek} ali \df{produkt} množic $A$ in $B$. Glede na to, da obstaja mnogo operacij, ki se imenujejo ``produkt'' (poznate že vsaj produkt števil, produkt števila z vektorjem, skalarni produkt vektorjev in vektorski produkt vektorjev, obstaja pa jih še precej več), je koristno produkt množic posebej poimenovati, da ga ločimo od drugih: zanj se je uveljavil izraz \df{kartezični produkt} (izhaja iz imena Cartesius, tj.~latinske različice priimka Renéja Descarta\footnote{René Descartes (1596 -- 1650) je bil francoski filozof, matematik in znanstvenik.}).

% Seštevanje potemtakem lahko razumemo kot preslikavo $+\colon \RR \times \RR \to \RR$. V tem smislu še vedno gre za preslikavo, ki dan vhodni podatek preslika v neki rezultat, le da je vhodni podatek dvojica števil, ne pa zgolj eno število. Kadar imamo produkt več enakih faktorjev, ga lahko (kot običajno) zapišemo v obliki potence; pisali bi lahko tudi $+\colon \RR^2 \to \RR$.

% Seveda nismo omejeni na preslikave samo ene ali dveh spremenljivk. Nič nam ne preprečuje definirati recimo $f(x, y, z) = 2x + y - 3z$. Smiselna domena te preslikave setoji iz \df{urejenih trojic} števil. V splošnem, če jemljemo elemente iz množic $A$, $B$, $C$, tedaj se množica vseh takih trojic označi z $A \times B \times C$. Prejšnji predpis določa potem preslikavo $f\colon \RR \times \RR \times \RR \to \RR$ (oziroma krajše $f\colon \RR^3 \to \RR$).

% Spremenljivk je lahko še več; poleg dvojic in trojic tako dobimo še četverice, peterice, šesterice\ldots V splošnem takšna končna zaporedja elementov imenujemo \df{urejene večterice}. Tudi število spremenljivk je lahko označeno s črko; na primer, preslikava, ki računa povprečje $n$ števil (kjer $n \in \NN_{\geq 1}$), je dana kot
% \begin{align*}
% \RR^n &\to \RR \\
% (x_1, x_2, \ldots, x_n) &\mapsto \frac{x_1 + x_2 + \ldots + x_n}{n}
% \end{align*}
% (če hočemo poudariti, da imajo naše večterice natanko $n$ komponent, jih imenujemo $n$-terice). Nadlega pri tem je sicer spet dvoumnost tropičja. Deloma jo je možno odpraviti tako, da celotno večterico označimo z eno spremenljivko. Pogosta izbira zapisa je $f(\mathbf{x})$ ali $f(\vec{x})$ (razlog za to je, da lahko večterico vidimo kot vektor).

% Marsikdaj želimo delati ne samo z eno preslikavo, pač pa s celo množico preslikav naenkrat. Zato uvedemo: množica vseh preslikav, ki slikajo iz $X$ v $Y$, se označi kot $Y^X$; temu se reče \df{eksponent} množic $X$ in $Y$ (\note{na primernem mestu kasneje} bomo razložili, od kod ta oznaka).

% \begin{zgled}
% Množico vseh preslikav, ki realna števila slikajo nazaj v realna števila, označimo z $\RR^\RR$. Če nas zanimajo realne preslikave, ki so definirana samo na intervalu $\intoo{-1}{1}$, opazujemo množico $\RR^{\intoo{-1}{1}}$. Definiramo lahko preslikavo
% \begin{align*}
% \RR^{\intoo{-1}{1}} &\to \RR \\
% f &\mapsto f(0),
% \end{align*}
% ki preslikavam priredi njihovo vrednost v točki $0$. Ta preslikava torej ima za argumente (tj.~vnose) celotne preslikave in ne števila! Sama po sebi je element množice $\RR^{\RR^{\intoo{-1}{1}}}$.
% \end{zgled}

% \begin{zgled}
% Za poljubne množice $A$, $B$, $C$ lahko definiramo sledečo preslikavo, katere argumenti so pari preslikav.
% \begin{align*}
% B^A \times C^B &\to C^A \\
% (f, g) &\mapsto g \circ f
% \end{align*}
% \end{zgled}


% \davorin{Glede na to, da gre za slovenski učbenik, dajem izrazu `preslikava' prednost pred izrazom `funkcija'. Seveda pa sem pojasnil tudi slednji izraz (v prvem poglavju).}

% \note{Uvod. Definicijsko območje in zaloga vrednosti \davorin{morda dodamo kot možno ime za zalogo vrednosti še prevod angleške besede `range', se pravi `razpon'?}. Zožitve (tako domene kot kodomene); oznake za to so $\rstr{f}_A$, $\rstr{f}^B$, $\rstr{f}_A^B$. Izvrednotenje (evalvacija) preslikave (če ne bomo tega pojasnili že pri eksponentih množic).}


\section{Zmnožek}
\label{sec:zmnozek}

Množice lahko \df{tvorimo} ali \df{konstruiramo} iz drugih množic na različne načine. V
tem poglavju bomo spoznali tri osnovne konstrukcije, ostale pa kasneje, ko bomo že nekaj
vedeli o logiki. Najprej obravnavajmo zmnožek ali kartezični produkt.

Takoj se zastavi vprašanje, kako sploh opisati novo konstrukcijo množic. Načelo
ekstenzionalnosti pove, da je množica opredeljena s svojimi elementi. Torej moramo
pojasniti, kaj so elementi nove množice, se pravi, kako jih vpeljemo, kaj lahko z njimi
počnemo in kakšne so njihove zakonitosti. Natančneje, novo konstrukcijo množic
določajo naslednja pravila:
%
\begin{enumerate}
\item pravilo \df{tvorbe}, ki vpelje novo množico,
\item pravila \df{vpeljave} elementov, ki podajo operacije, s katerimi gradimo elemente,
\item pravila \df{uporabe}, ki podajo opreacije, s katerimi razgradimo ali uporabimo elemente,
\item \df{enačbe}, ki opredeljujejo zakonitosti, ki veljajo za operacije vpeljave in uporabe.
\end{enumerate}
%
Najbolje je, da si postopek ogledamo na primeru.

\begin{pravilo}[Tvorba zmnožka]
  \label{pravilo:zmnozek-tvorba}
  Za vsaki množici $A$ in $B$ je $A \times B$ množica, ki se imenuje \df{zmnožek} ali
  \df{kartezični produkt} $A$ in $B$.
\end{pravilo}

\noindent
%
Pravilo tvorbe pove, da lahko tvorimo novo množico $A \times B$, ne pove pa, kakšne
elemente ima. To je vsebina naslednjih dveh pravil, ki povesta, kako sestavimo in
razstavimo elemente zmnožka.

\begin{pravilo}[Vpeljava urejenih parov]
  \label{pravilo:zmnozek-vpeljava}
  %
  Za vse $a \in A$ in $b \in B$ je $(a, b) \in A \times B$. Element $(a, b)$ imenujemo
  \df{urejeni par}.
\end{pravilo}

\begin{pravilo}[Uporaba urejenih parov]
  \label{pravilo:zmnozek-uporaba}
    %
  Za vsak $p \in A \times B$ je $\fst(p) \in A$ \df{prva projekcija} in $\snd(p) \in B$
  \df{druga projekcija} elementa~$p$.
\end{pravilo}

Nazadnje podamo še enačbe.

\begin{pravilo}[Računsko pravilo za urejene pare]
  \label{pravilo:zmnozek-racunanje}
  Za vse $a \in A$, $b \in B$ velja $\fst(a, b) = a$ in $\snd(a, b) = b$.
\end{pravilo}

\begin{pravilo}[Ekstenzionalnost urejenih parov]
  \label{pravilo:zmnozek-ekstenzionalnost}
  Za vse $p, q \in A \times B$ velja: če $\fst(p) = \fst(q)$ in $\snd(p) = \snd(q)$,
  potem $p = q$.
\end{pravilo}

\noindent
%
Računsko pravilo se tako imenuje, ker lahko z njim poenostavljamo izraze, drugo pa je
načelo ekstenzionalnosti, ker pravi, da je urejeni par določen s prvo in drugo projekcijo.

Kadar imamo opravka z večimi zmnožki, na primer $A \times B$ in $C \times D$, bi lahko
prišlo do zmede glede projekcij. Takrat jih opremimo še z dodatnimi oznakami množic, da
razločimo projekciji $\fst[A][B] : A \times B \to A$ in $\fst[C][D] : C \times D \to C$,
in podobno za~$\snd$.

Malo bolj naivna konstrukcija zmnožka bi se glasila takole: kartezični produkt
$A \times B$ je množica vseh urejenih parov $(a, b)$, kjer je $a \in A$ in $b \in B$. A
taka konstrukcija ni popolna, saj ne pove, kaj lahko z urejenim parom počnemo. Kako naj
vemo, da iz $(a, b)$ lahko izluščimo $a$ in $b$, in kako preverimo, ali sta dva urejena
para enaka? Če takih zadev ne določimo, bi lahko kdo mislil, da je urejeni par kaka druga
operacija, denimo seštevanje, unija, ali kdovekaj.

Dejstvo, da je vsak element zmnožka množic urejen par, in to celo na en sam način, lahko
dokažemo.

\begin{trditev}
  Naj bosta $A$ in $B$ množici. Za vsak element $p \in A \times B$ obstaja natanko en
  $a \in A$ in natanko en $b \in B$, da velja $p = (a, b)$.
\end{trditev}

\begin{proof}
  Naj bosta $A$ in $B$ množici in $p \in A \times B$. Najprej pokažimo, da $p$ res je enak
  nekemu urejenemu paru, namreč
  %
  \begin{equation*}
    p = (\fst(p), \snd(p)).
  \end{equation*}
  %
  Uporabimo načelo ekstenzionalnosti za pare, ki nam zagotavlja to enačbo, če dokažemo
  %
  \begin{equation*}
    \fst(p) = \fst(\fst(p), \snd(p))
    \qquad\text{in}\qquad
    \snd(p) = \snd(\fst(p), \snd(p)).
  \end{equation*}
  %
  Ti dve enačbi pa veljata, ker sta primerka računskih pravil za pare.

  Preveriti moramo še, da je $(\fst(p), \snd(p))$ edini urejeni par, ki je enak~$p$.
  Povedano z drugimi besedami, dokazati moramo: če je $p = (a, b)$ za neki $a \in A$ in
  $b \in B$, potem velja $a = \fst(p)$ in $b = \snd(p)$. Pa denimo, da bi za neki
  $a \in A$ in $B \in B$ veljalo $p = (a,b)$. Tedaj bi lahko uporabili računska pravila za
  pare in dobili
  %
  \begin{equation*}
    \fst(p) = \fst(a, b) = a
    \qquad\text{in}\qquad
    \snd(p) = \snd(a, b) = b,
  \end{equation*}
  %
  kar smo želeli dokazati.
\end{proof}

Trditev je prikladna, ko želimo podati funkcijsko pravilo za preslikavo, katere domena je
zmnožek množic. Primer take preslikave je
%
\begin{gather*}
  \RR \times \RR \to \RR \\
  p \mapsto \fst(p) + \snd(p)^2 \cdot \fst(p).
\end{gather*}
%
Ta zapis je precej nepregleden, a sledili smo navodilu, da mora stati na levi strani
funkcijskega predpisa simbol. Prejšnja trditev nam zagotavlja, da lahko vsak element
$\RR \times \RR$ na en sam način izrazimo kot urejeni par $(x, y)$, in zato ne bo nič
narobe, če zapišemo ta isti funkcijski predpis bolj pregledno tako, da upoštevamo, da
je $p$ enak $(x, y)$ za enolično določena $x$ in $y$:
%
\begin{gather*}
  \RR \times \RR \to \RR \\
  (x, y) \mapsto x + y^2 \cdot x.
\end{gather*}
%
Če bi funkcijo poimenovali, denimo $f$, bi dobili običajni zapis:
%
\begin{gather*}
  f : \RR \times \RR \to \RR \\
  f(x, y) = x + y^2 \cdot x.
\end{gather*}
%
Za tako preslikavo pravimo, da je ``funkcija dveh spremenljivk'', ker si mislimo, da smo
podali argumenta $x$ in $y$ ločeno drug od drugega. Tu pravzaprav vidimo, da bi lahko
rekli tudi, da je funkcija dveh spremenljivk pravzaprav običajna funkcija, katere
arugmenti so urejeni pari.

Poleg zmnožka dveh množic bi lahko tvorili tudi zmnožek treh ali več množic. Pravila bodo
podobna kot za zmnožek dveh množic, le da bi namesto urejenih parov tvorili \df{urejene
  večterice} in da bi imeli več projekcij. Za vsako projekcijo bi zapisali eno računsko
pravilo, princip ekstenzionalnosti pa bi bil tudi podoben tistemu za urejene pare.
Podorobnosti prepustimo za vajo.


\section{Vsota}
\label{sec:vsota}

Spoznali smo že unijo $A \cup B$ množic $A$ in $B$, ki vsebuje tiste elemente, ki so v $A$
ali v $B$. Če imata $A$ in $B$ skupne elemente, bodo ti v uniji seveda nastopili samo
enkrat. V skranjem primeru dobimo $A \cup A = A$. Včasih pa želimo združiti množici tako,
da ne pride do prekrivanja. Taka konstrukcija je \df{vsota} $A + B$ množic $A$ in $B$.
Prekrivanje preprečimo tako, da elemente, ki jih je prispevala~$A$ označimo z eno oznako,
tiste, ki jih je prispevala~$B$, pa z drugo.

\begin{pravilo}[Vsota]
  \label{vsota:tvorba}
  Za vsaki množici $A$ in $B$ je $A + B$ množica, ki se imenuje \df{vsota} ali
  \df{koprodukt} množic $A$ in $B$.
\end{pravilo}

\begin{pravilo}[Vpeljava elementov vsote]
  \label{vsota:vpeljava}
  Za vsaki množici $A$ in $B$ velja:
  %
  \begin{enumerate}
  \item za vsak $a \in A$ je $\inl(a) \in A + B$,
  \item za vsak $b \in B$ je $\inr(b) \in A + B$.
  \end{enumerate}
\end{pravilo}

S pravilom vpeljave smo pojasnili, da uporabljamo oznaki $\inl$ in $\inr$, prvo za
elemente iz~$A$ in drugo za elemente iz~$B$. Oznakama pravimo tudi
\df{injekciji}\footnote{Pravzaprav niti ni pomembno, kako poimenujemo oznaki, da sta le
  različni. V funkcijskem programiranju, kjer poznamo vsote podatkovnih tipov, programer
  sam določi, kakšne oznake bo uporabljal za injekcije.} in sta preslikavi
%
\begin{equation*}
  \iota_1 : A \to A + B
  \qquad\text{and}\qquad
  \iota_2 : B \to A + B.
\end{equation*}
%
Kadar imamo opravka z večimi vsotami, na primer $A + B$ in $C + D$, bi lahko prišlo do
zmede glede oznak. Takrat injekcije opremimo še z dodatnimi oznakami množic, da razločimo
injekciji $\inl[A][B] : A \to A + B$ in $\inl[C][D] : C \to C + D$, in podobno za~$\inr$.

Potrebujemo še pravili za uporabo in enakost elementov vsote, ki ju združimo v eno samo
pravilo.

\begin{pravilo}
  \label{vsota:uporaba}
  Za vsaki množici $A$ in $B$ in za vsak $u \in A + B$, bodisi obstaja natanko en
  $a \in A$, da je $u = \inl(a)$, bodisi obstaja natanko en $b \in B$, da je
  $u = \inr(b)$.
\end{pravilo}

Fraza ``bodisi \dots bodisi'' pomeni, da je vsak element $u \in A + B$ enak $\inl(a)$
za natanko en $a \in A$ ali $\inr(b)$ za natanko en $b \in B$, ne more pa se zgoditi oboje
hkrati ali nič od tega. Torej $\inl(a) = \inr(b)$ ne drži in celo v primeru, ko je
$A = B$ in $a = b$, je $\inl(a) \neq \inr(b)$. S tem smo v $A + B$ res ločili elemente $A$
od elementov $B$.
%
Fraza ``natanko en'' pove, da iz $u = \inl(a_1)$ in $u = \inl(a_2)$ sledi $a_1 = a_2$.
Povedano drugače, če velja $\inl(a_1) = \inl(a_2)$, potem je $a_1 = a_2$. Podobno iz
$\inr(b_1) = \inr(b_2)$ sledi $b_1 = b_2$.
%
Podajmo prepost primer, ki verjetno marsikaj pojasni:
%
\begin{equation*}
  \set{a, b, c} + \set{a, d, e} =
  \set{\inl(a), \inl(b), \inl(c), \inr(a), \inr(d), \inr(e)}.
\end{equation*}

Kako definiramo preslikavo $A + B \to C$? Ker je vsak element domene $A + B$ bodisi
$\inl(a)$ za neki $a \in A$ bodisi $\inr(b)$ za neki $b \in B$, \emph{obravnavamo oba
  primera}. Tako funkcijski zapis za preslikavo $A + B \to C$ zapišemo kot
%
\begin{equation*}
  u \mapsto
  \begin{cases}
    \cdots a \cdots & \text{če $u = \inl(a)$,}\\
    \cdots b \cdots & \text{če $u = \inr(b)$,}
  \end{cases}
\end{equation*}
%
kjer smemo v $\cdots a \cdots$ zapisati izraz, ki vsebuje simbol~$a$, in v
$\cdots b \cdots$ izraz, ki vsebuje simbol~$b$. Ker je tak zapis nekoliko neroden, se
dogovorimo, da ga lahko zapišemo tudi s \emph{večdelnim} funkcijskim predpisom:
%
\begin{align*}
  \inl(a) &\mapsto \cdots a \cdots, \\
  \inr(b) &\mapsto \cdots b \cdots.
\end{align*}
%
Če želimo preslikavo poimenovati, zapišemo
%
\begin{align*}
  f &: A + B \to C, \\
  f(\inl(a)) &= \cdots a \cdots \\
  f(\inr(b)) &= \cdots b \cdots.
\end{align*}
%
Vsi ti zapisi res določajo celovito in enolično prirejanje, saj nam pravila za vsoto
zagotavljajo, da vedno obvelja natanko en primer. Na sploh lahko podamo funkcijski zapis z
večimi primeri, če le pazimo, da obravnavamo vse možnosti, in da se le-te ne prekrivajo.
Na primer, predpis
%
\begin{align*}
  (A + B) \times C &\to B + A \\
  (\inl[A][B](a), c) &\mapsto \inr[B][A](a) \\
  (\inr[A][B](b), c) &\mapsto \inl[B][A](b)
\end{align*}
%
je celovit in enoličen, medtem ko predpis
%
\begin{align*}
  (A \times A) + B &\to A \\
  \inl(a_1, a_2) &\mapsto a_2
\end{align*}
%
ni veljaven, ker ni celovit, saj manjka primer $\inr(b) \mapsto \cdots$.

Poleg vsote dveh množic bi lahko tvorili vsoto treh ali več množic. Pravila bi bila
podobna, le da bi imeli več injekcij in več primerov.

\section{Eksponent}
\label{sec:eksponent}

Denimo, da sta $A$ in $B$ množici. Tedaj lahko obravnavamo preslikave
%
\begin{equation*}
  A \to B
\end{equation*}
%
z domeno $A$ in kodomeno $B$. Ali vse take preslikave tvorijo množico? Russellov paradoks
nas je izučil, da moramo pazljivo postaviti pravila za konstrukcije množic, nato pa jih
strogo držati. Pravila, ki smo jih podali do sedaj, ne zagotavljajo knostrukcije množic
vseh preslikav iz $A$ v $B$. Potrebujemo novo pravilo.

\begin{pravilo}[Eksponent]
  Za vsaki množici $A$ in $B$ ima \df{eksponent} ali \df{eksponentna množica $B^A$} za
  elemente natanko vse preslikave iz~$A$ v~$B$.
\end{pravilo}

Potemtakem je zapis $f : A \to B$ enakovreden zapisu $f \in B^A$.

Pravila, ki opredeljujejo elemente množice $B^A$ smo že spoznali. Pravilo vpeljave pravi,
da je preslikava podana z domeno, kodomeno ter celovitim in enoličnim prirejanjem med
njima. Pravilo uporabe je kar aplikacija: če je $f \in B^A$ in $a \in A$, lahko tvorimo
$f(a) \in B$. Tudi računsko pravilo za preslikave smo že spoznali, saj je to kar pravilo
zamenjave: funkcijski predpis uporabimo na argumentu tako, da vezano spremenljivko v
predpisu zamenjamo z argumentom. In ekstenzionalnost preslikav pove, kdaj sta dve
preslikavi enaki.

Preslikavi, ki sprejme kot argument preslikavo, pravimo \df{funkcional} ali \df{preslikava
  višjega reda}. Primer take preslikave je \df{kompozicija}:
%
\begin{align*}
  {\circ} &: C^B \times B^A \to C^A \\
  {\circ} &: (g, f) \mapsto (x \mapsto g(f(x))).
\end{align*}
%
Pišemo jo kot operacijo, torej $g \circ f$ namesto ${\circ}(g, f)$. V zgornjem zapisu smo
uporabili eksponente, a v tem primeru je bolj pregleden diagram:
%
\begin{equation*}
  \xymatrix{
    {A}
    \ar[r]^{f}
    \ar@/_2ex/[rr]_{g \circ f}
    &
    {B}
    \ar[r]^{g}
    &
    {C}
  }
\end{equation*}
%
Zakaj smo $\circ$ definirali tako, da kompozicijo $f$ in $g$ pišemo $g \circ f$ namesto
$f \circ g$? Ker si je mnogo lažje zapomniti računsko pravilo
%
\begin{equation*}
  (g \circ f)(x) = g(f(x)),
\end{equation*}
%
ki velja z našo definicijo, kot pa $(f \circ g)(x) = g(f(x))$, kar bi veljalo, če bi
zamenjali vlogi~$f$ in~$g$.

\begin{trditev}
  \parbox{0pt}{}
  %
  \begin{enumerate}
  \item Identiteta je nevtralna za kompozicijo: $\id[B] \circ f = f = f \circ \id[A]$.
  \item Kompozicija je asociativna: $(h \circ g) \circ f = h \circ (g \circ f)$.
  \end{enumerate}
\end{trditev}

\begin{proof}
  Trditev je zapisana pomanjkljivo, saj ne piše, kaj so $A$, $B$,$ f$ in~$g$. Avtorja
  trditve bi lahko vprašali, kaj je hotel povedati, a je bolje, da poskusimo to razvozlati
  sami, ker je to odlična vaja iz razumevanja matematičnih besedil.

  Takoj vidimo, da je $A$ množica, sicer zapis $\id[A]$ ne bi bil smiselen, in podobno je
  tudi $B$ množica. Simboli $f$, $g$ in $h$ zagotovo označujejo preslikave, saj nastopajo
  v kompoziciji. Kaj pa njihove domene in kodomene? Preslikava $f$ mora imeti domeno $A$,
  sicer ne bi bilo dovoljeno komponirati $f \circ \id[A]$, in mora imeti kodomeno $B$,
  sicer ne bi bilo dovoljeno komponirati $\id[B] \circ f$. Ostaneta še domeni in kodomeni
  preslikav $g$ in~$h$. Kompozicija $g \circ f$ kaže, da mora biti domena $g$ enaka
  kodomeni~$f$, torej $B$. Kompozicija $h \circ g$ pa pove, da je kodomena $C$ enaka
  domeni $h$. Če vse to zložimo v diagram, dobimo
  %
  \begin{equation*}
    \xymatrix{
      {A} \ar[r]^{f}
      &
      {B} \ar[r]^{g}
      &
      {\text{?}} \ar[r]^{h}
      &
      {\text{?}}
    }
  \end{equation*}
  %
  Trditev moramo razumeti tako, da bo čim bolj splošna in smiselna. Torej bomo za neznani
  množici vzeli kar poljubni množici $C$ in $D$:
  %
  \begin{equation*}
    \xymatrix{
      {A} \ar[r]^{f}
      &
      {B} \ar[r]^{g}
      &
      {C} \ar[r]^{h}
      &
      {D}
    }
  \end{equation*}
  %
  Preverimo, ali smo trditev pravilno razumeli. Ko vstavimo podrobnosti, se prvi del
  glasi: ``Za vse množice $A$ in $B$ ter preslikavo $f : A \to B$ velja
  $\id[B] \circ f = f = f \circ \id[A]$.''
  %
  Ker je to smiselna izjava, jo dokažimo. Enakost preslikav se dokaže z ekstenzionalnostjo
  preslikav, torej preverimo, ali imajo $\id[B] \circ f$, $f$ in $f \circ \id[A]$ enako
  vrednost za poljuben $x \in A$:
  %
  \begin{align*}
    (\id[B] \circ f)(x) &= \id[B] (f (x)) = f (x), \\
    f (x) &= f(x), \\
    (f \circ \id[A])(x) &= f (\id[A](x)) = f(x).\\
  \end{align*}
  %
  Zapišimo podrobno še drugi del: ``Za vse množice $A$, $B$, $C$ in $D$ ter preslikave
  $f : A \to B$, $g : B \to C$ in $h : C \to D$ velja
  $(h \circ g) \circ f = h \circ (g \circ f)$. To spet dokažemo tako, da uporabimo levo in
  desno stran enačbe na poljubnem $x \in A$:
  %
  \begin{align*}
    ((h \circ g) \circ f)(x) &= (h \circ g)(f(x)) = h(g(f(x))) \\
    (h \circ (g \circ f))(x) &= h((g \circ f)(x)) = h(g(f(x))). \qedhere
  \end{align*}
\end{proof}


Kompozicijo smo zapisali z \emph{vgnezdenim} funkcijskim predpisom, ki argumentu priredi
preslikavo, ki je spet podana s funkcijskim predpisom. V splošnem je vgnezdeni funkcijski
predpis oblike
%
\begin{align*}
  A &\mapsto C^B \\
  a &\mapsto (b \mapsto \cdots),
\end{align*}
%
kjer se lahko v $\cdots$ pojavita~$a$ in~$b$. Na tak zapis se je treba navaditi, a je zelo
prikladen, še posebej v funkcijskem programiranju. V matematiki ni zelo pogost, a mi se ga
ne bomo bali.

Pri računanju s preslikavami višjega reda včasih hkrati obravnavamo več funkicjskih
predpisov in lahko pride do zmede, če za vse uporabimo isto vezano spremenljivko. Na
primer, kompozitum preslikav
%
\begin{equation*}
  \begin{aligned}
    \RR &\to \RR \\
    x &\mapsto x^2 - 4
  \end{aligned}
  %
  \qquad\text{in}\qquad
  %
  \begin{aligned}
    \RR &\to \RR \\
    x &\mapsto 2 - x
  \end{aligned}
\end{equation*}
%
bi lahko izračunali takole:
%
\begin{align*}
  (x \mapsto x^2  - 4) \circ (x \mapsto 2 - x)
  &= (x \mapsto (x \mapsto x^2 - 4) ((x \mapsto 2 - x) x)) \\
  &= (x \mapsto (x \mapsto x^2 - 4) (2 - x)) \\
  &= (x \mapsto (2 - x)^2 - 4) \\
  &= (x \mapsto x^2 - 4 x).
\end{align*}
%
Tu imamo tri pojavitve $x$, ki bi jih morali ločiti, ker vsaka nastopa kot vezana
spremenljivka v svojem funkcijskem predpisu. Še posebej nejasen je računski korak
$(x \mapsto (x \mapsto x^2 - 4) (2 - x)) = (x \mapsto (2 - x)^2 - 4)$, ko vezano
spremenljivko~$x$ v funkcijskem predpisu zamenjamo z izrazom $2 - x$, ki tudi vsebuje~$x$.
To sta dva različna $x$-a! Spomnimo se, da lahko vezane spremenljivke vedno preminujemo.
Ponovimo račun, a tokrat tako, da imajo različni funkcijski predpisi različne vezane
spremenljivke. Kompozitum
%
\begin{equation*}
  \begin{aligned}
    \RR &\to \RR \\
    y &\mapsto y^2 - 4
  \end{aligned}
  %
  \qquad\text{in}\qquad
  %
  \begin{aligned}
    \RR &\to \RR \\
    z &\mapsto 2 - z
  \end{aligned}
\end{equation*}
%
izračunamo takole:
%
\begin{align*}
  (y \mapsto y^2  - 4) \circ (z \mapsto 2 - z)
  &= (x \mapsto (y \mapsto y^2 - 4) ((z \mapsto 2 - z) x)) \\
  &= (x \mapsto (y \mapsto y^2 - 4) (2 - x)) \\
  &= (x \mapsto (2 - x)^2 - 4) \\
  &= (x \mapsto x^2 - 4 x).
\end{align*}
%
To je dosti bolj pregledno. Da ne bo prihajalo do zapletov z vezanimi spremenljivkami, se
dogovorimo: \emph{kadar imamo opravka z večimi vezanimi spremenljivkami, jih vedno
  preimenujemo tako, da so med seboj različne.}

Funkcionale srečamo v analizi in funkcijskem programiranju. Limita zaporedja je
funkcional, ker sprejme kot argument zaporedje realnih števil, se pravi element $\RR^\NN$,
in mu priredi realno število. Odvod je funkcional, ki sprejme element $\RR^\RR$ in mu
priredi element $\RR^\RR$. Če smo povsem natančni, limita kot preslikava $\RR^\NN \to \RR$
ni celovit funkcional, ker nekatera zaporedja ne konvergirajo. Prav tako odvod kot
preslikave $\RR^\RR \to \RR^\RR$ ni celovit, ker nekatere preslikave niso odvedljive.
Preslikavam, ki niso celovite, pravimo \emph{delne} in o njih bomo več povedali v
razdelku~\ref{delne-preslikve}.

\section{Izomorfizem množic}
\label{sec:izomorfizem-mnozic}

Ko otrok prvič spozna pojem števila, je ta zanimiv sam po sebi. Z vnemo šteje do sto in se
rad pogovarja se o tem, koliko je en miljon. Sčasoma se radovednost osredotoči na
aritmetične operacije in, če ima mladenič ali mladenka v sebi matematično žilico, na
\emph{zakonitosti} števil: množenje z~$1$ nima učinka, vrstni red seštevanja ni pomemben
itd. Ali tudi operacijam na množicah, ki smo jih spoznali do sedaj, vladajo kakšne
podobne zakonitosti?

Za števili $a$ in $b$ velja $a \cdot b = b \cdot a$. Nekaj podobnega velja tudi za množici
$A$ in $B$ in njuna zmnožka $A \times B$ in $B \times A$. V splošnem sicer nista enaka, a
sta v nekem smislu enakovredna, ker lahko par $(x, y) \in A \times B$ pretvorimo v par
$(y, x) \in B \times A$ in obratno. Ta razmislek vodi do pojma izomorfizma.

% Pojasnilo: izomorfnost $A \ism B$ je struktura, ki je naravno podana z dvema
% preslikavama $A \to B$ in $B \to A$ ter dvema enačbama med njima. Zato tu zapišemo
% definicijo, ki hkrati uvede vse te pojme.

\begin{definicija}
  Množici $A$ in $B$ sta \df{izomorfni} in pišemo $A \ism B$, kadar obstajata preslikavi
  %
  \begin{equation*}
    f : A \to B
    \qquad\text{in}\qquad
    g : B \to A,
  \end{equation*}
  %
  za kateri velja
  %
  \begin{equation*}
    g \circ f = \id[A]
    \qquad\text{in}\qquad
    f \circ g = \id[B].
  \end{equation*}
  %
  Pravimo, da je~$f$ \df{izomorfizem} med~$A$ in~$B$ in da je~$g$ \df{inverz} ali
  \df{obrat}~$f$.
\end{definicija}

Preverimo, da velja $A \times B \cong B \times A$ za poljubni množici $A$ in $B$. To
storimo tako, da zapišemo preslikavi med zmnožkoma in preverimo, da tvorita
izomorfizem:\footnote{Držimo se pravila, da nikoli ne uporabimo iste vezane spremenljivke
  dvakrat, zato pravilo za $f$ zapišemo z $x$ in $y$ in pravilo za $g$ z $v$ in $u$.
  Marsikdo bi oba funkcisjka predpisa zapisal z $x$ in $y$, torej
  $f : (x, y) \mapsto (y, x)$ in $g : (y, x) \mapsto (x, y)$. To zmede nekatere študente,
  ker mislijo, da ``sta je $x$ v definiciji $f$ isti kot v definiciji $g$'', karkoli že
  naj bi to pomenilo. Poudarimo še enkat: vezana spremenljivka v funkcijskem predpisu nima
  nikakršne zveze z nobeno drugo pojavitvijo iste spremenljivke kje druge.}
%
\begin{align*}
  f &: A \times B \to B \times A &
  g &: B \times A \to A \times B \\
  f &: (x, y) \mapsto (y, x) &
  g &: (v, u) \mapsto (u, v).
\end{align*}
%
Treba je preveriti, da velja $g \circ f = \id[A \times B]$ in
$f \circ g = \id[B \times A]$. To naredimo z uporabo ekstenzionalnosti preslikav, ki pravi
da $g \circ f = \id[A]$ velja, če velja $(g \circ f)(a,b) = \id[A](a,b)$ za vse $a \in A$
in $b \in B$, in podobno za $f \circ g$. Obravnavajmo torej poljubna $a \in A$ in
$b \in B$ in izračunajmo:
%
\begin{equation*}
  (g \circ f)(a, b) =
  g (f (a, b)) = g (b, a) = (a, b).
\end{equation*}
%
Na podoben način preverimo $f \circ g = \id[B \times A]$.

\begin{zgled}\label{zgled:logaritmiranje-je-obratno-od-eksponenciranja}
  Primere izmorfizmov poznamo že iz srednje šole. Naj bo $\RR$ množica vseh realnih števil
  (glej razdelek~\ref{sec:realna-stevila}) in $\RR_{>0}$ množica vseh pozitivnih realnih
  števil. Tedaj logaritem in eksponentna funkcija,
  %
  \begin{equation*}
    \log : \RR_{>0} \to \RR
    \qquad\text{in}\qquad
    \exp : \RR \to \RR_{>0}
  \end{equation*}
  %
  tvorita izomorfizem, saj za $x \in \RR$ velja $\log (\exp x) = x$ in za $y \in \RR_{>0}$
  velja $\exp (\log y) = y$. Eksponentna funkcija seštevanje slika v množenje:
  $\exp 0 = 1$ in $\exp (x + y) = \exp x \cdot \exp y$, zato ni samo izomorfizem med
  množicama, ampak celo izomorfizem med grupama $(\RR, {+}, 0)$ in
  $(\RR_{>0}, {\cdot}, 1)$.

  Če ne veste, kaj je grupa in izomorfizem grup, nikar ne obupavajte. Vsak matematik se v
  vsakdanjem delu nenehno srečuje z neznanimi pojmi. Veste, da je znameniti profesor
  France Križanič\footnote{France Križanič (1928--2002), slovenski matematik} v enega od
  svojih učbenikov zapisal, da naj tisti, ki mu je branje dokazov odveč, ravna tako kot Du
  Fu:\footnote{Du Fu (712--770 pr.~n.~š), kitajski pesnik}
  %
  \begin{center}
    \begin{tabular}{l}
      Ko berem knjige,\\
      z vinom se krepčam\\
      in znak preskočim,\\
      če ga ne poznam.
    \end{tabular}
  \end{center}
\end{zgled}

\begin{naloga}
  Odkorakajte v knjižnico, izposodite si knjigo profesorja Križaniča in jo preberite.
\end{naloga}

Dokažimo nekaj osnovih lastnosti izmorfnosti in izomorizmov. Tokrat ne bomo zapisali
podrobnih dokazov. Za vajo jih dopolnite do tolikšnih podrobnosti, da boste sami sebe
prepričali, da trditve držijo.

\begin{trditev}
  Če je $f : A \to B$ izomorfizem med množicama $A$ in $B$ ter sta preslikavi
  $g : B \to A$ in $h : B \to A$ obe obrata~$f$, potem je $g = h$.
\end{trditev}

\begin{proof}
  Ker je $g$ obrat $f$, velja
  %
  \begin{equation*}
    g \circ f = \id[A]
    \qquad\text{in}\qquad
    f \circ g = \id[B],
  \end{equation*}
  %
  in ker je $h$ obrat $f$, velja
  %
  \begin{equation*}
    h \circ f = \id[A]
    \qquad\text{in}\qquad
    f \circ h = \id[B].
  \end{equation*}
  %
  Dokazati moramo, da iz teh štirih predpostavk sledi $g = h$, kar storimo z naslednjim
  računom:
  %
  \begin{align*}
    g
    &= \id[A] \circ g \tag{kompozicija z $\id[A]$ nima učinka} \\
    &= (h \circ f) \circ g \tag{predpostavka $h \circ f = \id[A]$} \\
    &= h \circ (f \circ g) \tag{kompozicija je asociativna} \\
    &= h \circ \id[B] \tag{predpostavka $f \circ g = \id[B]$} \\
    &= h. \tag{kompozicija z $\id[B]$ nima učinka}
  \end{align*}
\end{proof}

Če je $f : A \to B$ izomorfizem, potem ima natanko en obrat, ki ga označimo $\inv{f}$. Če
$f$ ni izomorfizem, zapis $f^{-1}$ ni veljaven izraz.

Oznaka za obrat je nekoliko nerodna, ker se prekriva z zapisom za obratno vrednost
števila: če je $x \in \RR$ neničelno realno število, potem je $\inv{x}$ tisto realno
število, za katerega velja $x \cdot \inv{x} = 1$. Torej moramo paziti: če je
$f : \RR \to \RR$ izomorfizem in $x \in \RR$, je $\inv{(f(x))}$ obrat števila $f(x)$,
medtem ko je $\inv{f}(x)$ število, ki ga dobimo, ko obrat preslikave $f$ uporabimo na~$x$.
Sami premislite, kaj je $\inv{(\inv{f}(x))}$.

\begin{naloga}
  Podajte primer izomorfizma $f : \RR \to \RR$ in števila $x \in \RR$, da velja
  $\inv{f}(x) = \inv{(f(x))}$. Nato podajte še primer, ko velja
  $\inv{f}(x) \neq \inv{(f(x))}$.
\end{naloga}

\begin{naloga}
  Ozrimo se še enkrat na dokaz prejšnje trditve. Ali smo uporabili vse štiri predpostavke?
  Zapišite \emph{bolj splošno trditev}, se pravi tako, ki navede samo tiste predpostavke,
  ki jih res potrebujemo v dokazu.
\end{naloga}

\begin{trditev}
  Za vse izmorfizme $f : A \to B$ in $g : B \to C$ velja
  %
  \begin{equation*}
    \inv{(\inv{f})} = f
    \qquad\text{in}\qquad
    \inv{(g \circ f)} = \inv{f} \circ \inv{g}.
  \end{equation*}
\end{trditev}

\begin{proof}
  Dokaz prepuščamo za vajo. Pozor, v desni enakosti se je zamenjal vrstni red $f$ in $g$!
  Nadalje opazimo še to: zapisali smo $\inv{(\inv{f})}$ in $\inv{(g \circ f)}$, ne da bi
  predhodno preverili, ali sta $\inv{f}$ in $g \circ f$ izomorfizma. Torej morate v dokazu
  najprej preveriti, da je sta $\inv{f}$ in $g \circ f$ izomorfizma, če sta $f$ in $g$
  izomorfizma.
\end{proof}

\begin{trditev}
  Za vse množice $A$, $B$ in $C$ velja:
  %
  \begin{enumerate}
  \item $A \ism A$,
  \item če $A \ism B$, potem $B \ism A$,
  \item če $A \ism B$ in $B \ism C$, potem $A \ism C$.
  \end{enumerate}
\end{trditev}

\begin{proof}
  \parbox{0pt}{}
  %
  \begin{enumerate}
  \item $\id[A]$ je izomorfizem iz $A$ v $A$, ki je sam svoj obrat,
  \item če je $f : A \to B$ izomorfizem iz $A$ v $B$, potem je $\inv{f}$ izomorfizem iz
    $B$ v $A$ in $f$,
  \item če je $f : A \to B$ izomorfizem iz $A$ v $B$ in $g : B \to C$ izomorfizem iz $B$ v
    $C$, potem je $g \circ f$ izomorfizem iz $A \to C$. \qedhere
  \end{enumerate}
\end{proof}

\begin{trditev}
  Preslikava ima največ en inverz.
\end{trditev}

\begin{naloga}
  Pogosto rečemo, da sta seštevanje in odštevanje obratni operaciji. Strogo vzeto, ti dve
  operaciji nista obratni kot preslikavi, saj obe slikata (recimo, da ju gledamo na
  realnih številih) $\RR \times \RR \to \RR$, tj.~ne slikata v nasprotnih smereh. Ugotovi,
  v kakšnem smislu točno sta seštevanje in odštevanje obratni, tj.~kateri dve preslikavi
  sta pravzaprav druga drugi obratni.
\end{naloga}

% TODO Izomorfnost je kongruenca za produkt, vsoto in eksponent.

\section{Algebra množic}
\label{sec:algebra-mnozic}

Kot že veste, seštevanje, množenje in potenciranje števil zadoščajo naslednjim
algebrajskim zakonom:
%
\begin{align*}
  a + 0 &= a                   &     a \cdot 1 &= a \\
  a + b &= b + a               &     a \cdot b &= b \cdot a \\
  a + (b + c) &= (a + b) + c   &     a \cdot (b \cdot c) &= (a \cdot b) \cdot c \\[1ex]
  0 \cdot a &= 0                           &   1^a &= 1 \\
  (a + b) \cdot c &= a \cdot c + b \cdot c &   (a \cdot b)^c &= a^c \cdot b^c \\[1ex]
  a^0 &= 1                     &   a^1 &= a \\
  a^{b + c} &= a^b \cdot a^c   &   a^{b \cdot c} &= (a^b)^c \\[1ex]
  0^a &= 0 \quad \text{če $a \neq 0$.}
\end{align*}
%
Že prej smo opazili, da je zakon $a \cdot b = b \cdot a$ podoben izomorfizmu
$A \times B \ism B \times A$. Kaj pa ostali zakoni?

\begin{izrek}
  \label{izrek:algebra-mnozic}
  Za vse množice $A$, $B$ in $C$ velja:
  %
  \begin{align*}
    A + \emptyset &\ism A                   &     A \times \one &\ism A \\
    A + B &\ism B + A               &     A \times B &\ism B \times A \\
    A + (B + C) &\ism (A + B) + C   &     A \times (B \times C) &\ism (A \times B) \times C \\[1ex]
    \emptyset \times A &\ism \emptyset                           &   \one^A &\ism \one \\
    (A + B) \times C &\ism A \times C + B \times C &   (A \times B)^C &\ism A^C \times B^C \\[1ex]
    A^\emptyset &\ism \one                     &   A^\one &\ism A \\
    A^{B + C} &\ism A^B \times A^C   &   A^{B \times C} &\ism (A^B)^C \\[1ex]
    \emptyset^A &\ism \emptyset \quad \text{če $A \neq \emptyset$.}
  \end{align*}
\end{izrek}

Izrek ni sam sebi namen, ampak je v njem nauk: \emph{z množicami lahko računamo}, tako kot
s števili. Preostanek razdelka je posvečen dokazu izreka.

\subsubsection{Asociativnost}
\label{sec:asociativnost}

Za ogrevanje dokažimo asociativnost zmnožkov,
$A \times (B \times C) \ism (A \times B) \times C$. Splošni element
$A \times (B \times C)$ je urejeni par oblike $(x, (y, z))$, kjer je $x \in A$, $y \in B$
in $z \in C$, med tem ko je splošni element $(A \times B) \times C$ oblike $((u, v), w)$,
kjer je $u \in A$, $v \in B$ in $w \in C$. Izomorfizmov ni težko zapisati:
%
\begin{align*}
  f &:  A \times (B \times C) \to (A \times B) \times C &
  g &: (A \times B) \times C \to A \times (B \times C) \\
  f &: (x, (y, z)) \mapsto ((x, y), z) &
  g &: ((u, v), w) \mapsto (u, (v, w)).
\end{align*}
%
Preverimo, da je $g$ obrat $f$. Za vse $x \in A$, $y \in B$ in $z \in C$ velja:
%
\begin{equation*}
  g(f(x, (y, z))) = g((x, y), z) = (x, (y, z))
\end{equation*}
%
in za vse $u \in A$, $v \in B$ in $w \in C$ velja
%
\begin{equation*}
  f(g((u, v), w)) = f(u, (v, w)) = ((u, v), w).
\end{equation*}
%
Tudi asociativnost vsote, $A + (B + C) \cong (A + B) + C$ ni nič bolj zapletena, le da
imamo opravka z injekcijami in obravnavanjem primerov. Najprej zapišimo izomorfizma s
popolnoma natančnim zapisom, kjer vse injekcije opremimo z oznakami množic:
%
\begin{align*}
  f &:  A + (B + C) \to (A + B) + C &
  g &: (A + B) + C \to A + (B + C) \\
  f &: \inl[A][B+C](x)             \mapsto \inl[A+B][C](\inl[A][B](x)) &
  g &: \inl[A+B][C](\inl[A][B](u)) \mapsto \inl[A][B+C](u)\\
  f &: \inr[A][B+C](\inl[B][C](y)) \mapsto \inl[A+B][C](\inr[A][B](y)) &
  g &: \inl[A+B][C](\inr[A][B](v)) \mapsto \inr[A][B+C](\inl[B][C](v))  \\
  f &: \inr[A][B+C](\inr[B][C](z)) \mapsto \inr[A+B][C](z) &
  g &: \inr[A+B][C](w)              \mapsto \inr[A][B+C](\inr[B][C](w))
\end{align*}
%
Isti zapis brez oznak množic je precej bolj čitljiv:
%
\begin{align*}
  f &:  A + (B + C) \to (A + B) + C &
  g &: (A + B) + C \to A + (B + C) \\
  f &: \inl(x)             \mapsto \inl(\inl(x)) &
  g &: \inl(\inl(u)) \mapsto \inl(u)\\
  f &: \inr(\inl(y)) \mapsto \inl(\inr(y)) &
  g &: \inl(\inr(v)) \mapsto \inr(\inl(v))  \\
  f &: \inr(\inr(z)) \mapsto \inr(z) &
  g &: \inr(w)              \mapsto \inr(\inr(w))
\end{align*}
%
Ali vidite, zakaj matematiki cenimo kratek in pregleden zapis? Preveč podrobnosti lahko
zakrije bistvo ideje. Preverjanje, da je $g$ obrat $f$, prepustimo tistim, ki radi veliko
pišejo.

\subsubsection{Preslikave in enojec}
\label{sec:preslikave-enojec}

Preslikavi
%
\begin{align*}
  f &: A \times \one \to A &
  g &: A \to A \times \one \\
  f &: (x, u) \mapsto x &
  g &: y \mapsto (y, \unit)
\end{align*}
%
tvorita izomorfizem $A \times \one \ism A$, saj za vsak $a \in A$ in $t \in \one$ velja,
upoštevaje da so vsi elementi~$\one$ enaki~$\unit$,
%
\begin{equation*}
  g(f(a, t)) = g(a) = (a, \unit) = (a, t)
  \qquad\text{in}\qquad
  f(g(a)) = f(a, t) = a.
\end{equation*}
%
Lahko bi rekli, da je $\one$ nevtralni element za zmnožek \emph{do izomorfizma natančno},
s čimer povemo, da ne velja \emph{enakost} $A \times \one = A$, ampak le
\emph{izomorfizem} $A \times \one \ism A$. Na tem mestu lahko tudi pojasnimo nenavadni
zapis edinega elementa~$\one$. Elementi zmnožka dveh množic so urejene dvojice, zmnožka
treh množic urejene trojice itd. Zmnožek nič množic je nevtralni element za množenje,
torej so njegovi elementi urejen ničterice, oziroma urejena ničterica~$\unit$, ker je ena
sama.

Izomorfizma $A^\one \ism A$ ni težko zapisati:
%
\begin{align*}
  f &: A^\one \to A &
  g &: A \to A^\one \\
  f &: h \mapsto h(\unit) &
  g &: x \mapsto (y \mapsto x)
\end{align*}
%
Preverimo, da je $g$ inverz $f$. Za vsak $x \in A$ velja
%
\begin{equation*}
  f(g(x)) = f(y \mapsto x) = x,
\end{equation*}
%
zato je $f \circ g = \id[A]$. Za vsak $h \in A^\one$ velja
%
\begin{equation*}
  g(f(h)) = g(h(\unit)) = (y \mapsto h(\unit)).
\end{equation*}
%
Ali sta $h$ in $y \mapsto h(\unit)$ enaki preslikavi? Kot vsakič, uporabimo
ekstenzionalnost preslikav, le da je tokrat še posebej preprosta: preslikavi z
domeno~$\one$ sta enaki, če imata enako vrednost pri argumentu $\unit$, saj je to edini
element~$\one$. Torej je $h = (y \mapsto h(\unit))$, saj velja
%
\begin{equation*}
  (y \mapsto h(\unit))(\unit) = h(\unit).
\end{equation*}
%
Izomorfnost $A$ in $A^\one$ pravzaprav pove nekaj zanimivega: preslikave $\one \to A$
lahko obravnavamo kot elemente~$A$ in obratno.


\subsubsection{Preslikave in prazna množica}
\label{sec:presl-prazna-mnozica}

Lotimo se izomorfizmov, v katere je vpletena prazna množica. Tu se ne moremo več zanašati
le na prirojen občutek za logiko, saj s prazno množico nimamo vsakdanjih izkušenj, oziroma
jo obravnavamo kot posebnost. Kako bi odgovorili na vprašanje, ali so vsi elementi prazne
množice praštevila? Pravilni odgovor je ``da''. In hkrati so vsi elementi prazne množice
sestavljena števila. Zakaj je to res bomo spoznali v
razdelku~\ref{sec:logika-prazna-mnozica}, ko bomo podrobno obravnavali pravila sklepanja.
Zaenkrat si zapomnimo, da je pravilna vsaka izjava ``za vse elemente prazne množice velja
\dots''. Pravimo, da je taka izjava \emph{na prazno izpolnjena}

Začnimo z vprašanjem, ali lahko tvorimo kako preslikavo $\emptyset \to A$. Najprej
ugotovimo, da so vse preslikave $\emptyset \to A$ enake. Res, za $f, g : \emptyset \to A$
velja $f = g$ natanko tedaj, ko za vse $x \in \emptyset$ velja $f(x) = g(x)$. A ravnokar
smo povedali, da je vsaka izjava oblike ``za vse $x \in \emptyset$ \dots'' veljavna. Pa
imamo kako preslikavo $\emptyset \to A$? Odgovor je pritrdilen, če lahko podamo kako
celovito in enolično prirejanje med elementi~$\emptyset$ in~$A$. Ker sta celovitost in
enoličnost spet izavi oblike ``za vse $x \in \emptyset$ \dots'', sta na prazno izpolnjena,
zato bo zadoščalo kakršnokoli prirejanje, denimo: nobenemu elementu ne priredimo nobenega
elementa. S tem smo utemeljili naslednjo trditev.

\begin{trditev}
  Za vsako množico $A$ obstaja natanko ena preslikava $\emptyset \to A$.
\end{trditev}

Edini preslikavi $\emptyset \to A$ pravimo \df{prazna preslikava}. S tem smo utemeljili
$A^\emptyset \ism \one$, saj izomorfizem prazni preslikavi priredi
$\unit$, njegov obrat pa priredi $\unit$ prazno preslikavo.

\subsubsection{Izomorfizmi in eksponenti}
\label{sec:izomorfizmi-in-eksponenti}

Nazadnje se posvetimo še zakonu $A^{B \times C} \ism (A^B)^C$.
Preverimo, da preslikavi\footnote{Saj ste se že naučili grške črke, ali ne?}
%
\begin{align*}
  \Lambda &: A^{B \times C} \to (A^B)^C
  &
  \Theta &: (A^B)^C \to A^{B \times C}
  \\
  \Lambda &: f \mapsto (c \mapsto (b \mapsto f(b, c)))
  &
  \Theta &: g \mapsto ((b, c) \mapsto g(c)(b))
\end{align*}
%
tvorita izomorfizem. Za vse $f \in A^{B \times C}$, $x \in B$ in $y \in C$ velja
%
\begin{align*}
  \Theta(\Lambda(f))(x, y)
  &= ((b, c) \mapsto \Lambda(f)(c)(b)) (x, y)  \\
  &= \Lambda(f)(y)(x) \\
  &= (c \mapsto (b \mapsto f(b, c)))(y)(x) \\
  &= (b \mapsto f(b, y)(x) \\
  &= f(x, y),
\end{align*}
%
zato je $\Theta(\Lambda(f)) = f$. Prav tako za vse $g \in (A^B)^C$ in $x \in B$ in
$y \in C$ velja
%
\begin{align*}
  \Lambda(\Theta(g))(y)(x)
  &= (c \mapsto (b \mapsto \Theta(g)(b, c)))(y)(x) \\
  &= (b \mapsto \Theta(g)(b, y))(x) \\
  &= \Theta(g)(x, y) \\
  &= ((b, c) \mapsto g(c)(b)) (x, y) \\
  &= g(y)(x)
\end{align*}
%
in zato $\Lambda(\Theta(g)) = g$.
%
Preslikavi $\Lambda(f)$ pravimo \df{transpozicija} preslikave~$f$, in prav tako preslikavi
$\Theta(g)$ pravimo transpozicija preslikave~$g$.

Izomorfizem $A^{B \times C} \ism (A^B)^C$ je zanimiv, ker pove, da lahko preslikavo dveh
argumentov vedno prevedemo na preslikavo enega argumenta. Natančneje, če je
$f : B \times C \to A$ preslikava dveh argumentov, je njena transpozicija
$\Lambda(f) : C \to A^B$ preslikava enega argumenta, njena vrednost pa je preslikava, ki
pričakuje še en argument. To dejstvo se s pridom izkorišča v funkcijskem programiranju:
namesto, da bi definirali preslikavo $f : B \times C \to A$, ki sprejme urejeni par
$(b, c)$ in vrne vrednost $f(b,c)$, raje definiramo enakovredno preslikavo
$\tilde{f} : B \to C \to A$, ki sprejme $b$ in vrne preslikavo $\tilde{f}(b)$, ta pa
sprejme še $c$ in vrne vrednost $\tilde{f}(b)(c)$.


% TODO Potence A^n in binomski izrek.

% TODO zmnozek, vsota in eksponent zo kongruenca za izomorfizme

% Pri asociativnosti produkta obravnavamo $A_1 \times A_2 \times \cdots \times A_n$ in
% enojec kot produkt nič množic. Podobno za vsote.

% Tu je treba pojasniti, zakaj pišemo $\unit$ za element $\one$.


% \section{Kar je že Davorin napisal}

% Interval realnih števil podamo s krajiščema intervala v oklepajih --- okrogli oklepaji ( ) označujejo odprtost intervala (krajišče ni vključeno v interval), oglati oklepaji [ ] pa zaprtost (krajišče je vključeno). Tako se npr.~interval realnih števil od $0$ do $1$, ki ne vsebuje krajišč, označi z $(0, 1)$, če jih vsebuje, pa z $[0, 1]$.

% Včasih pridejo prav tudi intervali na drugih množicah kot $\RR$. Zato se dogovorimo, da bomo intervale označevali tako, da podamo množico, ob kateri v indeksu zapišemo krajišči v oklepajih, npr.~$\intco[\NN]{1}{5} = \set{1, 2, 3, 4}$. Realna intervala iz prejšnjega odstavka tako zapišemo kot $\intoo{0}{1}$ in $\intcc{0}{1}$.

% Če interval v katero smer gre v nedogled, preprosto zapišemo množico z ustreznim simbolom za urejenost in krajiščem v indeksu. Na primer, $\RR_{> 0}$ označuje množico pozitivnih realnih števil, $\RR_{\geq 0}$ pa množico nenegativnih realnih števil.

% Primerjave med elementi, kot npr.~pravkar podani $>$ in $\geq$, imenujemo \df{relacije} (podrobneje jih bomo spoznali v poglavju~\ref{poglavje:relacije}). Zgornji zapis bomo uporabljali tudi za druge vrste relacij, ne samo za relacije urejenosti. Na primer, množico vseh neničelnih realnih števil zapišemo kot $\RR_{\neq 0}$.

% \davorin{To bi vsaj bil moj predlog. Na ta način se izognemo dvoumnostim (kar je namen). Na primer, kaj pomeni $\forall\, a > 0$? Če zapišemo $\forall\, a \in \NN_{> 0}$ ali $\forall\, a \in \RR_{> 0}$, je jasno. Razlog, da matematiki ``goljufajo'' in pridejo skozi brez tega, je (napol dogovorjena in ponotranjena, ampak arbitrarna) izbira črk; vsak izkušen matematik ve, da $\forall\, \epsilon > 0$ pomeni $\forall\, \epsilon \in \RR_{> 0}$. Dodaten problem je, da kasneje uporabljamo urejene pare, ki jih vsi na naši fakulteti pišejo z okroglimi oklepaji. Poskusimo se izogniti zmedi, ali $(a, b)$ pomeni urejeni par ali odprti interval. Če se ne strinjate, popravite in pustite komentar.}

% Če imamo dan neki element in neko množico, potem pripadnost tega elementa tej množici izrazimo s simbolom $\in$. Na primer, da je štiri naravno število, zapišemo $4 \in \NN$ (beri: ``štiri pripada množici naravnih števil'').

% Elementi množic lahko zadoščajo raznim lastnostim. Na primer, recimo, da $\phi$ označuje lastnost ``biti manj od pet''; to potem zapišemo
% \[\phi(x) \ = \ \ x < 5.\]
% V tem primeru $x$ imenujemo \df{spremenljivka}, saj ne gre za točno določeno vrednost, pač pa predstavlja splošno število (recimo, da se dogovorimo, da s $\phi$ označujemo lastnost na realnih številih).

% Tovrstne lastnosti nam omogočajo, da iz neke množice odberemo elemente z dano lastnostjo in na ta način dobimo novo množico, ki je podmnožica prejšnje. Množico vseh realnih števil, ki so manjša od pet, zapišemo na naslednji način.
% \[\set{x \in \RR}{x < 5}\]
% Seveda, ker je primerjava s števili zelo pogosta lastnost, je uporabno, če uvedemo krajše oznake, ki povejo isto; že prej smo se dogovorili, da tako množico označimo z $\RR_{< 5}$. Za povsem splošne lastnosti pa ne bomo imeli vnaprej dogovorjenih oznak, zato je dobro, da poznamo splošni zapis. Torej, če je $X$ poljubna množica in $\phi$ poljubna lastnost njenih elementov, tedaj podmnožico, ki vsebuje točno tiste elemente množice $X$, ki zadoščajo lastnosti $\phi$, označimo takole.
% \[\set[1]{x \in X}{\phi(x)}\]

% Pri tem se zavedajmo: ni pomembno, da spremenljivko označimo ravno z $x$. Zapis
% \[\set[1]{y \in X}{\phi(y)}\]
% še vedno označuje isto množico. V vsakem primeru gre za množico vseh elementov iz $X$ z lastnostjo $\phi$. Pravzaprav sploh ni nujno, da uporabimo črko; poslužimo se lahko kateregakoli simbola (ki mu nismo predtem že predpisali določenega pomena). Taisto množico lahko zapišemo tudi $\set{\heartsuit \in X}{\phi(\heartsuit)}$.

% Kadar imamo spremenljivko, ki jo lahko preimenujemo, ne da bi spremenili pomen izraza, jo imenujemo \df{nema spremenljivka}. Takšne primere že dobro poznate; na primer, integral $\int_0^1 x^2 \,dx$ se ne spremeni, če preimenujete spremenljivko in zapišete $\int_0^1 y^2 \,dy$.

% \begin{zgled}
% Kako bi zapisali množico vseh sodih naravnih števil? Spomnimo se, da je število sodo, kadar je deljivo z $2$. Za $n \in \NN$ to zapišemo takole: $2 \divides n$ (beri: ``dve deli $n$''). Množica sodih naravnih števil se potem zapiše kot
% \[\set[1]{n \in \NN}{2 \divides n}.\]
% \end{zgled}


\section{Vaje}

\begin{vaja}
Kaj veste povedati o množici~$A$, če zanjo velja, da so vsi njeni elementi enaki?
\begin{resitev}
Množica~$A$ ima kvečjemu en element, tj.~množica~$A$ je bodisi prazna bodisi enojec. Tudi: množica~$A$ je podmnožica kakega enojca oz.~edina preslikava $A \to \one$ je injektivna.
\end{resitev}
\end{vaja}

\begin{vaja}
  Načelo ekstenzionalnosti preslikav bi lahko zapisali tudi takole:
  %
  \begin{quote}
    Preslikavi $f : A \to B$ in $g : C \to D$ sta enaki, če velja $A = C$, $B = D$ in za
    vse $x_1, x_2 \in A$ velja, da iz $x_1 = x_2$ sledi $f(x_1) = g(x_2)$.
  \end{quote}
  %
  Dokažite, da je ta različica enakovredna običajnem načelu ekstenzionalnosti.
\end{vaja}

\begin{vaja}
  Zapišite pravila za zmnožek treh množic. Nato premislite še, kako bi podali pravila za
  zmnožek $n$ množic, kjer je~$n$ naravno število.
\end{vaja}

\begin{vaja}
  Naštejte vse elemente množice $\one + \one + \one$.
\end{vaja}

\begin{vaja}
  Preveri tiste izomorfnosti iz izreka~\ref{izrek:algebra-mnozic}, ki jih v
  razdelku~\ref{sec:algebra-mnozic} nismo utemeljili.
\end{vaja}

%%% Local Variables:
%%% mode: latex
%%% TeX-master: "ucbenik-lmn"
%%% End:


% Pod to vrstico so poglavja, ki še niso pripravljena
% za javno dostopno verzijo
\ifOPTincludeall
\chapter{Logika}\label{poglavje:logika}





\section{Logični simboli}\label{razdelek:logicni-simboli}

Preproste izjave, kot na primer \nls{$n$ je sodo število.}, že znamo zapisati s simboli: $2 \divides n$. Povečini pa delamo z bolj kompleksnimi, sestavljenimi izjavami. Tudi za te obstaja simbolni zapis; na primer, izjavo \nls{Če je $n$ sodo število, je tudi kvadrat števila $n$ sod.}, zapišemo kot $2 \divides n \implies 2 \divides n^2$. Seveda ta izjava velja za vsa naravna števila (znaš to dokazati?). To zapišemo takole: $\all{n \in \NN} (2 \divides n \implies 2 \divides n^2)$. V tem razdelku si bomo ogledali, kako povezati preprostejše izjave v bolj sestavljene in kako to v splošnem simbolno zapisati.

Kot smo navajeni iz naravnih jezikov, posamične stavke povežemo v sestavljeno poved z \emph{vezniki}. Najpogosteje uporabljeni matematični vezniki so v tabeli~\ref{tabela:standardni-izjavni-vezniki}.

\begin{table}[!ht]
\centering
\begin{tabular}{|ccc|}
\hline
\textbf{Izjavni veznik} & \textbf{Oznaka} & \textbf{Kako preberemo} \\
\hline
negacija & $\lnot{p}$ & ne $p$ \\
konjunkcija & $p \land q$ & $p$ in $q$ \\
disjunkcija & $p \lor q$ & $p$ ali $q$ \\
implikacija & $p \impl q$ & če $p$, potem $q$ \\
ekvivalenca & $p \lequ q$ & $p$ natanko tedaj, ko $q$ \\
\hline
\end{tabular}
\caption{Standardni izjavni vezniki}\label{tabela:standardni-izjavni-vezniki}
\end{table}

\begin{opomba}
V matematiki se za izjavne veznike običajno uporabljajo zgoraj navedene tujke, ampak vsaka od njih seveda ima svoj pomen. Dobesedni prevodi teh tujk so:
\begin{itemize}
\item
negacija $\to$ zanikanje,
\item
konjunkcija $\to$ vezava,
\item
disjunkcija $\to$ ločitev,
\item
implikacija $\to$ vpletenost,
\item
ekvivalenca $\to$ enakovrednost.
\end{itemize}
Za primerjavo: spomnite se vezalnega in ločnega priredja iz slovenščine!
\end{opomba}

\begin{zgled}
Naj $p$ označuje stavek \nls{Zunaj dežuje.} in $q$ stavek \nls{Vzamem dežnik.}. Tedaj $\lnot{p}$ pomeni \nls{Zunaj ne dežuje.} in $p \impl q$ pomeni \nls{Če zunaj dežuje, potem vzamem dežnik.}.
\end{zgled}

Kose sestavljene izjave lahko veže več kot en veznik. V tem primeru se (tako kot pri računanju s števili) dogovorimo o prednosti veznikov. Po dogovoru je vrstni red veznikov tak, kot v tabeli~\ref{tabela:standardni-izjavni-vezniki}, tj.~najmočneje veže negacija, nato konjunkcija, nato disjunkcija, nato implikacija, nato ekvivalenca. Kadar želimo, da se najprej izvede veznik z nižjo prednostjo, uporabimo oklepaje.

\begin{zgled}
Označimo sledeče stavke:
\begin{quote}
$p$ \ \ldots\ldots\ \nls{Imam čas.} \\
$q$ \ \ldots\ldots\ \nls{Ostanem doma.}
\end{quote}
Tedaj $\lnot{p} \land q$ pomeni isto kot $(\lnot{p}) \land q$, to je \nls{Nimam časa in ostanem doma.}, medtem ko $\lnot(p \land q)$ pomeni \nls{Ni res, da imam čas in ostanem doma.}.
\end{zgled}
\davorin{Če komu pade na pamet primer boljših stavkov, je zaželjeno, da popravi\ldots}

Poleg zgoraj navedenih izjavnih veznikov se včasih uporabljajo še sledeči (tabela~\ref{tabela:nadaljnji-izjavni-vezniki}).

\begin{table}[!ht]
\centering
\begin{tabular}{|ccc|}
\hline
\textbf{Izjavni veznik} & \textbf{Oznaka} & \textbf{Kako preberemo} \\
\hline
stroga disjunkcija & $p \xor q$ & bodisi $p$ bodisi $q$ \\
Shefferjev\tablefootnote{Henry Maurice Sheffer (1882 -- 1964) je bil ameriški logik.} veznik & $p \shf q$ & ne hkrati $p$ in $q$ \\
Łukasiewiczev\tablefootnote{Jan Łukasiewicz (beri: \hill{u}ukaśj\^{e}vič) (1878 -- 1956) je bil poljski logik in filozof.} veznik & $p \luk q$ & niti $p$ niti $q$ \\
\hline
\end{tabular}
\caption{Nekateri nadaljnji izjavni vezniki}\label{tabela:nadaljnji-izjavni-vezniki}
\end{table}

Za strogo disjunkcijo (tudi: ekskluzivna disjunkcija, izključitvena disjunkcija) se uporabljajo še druge oznake: $p \oplus q$, $p + q$. Razlika med navadno in strogo disjunkcijo je sledeča: $p \lor q$ pomeni, da \emph{vsaj eden} od $p$ in $q$ velja, medtem ko $p \xor q$ pomeni, da velja \emph{natanko eden}.

\begin{zgled}
Stavek \nls{Pisni del predmeta je potrebno opraviti s kolokviji ali pisnim izpitom.} je primer navadne disjunkcije (seveda se vam prizna pisni del predmeta tudi, če uspešno odpišete tako kolokvije kot pisni izpit), stavek \nls{Grem bodisi na morje bodisi v hribe.} pa je primer stroge disjunkcije (ne da se biti na dveh mestih hkrati).
\end{zgled}

Pogosto veznike iz tabele~\ref{tabela:nadaljnji-izjavni-vezniki} (in vse preostale, ki jih nismo navedli) kar izrazimo s standardnimi na sledeči način.
\begin{center}
\begin{tabular}{|ccc|}
\hline
\textbf{Izjavni veznik} & \multicolumn{2}{c|}{\textbf{Nekatere izražave s standardnimi vezniki}} \\
\hline
$p \xor q$ & $(p \lor q) \land \lnot(p \land q)$ & $(p \land \lnot{q}) \lor (\lnot{p} \land q)$ \\
$p \shf q$ & $\lnot(p \land q)$ & $\lnot{p} \lor \lnot{q}$ \\
$p \luk q$ & $\lnot(p \lor q)$ & $\lnot{p} \land \lnot{q}$ \\
\hline
\end{tabular}
\end{center}

Včasih pa vendarle raje delamo neposredno z dodatnimi vezniki. Služijo lahko kot koristna okrajšava, so pa še drugi razlogi. Na primer, stroga disjunkcija igra vlogo seštevanja v Boolovem kolobarju (glej~\note{razdelek o Boolovih kolobarjih}), Shefferjev in Łukasiewiczev veznik pa se uporabljata pri preklopnih vezjih, saj je z vsakim od njiju možno izraziti vse izjavne veznike (glej vajo~\ref{naloga:polni-nabori-z-enim-veznikom}). V računalništvu imajo ti trije vezniki standardne oznake XOR, NAND, NOR.

\davorin{Nekje tukaj povejmo, kakšno prednost damo tem trem veznikom v primerjavi s standardnimi. Kateremu dogovoru sledimo?}

Včasih so izjave odvisne od kakšnih parametrov. Na primer, naj $\phi(x)$ pomeni \nls{$x$ je zelen.}; tedaj $\phi(\text{trava})$ pomeni \nls{Trava je zelena.}. Simbol $\phi$ torej predstavlja lastnost določenih objektov. Takšne primere smo imeli že v razdelku~\ref{razdelek:mnozice}, kjer smo navedli oznako za podmnožico tistih elementov, ki zadoščajo dani lastnosti.

Lastnosti, odvisne od spremenljivk, lahko \emph{kvantificiramo} po njihovih spremenljivkah, tj.~povemo, ``kako pogosto'' velja lastnost. Tabela~\ref{tabela:kvantifikatorji} podaja najpogosteje uporabljane kvantifikatorje in njihove oznake.

\begin{table}[!ht]
\centering
\begin{tabular}{|ccc|}
\hline
\textbf{Kvantifikator} & \textbf{Oznaka} & \textbf{Kako preberemo} \\
\hline
univerzalni kvantifikator & $\all{x \in X} \phi(x)$ & za vsak $x$ iz $X$ velja lastnost $\phi$ \\
eksistenčni kvantifikator & $\some{x \in X} \phi(x)$ & obstaja $x$ iz $X$ z lastnostjo $\phi$ \\
\note{enolični eksistenčni kvantifikator?} & $\exactlyone{x \in X} \phi(x)$ & obstaja natanko en $x$ iz $X$ z lastnostjo $\phi$ \\
\hline
\end{tabular}
\caption{Kvantifikatorji}\label{tabela:kvantifikatorji}
\end{table}

Oznaki $\forall$ in $\exists$ sta narobe obrnjena A in E in izhajata iz nemščine (\textbf{a}ll, \textbf{e}xistiert).

Seveda je tudi kvantificirana spremenljivka vezana in jo lahko poljubno preimenujemo. Izjavi $\all{x \in X} \phi(x)$ in $\all{y \in X} \phi(y)$ povesta natanko isto: vsi elementi množice $X$ imajo lastnost $\phi$.

\begin{zgled}
Vemo, da za vsako nenegativno realno število obstaja enolično določen nenegativen kvadratni koren; to izjavo lahko zapišemo na sledeči način.
\[\all{a \in \RR_{\geq 0}} \exactlyone{b \in \RR_{\geq 0}} b^2 = a\]
Zaradi tega lahko definiramo kvadratni koren kot funkcijo $\sqrt{\phantom{I}}\colon \RR_{\geq 0} \to \RR_{\geq 0}$ \note{več o tem kasneje}.
\end{zgled}

Po dogovoru kvantifikatorji vežejo močneje kot izjavni vezniki. Izjavo, da je vsako celo število bodisi sodo bodisi liho, torej zapišemo takole.
\[\all{a \in \ZZ} (2 \divides a \xor 2 \divides (a-1))\]

\begin{zgled}
Za poljubno naravno število $n \in \NN$ naj $P(n)$ označuje izjavo, da je $n$ praštevilo. Torej, $P$ definiramo takole.
\[P(n) \dfeq \all{x \in \NN_{\geq 1}} (x \divides n \implies x = 1 \xor x = n)\]
(Premisli, kaj bi se zgodilo, če bi namesto stroge disjunkcije vzeli navadno. Bi še vedno dobili pravilni pojem praštevila?)

Naj $S(n)$ označuje, da je $n$ sestavljeno število.
\[S(n) \dfeq \some{x, y \in \intoo[\NN]{1}{n}} x \cdot y = n\]
(Kadar imamo več zaporednih kvantifikatorjev iste vrste, jih po dogovoru lahko strnemo kot zgoraj. Dana formula za $S(n)$ je krajši zapis za $\some{x \in \intoo[\NN]{1}{n}} \some{y \in \intoo[\NN]{1}{n}} x \cdot y = n$.)

Zdaj lahko na pregleden način zapišemo, da je vsako naravno število od $2$ naprej bodisi praštevilo bodisi sestavljeno.
\[\all{n \in \NN_{\geq 2}} (P(n) \xor S(n))\]
\end{zgled}

\section{Definicije}

\davorin{Predlagam, da v definicijah konsistentno uporabljamo `kadar' namesto `če' (``Funkcija je zvezna, kadar velja to in to.''). V definicijah gre za ekvivalenco, ne implikacijo.}

\davorin{Verjetno je smiselno v tem razdelku razložiti definicijsko enakost $\dfeq$ (oz.~$\revdfeq$). Če se tako odločimo, odstranimo zgornje uporabe teh simbolov.}


% TU SE KONČA PRESTAVLJENA ROBA


        \note{uvod}


        \section{Izjavni vezniki}

                V razdelku~\ref{razdelek:logicni-simboli} smo omenili nekaj izjavnih veznikov, podali oznake zanje in opisali njihov intuitivni pomen. Ampak če se hočemo zanašati na pravilnost naših sklepov, moramo tem oznakam dati \emph{formalni matematični pomen}.

                Če imamo neko izjavo, lahko določimo njeno resničnost, tj.~povemo, do kolikšne mere je resnična. Temu rečemo \df{resničnostna vrednost} izjave. Množico vseh možnih resničnostnih vrednosti označimo z $\tvs$. Seveda ni kaj dosti možnih resničnostnih vrednosti: to sta \df{resnica} (dogovorimo se, da bomo zanjo uporabljali oznako $\true$) in \df{neresnica} (oznaka $\false$). Se pravi, $\tvs = \set{\true, \false}$.

                \begin{opomba}
                        Logiki, kjer sta edini resničnostni vrednosti resnica in neresnica, rečemo \df{dvovrednostna} oziroma \df{klasična logika}. Obstajajo splošnejše vrste logike, kjer je $\set{\true, \false}$ prava podmnožica $\tvs$, ampak v tej knjigi se bomo omejili na klasično logiko, na katero ste navajeni in ki se uporablja v večjem delu matematike.
                \end{opomba}

                \davorin{Kako izrecno bomo ločevali med izjavami in njihovimi logičnimi vrednostmi?}

                Izjavne veznike lahko potem formalno podamo kot preslikave. Na primer, negacija je preslikava $\lnot\colon \tvs \to \tvs$ (vsaki resničnostni vrednosti pripišemo njeno nasprotno vrednost). Preslikavo, definirano na majhni končni množici, lahko preprosto podamo s tabelo vseh njenih vrednosti. V primeru izjavnih veznikov takim tabelam rečemo \df{resničnostne tabele}. Resničnostna tabela za negacijo je videti takole.
                \begin{center}
                        \begin{tabular}{c|c}
                                $p$ & $\lnot{p}$ \\
                                \hline
                                $\true$ & $\false$ \\
                                $\false$ & $\true$
                        \end{tabular}
                \end{center}
                Ta tabela povsem natančno definira negacijo kot preslikavo $\lnot\colon \tvs \to \tvs$. Seveda smo negacijo definirali tako, kot bi pričakovali: negacija resnice je neresnica, negacija neresnice je resnica.

                Podobno lahko naredimo z ostalimi izjavnimi vezniki, le da preostali vežejo dve izjavi. Se pravi, npr.~konjunkcija vzame dve resničnostni vrednosti in vrne resničnostno vrednost, ki pove, ali sta obe dani vrednosti resnični. Konjunkcijo lahko torej interpretiramo kot preslikavo $\land\colon \tvs \times \tvs \to \tvs$ (ali na kratko $\land\colon \tvs^2 \to \tvs$).

                V splošnem definiramo, da je \df{$n$-mestni izjavni veznik} preslikava oblike $\tvs^n \to \tvs$. Negacija je torej enomestni izjavni veznik, ostali vezniki, ki smo jih do zdaj omenili, pa so dvomestni.

                Definirajmo zdaj konjunkcijo natančno s pomočjo resničnostne tabele. Množica $\tvs \times \tvs$ ima štiri elemente --- vse možne pare, sestavljene iz $\true$ oz.~$\false$. Intuitivni pomen konjunkcije razumemo: konjunkcija dveh izjav je resnična natanko tedaj, ko sta obe izjavi resnični. To nas vodi do naslednje tabele.
                \begin{center}
                        \begin{tabular}{cc|c}
                                $p$ & $q$ & $p \land q$ \\
                                \hline
                                $\true$ & $\true$ & $\true$ \\
                                $\true$ & $\false$ & $\false$ \\
                                $\false$ & $\true$ & $\false$ \\
                                $\false$ & $\false$ & $\false$
                        \end{tabular}
                \end{center}

                Za disjunkcijo smo že rekli, da pride v dveh različicah: navadna pomeni, da vsaj ena od izjav velja, izključitvena pa pomeni, da velja natanko ena od izjav. Posledično je torej smiselno definirati funkciji $\lor, \xor\colon \tvs \times \tvs \to \tvs$ na sledeči način.
                \begin{center}
                        \begin{tabular}{cc|cc}
                                $p$ & $q$ & $p \lor q$ & $p \xor q$ \\
                                \hline
                                $\true$ & $\true$ & $\true$ & $\false$ \\
                                $\true$ & $\false$ & $\true$ & $\true$ \\
                                $\false$ & $\true$ & $\true$ & $\true$ \\
                                $\false$ & $\false$ & $\false$ & $\false$
                        \end{tabular}
                \end{center}
                Bodi pozoren na razliko med zadnjima dvema stolpcema!

                Obenem lahko še na hitro opravimo z veznikoma $\shf$ in $\luk$. Spomnimo se, da $p \shf q$ pomeni ``ne hkrati $p$ in $q$'', medtem ko $p \luk q$ pomeni ``niti $p$ niti $q$''.
                \begin{center}
                        \begin{tabular}{cc|cc}
                                $p$ & $q$ & $p \shf q$ & $p \luk q$ \\
                                \hline
                                $\true$ & $\true$ & $\false$ & $\false$ \\
                                $\true$ & $\false$ & $\true$ & $\false$ \\
                                $\false$ & $\true$ & $\true$ & $\false$ \\
                                $\false$ & $\false$ & $\true$ & $\true$
                        \end{tabular}
                \end{center}

                Implikacija je nekoliko bolj subtilna. Kaj točno trdimo z izjavo $p \impl q$, se pravi, kakor hitro velja $p$, mora veljati tudi $q$? No, če $p$ ne velja, potem sploh nismo postavili nobenega pogoja --- izjava je avtomatično izpolnjena. Če $p$ velja, pa zraven zahtevamo še $q$. Resničnostna tabela za implikacijo je potemtakem sledeča.
                \begin{center}
                        \begin{tabular}{cc|c}
                                $p$ & $q$ & $p \impl q$ \\
                                \hline
                                $\true$ & $\true$ & $\true$ \\
                                $\true$ & $\false$ & $\false$ \\
                                $\false$ & $\true$ & $\true$ \\
                                $\false$ & $\false$ & $\true$
                        \end{tabular}
                \end{center}

                Ekvivalenca je spet preprosta --- izjavi sta ekvivalentni, kadar imata isto resničnostno vrednost. Od tod dobimo sledečo resničnostno tabelo.
                \begin{center}
                        \begin{tabular}{cc|c}
                                $p$ & $q$ & $p \lequ q$ \\
                                \hline
                                $\true$ & $\true$ & $\true$ \\
                                $\true$ & $\false$ & $\false$ \\
                                $\false$ & $\true$ & $\false$ \\
                                $\false$ & $\false$ & $\true$
                        \end{tabular}
                \end{center}

                Za lažjo referenco zberimo resničnostne tabele vseh do zdaj omenjenih veznikov na eno mesto (tabela~\ref{tabela:resnicnostna-tabela-osnovnih-izjavnih-veznikov}).

                \begin{table}[!ht]
                        \centering
                        \begin{tabular}{c|c}
                                $p$ & $\lnot{p}$ \\
                                \hline
                                $\true$ & $\false$ \\
                                $\false$ & $\true$
                        \end{tabular}
                        \qquad\quad
                        \begin{tabular}{cc|ccccccc}
                                $p$ & $q$ & $p \land q$ & $p \lor q$ & $p \xor q$ & $p \shf q$ & $p \luk q$ & $p \impl q$ & $p \lequ q$ \\
                                \hline
                                $\true$ & $\true$ & $\true$ & $\true$ & $\false$ & $\false$ & $\false$ & $\true$ & $\true$ \\
                                $\true$ & $\false$ & $\false$ & $\true$ & $\true$ & $\true$ & $\false$ & $\false$ & $\false$ \\
                                $\false$ & $\true$ & $\false$ & $\true$ & $\true$ & $\true$ & $\false$ & $\true$ & $\false$ \\
                                $\false$ & $\false$ & $\false$ & $\false$ & $\false$ & $\true$ & $\true$ & $\true$ & $\true$
                        \end{tabular}
                        \caption{Resničnostna tabela osnovnih izjavnih veznikov}\label{tabela:resnicnostna-tabela-osnovnih-izjavnih-veznikov}
                \end{table}

                Zdaj ko imamo natančno definicijo izjavnih veznikov, lahko trditve v zvezi z njimi tudi formalno utemeljimo. Na primer, spomnimo se, da smo že malo po omembi veznikov $\xor$, $\shf$, $\luk$ podali njihovo izražavo z vezniki $\lnot$, $\land$, $\lor$. Če na glas preberemo vse izjave, nam je intuitivno jasno, katere se ujemajo in zakaj, ampak zdaj lahko dejansko preverimo, da te izražave veljajo.

                Na primer, kaj pomeni, da se $p \luk q$ lahko izrazi kot $\lnot(p \lor q)$? To pomeni, da sta funkciji $\tvs \times \tvs \to \tvs$, dani s predpisoma $(p, q) \mapsto p \luk q$ in $(p, q) \mapsto \lnot(p \lor q)$, enaki. (Slednja funkcija je sestavljena, tj.~sklop dveh funkcij. Lahko bi tudi zapisali, da velja $\luk = \lnot \circ \lor$.) Funkciji z isto domeno in kodomeno sta enaki, kadar pri vsakem argumentu vrneta isti vrednosti, kar v našem primeru pomeni, da imata enaka stolpca v resničnostni tabeli. Poračunajmo torej vse izraze v danih izražavah. Ko dobimo enake rezultate, bomo vedeli, da izražave dejansko veljajo.

                \begin{center}
                        \begin{tabular}{cc|cccccc}
                                $p$ & $q$ & $p \shf q$ & $p \land q$ & $\lnot(p \land q)$ & $\lnot{p}$ & $\lnot{q}$ & $\lnot{p} \lor \lnot{q}$ \\
                                \hline
                                $\true$ & $\true$ & $\efalse$ & $\true$ & $\efalse$ & $\false$ & $\false$ & $\efalse$ \\
                                $\true$ & $\false$ & $\etrue$ & $\false$ & $\etrue$ & $\false$ & $\true$ & $\etrue$ \\
                                $\false$ & $\true$ & $\etrue$ & $\false$ & $\etrue$ & $\true$ & $\false$ & $\etrue$ \\
                                $\false$ & $\false$ & $\etrue$ & $\false$ & $\etrue$ & $\true$ & $\true$ & $\etrue$
                        \end{tabular}
                \end{center}

                \begin{center}
                        \begin{tabular}{cc|cccccc}
                                $p$ & $q$ & $p \luk q$ & $p \lor q$ & $\lnot(p \lor q)$ & $\lnot{p}$ & $\lnot{q}$ & $\lnot{p} \land \lnot{q}$ \\
                                \hline
                                $\true$ & $\true$ & $\efalse$ & $\true$ & $\efalse$ & $\false$ & $\false$ & $\efalse$ \\
                                $\true$ & $\false$ & $\efalse$ & $\true$ & $\efalse$ & $\false$ & $\true$ & $\efalse$ \\
                                $\false$ & $\true$ & $\efalse$ & $\true$ & $\efalse$ & $\true$ & $\false$ & $\efalse$ \\
                                $\false$ & $\false$ & $\etrue$ & $\false$ & $\etrue$ & $\true$ & $\true$ & $\etrue$
                        \end{tabular}
                \end{center}

                \begin{center}
                        \begin{tabular}{cc|ccccc}
                                $p$ & $q$ & $p \xor q$ & $p \lor q$ & $p \land q$ & $\lnot(p \land q)$ & $(p \lor q) \land \lnot(p \land q)$  \\
                                \hline
                                $\true$ & $\true$ & $\efalse$ & $\true$ & $\true$ & $\false$ & $\efalse$ \\
                                $\true$ & $\false$ & $\etrue$ & $\true$ & $\false$ & $\true$ & $\etrue$ \\
                                $\false$ & $\true$ & $\etrue$ & $\true$ & $\false$ & $\true$ & $\etrue$ \\
                                $\false$ & $\false$ & $\efalse$ & $\false$ & $\false$ & $\true$ & $\efalse$
                        \end{tabular}
                \end{center}

                \begin{center}
                        \begin{tabular}{cc|cccccc}
                                $p$ & $q$ & $p \xor q$ & $\lnot{q}$ & $p \land \lnot{q}$ & $\lnot{p}$ & $\lnot{p} \land q$ & $(p \land \lnot{q}) \lor (\lnot{p} \land q)$  \\
                                \hline
                                $\true$ & $\true$ & $\efalse$ & $\false$ & $\false$ & $\false$ & $\false$ & $\efalse$ \\
                                $\true$ & $\false$ & $\etrue$ & $\true$ & $\true$ & $\false$ & $\false$ & $\etrue$ \\
                                $\false$ & $\true$ & $\etrue$ & $\false$ & $\false$ & $\true$ & $\true$ & $\etrue$ \\
                                $\false$ & $\false$ & $\efalse$ & $\true$ & $\false$ & $\true$ & $\false$ & $\efalse$
                        \end{tabular}
                \end{center}

                Kako simbolno zapisati, da sta dve izražavi enaki? Lahko bi pisali
                \[\big((p, q) \mapsto p \shf q\big) = \big((p, q) \mapsto \lnot(p \land q)\big),\]
                ampak to je nekoliko nerodno in nepregledno. Kasneje (v razdelku~\note{o anonimnih funkcijah}) se bomo naučili $\lambda$-notacijo, s katero dobimo
                \[(\lam{(p, q) \in \tvs^2} p \shf q) = (\lam{(p, q) \in \tvs^2} \lnot(p \land q)),\]
                ampak to je še vedno nepregledno. Uveljavil se je običaj, da se izraze, ki so enakovredni v smislu, da dajo isti rezultat pri vsaki izbiri argumentov, poveže s simbolom $\equiv$, torej zapišemo
                \[p \shf q \equiv \lnot(p \land q).\]
                Konkretno za izraze v logiki se uporablja tudi $\sim$, se pravi, zapišemo lahko tudi
                \[p \shf q \sim \lnot(p \land q).\]
                V tej knjigi se bomo držali uporabe simbola $\equiv$. \davorin{Recimo. Po mojem je to boljše, ker lahko $\equiv$ uporabljamo še za druge funkcije (npr.~$f(x) \equiv 0$ pomeni, da je $f$ konstantno enaka $0$, medtem ko $f(x) = 0$ predstavlja enačbo, s katero iščemo ničle funkcije) in ker bomo kasneje $\sim$ uporabljali za ekvivalenčne relacije.}

                Med drugim smo s temi tabelami izpeljali tako imenovana \df{de Morganova zakona} za izjavno logiko \davorin{Verjetno je smiselno specificirati ``za izjavno logiko''. Imeli bomo namreč še zakona za predikatno logiko (za $\forall$ in $\exists$) ter za množice (za preseke in unije).}, ki povesta, kako negacija vpliva na konjunkcijo in disjunkcijo:
                \[\lnot(p \land q) \equiv \lnot{p} \lor \lnot{q},\]
                \[\lnot(p \lor q) \equiv \lnot{p} \land \lnot{q}.\]
                To je smiselno: kadar ni res, da veljata oba $p$ in $q$, vsaj eden od njiju ne velja. Kadar ni res, da velja vsaj eden od njiju, nobeden od njiju ne velja.

                Z resničnostnimi tabelami lahko preverimo še mnoge druge formule. \df{Zakon dvojne negacije} pravi $\lnot\lnot{p} \equiv p$, tj.~če dvakrat zanikamo izjavo, dobimo izjavo, enakovredno začetni. Poračunajmo tabelo.

                \begin{center}
                        \begin{tabular}{c|ccc}
                                $p$ & $\lnot{p}$ & $\lnot\lnot{p}$ & $p$ \\
                                \hline
                                $\true$ & $\false$ & $\etrue$ & $\etrue$ \\
                                $\false$ & $\true$ & $\efalse$ & $\efalse$
                        \end{tabular}
                \end{center}

                Spomnimo se: za poljubno dvomestno operacijo $\oper$ na neki množici $X$ rečemo, da je
                \begin{itemize}
                        \item
                                \df{izmenljiva} ali \df{komutativna}, kadar velja $a \oper b = b \oper a$ za vse $a, b \in X$ (na kratko: $a \oper b \equiv b \oper a$),
                        \item
                                \df{družilna} ali \df{asociativna}, kadar velja $(a \oper b) \oper c = a \oper (b \oper c)$ za vse $a, b, c \in X$ (na kratko: $(a \oper b) \oper c \equiv a \oper (b \oper c)$),
                        \item
                                \df{idempotentna} \davorin{a imamo slovenski izraz za to?}, kadar velja $a \oper a = a$ za vse $a \in X$ (torej $a \oper a \equiv a$).
                \end{itemize}

                Preverimo z resničnostno tabelo, da je konjunkcija komutativna, torej $p \land q \equiv q \land p$.

                \begin{center}
                        \begin{tabular}{cc|ccccc}
                                $p$ & $q$ & $p \land q$ & $q \land p$ \\
                                \hline
                                $\true$ & $\true$ & $\etrue$ & $\etrue$ \\
                                $\true$ & $\false$ & $\efalse$ & $\efalse$ \\
                                $\false$ & $\true$ & $\efalse$ & $\efalse$ \\
                                $\false$ & $\false$ & $\efalse$ & $\efalse$
                        \end{tabular}
                \end{center}

                Še hitreje lahko preverimo, da je konjunkcija idempotentna.

                \begin{center}
                        \begin{tabular}{c|cc}
                                $p$ & $p \land p$ & $p$ \\
                                \hline
                                $\true$ & $\etrue$ & $\etrue$ \\
                                $\false$ & $\efalse$ & $\efalse$
                        \end{tabular}
                \end{center}

                Kako pa preveriti, da je konjunkcija asociativna, torej $(p \land q) \land r \equiv p \land (q \land r)$? Vidimo, da v teh izrazih nastopajo tri spremenljivke in torej potrebujemo resničnostno tabelo, kjer upoštevamo vseh osem možnosti za izbiro $p$, $q$, $r$.

                \begin{center}
                        \begin{tabular}{ccc|cccc}
                                $p$ & $q$ & $r$ & $p \land q$ & $(p \land q) \land r$ & $q \land r$ & $p \land (q \land r)$ \\
                                \hline
                                $\true$ & $\true$ & $\true$ & $\true$ & $\etrue$ & $\true$ & $\etrue$ \\
                                $\true$ & $\true$ & $\false$ & $\true$ & $\efalse$ & $\false$ & $\efalse$ \\
                                $\true$ & $\false$ & $\true$ & $\false$ & $\efalse$ & $\false$ & $\efalse$ \\
                                $\true$ & $\false$ & $\false$ & $\false$ & $\efalse$ & $\false$ & $\efalse$ \\
                                $\false$ & $\true$ & $\true$ & $\false$ & $\efalse$ & $\true$ & $\efalse$ \\
                                $\false$ & $\true$ & $\false$ & $\false$ & $\efalse$ & $\false$ & $\efalse$ \\
                                $\false$ & $\false$ & $\true$ & $\false$ & $\efalse$ & $\false$ & $\efalse$ \\
                                $\false$ & $\false$ & $\false$ & $\false$ & $\efalse$ & $\false$ & $\efalse$
                        \end{tabular}
                \end{center}

                To pomeni, da lahko v izrazih, kjer nastopa več zaporednih konjunkcij, spuščamo oklepaje: namesto $p \land (\lnot{q} \land r)$ pišemo kar $p \land \lnot{q} \land r$.

                Enako velja tudi za disjunkcijo.

                \begin{naloga}
                        Dokaži, da je disjunkcija komutativna, asociativna in idempotentna!
                \end{naloga}

                Preostali dvomestni vezniki, ki smo jih omenili, ne zadoščajo vsem trem lastnostim naenkrat.

                \begin{naloga}
                        Preveri, kateri znani dvomestni izjavni vezniki so komutativni, asociativni oziroma idempotentni!
                \end{naloga}

                Ko rešite zgornjo vajo, boste med drugim opazili: implikacija ni komutativna. To pomeni, da lahko definiramo nov izjavni veznik $\revimpl$ na naslednji način: $p \revimpl q \dfeq q \impl p$ za vse $p, q \in \tvs$. Z drugimi besedami, $\revimpl$ je dan s sledečo resničnostno tabelo.
                \begin{center}
                        \begin{tabular}{cc|c}
                                $p$ & $q$ & $p \revimpl q$ \\
                                \hline
                                $\true$ & $\true$ & $\true$ \\
                                $\true$ & $\false$ & $\true$ \\
                                $\false$ & $\true$ & $\false$ \\
                                $\false$ & $\false$ & $\true$
                        \end{tabular}
                \end{center}

                \note{dokazi s pomočjo resničnostnih tabel še vseh ostalih formul, ki jih hočemo imeti, med drugim distributivnosti}

                Do zdaj smo omenili zgolj nekaj posamičnih izjavnih veznikov. Koliko pa je vseh skupaj? Spomnimo se, da je $n$-mestni izjavni veznik definiran kot preslikava $\tvs^n \to \tvs$. Množica $\tvs^n$ vsebuje vse urejene $n$-terice elementov $\true$ in $\false$; teh je $2^n$ (za vsako od $n$ mest v $n$-terici imamo dve možnosti in vse te izbire so neodvisne med sabo). Za vsako od teh $2^n$ večteric imamo dve možnosti, kam jo preslikamo: v $\true$ ali v $\false$. Vseh možnosti --- torej vseh $n$-mestnih veznikov --- je potemtakem $2^{2^n}$. (Vseh izjavnih veznikov, ko dopuščamo vse možne $n$, je seveda neskončno.)

                Za boljšo predstavo si oglejmo vse $n$-mestne veznike za majhne $n \in \NN$. Prva možnost je $n = 0$. Formula nam pravi, da je število ničmestnih izjavnih veznikov enako $2^{2^0} = 2^1 = 2$. Kaj pomeni, da pri nič vhodnih podatkih vrnemo $\true$ ali $\false$? To pomeni, da preprosto izberemo resničnostno vrednost --- z drugimi besedami, ničmestni izjavni vezniki so isto kot resničnostne vrednosti.

                Koliko je vseh enomestnih izjavnih veznikov? Formula pravi $2^{2^1} = 2^2 = 4$. Zapišimo vse možnosti.

                \begin{center}
                        \begin{tabular}{c|cccc}
                                $p$ &&&& \\
                                \hline
                                $\true$ & $\true$ & $\false$ & $\true$ & $\false$ \\
                                $\false$ & $\true$ & $\false$ & $\false$ & $\true$
                        \end{tabular}
                \end{center}

                Vidimo: enomestni izjavni vezniki so obe konstantni funkciji na $\tvs$, identiteta na $\tvs$ in negacija.

                Kar se dvomestnih veznikov tiče, vidimo, da jih je $2^{2^2} = 2^4 = 16$.

                \begin{naloga}
                        Preveri, da so vsi dvomestni vezniki natanko: konstanta z vrednostjo $\top$, projekcija na prvo komponento (tj.~$(p, q) \mapsto p$), projekcija na drugo komponento (tj.~$(p, q) \mapsto q$), konjunkcija $\land$, disjunkcija $\lor$, implikacija $\impl$, povratna implikacija $\revimpl$, ekvivalenca $\lequ$ in negacije vseh teh.
                \end{naloga}

                Tromestnih veznikov je že $2^{2^3} = 2^8 = 256$ in ne bomo vseh naštevali. Kako pa bi kakega dobili? Preprost način je, da vzamemo tri spremenljivke in jih združimo z večimi znanimi vezniki, na primer $(p, q, r) \mapsto p \land \lnot{q} \impl r$.\footnote{Načeloma sploh ni nujno, da vse tri spremenljivke dejansko uporabimo. Na primer, $(p, q, r) \mapsto p \land q$ še vedno podaja tromestni veznik, saj gre za preslikavo $\tvs^3 \to \tvs$.}

                Seveda se pojavi vprašanje, ali obstajajo izjavni vezniki, ki jih ne bi mogli sestaviti iz osnovnih. Izkaže se, da je odgovor nikalen: \emph{vsak veznik (ne glede na mestnost) je možno izraziti z osnovnimi}; pravzaprav zadostujejo že $\lnot$, $\land$ in $\lor$.

                Ideja je sledeča. Katerikoli izjavni veznik je oblike $V\colon \tvs^n \to \tvs$ in v celoti podan z resničnostno tabelo. Vzemimo konkreten primer; naj bo $V$ tromestni veznik, podan z naslednjo tabelo.

                \begin{center}
                        \begin{tabular}{ccc|c}
                                $p$ & $q$ & $r$ & $V(p, q, r)$ \\
                                \hline
                                $\true$ & $\true$ & $\true$ & $\false$ \\
                                $\true$ & $\true$ & $\false$ & $\true$ \\
                                $\true$ & $\false$ & $\true$ & $\true$ \\
                                $\true$ & $\false$ & $\false$ & $\false$ \\
                                $\false$ & $\true$ & $\true$ & $\true$ \\
                                $\false$ & $\true$ & $\false$ & $\true$ \\
                                $\false$ & $\false$ & $\true$ & $\false$ \\
                                $\false$ & $\false$ & $\false$ & $\false$
                        \end{tabular}
                \end{center}

                Tedaj lahko rečemo: $V$ je resničen tedaj, ko smo v 2., 3., 5.~ali 6.~vrstici. Kdaj smo v drugi vrstici? Točno tedaj, ko $p$ in $q$ veljata, $r$ pa ne, se pravi, ko velja $p \land q \land \lnot{r}$. Podobno naredimo še za preostale vrstice: tretja je določena s $p \land \lnot{q} \land r$, peta z $\lnot{p} \land q \land r$ in šesta z $\lnot{p} \land q \land \lnot{r}$. Potemtakem lahko zapišemo:
                \[V(p, q, r) \equiv (p \land q \land \lnot{r}) \lor (p \land \lnot{q} \land r) \lor (\lnot{p} \land q \land r) \lor (\lnot{p} \land q \land \lnot{r}).\]
                Temu rečemo \df{disjunktivna normalna oblika} (s kratico DNO) veznika $V$.

                Obstaja še dualna oblika take izražave. Lahko si rečemo tudi, da je $V$ resničen, kadar nismo v 1., 4., 7.~oz.~8.~vrstici. Kdaj nismo v prvi vrstici? Kadar niso vsi $p$, $q$, $r$ resnični, torej ko je vsaj eden od njih neresničen --- s formulo $\lnot{p} \lor \lnot{q} \lor \lnot{r}$. Kdaj nismo v četrti vrstici? Ko ni res, da je $p$ resničen, $q$ in $r$ pa ne, torej ko prekršimo vsaj enega teh pogojev, kar nam da formulo $\lnot{p} \lor q \lor r$. Podobno sklepamo, da nismo v sedmi vrstici, kadar velja $p \lor q \lor \lnot{r}$, in da nismo v osmi vrstici, kadar velja $p \lor q \lor r$. To nam da sledečo izražavo za $V$:
                \[V(p, q, r) \equiv (\lnot{p} \lor \lnot{q} \lor \lnot{r}) \land (\lnot{p} \lor q \lor r) \land (p \lor q \lor \lnot{r}) \land (p \lor q \lor r).\]
                Temu rečemo \df{konjunktivna normalna oblika} (s kratico KNO) veznika $V$.

                Spremenljivkam in njihovim negacijam z eno besedo rečemo \df{literali}. Disjunktivna normalna oblika je torej disjunkcija konjunkcij literalov, konjunktivna normalna oblika pa konjunkcija disjunkcij literalov.

                Iz tega primera je jasno, kako postopamo za poljuben izjavni veznik in zanj zapišemo DNO ali KNO. Opazimo: dolžina posamičnega člena, ki ga omejujejo oklepaji, je vedno enaka (vsebuje toliko literalov, kolikor je mestnost veznika), število teh členov pa razberemo iz stolpca, ki podaja vrednosti veznika v resničnostni tabeli. V primeru DNO je to število enako številu resnic $\true$, v primeru KNO pa številu neresnic $\false$. V zgornjem primeru sta bili DNO in KNO enako dolgi, ker smo imeli štiri $\true$ in $\false$, v splošnem pa se nam morda bolj splača uporabiti eno obliko kot drugo. Na primer, DNO implikacije se glasi $p \impl q \equiv (p \land q) \lor (\lnot{p} \land q) \lor (\lnot{p} \land \lnot{q})$, KNO pa je precej krajša: $p \impl q \equiv \lnot{p} \lor q$.

                Vidimo pa, da tu naletimo na problem: kaj se zgodi, če se katera resničnostna vrednost v stolpcu veznika sploh ne pojavi --- z drugimi besedami, kaj če je funkcija, ki podaja veznik, konstantna? Najprej dajmo takim veznikom ime: izjavni veznik, ki je pri vseh argumentih resničen, se imenuje \df{istorečje} ali \df{tavtologija}, izjavni veznik, ki je vedno neresničen, pa se imenuje \df{protislovje} ali \df{kontradikcija}.

                Za istorečje lahko vedno (ne glede na mestnost) zapišemo DNO (ki je sicer najdaljša možna), medtem ko bi KNO načeloma bila konjunkcija nič členov. Je to smiselno? V bistvu ja: če zahtevamo, da hkrati velja nič pogojev, je naša zahteva vedno izpolnjena. V tem smislu je konjunkcija nič členov enaka $\true$.

                Poglejmo podobne primere iz računstva. Kaj je vsota nič členov? Odgovor je seveda $0$. To je enota za seštevanje, kar je smiselno: če nič členom prištejemo en člen, moramo imeti zgolj ta člen. Podobno sklepamo: zmnožek nič členov je enota za množenje $1$ --- če nič faktorjem dodamo še en faktor, imamo skupaj zgolj ta faktor. Spomni se tudi: $a^0 = 1$ in $0! = 1$. To, da je ničkratna uporabe neke operacije enaka enoti za to operacijo, se izide tudi za konjunkcijo: dejansko velja $p \land \true \equiv p \equiv \true \land p$ (preveri z resničnostno tabelo!).

                Enak razmislek velja za protislovje. Zanj lahko zapišemo KNO na običajen način, medtem ko bi DNO bila disjunkcija nič členov. Smiselno je, da je disjunkcija nič členov enaka $\false$, tako zaradi tega, ker je $\false$ enota za disjunkcijo (preveri!), kot zaradi čisto intuitivnega razmisleka: kdaj je vsaj en člen od nič členov resničen? Nikoli.

                Vseeno je nekoliko nerodno delati s konjunkcijo ali disjunkcijo nič členov --- kako točno bi to zapisali? Da velja $V(p_1, p_2, \ldots, p_n) \equiv $? Če nič ne zapišemo, kako sploh vemo, ali smo mislili na ničkratno konjunkcijo, disjunkcijo ali katerokoli drugo operacijo? Nekateri se zato preprosto dogovorijo, da ne dopuščajo ničkratnih operacij v DNO oz.~KNO in potem štejejo, da istorečja nimajo KNO, protislovja pa ne DNO.

                Tudi če ne dopuščamo ničkratnih operacij, pa še vedno velja: vsak izjavni veznik z mestnostjo vsaj $1$ ima vsaj eno od DNO oz.~KNO in ga torej lahko izrazimo samo z negacijo, konjunkcijo in disjunkcijo. Družini izjavnih veznikov, s katerimi lahko izrazimo vse veznike z mestnostjo vsaj $1$, rečemo \df{poln nabor}. Na kratko lahko torej rečemo, da je $\set{\lnot, \land, \lor}$ poln nabor.

                Jasno, če je neka množica veznikov poln nabor, je tudi vsaka njena nadmnožica poln nabor. Sledi, da je tudi na primer $\set{\lnot, \land, \lor, \impl}$ poln nabor.

                Spomnimo se zdaj de Morganovih zakonov in zakona o dvojni negaciji --- iz njih lahko izpeljemo $p \land q \equiv \lnot(\lnot{p} \lor \lnot{q})$ in $p \lor q \equiv \lnot(\lnot{p} \land \lnot{q})$. Se pravi, konjunkcijo lahko izrazimo z disjunkcijo in negacijo in prav tako lahko disjunkcijo izrazimo s konjunkcijo in negacijo. To pomeni, da sta že $\set{\lnot, \lor}$ in $\set{\lnot, \land}$ polna nabora! Se pravi, vse veznike s pozitivno mestnostjo je možno izraziti že samo z dvema.

                Je možno iti še dlje in najti en sam veznik, s katerim lahko izrazimo ostale? Odgovor je da: $\set{\shf}$ in $\set{\luk}$ sta polna nabora. (Izkaže se, da sta to edina taka veznika med dvomestnimi vezniki.)

                \begin{naloga}\label{naloga:polni-nabori-z-enim-veznikom}
                        \
                        \begin{enumerate}
                                \item
                                        Izrazi negacijo samo z veznikom $\shf$. Izrazi še konjunkcijo ali disjunkcijo samo z veznikom $\shf$. Sklepaj, da je $\set{\shf}$ poln nabor.
                                \item
                                        Izrazi negacijo samo z veznikom $\luk$. Izrazi še konjunkcijo ali disjunkcijo samo z veznikom $\luk$. Sklepaj, da je $\set{\luk}$ poln nabor.
                        \end{enumerate}
                \end{naloga}

                \davorin{Bi na tem mestu predebatirali preklopna vezja?}

                \davorin{Mogoče lahko zavoljo celovitosti podamo karakterizacijo polnih naborov kot izrek (in se za dokaz skličemo na literaturo). Nabor je poln, kadar za vsako sledečih lastnosti obstaja veznik v njem, ki jo prekrši: ohranjanje resnice, ohranjanje neresnice, monotonost, sebi-dualnost, afinost (kot polinom Žegalkina).}


        \section{Predikati in kvantifikatorji}

                \note{``Lastnostim'' elementov množic, ki smo jih prej uporabljali za podajanje podmnožic in pri kvantifikatorjih, zdaj ``uradno'' rečemo \df{predikati} in jih formalno definiramo: predikat na množici $X$ je preslikava $X \to \tvs$. Karakteristične preslikave podmnožic. Spomnimo se kvantifikatorjev in jih definiramo kot preslikave $\tvs^X \to \tvs$. Povemo, da lahko imajo predikati več spremenljivk in da lahko kvantificiramo po samo nekaterih (dobimo torej preslikave oblike $\tvs^{X \times Y} \to \tvs^Y$). Vezane, proste spremenljivke. Pravila, ki veljajo za kvantifikatorje (de Morgan itd.).}


\section{Vaje}

\begin{vaja}
  Preverite, da je $(p \Rightarrow q) \lor (q \Rightarrow p)$ tavtologija z resničnostno tabelo in s
  poenostavljanjem.
\end{vaja}

\anja{Ali želimo imeti toliko nalog iz polnih naborov? Jaz sem samo skopirala te naloge od prejšnjih vaj.}

\begin{vaja}
Pokaži, da so naslednji nabori izjavnih povezav polni.
\begin{enumerate}
 \item $\set{ \land, \xor, \true }$
 \item $\set{ \impl, \lnot }$
 \item $\set{ \impl, \false }$
 \item $\set{ \land, \false, \true, \Delta }$, kjer je $\Delta(p,q,r) \equiv p \xor q \xor r$.
\end{enumerate}
\end{vaja}

\begin{vaja}
 Naslednje izjave izrazi le z veznikoma $\lnot$ in $\impl$.
\begin{itemize}
  \item $p\land q$
  \item $(p\xor q)\lequ (p\lor q)$
  \item $p\luk q$
\end{itemize}
\end{vaja}

\begin{vaja}
Pokaži, da spodnja nabora izjavnih veznikov {\em nista} polna:
\begin{itemize}
 \item  $\set{\land, \lequ }$,
 \item  $\set{ \land, \xor }$.
\end{itemize}
\end{vaja}

%\begin{vaja}
%Trimestna izjavna povezava $\ifthen{p}{q}{r}$ je določena takole:
%\[
%\ifthen{p}{q}{r} = \left\{ \begin{array}{ll}
% q, & p = 1 \\
% r, & p = 0.
%\end{array} \right.
%\]
%\begin{itemize}
% \item Izrazi povezavo $\ifthen{p}{q}{r}$ čim krajše z običajnimi izjavnimi povezavami.
%  Pokaži, da velja enakovrednost $\ifthen{p}{q}{r} \sim (p \Rightarrow q) \land (\lnot p \Rightarrow r)$.
% \item Pokaži, da je nabor $\{ 0, 1, \ifthen{p}{q}{r} \}$ poln.
% \item Izrazi z naborom iz prejšnje točke izjavo $p \Leftrightarrow q$.
%\end{itemize}
%\end{vaja}

\begin{vaja}
Kateri izmed naslednjih izjavnih veznikov sestavlja poln nabor?
\begin{itemize}
 \item $\Lambda(p,q,r) \equiv p \impl (q \lor r)$
 \item $\Lambda(p,q,r) \equiv (p \shf q) \luk r$
 \item $\Lambda(p,q,r) \equiv (\lnot p \land \lnot r) \impl q$
 \item $\Lambda(p,q,r) \equiv p \impl (q \impl \lnot r)$
\end{itemize}
\end{vaja}


\begin{vaja}
Ali sestavljata izjavni povezavi $\set{ \impl, \not\impl }$, kjer je $p \not\impl q \equiv \lnot (p \impl q)$, poln nabor?
\end{vaja}

\begin{vaja}
Izjavna povezava $\square$ je določena z $p \square q \equiv p \land \lnot q$. Ugotovi, kateri nabori od spodnjih naborov izjavnih povezav so polni.
\begin{itemize}
 \item $\set{ \square }$
 \item $\set{ \square, \lnot }$
 \item $\set{ \square, \impl }$
\end{itemize}
\end{vaja}

\begin{vaja}
Preklopna vezja so sestavljena iz stikal in žic. Stikala so lahko vklopljena ali izklopljena, glede na njihovo stanje pa je odvisno, ali bo tok tekel po žici ali ne. Denimo, da imamo dve stikali $A$ in $B$. Če sta stikali vezani zaporedno, tj. 
\begin{center}
\begin{circuitikz} \draw
(0,0) to [switch, l^=$A$] (2,0) to[switch, l^=$B$] (4,0);
\end{circuitikz}
\end{center}
potem tok teče, kadar sta obe stikali vklopljeni. Če pa sta stikali vezani vzporedno, tj.
\begin{center}
\begin{circuitikz} \draw
(0,0) to [short, -*] (1,0) to [short] (1,1) to [switch, l^=$A$] (3,1) to[short] (3,0)
(1,0) to [short] (1,-1) to [switch, l^=$B$] (3,-1) to[short] (3,0) to[short, *-] (4,0);
\end{circuitikz}
\end{center}
potem tok teče, če je vklopljeno stikalo $A$ ali stikalo $B$. Tako lahko simuliramo logične veznike. Stikala so izjavne spremenljivke, takšni bloki pa predstavljajo vrata ``in'' ter ``ali''. Vrata za logične veznike predstavljamo z naslednjimi simboli:
\begin{center}
\begin{circuitikz} \draw
(0,0) node[or port] (myor1) {}
(0,2) node[and port] (myand1) {}
(6,0) node[not port](mynot1){}
(6,2) node[nor port](mynor){}
(2,0) node(o) {ali}
(2,2) node(a) {in}
(9,0) node(n) {ne}
(9,2) node(no) { niti};
%(myand1.out) -- (myxnor.in 1)
%(myand2.out) -- (myxnor.in 2);
\end{circuitikz}
\end{center}

Prvi dve vezji lahko torej z vrati zapišemo takole
\begin{center}
\begin{circuitikz} \draw
(0,4) to[switch, l^=$A$] (1,4)
(3,2) node[and port] (myand1) {}
(0,0) to[switch,  l^=$B$] (1,0)
(1,4) -- (myand1.in 1)
(1,0)-- (myand1.in 2)
(5,2) node(in){ter}
(6,4) to[switch, l^=$A$] (7,4)
(9,2) node[or port] (myand1) {}
(6,0) to[switch,  l^=$B$] (7,0)
(7,4) -- (myand1.in 1)
(7,0)-- (myand1.in 2);
\end{circuitikz}
\end{center}

Andrej prenavlja stanovanje in načrtuje električno napeljavo. Ker se mu pred spanjem ne ljubi vstajati, da bi ugasnil luč, si želi v spalnici imeti dve stikali, eno ob postelji in eno pri vhodu. Seveda pa morata obe stikali delovati, torej ko pritisnemo na katero koli stikalo, se mora luč prižgati ali pa ugasniti, če je že prižgana. Pri izdelavi električnega omrežja sme Andrej uporabiti le vrata ``in'', ``ali'' ter ``ne''. Ker pa so vrata draga, si želi uporabiti čim manj vrat. Pomagajte Andreju načrtovati vezje za njegovo spalnico. Kaj pa če lahko uporabi samo vrata ``in'' ter ``ali''? Ali lahko uporabi le vrata ``niti'' ($\downarrow$)?
\begin{resitev}
Imamo dve stikali, imenujmo ju $p$ in $q$. Opazujemo, kdaj luč sveti. Na začetku sta obe stikali ugasnjeni in luč ne sveti. Če prižgemo eno stikalo, mora luč zasvetiti. Če prižgemo nato še drugo stikalo mora luč ugasniti. Ugotovimo, da je luč prižgana, ko je prižgano natanko eno stikalo. To ponazorimo v naslednji tabeli:

\begin{center}
                        \begin{tabular}{cc|c}
                                $p$ & $q$ & \text{ luč sveti } \\
                                \hline
                                $\true$ & $\true$& $\false$ \\
                                $\true$ & $\false$  & $\true$ \\
                                $\false$ & $\true$ & $\true$ \\
                                $\false$ & $\false$  & $\false$
                        \end{tabular}
\end{center}
Opazimo, da ima to enako tabelo, kot izjava $p  \xor q$. Torej moramo to izjavo izraziti z izjavnimi vezniki $\land, \lor$ in $\neg$. En način, kako to naredimo je, da zapišemo $p \xor q \equiv (p \lor q) \land \neg (p \land q)$, in tako konstruiramo vezja z vrati ``in'', ``ali'' in negacijo takole:
\begin{center}
\begin{circuitikz} \draw
(2,0) node[anchor=north] (q) {}
(2,8) node[anchor=south] (p) {}
(4 ,2) node[or port] (myor1) {}
(4,6) node[and port] (myand1) {}
(6,6) node[not port](mynot1){}
(8,4) node[and port](myand2){}
(0,0) to[switch, l^=$q$, -*] (2,0)
(0,8) to[switch,  l^=$p$, -*] (2,8)
(p) -- (myor1.in 1)
(q) -- (myor1.in 2)
(p) -- (myand1.in 1)
(q) -- (myand1.in 2)
(myand1.out) -- (mynot1.in)
(mynot1.out) -- (myand2.in 1)
(myor1.out) -- (myand2.in 2)
(myand2.out) to [lamp] (10,4);
%(myand1.out) -- (myxnor.in 1)
%(myand2.out) -- (myxnor.in 2);
\end{circuitikz}
\end{center}
Le z veznikoma $\land$ in $\lor$ tega ne moremo storiti, saj veznika ne predstavljata polnega nabora. Z uporabo zgolj Łukasiewiczevega veznika pa je to mogoče, saj predstavlja poln nabor. 
\end{resitev}
\end{vaja}


%%% Local Variables:
%%% mode: latex
%%% TeX-master: "ucbenik-lmn"
%%% End:

\input{dokazovanje.tex}
\input{konstrukcije.tex}
\input{preslikave.tex}
\input{relacije.tex}
\chapter{Diskretne strukture}
\label{chap:strukture}


%%%%%%%%%%%%%%%%%%%%%%%%%%%%%%%%%%%%%%%%%%%%%%%%%%%%%%%%%%%%%%%%%%%%%%
\section{Relacije}
\label{sec:relacije}

%%%%%%%%%%%%%%%%%%%%%%%%%%%%%%%%%%%%%%%%%%%%%%%%%%%%%%%%%%%%%%%%%%%%%%
\section{Grafi}
\label{sec:grafi}

%%%%%%%%%%%%%%%%%%%%%%%%%%%%%%%%%%%%%%%%%%%%%%%%%%%%%%%%%%%%%%%%%%%%%%
\section{Urejenost}
\label{sec:urejenost}




%%% Local Variables: 
%%% mode: latex
%%% TeX-master: "lmn"
%%% End: 

\input{stevilske-mnozice.tex}
\input{indukcija.tex}
\input{kumulativna-hierarhija.tex}
\input{kardinalna-stevila.tex}
\input{ordinalna-stevila.tex}
\Closesolutionfile{resitve}
\chapter{Rešitve vaj}
\begin{Resitev}{2.1}
Množica~$A$ ima kvečjemu en element, tj.~množica~$A$ je bodisi prazna bodisi enojec. Tudi: množica~$A$ je podmnožica kakega enojca oz.~edina preslikava $A \to \one$ je injektivna.
\end{Resitev}

\appendix
\input{razlaga-makrojev.tex}
\else
\fi

\end{document}

%%% Local Variables:
%%% mode: latex
%%% TeX-master: "ucbenik-lmn"
%%% End:
